\documentclass[twoside]{MATH77}
\usepackage{multicol}
\usepackage[fleqn,reqno,centertags]{amsmath}
\begin{document}
\begmath 3.3 Random Numbers: Exponential, Rayleigh, and Poisson

\silentfootnote{$^\copyright$1997 Calif. Inst. of Technology, \thisyear \ Math \`a la Carte, Inc.}

\subsection{Purpose}

Generate pseudorandom numbers from the exponential and Rayleigh
distributions and pseudorandom integers from the Poisson distribution.

\subsection{Usage}

\subsubsection{Generating exponential pseudorandom numbers}

The density function for the exponential distribution with mean and standard
deviation, $\mu $, has the value zero for $x < 0$ and $\mu^{-1} \exp (-x/\mu )$ for $%
x \geq 0$. The cumulative distribution function has the value zero for $x <
0 $ and $1- \exp (- x/\mu )$ for $x \geq 0$. If $u$ is a random variable
having the uniform distribution on [0,~1] then $x = - \mu \ \log u$ is a
random variable having the exponential distribution with mean and standard
deviation, $\mu .$

\paragraph{Program Prototype, Single Precision}

\begin{description}
\item[REAL]  \ {\bf SRANE, XMEAN, X}
\end{description}

Assign a value to XMEAN.
$$
\fbox{{\bf X = SRANE(XMEAN)}}
$$
\paragraph{Argument Definitions}

\begin{description}
\item[XMEAN]  \ [in] Specifies the mean and standard deviation of the
desired exponential distribution. Require XMEAN $>0.$

\item[SRANE]  \ [out] The function returns a nonnegative pseudorandom number
from the exponential distribution with mean and standard deviation equal to
XMEAN.
\end{description}

\subsubsection{Generating Rayleigh pseudorandom numbers}

The density function for the Rayleigh distribution with scaling parameter, $%
\alpha $, has the value zero for $x < 0$ and
\begin{equation*}
(x/\alpha ^2) \exp (-x^2/2\alpha ^2)
\end{equation*}
for $x \geq 0$. The cumulative distribution function has the value zero for $%
x < 0$ and
\begin{equation*}
1 - \exp(-x^2/2 \alpha ^2)
\end{equation*}
for $x \geq 0$. The mean and standard deviation of this distribution are
\begin{equation*}
\mu = \alpha \sqrt{\pi /2} \approx 1.25331\ \alpha
\end{equation*}
and
\begin{equation*}
\sigma = \alpha \sqrt{2 - \pi /2} \approx 0.655136\ \alpha
\end{equation*}
If $u$ is a random variable having the uniform distribution on [0,~1] then
\begin{equation*}
x = \alpha \sqrt{-2 \log u}
\end{equation*}
is a random variable having the Rayleigh distribution with scaling
parameter, $\alpha .$

\paragraph{Program Prototype, Single Precision}

\begin{description}
\item[REAL]  \ {\bf SRANR, ALPHA, X}
\end{description}

Assign a value to ALPHA.
$$
\fbox{{\bf X = SRANR(ALPHA)}}
$$
\paragraph{Argument Definitions}

\begin{description}
\item[ALPHA]  \ [in] Specifies the scaling of the desired Rayleigh
distribution. Require ALPHA $>0$. The distribution will have mean = ALPHA $%
\times \sqrt{\pi /2}$ and variance = ALPHA$^2\times (2-\pi /2).$

\item[SRANR]  \ [out] The function returns a nonnegative pseudorandom number
from the Rayleigh distribution with scaling parameter, ALPHA.
\end{description}

\subsubsection{Generating Poisson pseudorandom integers}

The Poisson distribution with mean and variance, $\mu $, is defined over
nonnegative integers. The nonnegative integer, $k$, occurs with probability $%
p_k$ given by%
\begin{equation*}
p_k=e^{-\mu }\frac{\mu ^k}{k!}
\end{equation*}
\paragraph{Program Prototype, Single Precision}

\begin{description}
\item[REAL]  \ {\bf XMEAN}

\item[INTEGER]  \ {\bf ISRANP, K}
\end{description}

Assign a value to XMEAN.
$$
\fbox{{\bf K = ISRANP(XMEAN)}}
$$

\paragraph{Argument Definitions}

\begin{description}
\item[XMEAN]  \ [in] Specifies the mean and variance of the desired Poisson
distribution. XMEAN must be positive and not so large that exp($-$XMEAN)
would underflow. For example if the underflow limit is $10^{-38}$, XMEAN
must not exceed~87. This subprogram requires more computing time for larger
values of XMEAN or if XMEAN is changed frequently. See Section D.

\item[ISRANP]  \ [out] The function returns a nonnegative pseudorandom
integer from the Poisson distribution with mean XMEAN.
\end{description}

\subsubsection{Modifications for Double Precision}

Change the names SRANE, SRANR, and ISRANP to DRANE, DRANR, and IDRANP
respectively, and change the REAL type statements above to
DOUBLE PRECISION. Note particularly that if either of the function names
DRANE or DRANR is used it must be typed DOUBLE PRECISION either explicitly
or via an IMPLICIT statement.

\subsection{Examples and Remarks}

The programs DRSRANE, DRISRANP, and DRSRANR demonstrate, respectively, the
use of SRANE, ISRANP, and SRANR. These programs use SSTAT1 and SSTAT2, or
ISSTA1 and ISSTA2 to compute and print statistics and a histogram based on a
sample of~10000 numbers each.

To fetch or set the seed used in the underlying pseudorandom integer
sequence use the subroutines described in Chapter~3.1.

\subsection{Functional Description}

\subparagraph{Method}

The exponential random number is computed as $x = -\mu \ \log u$ where $u$
is a random number from the uniform distribution on [0,~1].

The Rayleigh random number is computed as $x = \alpha \sqrt{-2 \log u}$
where $u$ is a random number from the uniform distribution on [0,~1].

The Poisson subprogram uses ideas from \cite{Snow:1968:A342}.  The method
begins by obtaining a random number, $u$, from the uniform distribution on
[0,~1].  Then the probabilities $p_0$, $p_1$, ..., defined above in
Section B.3, are summed until the sum reaches or exceeds $u$.  The index
of the last term in the sum is then returned as the Poisson random
integer.

To improve efficiency on the assumption that the subprogram may be
referenced many successive times with the value of XMEAN remaining
unchanged, the partial sums through at most the term $p_{84}$ are stored in
an internal array as they are computed. On subsequent references, if the
value of XMEAN has not been changed, previously computed partial sums can be
tested without the need to recompute them. The testing starts at the index
nearest to the value, XMEAN, since these indices have the highest
probabilities of being selected.

These subprograms obtain uniform pseudorandom numbers by calling SRANUA or
DRANUA, using the array in common block /RANCMS/ or /RANCMD/ as a buffer as
described in Chapter~3.1.

Values returned as double-precision random numbers will have random bits
throughout the word, however the quality of randomness should not be
expected to be as good in a low-order segment of the word as in a high-order
part.

\bibliography{math77}
\bibliographystyle{math77}

\subsection{Error Procedures and Restrictions}

In subprograms SRANE, DRANE, SRANR, and DRANR the input parameter should be
positive, however no test is made of this. The input parameter is simply
used as a multiplicative factor.

Subprogram ISRANP will issue an error message and return the value $-1$ if
XMEAN\ $\leq $ 0 or if XMEAN $\geq -0.5\times \log \,($ underflow limit ).

\subsection{Supporting Information}

\begin{tabular}{@{\bf}l@{\hspace{5pt}}l}
\bf Entry & \hspace{.35in} {\bf Required Files}\vspace{2pt} \\
DRANE & \parbox[t]{2.7in}{\hyphenpenalty10000 \raggedright
DRANE, ERFIN, ERMSG, RANPK1, RANPK2\rule[-5pt]{0pt}{8pt}}\\
DRANR & \parbox[t]{2.7in}{\hyphenpenalty10000 \raggedright
DRANR, ERFIN, ERMSG, RANPK1, RANPK2\rule[-5pt]{0pt}{8pt}}\\
IDRANP & \parbox[t]{2.7in}{\hyphenpenalty10000 \raggedright
AMACH, ERFIN, ERMSG, IDRANP, RANPK1, RANPK2, DERM1, DERV1\rule[-5pt]{0pt}{8pt}}\\
ISRANP & \parbox[t]{2.7in}{\hyphenpenalty10000 \raggedright
AMACH, ERFIN, ERMSG, ISRANP, RANPK1, RANPK2, SERM1, SERV1\rule[-5pt]{0pt}{8pt}}\\
SRANE & \parbox[t]{2.7in}{\hyphenpenalty10000 \raggedright
ERFIN, ERMSG, RANPK1, RANPK2, SRANE\rule[-5pt]{0pt}{8pt}}\\
SRANR & \parbox[t]{2.7in}{\hyphenpenalty10000 \raggedright
ERFIN, ERMSG, RANPK1, RANPK2, SRANR}\\\end{tabular}

Based on subprograms written for JPL by Stephen L. Richie, Heliodyne Corp.,
and Wiley R. Bunton, JPL, 1969. Adapted to Fortran~77 for the JPL MATH77
library by C. L. Lawson and S. Y. Chiu, JPL, April~1987.

1991 November: Lawson reorganized and renamed common blocks.


\begcodenp
\lstset{language=[77]Fortran,showstringspaces=false}
\lstset{xleftmargin=.8in}

\centerline{\bf \large DRSRANE}\vspace{10pt}
\lstinputlisting{\codeloc{srane}}
\newpage

\vspace{30pt}\centerline{\bf \large ODSRANE}\vspace{10pt}
\lstset{language={}}
\lstinputlisting{\outputloc{srane}}
\newpage

\lstset{language=[77]Fortran,showstringspaces=false}
\lstset{xleftmargin=.8in}

\centerline{\bf \large DRSRANR}\vspace{10pt}
\lstinputlisting{\codeloc{sranr}}
\newpage

\vspace{30pt}\centerline{\bf \large ODSRANR}\vspace{10pt}
\lstset{language={}}
\lstinputlisting{\outputloc{sranr}}

\newpage
\lstset{language=[77]Fortran,showstringspaces=false}
\lstset{xleftmargin=.8in}

\centerline{\bf \large DRISRANP}\vspace{0pt}
\lstinputlisting{\codeloc{isranp}}

\vspace{20pt}\centerline{\bf \large ODISRANP}\vspace{5pt}
\lstset{language={}}
\lstinputlisting{\outputloc{isranp}}
\end{document}
