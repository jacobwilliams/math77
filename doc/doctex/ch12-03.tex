\documentclass[twoside]{MATH77}
\usepackage{multicol}
\usepackage[fleqn,reqno,centertags]{amsmath}
\begin{document}
\begmath 12.3 Table Look-up With Hermite Cubic Interpolation

\silentfootnote{$^\copyright$1997 Calif. Inst. of Technology, \thisyear \ Math \`a la Carte, Inc.}

\subsection{Purpose}

Given a table of values of a function and its first derivative, this
subroutine does table look-up and Hermite cubic interpolation to compute an
interpolated value of the function for an arbitrary argument. Optionally the
subroutine can compute values of the first, second, or third derivative of
the interpolation function.  All features here are also available,
with more generality, in Chapter~12.1.

The interpolated function will match the given data at the tabular points,
and thus will have a continuous value and first derivative. In general the
second and third derivatives will have jump discontinuities at the tabular
points. It is possible, of course, for the data to satisfy relations that
give rise to continuity of second or third derivatives in the interpolated
function. In particular, data from the cubic spline fitting programs, DC2FIT
or SC2FIT of Chapter~11.4, will produce continuous second derivatives.

\subsection{Usage}

\subsubsection{Program Prototype, Single Precision}

\begin{description}
\item[INTEGER]  \ {\bf NDERIV, NTAB}

\item[REAL]  \ {\bf SHINT, X, XTAB}($\geq $NTAB){\bf ,\newline
YTAB}($\geq $ NTAB){\bf , YPTAB}($\geq $NTAB){\bf , YVAL}
\end{description}

Assign values to X, NDERIV, NTAB, XTAB(), YTAB(), and YPTAB(). In
particular, NTAB, XTAB(), YTAB(), and YPTAB() may have the same values as on
return from a previous call to SC2FIT.

\begin{center}
\fbox{\begin{tabular}{@{\bf }c}
YVAL = SHINT(X, NDERIV, NTAB,\\
XTAB, YTAB, YPTAB)\\
\end{tabular}}
\end{center}

This reference will set YVAL to the value (or a derivative value as selected
by NDERIV) of the curve defined by NTAB, XTAB(), YTAB(), and YPTAB().

\subsubsection{Argument Definitions}

\begin{description}
\item[X]  \ [in] Abscissa at which interpolation is to be done. If X is
outside the domain of the table given in XTAB(), extrapolation will be done
using the two tabular points at the nearest end of the table.

\item[NDERIV]  \ [in] Set by the user to 0, 1, 2, or~3 to select computation
of the value, or the first, second, or third derivative of the interpolation
function.

\item[NTAB]  \ [in] Number of values in each of the arrays, XTAB(), YTAB(),
and YPTAB(). Require NTAB\ $\geq 2.$

\item[XTAB()]  \ [in] Abscissae of the given data. These values must be
either strictly increasing or strictly decreasing.

\item[YTAB(), YPTAB()]  \ [in] Values and first derivatives associated with
the abscissae given in XTAB().\vspace{-4pt}%
\begin{gather*}
\text{YTAB}(i)=f(\text{XTAB}(i)),\ i=1,...,\text{NTAB}\\
\text{YPTAB}(i)=f^{\prime }(\text{XTAB}(i)),\ i=1,...,\text{NTAB}
\end{gather*}
\item[SHINT]  \ [out] Returns the value or selected derivative value
computed by interpolation.
\end{description}

\subsubsection{Modifications for double precision}

For double-precision usage change the REAL type statements to DOUBLE
PRECISION and change the function name from SHINT to DHINT. Note
particularly that since this is a function subprogram the name DHINT
must be typed either explicitly or via an IMPLICIT statement if it is used.

\subsection{Examples and Remarks}

The program DRSHINT illustrates the use of SHINT. This program first builds
a table of values of the exponential function and its first derivative at
three points, XTAB$()=0$, 0.5, and 0.75. The program DRSHINT then uses SHINT
to interpolate for the value and first derivative at a set of 21~points, $x=0
$ to 1.0 in steps of~0.05. Output is shown in ODSHINT. The first two columns
show $x$ and $\exp (x)$. Columns headed YINT and YPINT contain the
interpolated value and the interpolated first derivative, respectively. The
exponential function was chosen for convenience in this demonstration since
at any point the true function value and all derivatives have the same
value. Note that the function values are approximated more accurately than
are the derivative values. Also note that the approximations are better on
the interval from~0.5 to~0.75 than on the longer interval from~0 to~0.5. For
$x>0.75$ extrapolation is taking place, leading to much larger errors.

Interpolation error bounds are given in Section D. Applying these bounds to
this example, ignoring the O($h^4)$ term in the bound for the first
derivative error, the error bounds are:
\begin{tabbing}
 for $f$ in [0, $0.5] = (0.00260)(1.65)(0.5)^4 = 0.000268$\\
 for $f$ in [0.5, $0.75] = (0.00260)(2.12)(0.25)^4 = 0.000022$\\
 for $f^{\prime}$ in [0, $0.5] = (0.00802)(1.65)(0.5)^3 = 0.00165$\\
 for $f^{\prime}$ in [0.5, $0.75] = (0.00802)(2.12)(0.25)^3 = 0.000266$
\end{tabbing}
Note that these bounds are consistent with the results shown in ODSHINT.

\subsection{Functional Description}

Hermite interpolation together with the error in such interpolation is
described in \cite[pp.~314--317]{Hildebrand:1956:INA}.

SHINT maintains a saved variable, LOOK, with an initial value of~1. The
look-up procedure always starts with the tabular interval indexed by the
saved value of\pagebreak[2] LOOK, unless the saved value is not in
{[1, $\text{NTAB} - 1$]}.
The look-up procedure is a linear search, either forward or backward, from the
starting value of LOOK. This is efficient for cases in which a succession of
look-ups are done for successive arguments that are not far apart relative
to the tabular spacing.

Each interpolation or extrapolation uses function values and first
derivative values at two adjacent tabular points and is exact for
cubic polynomial data.  If the interpolation abscissa, $x$, is equal
to an interior tabular abscissa, say $x_i$, then tabular data
associated with $x_{i-1}$ and $x_i$ will be used for the computation.
Extrapolation uses the first or last two tabular points, as
appropriate.

If the given data are samples from a function having at least a fourth order
derivative, and this derivative is bounded in magnitude by $M_4$ over the
interval $[x_i$, $x_{i+1}]$, of length, $h$, then the error, $E_i$, in
computing the interpolated derivative of order $i$ for any argument in this
interval is bounded as follows:
\begin{alignat*}{3}
\hspace{-8pt}
|E_0| &\leq c_0M_4h^4k,&\ \ c_0 &= 1/384\:&\approx \:&0.00260\\
\hspace{-8pt}
|E_1| &\leq c_1M_4h^3 + O(h^4),&\ \ c_1 &= \sqrt 3/216\:
&\approx \:&0.00802\\
\hspace{-8pt}
|E_2| &\leq c_2M_4h^2 + O(h^3),&\ \ c_2&= 1/12 \:&\approx \:&0.0833\\
\hspace{-8pt}
|E_3| &\leq c_3M_4h + O(h^2),&\ \ c_3 & = 0.5
\end{alignat*}

\bibliography{math77}
\bibliographystyle{math77}

\subsection{Error Procedures and Restrictions}

This subprogram requires NTAB $\geq 2$ and $0 \leq \text{NDERIV} \leq 3$.
The values in the XTAB() array must either be strictly increasing or
strictly decreasing. These conditions are not checked. Violation of these
conditions will cause unpredictable effects.

\subsection{Supporting Information}

The source language is ANSI Fortran~77.

\begin{tabular}{@{\bf}l@{\hspace{5pt}}l}
\bf Entry & \hspace{.15in} {\bf Required Files}\vspace{2pt} \\
DHINT & \parbox[t]{2.7in}{\hspace{.35in} DHINT\rule[-5pt]{0pt}{8pt}}\\
SHINT & \parbox[t]{2.7in}{\hspace{.35in} SHINT}\\
\end{tabular}

Original code designed by C. L. Lawson and R. J. Hanson, JPL, 1968.
Programmed and modified by Lawson, Hanson, T. Lang, and D. Campbell, JPL,
1968--1974. Adapted to Fortran~77 by Lawson and S. Y. Chiu, July~1987.


\begcode

\medskip\
\lstset{language=[77]Fortran,showstringspaces=false}
\lstset{xleftmargin=.8in}

\centerline{\bf \large DRSHINT}\vspace{10pt}
\lstinputlisting{\codeloc{shint}}

\vspace{30pt}\centerline{\bf \large ODSHINT}\vspace{10pt}
\lstset{language={}}
\lstinputlisting{\outputloc{shint}}
\end{document}
