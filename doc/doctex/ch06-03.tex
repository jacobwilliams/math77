\documentclass[twoside]{MATH77}
\usepackage{multicol}
\usepackage[fleqn,reqno,centertags]{amsmath}
\begin{document}
\begmath 6.3 Basic Linear Algebra Subprograms (BLAS1)

\silentfootnote{$^\copyright$1997 Calif. Inst. of Technology, \thisyear \ Math \`a la Carte, Inc.}

\subsection{Purpose}

This is a set of subroutine and function subprograms for basic mathematical
operations on a single vector or a pair of vectors. The operations provided
are those commonly used in algorithms for numerical linear algebra problems,
$e.g.$, problems involving systems of equations, least-squares, matrix
eigenvalues, optimization, etc.

\subsection{Usage}

Described below under B.1 through B.13 are:

\begin{tabular*}{3.3in}{@{}l@{~~}l}
B.1 & Vector Arguments \dotfill \pageref{B1}\\
B.2 & Dot Product [SDOT, DDOT, CDOTC,\\
 & CDOTU, DSDOT, SDSDOT] \dotfill \pageref{B2}\\
B.3 & Scalar times a Vector Plus a Vector\\
 & [SAXPY, DAXPY, CAXPY] \dotfill \pageref{B3}\\
B.4 & Set up Givens Rotation [SROTG, DROTG] \dotfill \pageref{B4}\\
B.5 & Apply Givens Rotation [SROT, DROT] \dotfill \pageref{B5}\\
B.6 & Set up Modified Rotation\\
 & [SROTMG, DROTMG] \dotfill \pageref{B6}\\
B.7 & Apply Modified Rotation [SROTM, DROTM] \dotfill \pageref{B7}\\
B.8 & Copy X into Y [SCOPY, DCOPY, CCOPY] \dotfill \pageref{B8}\\
B.9 & Swap X and Y [SSWAP, DSWAP, CSWAP] \dotfill \pageref{B9}\\
B.10 & Euclidean Norm\\
 & [SNRM2, DNRM2, SCNRM2] \dotfill \pageref{B10}\\
B.11 & Sum of Absolute Values\\
 & [SASUM, DASUM, SCASUM] \dotfill \pageref {B11}\\
B.12 & Constant Times a Vector\\
 & [SSCAL, DSCAL, CSCAL, CSSCAL] \dotfill \pageref{B12}\\
B.13 & Index of Element Having Maximum Absolute\quad \quad ~\\
 & Value [ISAMAX, IDAMAX, ICAMAX] \dotfill \pageref{B13}
\end{tabular*}

\subsubsection{Vector Arguments\label{B1}}

For subprograms of this package a vector is specified by three arguments,
say N, SX, and INCX, where

\begin{description}
\item[N]  \ denotes the number of elements in the vector,

\item[SX (or DX or CX)]  \ identifies the array containing the vector, and

\item[INCX]  \ is the (signed) storage increment between successive elements
of the vector.
\end{description}

Let $x_i$, $i = 1$, ..., N, denote the vector stored in the array SX().
Within a subprogram of this package the array argument, SX, will be declared
as
\begin{description}
\item[REAL]  \ {\bf SX($*$) }
\end{description}
In the common case of INCX\ $= 1$, $x_i$ is stored in SX(i). More generally,
if INCX\ $\geq 0$, $x_i$ is stored in SX($1 + (i-1) \times \text{INCX})$, and
if INCX $< 0$, $x_i$ is stored in X($1 + (\text{N}-i) \times |\text{INCX}|).$

With indexing as specified above, looping operations within the
subprograms are performed in the order i=1, 2, ...,~N.

Only positive values of INCX are allowed for subprograms that have a single
vector argument. For subprograms having two vector arguments, the two
increment parameters, INCX and INCY, may independently be positive, zero, or
negative.

Due to the Fortran~77 rules for argument association and array storage, the
actual argument playing the role of SX is not limited to being a singly
dimensioned array. It may be an array of any number of dimensions, or it may
be an array element with any number of subscripts. The standard does not
permit this argument to be a simple variable, however.

If the actual argument is an array element, $i.e.$, a subscripted array
name, then it identifies $x_1$, the first element of the vector, if INCX\ $%
\geq 0$, and $x_N$, the last element of the vector, if INCX\ $< 0.$

The most common reason for using an increment value different from one is to
operate on a row of a matrix. For example, if an array is declared as
\begin{description}
\item[DOUBLE PRECISION]  \ {\bf A}(5, 10)
\end{description}
The third row of A(,) begins at element A(3,1) and has a storage spacing of~5
(double precision) storage locations between successive elements. Thus this
row-vector of length~10 would be described by the parameter triple
(N,~DX,~INCX) = (10,~A(3,1),~5) in calls to subprograms of this package.

See Section C for examples illustrating the specifications of vector
arguments.

In the following subprogram descriptions we assume the following
declarations have been made for any subprograms that use these variable names
\begin{description}
\item[INTEGER]  {\bf N, INCX, INCY }

\item[REAL]  {\bf \ SX($mx$), SY($my$) }

\item[DOUBLE PRECISION]  {\bf \ DX($mx$), DY($my$) }

\item[COMPLEX]  {\bf \ CX($mx$), CY($my$) }
\end{description}
In the common case of INCX\ $= 1$, $mx$ must satisfy $mx \geq\ $ N. More
generally, $mx$ must satisfy $mx \geq 1+(\text{N}-1)\times |\text{INCX}|$.
Similarly, $my$ must be consistent with N and INCY.

In Sections B.2 through B.13 we use the convention that ${\bf x}$ denotes
the vector contained in the storage array SX(), DX(), or CX(), $a$ denotes the
scalar value contained in SA, DA, or CA, etc.

\subsubsection{Dot Product Subprograms\label{B2}}
\begin{description}
\item[REAL]  \ {\bf SDOT, SDSDOT, SB, SW }

\item[DOUBLE PRECISION]  {\bf \ DDOT, DSDOT, DW }

\item[COMPLEX]  {\bf \ CDOTC, CDOTU, CW }
\end{description}
The first four subprograms each compute
\begin{equation*}
w=\sum_{i=1}^Nx_iy_i
\end{equation*}
Single precision:
$$
\fbox{{\bf SW = SDOT(N, SX, INCX, SY, INCY)}}
$$
Double precision:
$$
\fbox{{\bf DW = DDOT(N, DX, INCX, DY, INCY)}}
$$
Single precision data. Uses double precision arithmetic internally and
returns a double precision result:
$$
\fbox{{\bf DW = DSDOT(N, SX, INCX, SY, INCY)}}
$$
Complex (unconjugated):
$$
\fbox{{\bf CW = CDOTU(N, CX, INCX, CY, INCY)}}
$$
CDOTC computes%
\begin{equation*}
w=\sum_{i=1}^N\bar x_iy_i
\end{equation*}
where $\bar x_i$ denotes the complex conjugate of the given $x_i$. This is
the usual inner product of complex N-space.
$$
\fbox{{\bf CW = CDOTC(N, CX, INCX, CY, INCY)}}
$$
SDSDOT computes%
\begin{equation*}
w=b+\sum_{i=1}^Nx_iy_i
\end{equation*}
using single precision data, double precision internal arithmetic, and
converting the final result to single precision:
$$
\fbox{{\bf SW = SDSDOT(N, SB, SX, INCX, SY, INCY)}}
$$
In each of the above six subprograms the value of the summation from~1 to~N
will be set to zero if N $\leq 0.$

\subsubsection{Scalar Times a Vector Plus a Vector\label{B3}}
\begin{description}
\item[REAL]  \ {\bf SA }

\item[DOUBLE PRECISION]  {\bf \ DA }

\item[COMPLEX]  {\bf \ CA }
\end{description}
Given a scalar, $a$, and vectors, ${\bf x}$ and ${\bf y}$, each of these
subroutines replaces ${\bf y}$ by $a{\bf x} + {\bf y}$. If $a = 0$ or N $%
\leq 0$, each subroutine returns, doing no computation.

Single precision:
$$
\fbox{{\bf CALL SAXPY (N, SA, SX, INCX, SY, INCY)}}
$$
Double precision:
$$
\fbox{{\bf CALL DAXPY (N, DA, DX, INCX, DY, INCY)}}
$$
Complex:
$$
\fbox{{\bf CALL CAXPY (N, CA, CX, INCX, CY, INCY)}}
$$
\subsubsection{Construct a Givens Plane Rotation\label{B4}}
\begin{description}
\item[REAL]  \ {\bf SA, SB, SC, SS }

\item[DOUBLE PRECISION]  {\bf \ DA, DB, DC, DS }
\end{description}
Single precision:
$$
\fbox{{\bf CALL SROTG (SA, SB, SC, SS)}}
$$
Double precision:
$$
\fbox{{\bf CALL DROTG (DA, DB, DC, DS)}}
$$
Given $a$ and $b$, each of these subroutines computes $c$ and $s$, satisfying%
\begin{equation*}
\left[
\begin{array}{rr}
c & s \\
-s & c
\end{array}
\right] \cdot \left[
\begin{array}{c}
a \\
b
\end{array}
\right] =\left[
\begin{array}{c}
r \\
0
\end{array}
\right]
\end{equation*}
subject to $c^2+s^2=1$ and $r^2=a^2+b^2$. Thus the matrix involving $c$ and $%
s$ is an orthogonal (rotation) matrix that transforms the second component
of the vector $[a,b]^t$ to zero. This matrix is used in certain least-%
squares and eigenvalue algorithms.

If $r = 0$ the subroutine sets $c = 1$ and $s = 0$. Otherwise the sign of $r$
is set so that $sgn(r) = sgn(a)$ if $|a| > |b|$, and $sgn(r) = sgn(b)$ if $%
|a| \leq |b|$. Then, $c = a/r$ and $s = b/r.$

Besides setting $c$ and $s$, the subroutine stores $r$ in place of $a$ and
another number, $z$, in place of $b$.  The number $z$ is rarely needed.
See
\cite{Lawson:1979:BLA}--\nocite{Lawson:1979:ABL}\nocite{Dodson:1982:RBL}
\cite{Dodson:1983:CRB} in Section D for a description of $z.$

These subroutines are designed to avoid extraneous overflow or underflow in
cases where $r^2$ is outside the exponent range of the computer arithmetic
but $r$ is within the range.

\subsubsection{Apply a Plane Rotation\label{B5}}
\begin{description}
\item[REAL]  \ {\bf SC, SS }

\item[DOUBLE PRECISION]  {\bf \ DC, DS }
\end{description}
Single precision:

\begin{center}
\fbox{\begin{tabular}{@{\bf }c}
CALL SROT (N, SX, INCX,\\
SY, INCY, SC, SS)\\
\end{tabular}}
\end{center}

Double precision:

\begin{center}
\fbox{\begin{tabular}{@{\bf }c}
CALL DROT (N, DX, INCX,\\
DY, INCY, DC, DS)\\
\end{tabular}}
\end{center}

Given vectors, ${\bf x}$ and ${\bf y}$, and scalars, $c$ and $s$, this
subroutine replaces the 2$\times $N matrix%
\begin{equation*}
\left[
\begin{array}{c}
{\bf x}^t \\ {\bf y}^t
\end{array}
\right] \ \ \text{by the 2}\times \text{N matrix}\ \ \left[
\begin{array}{rr}
c & s \\
-s & c
\end{array}
\right] \cdot \left[
\begin{array}{c}
{\bf x}^t \\ {\bf y}^t
\end{array}
\right] .
\end{equation*}
If N $\leq 0$ or if $c=1$ and $s=0$, these subroutines return, doing no
computation.

\subsubsection{Construct a Modified Givens Transformation\label{B6}}
\begin{description}
\item[REAL]  \ {\bf SD1, SD2, SX1, SX2, SPARAM}(5)

\item[DOUBLE PRECISION]  \ {\bf DD1, DD2, DX1, DX2, DPARAM}(5)
\end{description}
Single precision:

\begin{center}
\fbox{\begin{tabular}{@{\bf }c}
CALL SROTMG (SD1, SD2,\\
SX1, SX2, SPARAM)\\
\end{tabular}}
\end{center}

Double precision:

\begin{center}
\fbox{\begin{tabular}{@{\bf }c}
CALL DROTMG (DD1, DD2,\\
DX1, DX2, DPARAM)\\
\end{tabular}}
\end{center}

The input quantities $d_1$, $d_2$, $x_1$, and $x_2$, define a
2-vector $[w_1\ w_2]^t$ in partitioned form as%
\begin{equation*}
\left[
\begin{array}{c}
w_1 \\
w_2
\end{array}
\right] =\left[
\begin{array}{cc}
d_1^{1/2} & 0 \\
0 & d_2^{1/2}
\end{array}
\right] \cdot \left[
\begin{array}{c}
x_1 \\
x_2
\end{array}
\right] .
\end{equation*}
The subroutine determines the modified Givens rotation matrix H that
transforms $x_2$ and thus $w_2$ to zero. It also replaces $d_1$, $d_2$, and $x_1$
with $\delta _1$, $\delta _2$, and $\xi _1$, respectively. These quantities
satisfy

\begin{align*}
\left[ \begin{array}{c} \omega  \\ 0 \end{array}
\right] &= \left[ \begin{array}{cc}
\delta _1^{1/2} & 0 \\ 0 & \delta _2^{1/2} \end{array} \right]
\cdot H\cdot \left[ \begin{array}{c} x_1 \\ x_2 \end{array} \right] \\
 &=\left[ \begin{array}{cc}
\delta _1^{1/2} & 0 \\ 0 & \delta _2^{1/2}
\end{array} \right]
\cdot \left[ \begin{array}{c} \xi _1 \\ 0
\end{array} \right] ,
\end{align*}

with $\omega =\pm (w_1^2+w_2^2)^{1/2}$. A representation of the matrix H will be
stored by this subroutine into SPARAM() or DPARAM() for subsequent use by
subroutines SROTM or DROTM.

See the Appendix in \cite{Lawson:1979:BLA} for more details on the
computation and storage of $H.$

Most of the time the matrix H will be constructed to have two elements equal
to +1 or $-$1. Thus multiplication of a 2-vector by H can be programmed to
be faster than multiplication by a general $2\times 2$ matrix. This is the
motivation for using a modified Givens matrix rather than a standard Givens
matrix; however, these matrix multiplications must represent a very
significant percentage of the execution time of an application program in
order for the extra complexity of using the modified Givens matrix to be
worthwhile.

\subsubsection{Apply a Modified Givens Transformation\label{B7}}
\begin{description}
\item[REAL]  \ {\bf SPARAM}(5)

\item[DOUBLE PRECISION]  \ {\bf DPARAM}(5)
\end{description}
Single precision:

\begin{center}
\fbox{\begin{tabular}{@{\bf }c}
CALL SROTM (N, SX, INCX,\\
SY, INCY, SPARAM)\\
\end{tabular}}
\end{center}

Double precision:

\begin{center}
\fbox{\begin{tabular}{@{\bf }c}
CALL DROTM (N, DX, INCX,\\
DY, INCY, DPARAM)\\
\end{tabular}}
\end{center}

Given vectors, ${\bf x}$ and ${\bf y}$, and a representation of an H matrix
constructed by SROTMG or DROTMG in SPARAM() or DPARAM(), this subroutine
replaces the 2$\times $N matrix%
\begin{equation*}
\left[
\begin{array}{c}
{\bf x}^t \\ {\bf y}^t
\end{array}
\right] \quad \text{by }\quad H\cdot \left[
\begin{array}{c}
{\bf x}^t \\ {\bf y}^t
\end{array}
\right] .
\end{equation*}
Due to the special form of the matrix, H, this matrix multiplication
generally requires only 2N multiplications and 2N additions rather than the
4N multiplications and 2N additions that would be required if H were an
arbitrary $2\times 2$ matrix.

If N $\leq 0$ or H is the identity matrix, these subroutines return
immediately.

\subsubsection{Copy a Vector ${\bf x}$ to ${\bf y}$\label{B8}}

Single precision:
$$
\fbox{{\bf CALL SCOPY (N, SX, INCX, SY, INCY)}}
$$

Double precision:
$$
\fbox{{\bf CALL DCOPY (N, DX, INCX, DY, INCY)}}
$$
Complex:
$$
\fbox{\bf CALL CCOPY (N, CX, INCX, CY, INCY)}
$$
Each of these subroutines copies the vector ${\bf x}$ to ${\bf y}$. If N $%
\leq 0$ the subroutine returns immediately.

\subsubsection{Swap Vectors ${\bf x}$ and ${\bf y}$\label{B9}}

Single precision:
$$
\fbox{{\bf CALL SSWAP (N, SX, INCX, SY, INCY)}}
$$
Double precision:
$$
\fbox{{\bf CALL DSWAP (N, DX, INCX, DY, INCY)}}
$$
Complex:
$$
\fbox{{\bf CALL CSWAP (N, CX, INCX, CY, INCY)}}
$$
This subroutine interchanges the vectors ${\bf x}$ and ${\bf y}$. If N $\leq
0 $, the subroutine returns immediately.

\subsubsection{Euclidean Norm of a Vector\label{B10}}
\begin{description}
\item[REAL]  \ {\bf SNRM2, SCNRM2, SW }

\item[DOUBLE PRECISION]  {\bf \ DNRM2, DW }
\end{description}
Single precision:
$$
\fbox{{\bf SW = SNRM2(N, SX, INCX)}}
$$
Double precision:
$$
\fbox{{\bf DW = DNRM2(N, DX, INCX)}}
$$
Complex data, REAL result:
$$
\fbox{{\bf SW = SCNRM2(N, CX, INCX)}}
$$
Each of these subprograms computes%
\begin{equation*}
w=\left[ \sum_{i=1}^N|x_i|^2\right] ^{1/2}.
\end{equation*}
If N $\leq 0$, the result is set to zero.

These subprograms are designed to avoid overflow or underflow in cases in
which $w^2$ is outside the exponent range of the computer arithmetic but $w$
is within the range.

\subsubsection{Sum of Magnitude of Vector Components\label{B11}}
\begin{description}
\item[REAL]  \ {\bf SASUM, SCASUM, SW }

\item[DOUBLE PRECISION]  {\bf \ DASUM, DW }
\end{description}
SASUM and DASUM compute%
\begin{equation*}
w=\sum_{i=1}^N|x_i|.
\end{equation*}
Single precision:
$$
\fbox{{\bf SW = SASUM(N, SX, INCX)}}
$$
Double precision:
$$
\fbox{{\bf DW = DASUM(N, DX, INCX)}}
$$
Given a complex vector, ${\bf x}$, SCASUM computes a REAL result for the
expression:%
\begin{equation*}
w=\sum_{i=1}^N|\Re x_i|+|\Im x_i|.
\end{equation*}
$$
\fbox{{\bf SW = SCASUM(N, CX, INCX)}}
$$
If N $\leq 0$ these subprograms set the result to zero.

\subsubsection{Vector Scaling\label{B12}}
\begin{description}
\item[REAL]  \ {\bf SA }

\item[DOUBLE PRECISION]  {\bf \ DA }

\item[COMPLEX]  {\bf \ CA }
\end{description}
Given a scalar, $a$, and vector, ${\bf x}$, each of these subroutines replaces
the vector, ${\bf x}$, by the product $a{\bf x}$.

Single precision:
$$
\fbox{{\bf CALL SSCAL (N, SA, SX, INCX)}}
$$
Double precision:
$$
\fbox{{\bf CALL DSCAL (N, DA, DX, INCX)}}
$$
Complex:
$$
\fbox{{\bf CALL CSCAL (N, CA, CX, INCX)}}
$$
Given REAL $a$ and complex ${\bf x}$, the subroutine CSSCAL replaces ${\bf x}$
by the product $a{\bf x}$.
$$
\fbox{{\bf CALL CSSCAL (N, SA, CX, INCX)}}
$$
If N $\leq 0$ these subroutines return immediately.

\subsubsection{Find Vector Component of Largest Magnitude\label{B13}}
\begin{description}
\item[INTEGER]  \ {\bf ISAMAX, IDAMAX, ICAMAX, IMAX }
\end{description}
ISAMAX and IDAMAX each determine the smallest $i$ such that%
\begin{equation*}
|x_i|=\max \{|x_j|:j=1,...,N\}
\end{equation*}
REAL vector, integer result:
$$
\fbox{{\bf IMAX = ISAMAX(N, SX, INCX)}}
$$
Double precision vector, integer result:
$$
\fbox{{\bf IMAX = IDAMAX(N, DX, INCX)}}
$$
Given a complex vector, ${\bf x}$, ICAMAX determines the smallest $i$ such
that%
\begin{equation*}
|\Re x_i|+|\Im x_i|=\max \{|\Re x_j|+|\Im x_j|:j=1,...,N\}
\end{equation*}
$$
\fbox{{\bf IMAX = ICAMAX(N, CX, INCX)}}
$$
If N $\leq 0$, each of these subprograms sets the integer result to zero.

\subsection{Examples and Remarks}

The program, DRDBLAS1, and its output, ODDBLAS1, illustrate the use of
various BLAS1 subprograms to compute matrix-vector and matrix-matrix products.

Problems (1) and (2) show two ways of computing the matrix-vector
product, $A{\bf b}$. Using DDOT accesses the elements of $A$ by rows.
Using DAXPY accesses $A$ by columns. The column ordering has an
efficiency advantage on virtual memory systems since Fortran stores
arrays by columns and there will therefore be fewer page faults.

Problem (3) illustrates multiplication by the transpose of a matrix without
transposing the matrix in storage.

Problem (4) illustrates matrix-matrix multiplication. This operation could
also be programmed to use DAXPY rather than DDOT.

These examples also illustrate the feature that matrix dimensions and
storage array dimensions need not be the same. Thus, in DRDBLAS1, a $2\times
3$ matrix $A$ is stored in a $5\times 10$ storage array A(,).

The program, DRDBLAS2, with output, ODDBLAS2, illustrates a complete
algorithm for solving a linear least-squares problem. The algorithm first
performs sequential accumulation of data into a triangular matrix, using
DROTG and DROT to build and apply Givens orthogonal transformations. It then
solves the triangular system using DCOPY and DAXPY. This is a very reliable
algorithm.

The problem solved is the determination of coefficients $c_1$, $c_2$, and $%
c_3$ in the expression $c_1 + c_2x + c_3\exp (-x)$ to produce a least-%
squares fit to the data given in XTAB() and YTAB(). Note that by changing
dimension parameters, and the statements assigning values to the array W(),
DRDBLAS2 could be altered to solve any specific linear least-squares problem.

\subsection{Functional Description}

This set of subprograms is described in more detail in \cite{Lawson:1979:BLA}.
For discussion of the standard and modified Givens orthogonal transformations
and their use in least-squares computations see \cite{Lawson:1974:SLS}.

There is an error in \cite{Lawson:1979:BLA} in the discussion of the
parameter, $z$, which is returned by SROTG or DROTG.  This error was
corrected by \cite{Dodson:1982:RBL}, after which \cite{Dodson:1983:CRB}
corrected an error in \cite{Dodson:1982:RBL}.

\bibliography{math77}
\bibliographystyle{math77}

\subsection{Error Procedures and Restrictions}

These subprograms do not issue any error messages. If INCX $\leq $ 0 in any
of the subprograms having a single vector argument the results are
unpredictable.

A value of N $= 0$ is a valid special case. Values of N $< 0$ are treated
like N $= 0.$

\subsection{Supporting Information}

The source language is ANSI Fortran~77.

Each program unit has a single entry with the same name as the program unit.
No program units contain external references.

As a result of experiments done in \cite{Krogh:1972:OUA}, a BLAS-type
package was originally proposed in \cite{Hanson:1973:APS}.  The package
was subsequently developed and tested over the period 1973--77 as an ACM
SIGNUM committee project, with the final product being announced
in~\cite{Lawson:1979:BLA}.  This package was subsequently used as a
component in many other numerical software packages and optimized
machine-language versions were produced for a number of different computer
systems.

This package was identified as BLAS when it appeared in~1979
\cite{Lawson:1979:BLA}, however it is now identified as ``BLAS1", or
``Level~1 BLAS" since publication of BLAS2~\cite{Dongarra:1988:ESF} for
matrix-vector operations and BLAS3~\cite{Dongarra:1990:SLB} for
matrix-matrix operations.  These latter two packages support more
efficient use of vector registers, processor cache, and parallel
processors than is possible in BLAS1.\nocite{Hanson:1987:ATA}

Adapted to Fortran~77, by C. Lawson and S. Chiu, JPL, January~1984.
Replaced L2 norm routines with new versions which avoid undeflow in all
cases and which don't use assigned go to's, F. Krogh, May 1998.

All entries need one file of the same name, except for DNRM2, SCNRM2,
and SNRM2 all of which also require AMACH.
\begin{center}\bf \vspace{-20pt}
\begin{tabular}{llll}
\multicolumn{4}{c}{Entries}\\
CAXPY & CCOPY & CDOTC & CDOTU\\
CSCAL & CSSCAL & CSWAP & DASUM\\
DAXPY & DCOPY & DDOT & DNRM2\\
DROT & DROTG & DROTM & DROTMG\\
DSCAL & DSDOT & DSWAP & ICAMAX\\
IDAMAX & ISAMAX & SASUM & SAXPY\\
SCASUM & SCNRM2 & SCOPY & SDOT\\
SDSDOT & SNRM2 & SROT & SROTG\\
SROTM & SROTMG & SSCAL & SSWAP\\
\end{tabular}
\end{center}

\begcode

\medskip\
\lstset{language=[77]Fortran,showstringspaces=false}
\lstset{xleftmargin=.8in}

\centerline{\bf \large DRDBLAS1}\vspace{10pt}
\lstinputlisting{\codeloc{dblas1}}
\newpage

\vspace{30pt}\centerline{\bf \large ODDBLAS1}\vspace{10pt}
\lstset{language={}}
\lstinputlisting{\outputloc{dblas1}}

\lstset{language=[77]Fortran,showstringspaces=false}
\lstset{xleftmargin=.8in}
\vspace{.5in}
\centerline{\bf \large DRDBLAS2}\vspace{10pt}
\lstinputlisting{\codeloc{dblas2}}

\vspace{30pt}\centerline{\bf \large ODDBLAS2}\vspace{10pt}
\lstset{language={}}
\lstinputlisting{\outputloc{dblas2}}
\end{document}
