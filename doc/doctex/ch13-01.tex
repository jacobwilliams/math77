\documentclass[twoside]{MATH77}
\usepackage{multicol}
\usepackage[fleqn,reqno,centertags]{amsmath}
\begin{document}
\begmath 13.1 Numerical Evaluation of Integrals Over One Dimension

\silentfootnote{$^\copyright$1997 Calif. Inst. of Technology, \thisyear \ Math \`a la Carte, Inc.}

\subsection{Purpose}

This collection of subprograms estimates the value of the integral%
\begin{equation*}
\int_a^bf(x)\,dx
\end{equation*}
where the integrand $f(x)$ and the limits $a$ and $b$ are supplied by the
user.

\subsection{Usage}

Described below under B.1 through B.5 are:

\begin{tabular*}{3.3in}{@{}l@{~~}l}
B.1 & \hspace{-20pt} Program Prototype, Single Precision\dotfill \pageref{PPSP}\\
\quad B.1.a & The Calling Routine\dotfill \pageref{Calling}\\
\quad B.1.b & Argument Definitions for SINT1\dotfill
\pageref{ArgDef}\\
\quad B.1.c & The User-supplied Integrand\\
 & Subroutine SINTF\dotfill
\pageref{SINTF}\\
\quad B.1.d & Argument Definitions for SINTF\dotfill \pageref{ArgSINTF}\\
\quad B.1.e & Passing Extra Information into SINTF\dotfill \pageref{ExSINTF}\\
B.2 & \hspace{-20pt}  Program Prototype, Single Precision,\\
 & \hspace{-20pt} Reverse Communication\dotfill
\pageref{PPRC}\\
\quad B.2.a & The Calling Routine\dotfill \pageref{CallingRC}\\
\quad B.2.b & Argument Definitions\dotfill \pageref{ArgDefRC}\\
B.3 & \hspace{-20pt} Methods to Request Unusual Usage Through
\rule{.3in}{0pt}\\
 & \hspace{-20pt} the Arguments IOPT and WORK\dotfill \pageref{UnusualUse}\\
B.4 & \hspace{-20pt} Changing the Selection of Some Options\\
 &\hspace{-20pt} During the Computation\dotfill \pageref{ChangeSel}\\
B.5 & \hspace{-20pt} Modifications for Double Precision Usage\dotfill \pageref{DPuse}\\
\end{tabular*}

\subsubsection{Program Prototype, Single Precision.\label{PPSP}}

\paragraph{The Calling Routine\label{Calling}}

{\bf REAL} {\bf A, B, ANSWER, WORK$(\geq k_1)$}
[$k_1$ depends on options used ($\geq 1$).]


{\bf INTEGER} {\bf IOPT$(\geq k_2)$} [$k_2$ depends on options used
($\geq 2$).]


Assign values to A and B.

Assign values to IOPT() and elements of WORK() referenced using options (see
Section B.3). For simplest usage, set

\hspace{.2in}IOPT$(2)=0$%
$$
\fbox{{\bf CALL SINT1(A,B,ANSWER,WORK,IOPT)}}
$$
If the computation is successful (indicated by IOPT$(1)<0)$ the result is in
ANSWER and the estimate of the error in the result is in WORK(1).

\vspace{20pt}
\paragraph{Argument Definitions for SINT1\label{ArgDef}}

\begin{description}
\item[A]  [in] Lower limit of integration.

\item[B]  [in] Upper limit of integration. The relative values of A and B
are interpreted properly.  Thus if one exchanges A and B, the sign of
the answer is changed.  When the integrand is positive, the sign of
the result is the same as the sign of B $-$ A.

\item[ANSWER]  [out] Estimate of the integral.

\item[WORK()]  [inout] On completion, WORK(1) contains an estimate of the
upper bound of the magnitude of the difference between ANSWER and the
true value of the integral. Other elements of WORK may be referenced by
the option vector IOPT as described below and in Section B.3, and to pass
parameters to SINTF as described in Section B.1.e.

\item[IOPT()]  [inout] Used to return status information to the user, to
allow the selection of options, and to pass parameters to SINTF as described
in Section B.1.e. IOPT(1) returns a status indicator with the possible
values:

\begin{itemize}
\item[$-1$]  Normal termination with either the absolute or relative error
tolerance criteria satisfied (see option~3 in Section~B.3).

\item[$-2$]  Normal termination with neither the absolute nor relative error
tolerance criteria satisfied, but the error tolerance based on the locally
achievable precision is satisfied. (See option~3 in Section~B.3).
This is the normal status value when no tolerances are specified.

\item[$-3$]  Normal termination with none of the error tolerance criteria
satisfied. (See option~3 in Section~B.3).
\item[$+4$]  Bad value for an element of IOPT.

\item[$+5$]  Too many function values needed (see option~9 in
Section~B.3).

\item[$+6$]  The integrand apparently contains a nonintegrable singularity.\
The abscissa of the singularity is near ANSWER. WORK(1) contains a large
number.
\end{itemize}

The remainder of IOPT may be used to select options and to pass parameters
to SINTF. If no options are selected, set IOPT$(2)=0.$

See Section B.3 for a detailed description of options and the following
table for an overview.
\end{description}

\newpage\ \centerline{{\bf Table of Options for SINT1}\hspace{.5in}}
\ {\bf Option}\newline
{\bf Number\hspace{.4in}Brief Description}\vspace{-3pt}
\begin{itemize}
\item[0]  No more options.

\item[1]  No effect. Reserved for future use. Do not use option~1.

\item[2]  Select level of diagnostic output.

\item[3]  Specify error tolerances.

\item[4]  Specify an a posteriori estimate of the absolute error in the
calculated value of the integrand.

\item[5]  Specify an a priori estimate of relative error expected in the
calculated values of integrands.

\item[6]  Use reverse communication.

\item[7]  Specify minimum index of quadrature formula.

\item[8]  No effect. The parameter may provide information to SINTF.

\item[9]  Specify maximum number of integrand evaluations allowed.

\item[10]  Return number of integrand evaluations used.

\item[11]  Specify location of singularity or discontinuity in integrand.

\item[12]  Specify absolute errors in the limits.

\item[13]  Used for multi-dimensional integration. See Chapter~13.2.
\end{itemize}

\paragraph{The User-Supplied Integrand Subroutine SINTF\label{SINTF}}

SINT1 requires that the user provide values of the integrand. In the case
of simple usage these values are provided by a user-supplied subroutine of
the form:

{\tt \begin{tabbing}
SUBROUTINE SINTF(ANSWER, X, IFLAG)\\
REAL \ ANSWER, X\\
INTEGER \ IFLAG\\
ANSWER $=$ value of integrand at X\\
RETURN\\
END
\end{tabbing}}

\paragraph{Argument Definitions for SINTF\label{ArgSINTF}}

\begin{description}
\item[ANSWER]  \ [out] set by SINTF to be the value of the integrand at X.

\item[X]  \ [in] contains the abscissa at which the integrand is to be
evaluated. X may also be used to pass extra information into SINTF, as
described in Section B.1.e.

\item[IFLAG]  \ [in] is not needed for simple usage, but may be used to pass
extra information into SINTF, as described in Section B.1.e.
\end{description}

Most mathematical software that requires user defined function
information allows the user to pass in the name of a subprogram for
evaluating the function.  The approach used here has the advantage of not
requiring the user to declare his subprogram in an external statement
(It's not unusual for this to be forgotten.), and of giving a meaningful
diagnostic when no such routine is provided.  If one is solving different
problems with different programs, one can use different file names for the
different function subprograms, all of which would use the same entry
name.  The linker must then be told which of these function subprograms
should be used in forming the executable file.  If one wants to solve
several different problems in the same program, one should code the
separate cases in one subprogram and select the one desired by passing a
``case'' variable into the routine.  This can either be done as part of
IFLAG as mentioned above, or can be done through the use of named
common.  One can also use reverse communication and call subprograms with
different names for the different cases.

\paragraph{Passing Extra Information into SINTF\label{ExSINTF}}

The argument X of SINTF is the first element of the WORK vector passed from
the user's calling routine to SINT1. Elements of WORK other than the first
may be referenced by options, as described in Section B.3, or used to pass
extra floating point information to SINTF. If X is used in this way it must
be declared to be an array instead of a scalar.

The argument IFLAG of SINTF is the first element of the IOPT() vector
passed from the users calling routine to SINT1. Elements of IOPT
(after the first) that are not used for options as described in
Section B.3 may be used to pass integer valued information into
SINTF. In addition, the parameter of option~8 described in Section
B.3 may be examined by SINTF. If IFLAG is used in this way it must be
declared to be an array instead of a scalar.
\subsubsection{Program Prototype, Single Precision, Reverse Communication.%
\label{PPRC}}
\paragraph{The Calling Routine\label{CallingRC}}
\begin{description}
\item[REAL]  \ {\bf A, B, ANSWER, WORK$(\geq 1)$}
\item[INTEGER]  \ {\bf IOPT$(\geq 3)$}
\end{description}
Assign values to A, B, IOPT and elements of WORK referenced by options.
Option~6 must be selected (see Section~B.3). For simple usage, set
\begin{tabbing}
\hspace{.2in}\=IOPT$(2) = 6$\\
\>IOPT$(3) = 0$
\end{tabbing}
\begin{center}\vspace{-5pt}
\fbox{\begin{tabular}{@{\bf }c}
CALL SINT1 (A, B, ANSWER,\\
WORK, IOPT)
\end{tabular}}
\end{center}\vspace{-5pt}
\hspace{.2in}DO
$$
\quad \fbox{{\bf CALL SINTA (ANSWER, WORK, IOPT)}}
$$\vspace{-20pt}
\begin{tabbing}
\hspace{.2in}\=\ \ \ \ \=IF (IOPT(1) .NE.\ 0) EXIT\\
\>\>ANSWER $=$ value of integrand at $x$, where\\
\>\>\ \ \ \ \=$x =$ WORK(1)\\
\>END DO\\
\>$\{$ Integration is complete. $\}$
\end{tabbing}
\paragraph{Argument Definitions\label{ArgDefRC}}
\begin{description}
\item[A, B]  \ Used as described in Section B.1.b.
\item[ANSWER]  \ [inout] During integration, ANSWER provides a value of the
integrand to SINTA. On completion, ANSWER contains an estimate of the
integral.
\item[WORK()]  \ [inout] Used as described in Section B.1.b, except that
during the integration WORK(1) is the abscissa at which the integrand is to
be evaluated.
\item[IOPT()]  \ [inout] Used as described in Section B.1.b, except IOPT(1)
has additional uses. If IOPT$(1)=0$, then evaluate the integrand at the
abscissa specified in WORK(1) and put the result in ANSWER. If IOPT$(1)\neq
0$, integration has been completed. Refer to Section B.1 for the meaning of
nonzero values of IOPT(1).
\end{description}

\subsubsection{Methods to Request Unusual Usage Through the Arguments IOPT
and WORK\label{UnusualUse}}

Options are identified by codes ranging from~0 to~12. Options 0 and~6 have
no arguments, while the other options each require one integer argument. An
option is selected by placing its code in the array IOPT() with the
following element of the array containing its argument if it requires one.\
The option codes, with space for their arguments as required, must be stored
as a contiguous sequence beginning in IOPT(2) and ending with the option
code~0.

If a nonzero option code appears more than once, only the last occurrence
will be effective.

Depending on the option, an argument may be a location where an
integer value is provided or returned, or it may be an index into the
array WORK() indicating the location of a floating point datum or the
beginning of a sequence of floating point data.

If options involving indexes into WORK() are used, the index must exceed 1,
and WORK() must be dimensioned sufficiently large.

All unselected options are assigned default values.

The options are defined as follows:
\begin{itemize}
\item[0]  (No argument) Terminates the sequence of selected options.
\item[1]  (Argument K1) No effect. Reserved for future use. Do not use
option~1.
\item[2]  (Argument K2) K2 selects the level of diagnostic output:
\begin{itemize}
\item[0]  No printing.
\item[1]  Minimum printing --- error messages (default value).
\item[2]  Panel boundaries and error estimates in addition to (1).
\item[3]  Error estimates for each quadrature formula in addition to (2).
\item[4]  Maximum detail.
\end{itemize}
\item[3]  (Argument K3) K3 is an index into WORK(). The user provides the
absolute error tolerance in WORK(K3), the error tolerance T based on
the locally achievable precision in WORK(K3$+1)$, and the error tolerance
relative to the value of the integral in WORK(K3$+2)$. The error tolerance
T should be assigned a value $\epsilon ^f$, where $\epsilon $ is the
smallest number such that computing $1.0 + \epsilon $ gives a number
different from~1.0 ($\epsilon $ is the precision of the machine
arithmetic) and $f$ is the fraction (number of correct digits desired)
/ (digits of achievable precision).  (This
latter number is discussed in Section D where the method for
computing the error estimate is described.)  The internal value for T
is bounded between $\epsilon $ and $\sqrt {\epsilon }$, that is
$\frac 12 \leq f \leq 1$.  If this option is not selected, the
absolute error tolerance and the tolerance relative to the value of
the integral are both set to zero, and T is set using an $f$ of~3/4,
$i.e.$ T = $\epsilon ^{3/4}$. The default values usually provide
all the accuracy that can be obtained efficiently.

The error tolerance relative to the value of the integral is applied
globally (over the entire region of integration) rather than locally (one
step at a time). This policy provides true control of error relative to the
value of the integral when the integrand is not sign definite, as well as
when the integrand is sign definite. To apply the criterion of error
tolerance relative to the value of the integral, the value of the integral
over the entire region, estimated without refinement of the region, is used
to derive an absolute error tolerance that may be applied locally. If the
preliminary estimate of the value of the integral is significantly in error,
and the least restrictive error tolerance is relative to the value of the
integral, the cost of computing the integral will be larger than the cost of
computing the integral to the same degree of accuracy using appropriate
values of either of the other error tolerance criteria. The preliminary
estimate of the integral may be significantly in error if the integrand is
not sign definite or has large variation.

\item[4]  (Argument K4) In WORK(K4) the user provides an a posteriori
estimate of the absolute value of the error committed while evaluating the
integrand. This value may be computed during evaluation of the integrand.\
The default value is zero.

\item[5]  (Argument K5) In WORK(K5) the user provides an a priori estimate
of the relative error expected to be committed while evaluating integrands.\
Changes to this value are not detected during evaluation of the integral.\
The default value is the machine precision.

\item[6]  (No argument) Selects use of reverse communication. See Section
B.2.

\item[7]  (Argument K7) K7 specifies the minimum index of quadrature formula
used before SINTA does a more detailed analysis to determine whether a
higher order formula should be used or the interval subdivided. The default
value for this minimum index is~3. The number of integrand samples used for
a formula of index $k$ is $2^k-1$, and the algebraic order is $3/2$ the
number of samples. If the user knows from analysis or experience with this
program that an integral can only be computed to the desired accuracy by
using a formula of higher index than~3, some efficiency may be gained by so
notifying the program.

\item[8]  (Argument K8) No effect. K8 may be used to pass information to
SINTF.

\item[9]  (Argument K9) K9 specifies the maximum number of function values
to use to compute the integral. If this option is not selected, or if K9 $%
\leq $ 0, the number of function values is not bounded.

\item[10]  (Argument K10) On return the number of function values used to
calculate the integral is stored in K10.

\item[11]  (Argument K11) K11 is a signed index providing a means for the
user to give the subroutines information about the location and type of any
known singularities of the integrand. When an integrand appears to have
singular behavior at the end of the interval, a transformation of the
variable of integration is applied to reduce the strength of the
singularity. When an integrand appears to have singular behavior inside the
interval, the abscissa of the singularity is determined as precisely as
necessary, depending on the error tolerance, and the interval is subdivided.

The discovery of singular behavior and determination of the abscissa of
singular behavior are expensive. If the user knows of the existence
of a singularity, the efficiency of computation of the integral may be
improved by requesting an immediate transformation of the independent
variable or subdivision of the interval. In WORK($|$K11$|$) the user
provides the real part of the abscissa of a singularity or discontinuity
in the integrand. It is recommended that the user select this option for
all singularities, even those outside [A, B].

For singularities having a leading term of the form $x^\alpha $ where $%
\alpha $ is not an integer, if $\alpha $ is ``large'' or has the form $%
\alpha =(2n-1)/2$ where $n$ is a nonnegative integer, or the singularity is
well outside the interval, make K11 positive. Otherwise, make K11 negative.\
The meaning of ``large'' depends on the rest of the integrand and the length
of the interval. ``Large'' is probably about~2 for the typical case. For a
singularity of the form $x^\alpha \log \ x$ use the above rule, even if $%
\alpha $ is an integer. For other types of singularity make a reasonable
guess based on the above. If several similar integrals are to be computed,
some experimentation may be useful.

When K11 $>0$, a transformation of the form T $=$ TA $+$ (X $-$ TA$)^2/$ (TB
$-$ TA) is applied, where TA is the abscissa of the singularity and TB is
the end of the interval. If TA is outside the interval, TB will be the end
of the interval farthest from TA. If TA is inside the interval, the
interval will immediately be subdivided at TA, and both parts will be
separately integrated with TB equal to each end of the original interval,
respectively. When K11 $<0$, a transformation of the form T $=$ TA $+$ (X $-
$ TA$)^4/($TB $-$ TA$)^3$ is applied, with TA and TB as above.

If the integrand has singularities at more than one abscissa within the
region, or more than one pole near the real axis such that the real parts
are within the region of integration, the interval should be subdivided at
the abscissae of the singularities or the real parts of the poles, the
integrals should be computed as separate problems, and the answers summed.

\item[12]  (Argument K12) The user provides the error in the lower limit in
WORK(K12), and the error in the upper limit in WORK(K12$+1)$. Suppose the lower
limit $a$ is a function of several quantities $y_j$. Then WORK(K12) should
be $\Sigma \ |\partial a/\partial y_j|\ E(y_j)$, where E($y_j)$ is the error
in $y_j$. The error in the integral due to error in the limit is the error
in the limit times the integrand evaluated near the limit. WORK(K12$+1)$
should be evaluated similarly for the upper limit. If this option is not
selected, or K12 $=0$, then the limits will be assumed to be exact.
\end{itemize}

\subsubsection{Changing the Selection of Some Options During the Computation%
\label{ChangeSel}}

It is possible to change the selection of options 1, 2, 6, 7, and~9
during the integration. To do this, construct an option vector exactly as
for the initial call to SINT1. Then
$$
\fbox{{\bf CALL SINTOP (IOPT, WORK)}}
$$
IF (IOPT(1) .GT.\ 0) GO TO [Bad IOPT value routine]

This call does not result in any option selections being set to their
default values. Otherwise, the action of each option is as described in
Section B.3.

\subsubsection{Modifications for Double Precision Usage\label{DPuse}}

For double precision usage, change all REAL type statements to DOUBLE
PRECISION and change the prefix of all subroutine names from SINT to
DINT.

\subsection{Examples and Remarks}

See DRSINT1F and ODSINT1F for an example of the use of SINT1 to compute%
\begin{equation*}
\int_0^\pi \sin x\,dx-2=0
\end{equation*}

\subsection{Functional Description}

The integral is estimated using quadrature formulae due to T.  N.  L.
Patterson \cite{Patterson:1968:TOA}.  Patterson described a family of
formulae in which the $ k^{th} $ formula used all the integrand values
used in the $k-1^{st}$ formula, and added $2^{k-1}$ new integrand values
in an optimal way.  The first formula is the midpoint rule, the second is
the three point Gauss formula, and the third is the seven point Kronrod
formula.  Formulae of this family of higher degree had not previously been
described.  This program uses formulae up to $k=8.$

An error estimate is obtained by comparing the values of the integral
estimated by two adjacent formulae, examining differences up to the
fifteenth order, integrating round-off error, integrating error declared to
have been committed during computation of the integrand, integrating a
first order estimate of the effect round-off error in the abscissae has on
integrand values, and including errors in the limits. The latter four
methods are also used to derive a bound on the achievable precision.

If the integral over an interval cannot be estimated with sufficient
accuracy the interval is subdivided. The difference table is used to
discover whether the integral is difficult to compute because the integrand
is too complex or has singular behavior. In the former case, the estimated
error, requested error tolerance, and difference table are used to choose a
step size.

In the latter case, the difference table is used in a search algorithm to
find the abscissa of the singular behavior. If the singular behavior is
discovered on the end of an interval, a change of independent variable is
applied to reduce the strength of the singularity.

The program also uses the difference table to detect nonintegrable
singularities, jump discontinuities, and computational noise.

The extensive test results given in \cite{Krogh:1974:PTR} and
\cite{Krogh:1977:TRQI} suggest that SINT1 is more reliable and requires
fewer function evaluations than seven other generally available single
precision quadrature subprograms, and DINT1 is more reliable and requires
fewer function evaluations than three other generally available double
precision quadrature subprograms.

\bibliography{math77}
\bibliographystyle{math77}

\subsection{Error Procedures and Restrictions}

Error messages are printed using the extended error message processor
described in Chapter~19.3. If an error does not result in a ``stop,'' error
signals are returned to the user by values of the status flag in IOPT(1).\
Printing, when enabled by Option~2, is executed in subroutines SINTO for
single precision and DINTO for double precision. Both of these programs also
use the message processor described in Chapter~19.3. One can change the
action on errors and parameters affecting the output of messages, by calling
the message/error routine MESS before calling this routine.

\subsection{Supporting Information}

\begin{tabular}{@{\bf}l@{\hspace{5pt}}l}
Entry & \hspace{.35in} {\bf Required Files}\vspace{2pt} \\
DINT1 & \parbox[t]{2.7in}{\hyphenpenalty10000 \raggedright
      AMACH, DINT1, DINTA, DINTDL, DINTDU, DINTF, DINTNS, DINTO, DINTOP,
 DINTSM, DMESS, MESS\rule[-5pt]{0pt}{8pt}}\\
DINTA & \parbox[t]{2.7in}{\hyphenpenalty10000 \raggedright
      AMACH, DINTA, DINTDL, DINTDU, DINTF, DINTNS, DINTO, DINTSM, DMESS,
 MESS\rule[-5pt]{0pt}{8pt}}\\
DINTOP & \parbox[t]{2.7in}{\hyphenpenalty10000 \raggedright
      AMACH, DINTOP, MESS\rule[-5pt]{0pt}{8pt}}\\
SINT1 & \parbox[t]{2.7in}{\hyphenpenalty10000 \raggedright
      AMACH, MESS, SINT1, SINTA, SINTDL, SINTDU, SINTF, SINTNS,SINTO, SINTOP,
 SINTSM, SMESS\rule[-5pt]{0pt}{8pt}}\\
SINTA & \parbox[t]{2.7in}{\hyphenpenalty10000 \raggedright
      AMACH, MESS, SINTA, SINTDL, SINTDU, SINTF, SINTNS, SINTO, SINTSM,
 SMESS\rule[-5pt]{0pt}{8pt}}\\
SINTOP & \parbox[t]{2.7in}{\hyphenpenalty10000 \raggedright
      AMACH, MESS, SINTOP\rule[-5pt]{0pt}{8pt}}\\
\end{tabular}

The source language in ANSI Fortran 77.  Common blocks referenced: SINTC
and SINTEC in the single precision versions, and DINTC and DINTEC in the
double precision versions.  SINTC and DINTC are written using several
COMMON statements, for maintenance purposes.  This usage conforms to the
ANSI Fortran-77 standard, but at least one compiler interprets it
improperly.  If you experience inscrutable errors, try re-writing the
COMMON statements for SINTC (or DINTC) into a single statement.

Designed by Fred T.\ Krogh and W.\ Van Snyder, JPL, 1977. Programmed by W.\
Van Snyder, 1977.\\Revised by W. Van Snyder, 1986, 1988.\\
Error/Message handling revised by F. T. Krogh, March~1992.

Two demonstration drivers are shown below, with the output they produced
when run on an IBM PC/AT with an 80287 floating point coprocessor. The first
demonstrates forward communication usage; the second demonstrates reverse
communication usage.

\begcode

\medskip\
\lstset{language=[77]Fortran,showstringspaces=false}
\lstset{xleftmargin=.8in}

\centerline{\bf \large DRSINT1F}\vspace{10pt}
\lstinputlisting{\codeloc{sint1f}}

\vspace{30pt}\centerline{\bf \large ODSINT1F}\vspace{10pt}
\lstset{language={}}
\lstinputlisting{\outputloc{sint1f}}
\newpage

\lstset{language=[77]Fortran,showstringspaces=false}
\lstset{xleftmargin=.8in}
\centerline{\bf \large DRSINT1R}\vspace{10pt}
\lstinputlisting{\codeloc{sint1r}}

\vspace{10pt}\centerline{\bf \large ODSINT1R}\vspace{10pt}
\lstset{language={}}
\lstinputlisting{\outputloc{sint1r}}
\end{document}
