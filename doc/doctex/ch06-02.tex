\documentclass[twoside]{MATH77}
\usepackage{multicol}
\usepackage[fleqn,reqno,centertags]{amsmath}
\begin{document}
\begmath 6.2  Extended Vector and Matrix Output

\silentfootnote{$^\copyright$1997 Calif. Inst. of Technology, \thisyear \ Math \`a la Carte, Inc.}

\subsection{Purpose}

The subroutines described here print vectors and matrices in an attractive
format. They differ from those in Chapter~6.1 primarily by allowing the user
to indicate the length of the line, the unit number, and the number of
significant digits to print.

\subsection{Usage}

\subsubsection{Program Prototype, Vector Output, Single Precision}

\begin{description}

\item[INTEGER] \ {\bf N , LWIDTH, LUNIT, NUMDIG}

\item[REAL] \ {\bf V}($\geq $N)

\end{description}

Assign values to all subroutine arguments

\begin{center}
\fbox{\begin{tabular}{@{\bf }c}
CALL SVECPR(V, N, $^{\prime}$Text$^{\prime}$, LWIDTH,\\
LUNIT, NUMDIG)\\
\end{tabular}}
\end{center}

The Vector in V() will now have been printed.

\paragraph{Argument Definitions}

\begin{description}

\item[V()] \ [in] The vector to be printed.

\item[N] \ [in] The number of elements in the vector V().

\item[$^{\prime}$Text$^{\prime}$] \ [in] Text to be printed as a heading
for the output.  If the first character is ``0" it is used for Fortran vertical
format control, $i.e$. it gives an extra blank line, else the first
character is treated as part of the text.

\item[LWIDTH] \ [in] The width of the output line. Fortran vertical format
control is not counted as part of the line. Note that on many CRT displays
that have 80~character lines, if exactly 80~characters are printed an extra
blank line is generated. (If LWIDTH $\leq$ 20, the current default line width
is used; see the message routine MESS, Chapter~19.3.)

\item[LUNIT] \ [in] Fortran unit number for the output. If LUNIT is 0 output is
printed on the standard output unit with Fortran vertical format control,
else it is printed to a file without Fortran vertical format control.

(If LUNIT $< 0$, the current default is used; see the message routine
MESS.)

\item[NUMDIG] \ [in] Number of digits to print for each item.
(If NUMDIG $\leq $ 0, the current default is used, which is the full
floating point precision unless this default has been changed with a call
to the message routine MESS.)

\end{description}

\subsubsection{Program Prototype, Matrix Output, Single Precision}

\begin{description}

\item[INTEGER] \ {\bf IDIMA, M, N, LWIDTH, LUNIT, NUMDIG}

\item[REAL] \ {\bf A}(IDIMA, $\geq $N)

\end{description}

Assign values to all subroutine arguments

\begin{center}
\fbox{\begin{tabular}{@{\bf }c}
CALL SMATPR(A, IDIMA, M, N, $^{\prime}$Text$^{\prime}$,\\
LWIDTH, LUNIT, NUMDIG)\\
\end{tabular}}
\end{center}

The Matrix in A will now have been printed.

\paragraph{Argument Definitions}

\begin{description}

\item[A(,)] \ [in] The matrix to be printed.

\item[IDIMA] \ [in] The declared first dimension of A(,).

\item[M] \ [in] The number of rows from A(,) to be printed.

\item[N] \ [in] The number of columns from A(,) to be printed.

\item[$^{\prime}$Text$^{\prime}$] \ [in] Text to be printed as a heading for the
output, as for SVECPR above.

\item[LWIDTH, LUNIT, NUMDIG] \ [in] all are defined as for SVECPR above.
\end{description}

\subsubsection{Modifications for Double Precision}

For double precision usage change the names SVECPR and SMATPR to DVECPR and
DMATPR, and change the REAL declarations to DOUBLE PRECISION.

\subsubsection{Modifications for Integer}

For integer usage change the names SVECPR and SMATPR to IVECPR and IMATPR,
change the REAL declarations to INTEGER, and omit the argument NUMDIG.

\subsection{Examples and Remarks}

The programs DRVECPR and DRMATPR illustrate the use of these routines to
print vectors and matrices respectively. Output is shown in ODVECPR and
ODMATPR.

\subsection{Functional Description}

The contents of the character string are always printed. If no
heading is desired,  use $^\prime~~^\prime$. If one wants the heading on
a separate line from the first line of the vector output, one should pad the
text with enough blanks to fill all but the last character of the line.
Numbers are printed using the fewest number of characters possible, subject
to keeping the numbers from one line to the next aligned, and in the case of
floating point, printing all the digits requested.

The vector output prints $``i-j:"$ at the beginning of every line after the
first, where $i$ is the index for the first component of the vector printed
on that line, and $j$ the index of the last.  If there is more than one
line, the first line is offset so that numbers on following lines line
up with those on the first line.

The matrix output prints ``Col $i"$ at the head of each column, and ``Row $j"
$ at the left of each row, where $i$ is the index for the column and $j$ is
the index for the row.

These routines are short routines that call the message processors
described in Chapter~19.3.  More flexibility in the printing of text
can be obtained by calling SMESS, DMESS, or MESS directly.

\subsection{Error Procedures and Restrictions}

No vector is printed if N $\leq 0$, and no matrix is printed if N,
M, or IDIMA $\leq $ 0.

\subsection{Supporting Information}

The source language is ANSI Fortran~77.

Algorithm and code due to F. T. Krogh, JPL, November~1991.


\begin{tabular}{@{\bf}l@{\hspace{5pt}}l}
\bf Entry & \hspace{.35in} {\bf Required Files}\vspace{2pt} \\
DMATPR & \parbox[t]{2.7in}{\hyphenpenalty10000 \raggedright
AMACH, DMATPR, DMESS, MESS\rule[-5pt]{0pt}{8pt}}\\
DVECPR & \parbox[t]{2.7in}{\hyphenpenalty10000 \raggedright
AMACH, DMESS, DVECPR, MESS\rule[-5pt]{0pt}{8pt}}\\
IMATPR & \parbox[t]{2.7in}{\hyphenpenalty10000 \raggedright
IMATPR, MESS\rule[-5pt]{0pt}{8pt}}\\
IVECPR & \parbox[t]{2.7in}{\hyphenpenalty10000 \raggedright
IVECPR, MESS\rule[-5pt]{0pt}{8pt}}\\
SMATPR & \parbox[t]{2.7in}{\hyphenpenalty10000 \raggedright
AMACH, MESS, SMATPR, SMESS\rule[-5pt]{0pt}{8pt}}\\
SVECPR & \parbox[t]{2.7in}{\hyphenpenalty10000 \raggedright
AMACH, MESS, SMESS, SVECPR}\\
\end{tabular}

\begcode

\medskip\
\lstset{language=[77]Fortran,showstringspaces=false}
\lstset{xleftmargin=.8in}

\centerline{\bf \large DRVECPR}\vspace{10pt}
\lstinputlisting{\codeloc{vecpr}}

\vspace{30pt}\centerline{\bf \large ODVECPR}\vspace{10pt}
\lstset{language={}}
\lstinputlisting{\outputloc{vecpr}}

\newpage
\lstset{language=[77]Fortran,showstringspaces=false}
\lstset{xleftmargin=.8in}

\centerline{\bf \large DRMATPR}\vspace{10pt}
\lstinputlisting{\codeloc{matpr}}

\vspace{30pt}\centerline{\bf \large ODMATPR}\vspace{10pt}
\lstset{language={}}
\lstinputlisting{\outputloc{matpr}}
\end{document}
