\documentclass[twoside]{MATH77}
\usepackage[\graphtype]{mfpic}
\usepackage{multicol}
\usepackage[fleqn,reqno,centertags]{amsmath}
\begin{document}
\opengraphsfile{pl02-08}
\begmath 2.8 Complete Elliptic Integrals K and E

\silentfootnote{$^\copyright$1997 Calif. Inst. of Technology, \thisyear \ Math \`a la Carte, Inc.}

\subsection{Purpose}

These subprograms compute values of the complete elliptic integrals of the
first and second kinds which are defined respectively by%
\begin{align*}
\hspace{-15pt}K(m)&=\int_0^{\pi /2}\!\!\!\left( 1-m\sin ^2t\right)^{-1/2}
\!dt,\ \ \text{for }0\leq m<1,\text{ and}\\
\hspace{-15pt}E(m)&=\int_0^{\pi /2}\!\!\!\left( 1-m\sin ^2t\right)^{1/2}
\!dt,\ \ \text{for }0\leq m\leq 1.
\end{align*}
\subsection{Usage}

\subsubsection{Program Prototype}

\begin{description}
\item[REAL]  \ {\bf YK, YE, SCPLTK, SCPLTE, EM}
\end{description}

Assign a value to EM.

To compute the K() function:
$$
\fbox{{\bf YK =SCPLTK(EM)}}
$$
To compute the E() function:
$$
\fbox{{\bf YE =SCPLTE(EM)}}
$$

\subsubsection{Argument Definitions}

\begin{description}
\item[EM]  \ [in] Value of the parameter, $m$. Require $0\leq m<1$ to
compute K($m)$, and $0\leq m\leq 1$ to compute E($m$).
\end{description}

\subsubsection{Modifications for Double Precision}

For double precision usage, change the REAL statement to DOUBLE PRECISION
and change the subroutine names from SCPLTK and SCPLTE to DCPLTK and DCPLTE.

\subsection{Examples and Remarks}

{\bf Example}: Compute Legendre's relation

\hspace{.25in}$z=\pi /2 -(KE^{\prime}+K^{\prime}E-KK^{\prime}) = 0,$
\quad where

\hspace{.25in}$K=K(m),$\quad $K^{\prime}=K(1-m)$

\hspace{.25in}$E=E(m),$\quad $E^{\prime}=E(1-m).$

See DRSCPLTK and ODSCPLTK for an example of the use of these subprograms to
evaluate this identity.

\subsection{Functional Description}

\subsubsection{Properties of K and E}

These functions are discussed in the references.
\nocite{ams55:Ellip-Int}
\nocite{Hart:1968:CA:Ellip-Int}

The function K($m)$ increases from $\pi /2$ to infinity as $m$ varies from 0
to~1, and is asymptotic to $0.5 \ln (16/(1-m))$\linebreak
\vspace{10pt}

\hspace{5pt}\mbox{\input pl02-08 }

as $m \rightarrow 1.$ Although K($m) \rightarrow \infty $ as $m \rightarrow
1,$ the values of K($m)$ for computer representable values of $m$ close
to one are not extremely large. For example the value of K($m)$ at
computer representable values of $m$ is bounded by 10.6 on a machine
having $10^{-8}$ precision and by 22.1 on a machine having $10^{-18}$
precision.

The function E($m)$ decreases from $\pi /2$ to 1 as $m$ varies from 0 to~1.

The variable $m$ used here is generally called the $parameter$ of the elliptic
functions. Other common parameterizations make use of the $modulus$, $k =
\sqrt m$, or the $modular\ angle$, $\alpha $, satisfying $k = \sin \ \alpha .$

\subsubsection{Computation of K and E}

These subprograms use Chebyshev polynomial approximators due to W. J. Cody,
\cite{Cody:1965:CAC}. These are used in the form

\hspace{.4in}$P_n(1-m) - \ln (1-m) Q_n(1-m)$

where $P_n$ and $Q_n$ are different polynomials for K and E, and $n$ is the
degree of the polynomials.

The negative logarithm base ten of the maximum absolute error of these
approximators is given for degrees 5,~9, and~10 as follows:
\begin{center}
\begin{tabular}{rrr}
& \multicolumn{1}{c}{\bf Precision of} & \multicolumn{1}{c}{\bf Precision of}\\
\multicolumn{1}{c}{\bf Degree $n$} & \multicolumn{1}{c}{\bf K approximator} &
\multicolumn{1}{c}{\bf E approximator}\\
5 \rule{.25in}{0pt} & 9.50 \rule{.35in}{0pt} & 9.44 \rule{.35in}{0pt}\\
9 \rule{.25in}{0pt} & 15.87 \rule{.35in}{0pt} & 15.84 \rule{.35in}{0pt}\\
10 \rule{.25in}{0pt} &  17.45 \rule{.35in}{0pt} & 17.42 \rule{.35in}{0pt}
\end{tabular}
\end{center}
The subprograms use degree $n = 5$ on machines for which $%
-\log _{10}(\text{R1MACH}(3)) < 8.2$, degree $n = 9$ on machines for
which $-\log _{10}(\text{R1MACH}(3)) < 16.2$, and degree $n = 10$ on
machines having more precision. The accuracy of these subprograms is
limited to~17.4 decimal places even on machines having more precision.
See Chapter~19.1 for a description of R1MACH.

\subsubsection{Accuracy Tests}

Subprograms SCPLTK and SCPLTE were each tested on an IBM compatible
PC using IEEE arithmetic by
comparison with DCPLTK and DCPLTE, respectively, at~10,000 points in the
interval (0.0,~1.0). The relative precision of the IEEE single precision
arithmetic is $\rho = 2^{-23} \approx 1.192 \times 10^{-7}$.

For SCPLTK, 33\% of the test points gave relative errors less than $\rho $.
The maximum relative error observed was $2.0\rho .$

For SCPLTE, 68\% of the relative errors were less than $\rho $. The maximum
relative error observed was $1.9\rho .$

\bibliography{math77}
\bibliographystyle{math77}

\subsection{Error Procedures and Restrictions}

The K subprograms issue an error message if $m < 0$ or $m \geq 1$. The E
subprograms issue an error message if $m < 0$ or $m > 1$. On error
conditions the value zero is returned.  Error messages are processed
using the subroutines of Chapter~19.2 with an error level of zero.

\subsection{Supporting Information}

The source language is ANSI Fortran~77.

\begin{tabular}{@{\bf}l@{\hspace{5pt}}l}
\bf Entry & \hspace{.35in} {\bf Required Files}\vspace{2pt} \\
DCPLTE & \parbox[t]{2.7in}{\hyphenpenalty10000 \raggedright
AMACH, DCPLTE, DERM1, DERV1, ERFIN, ERMSG\rule[-5pt]{0pt}{8pt}}\\DCPLTK & \parbox[t]{2.7in}{\hyphenpenalty10000 \raggedright
AMACH, DCPLTK, DERM1, DERV1, ERFIN, ERMSG\rule[-5pt]{0pt}{8pt}}\\SCPLTE & \parbox[t]{2.7in}{\hyphenpenalty10000 \raggedright
AMACH, ERFIN, ERMSG, SCPLTE, SERM1, SERV1\rule[-5pt]{0pt}{8pt}}\\SCPLTK & \parbox[t]{2.7in}{\hyphenpenalty10000 \raggedright
AMACH, ERFIN, ERMSG, SCPLTK, SERM1, SERV1\rule[-5pt]{0pt}{8pt}}\\\end{tabular}

Designed and programmed by E. W. Ng, JPL, 1974. Modified by K. Stewart, JPL,
1981, C. L. Lawson and S. Y. Chiu, JPL, 1983.


\begcode

\medskip
\lstset{language=[77]Fortran,showstringspaces=false}
\lstset{xleftmargin=.8in}

\centerline{\bf \large DRSCPLTK}\vspace{10pt}
\lstinputlisting{\codeloc{scpltk}}

\vspace{30pt}\centerline{\bf \large ODSCPLTK}\vspace{10pt}
\lstset{language={}}
\lstinputlisting{\outputloc{scpltk}}
\closegraphsfile
\end{document}
