\documentclass[twoside]{MATH77}
\usepackage{multicol}
\usepackage[fleqn,reqno,centertags]{amsmath}
\begin{document}
\hyphenation{JTPREV}
\begmath 4.5 Sequential Solution of a Banded Least-Squares Problem

\silentfootnote{$^\copyright$1997 Calif. Inst. of Technology, \thisyear \ Math \`a la Carte, Inc.}

\subsection{Purpose}

Let a linear least-squares problem be denoted by
\begin{equation*}
A{\bf x}\simeq {\bf b}
\end{equation*}
where $A$ is a given $m\times n$ matrix with $m\geq n$, ${\bf b}$ is a given
$m$-vector, and it is required to find an $n$-vector, ${\bf x}$, that is an
approximate solution to this equation in the least-squares sense. The given
data for this problem can be regarded as the composite matrix, $[A:{\bf b}].$

Assume further that the matrix $A$ is {\em banded}, in the sense that there
is an integer, NB $\leq n$, such that in any row of $A$ all nonzero elements
occur within a set of NB contiguous positions. This set of subroutines is
intended for the case in which NB is significantly smaller than $n$ so there
is the possibility of achieving some significant saving of storage and
execution time by taking advantage of this band property.

Subroutine SBACC can be used to preprocess the data matrix, $[A:{\bf b}]$,
sequentially. Subroutine SBSOL can be used to solve the problem after the
transformation accomplished by SBACC. Subroutine SBSOL can also be used to
compute a covariance matrix for the solution vector.

\subsection{Usage}

\subsubsection{Sequential processing of data}

The user must make a sequence of calls to SBACC, sending a block of rows of $%
[A:{\bf b}]$ with each call. On the first call the user sets NB, the
bandwidth, which will remain unchanged after that. Also, on the first call
the user must set IR = 1, and JT = 1. After that IR will be updated by SBACC.

Along with each block the user sets two integers, MT and JT, indicating that
MT new rows are being provided, within these rows of $A$ only the NB columns
beginning with column JT are being provided, and that the elements in all
other columns in these rows of $A$ are zero. The user puts this block of
data in the array G(,), beginning with row IR of G(,). The data
corresponding to column JT of $A$ goes into the first column of G(,). Data
from the vector ${\bf b}$ of the problem goes into column NB + 1 of G(,).

The data blocks must be ordered so that from one call to the next, the value
of the column index JT either remains the same or increases.

\paragraph{Program Prototype, Single Precision}

\begin{description}
\item[INTEGER]  \ {\bf LDG, NB, IR, MT, JT, JTPREV, IERR2}

\item[REAL]  \ {\bf G}(LDG, $\geq $NB+1)
\end{description}

When starting a new problem set LDG, NB, and IR. On the initial call and all
subsequent calls for the same problem, set MT and JT and store MT rows of
data into G(,) beginning at row IR.

\begin{center}
\fbox{\begin{tabular}{@{\bf }c}
CALL SBACC(G, LDG, NB, IR, MT,\\
JT, JTPREV, IERR2)\\
\end{tabular}}
\end{center}

Following the call the contents of G(,) will have been altered, reflecting
the processing of the new data. The values of JTPREV and IR may have been
changed. IERR2 will be set.

\paragraph{Argument Definitions}

\begin{description}
\item[G(,)] \ [inout] The working array.  With MT, JT, and IR defined below,
  on each call to SBACC the user places columns JT to JT + NB $-$ 1 of MT rows
  of $A$ into G(IR:IR+MT$-$1, 1:NB) and the corresponding elements of
  $\mathbf{b}$ into G(IR:IR+MT$-$1, NB+1).  It is implied that within these MT
  rows of $A$ all elements not in columns JT through JT + NB $-$ 1 are zero.

\item[LDG]  \ [in] The leading dimensioning parameter for G(,). LDG must be
large enough so that on each call to SBACC one has $\text{IR}+\text{MT}-1\leq $
LDG.

MT can be set by the user on each call. Let $MT_{\max }$ denote the largest
value the user will assign to MT. If the user keeps JT $\leq n-\text{NB} +
1$ (see discussion in Section C) then the largest value of IR will be $n+2$,
so it would suffice to set LDG $=n+1+MT_{\max }.$

If the user permits JT to be as large as $n$, then LDG must be at least $n$
+ NB + $MT_{\max }.$

\item[NB]  \ [in] Set by user on the initial call for a problem. Must not be
changed during the processing for one problem. NB indicates the bandwidth of
the data matrix $A$. In any row of $A$ all nonzero elements must appear
in some set of NB consecutive columns.

\item[IR]  \ [inout] Index of first row of G(,) into which the user is to
place new data. The variable IR must be set by the user to the value~1 on
the initial call to SBACC and must not be altered after that by the user on
successive calls for the same problem. IR will be updated in SBACC by
(effectively) setting IR $=$ JT + $\min ($NB + 1, MT + $\max $(IR $-$ JT,~0)).

Let $IR_{in}$ denote the value of IR on entry to SBACC and $IR_{out}$ denote
its value on return. These quantities will satisfy%
\begin{equation*}
IR_{in}\leq IR_{out}\leq \text{JT + NB + 1}.
\end{equation*}
\item[MT]  \ [in] Set by user to indicate the number of new rows of data
being introduced by the current call to SBACC. SBACC will return immediately
if MT $\leq $ 0. MT will not be altered by SBACC.

\item[JT]  \ [in] Set by user to indicate the column of the new sub-block of
$A$ that the user is storing in the first column of the work array, G(,).
The user must set JT = 1 on the initial call to SBACC and must either leave
JT the same or increase it on each successive call for the same problem. Too
large an increase between successive calls can cause the problem to be
structurally singular. See Section E for more information on this. JT will not
be altered by SBACC.

\item[JTPREV]  \ [inout] Need not be set before the first call to
SBACC for a problem. Must not be altered by the user on later calls
for the same problem. JTPREV is used within SBACC to mark the row of
G(,) at which the method of packing data changes. Quantities in G(,)
at and below row JTPREV are subject to potential change during the
processing of additional data, whereas quantities above row JTPREV
are not.

On return, SBACC sets JTPREV = JT.  JTPREV, as set by SBACC, is needed
as input when calling SBSOL.

\item[IERR2]  \ [inout] Error status indicator. Need not be set before the
initial call for a problem. Will be set by SBACC to zero if no errors are
detected and to nonzero values, as described in Section E, if error
conditions are detected.
\end{description}

\paragraph{Modifications for Double Precision}

For double precision usage change the REAL statement to DOUBLE PRECISION
and change the subroutine name SBACC to DBACC.

\subsubsection{Computation of Solution Vector}

\paragraph{Program Prototype, Single Precision}

\begin{description}
\item[INTEGER]  \ {\bf MODE, LDG, NB, IR, JTPREV, N, IERR3}

\item[REAL]  \ {\bf G}(LDG, $\geq $NB+1){\bf , X}($\geq N)${\bf , RNORM}
\end{description}

Set MODE and N. If MODE is set to~2~or~3 then also store a vector, ${\bf p}$%
, into X(). G(,), NB, IR, and JTPREV should have values resulting from
previous calls to SBACC.

\begin{center}
\fbox{\begin{tabular}{@{\bf }c}
CALL SBSOL(MODE, G, LDG, NB, IR, \\
JTPREV, X, N, RNORM, IERR3)\\
\end{tabular}}
\end{center}

On return, X() and RNORM will contain computed results and IERR3 will be
set. No other quantities in the argument list will be modified.

\paragraph{Argument Definitions}

\begin{description}
\item[MODE]  \ [in] The previous processing of data by SBACC will have left
a representation of an upper triangular matrix, $R$, and a vector, ${\bf y}$%
, in the array, G(,). See Section~D for the interpretation of these quantities.
The user selects the desired solution process by setting MODE to~1,~2, or~3.

\begin{itemize}
\item[=1]  Solve $R{\bf x}={\bf y}$, where $R$ and ${\bf y}$ are
contained in G(,) as the result of previous calls to SBACC. The
solution vector, ${\bf x}$, will be stored in X(). This gives the solution
to the least-squares problem, $A{\bf x}\simeq {\bf b}.$

\item[=2]  Solve $R^t{\bf x}={\bf p}$, where $R$ is the matrix residing in
G(,) as the result of previous calls to SBACC, and ${\bf p}$ is a vector
placed in X() by the user. The solution vector, ${\bf x}$, will replace $%
{\bf p}$ in X().

\item[=3]  Solve $R{\bf x}={\bf p}$, where $R$ is the matrix residing in G(,)
as the result of previous calls to SBACC, and ${\bf p}$ is a vector placed
in X() by the user. The solution vector, ${\bf x}$, will replace ${\bf p}$
in X().
\end{itemize}

\item[G(,),LDG,NB,IR,JTPREV]  \ [in] These arguments must contain values as
they were defined upon the return from a preceding call to SBACC.

\item[X()]  \ [inout] On input, with MODE = 2 or 3, this array must contain
the N-dimensional right-side vector of the system to be solved. On return,
with MODE = 1, 2, or 3, this array will contain the N-dimensional solution
vector of the appropriate system that has been solved.

\item[N]  \ [in] Set by user to specify the dimensionality of the desired
solution vector. This causes the subroutine SBSOL to use only the leading N$%
\times $N submatrix of the triangular matrix currently represented in the
array G(,). An error is reported if this submatrix is singular.

\item[RNORM]  \ [out] If MODE = 1, RNORM is set by the subroutine to the
norm of the residual vector for the least-squares problem, $i.e.$, $\Vert
{\bf b}-A{\bf x}\Vert $. This number is computed as%
\begin{equation*}
\left[\, \sum_{I=\text{N}+1}^{\text{IR}-1}\text{G(}I\text{, NB + 1})^2\right]
^{1/2}
\end{equation*}
If MODE = 2 or 3, RNORM is set to zero.

\item[IERR3]  \ [out] Error status indicator set by SBSOL. Zero means no
errors detected. If a diagonal element of the leading N$\times $N submatrix
represented in G(,) is zero, an error message will be issued and IERR3 will
be set to the index of the first zero diagonal element. In this latter case
the solution vector X() will not be computed.
\end{description}

\paragraph{Changes for Double Precision}

For double precision usage change the REAL statement to DOUBLE PRECISION and
change the subroutine name SBSOL to DBSOL.

\subsection{Examples and Remarks}

\subsubsection{Computation of the covariance matrix for the solution vector}

The features provided by MODE = 2 and 3 are primarily intended to support
the computation of the unscaled covariance matrix, $C$, for the least-%
squares problem. This matrix, $C$, is defined by $C = (A^tA)^{-1}$. However,
since $R^tR = A^tA$ (See Section~D), $C$ is also given by $C = (R^tR)^{-1}$,
and thus $C$ satisfies the equation, $R^tRC = I$, where $I$ is the $n\times n
$ identity matrix. It follows that the matrix, $C$, can be computed in two
steps, first solving
\begin{equation*}
R^tZ = I
\end{equation*}
for $Z$, and then solving
\begin{equation*}
RC = Z
\end{equation*}
for $C$. Using SBSOL or DBSOL the matrices $Z$ and $C$ can be computed one
column at a time.

This matrix, $C$, must be multiplied by an estimate of the variance of the
data error to obtain the solution covariance matrix. This variance can be
estimated using N, MTOTAL, and RNORM as follows:

\hspace{.2in}DOF = MTOTAL $-$ N

\hspace{.2in}VAR = RNORM$**$2 / DOF

where MTOTAL is the total number of rows of $A$ introduced into the problem.

\subsubsection{Demonstration problem}

As a demonstration problem we compute the continuous piecewise linear
function that best fits a sample of 91~values of the sine function in the
least-squares sense. Sample values are values of the sine function at one
degree steps from zero to 90~degrees. The piecewise linear function will be
parameterized by a set of ten values, $y_i$, $i = 1$, 10, which will be the
values of the piecewise linear function at the arguments 0, 10, 20, ..., 90
degrees. The fitted function will be defined by linear interpolation between
adjacent pairs of these points.

Note that because the sine curve is concave down throughout this interval we
expect the knots, $y_i$, each to lie above the sine curve, with the linearly
interpolated segment between each adjacent pair of knots passing below the
sine curve.

Program DRDBACC illustrates the use of DBACC and DBSOL to compute the ten
values of $y_i$ for this problem. It also computes the (formal) $10\times 10$
covariance matrix for these quantities. The results are shown in ODDBACC.

Note from the listing of residuals that the residuals at $x = 0$, 10, 20,
... degrees are positive, while the residuals at $x = 5$, 15, 25, ...
degrees are negative, as expected.

This problem is not really statistical since the given data are essentially
exact. Thus the computed SIGFAC and covariance matrix do not really have a
statistical interpretation, but merely serve to illustrate how to use these
subroutines to compute these quantities. Note in particular that the
covariance matrix is symmetric (to some level of accuracy), as it should be,
even though this method computes each column of the covariance matrix
independently.

For an example of more general usage of DBACC see a listing of the MATH77
library subroutine, DC2FIT, Chapter~11.4.

\subsubsection{Data blocks having fewer than NB contiguous columns of nonzeros}

This subroutine requires the value of NB to be constant throughout the
processing of one problem. If the data block being sent to SBACC in one call
has all of its nonzeros in fewer than NB contiguous columns the user must
pad out the block to a width of NB columns by including zeros. The padding
columns may be either on the left or the right or both.

For example, suppose NB = 4 and one is introducing a row, or block of rows,
of $A$ having nonzeros only in columns 5, 6, and 7. One may either set JT =
4 and send columns 4, 5, 6, and 7, with column~4 containing zeros; or else
one may set JT = 5 and send columns 5, 6, 7, and 8, with column~8 containing
zeros.

In choosing between these alternatives there are two points to consider. JT
must be nondecreasing on successive calls to SBACC. Sending padding columns
that are beyond the last actual column of $A$ causes the algorithm to use
more rows of storage than would otherwise be necessary.

\subsection{Functional Description}

Subroutine SBACC uses Householder orthogonal transformations to process the
given data, producing an equivalent least-squares problem of the form%
\begin{equation*}
\left[
\begin{array}{c}
R \\
0
\end{array}
\right] \simeq \left[
\begin{array}{c}
{\bf y} \\ \alpha
\end{array}
\right]
\end{equation*}
where $R$ is an $n\times n$ upper triangular matrix with a bandwidth of NB, $%
{\bf y}$ is an $n$-vector, and $\alpha $ is a scalar quantity. These
quantities are related to the data, $[A:{\bf b}]$, by the relations, $%
R^tR=A^tA$, $R^t{\bf y}=A^t{\bf b}$, and ${\bf y}^t{\bf y}+\alpha ^2={\bf b}%
^t{\bf b}.$

The solution, ${\bf x}$, for this problem is also the solution for the given
least-squares problem, $A{\bf x}\simeq {\bf b}$, and can be computed by
solving the triangular system, $R{\bf x}={\bf y}$. This latter system is
solved for ${\bf x}$ when SBSOL is called with MODE = 1.

The residual vector for the transformed least-squares problem is
\begin{equation*}
\left[
\begin{array}{c}
\bf 0 \\
\alpha
\end{array}
\right].
\end{equation*}
The norm of this residual vector is $|\alpha |$ and this is also equal to $%
\Vert {\bf b}-A{\bf x}\Vert $. After $n$ linearly independent rows of $A$
have been accumulated, where $n$ is the number of columns of $A$, if a
solution is requested with N $=n$, then the value $|\alpha |$ will be
returned as RNORM. If a solution is requested with N $<n$, components of $%
{\bf y}$ beyond $y_N$ will also be used in computing RNORM.

Subroutine SBACC dynamically partitions the array G(,) into three segments
by groups of rows. These segments are rows~1 through JTPREV $-$ 1, rows JTPREV
through IR $-$ 1, and rows IR through LDG.

The first segment holds rows of $[R:{\bf y}]$ for which processing is
completed. In this segment element $r_{i,j}$ is stored in G($i,j-i+1).$

The second segment, consisting of at most NB + 1 rows, holds rows of%
\begin{equation*}
\left[
\begin{array}{ccc}
R & : & {\bf y} \\
0 & : & \alpha
\end{array}
\right]
\end{equation*}
that may be changed by data yet to be received. In this segment element $%
r_{i,j}$ is stored in G($i,j-\text{JTPREV}+1).$

The third segment is used to receive new rows of $[A:{\bf b}].$

The subroutine SBSOL alters only the contents of X() and RNORM. Thus it is
permissible to alternate between accumulating data using SBACC and obtaining
a current solution or covariance matrix using SBSOL. It is also permissible
to use SBSOL with N $< n$. The subroutine SBSOL can compute a solution
vector of length N whenever the portion of the $A$ matrix introduced to that
point has the property that its first N columns are linearly independent.


\bibliography{math77}
\bibliographystyle{math77}

\subsection{Error Procedures and Restrictions}

In response to detected error conditions, SBACC will set IERR2 nonzero and
issue an error message using the error message routines of Chapter~19.2. On
errors numbered 1 through~4, SBACC will return. Generally the calling program
should abandon the problem processing in these cases. When IERR2 = 3 it is
conceivable, but not likely, that the user might wish to continue the
processing, in which case the user must reset IERR2 to zero. If SBACC is
reentered with IERR2 $\neq 0$, (and IR $> 1)$ it will set IERR2 = 5 and
execute a STOP.

{\bf IERR2 \hspace{.2 in} Explanation}\vspace{-10pt}
\begin{itemize}
\item[1]  MT $>$ LDG $-$ IR + 1. To correct this problem either use a
larger dimension LDG, or a smaller block size MT.

\item[2]  JT $<$ JTPREV. This occurs if the data blocks are out of order, in
the sense that the current JT is smaller than JT on the previous call.

\item[3]  JT $>\min ($JTPREV + NB, IR). This occurs when the difference
between the current JT and JT on the previous call is so large that the
matrix would be structurally singular. This condition does not, however,
preclude the processing of the data to triangular form, and thus SBACC will
do the processing, unless there is a storage limitation, in which case IERR2
will be set to~4.

This condition is likely to be due to a program usage error.  If not, then
the problem needs either more data in preceding blocks or a mathematical
model with fewer or differently defined free parameters.  Alternatively,
see pp.~218--219 of \cite{Lawson:1974:SLS} for ideas on stabilizing such a
structurally singular problem.

\item[4]  The condition described above for IERR2 = 3 holds and there is a
storage limitation indicated by MT $>\text{LDG}-\text{JT}+1$. The storage
problem could be relieved by increasing LDG or decreasing MT. However, the
more fundamental problem indicated by IERR2 = 3 must still be dealt with.

\item[5]  SBACC has been entered with IERR2 $\neq 0$. SBACC will execute a
STOP since the calling program has ignored a nonzero setting of IERR2 on the
previous call.
\end{itemize}

The following conditions are not tested in SBACC and violation will have
unpredictable effects: IR, and JT must be set to~1 on the first call to
SBACC. JTPREV and IR must not subsequently be altered by the user during the
processing for one problem.

SBACC will return immediately if MT $\leq $ 0. This is not regarded as an
error condition.

If SBSOL encounters a zero diagonal term in the N$\times $N matrix R, IERR3
will be set to the index of the (first) zero term and an error message will
be issued. The solution X() will not be computed in this case.

\subsection{Supporting Information}

The source language is ANSI Fortran~77.

These subroutines are adaptations to the JPL MATH77 library of the
algorithms and subroutines BNDACC and BNDSOL that were developed by C.  L.
Lawson and R.  J.  Hanson at JPL in~1972 and described in detail in
\cite{Lawson:1974:SLS}.


\begin{tabular}{@{\bf}l@{\hspace{5pt}}l}
\bf Entry & \hspace{.35in} {\bf Required Files}\vspace{2pt} \\
DBACC & \parbox[t]{2.7in}{\hyphenpenalty10000 \raggedright
DBACC, DHTCC, DNRM2, ERFIN, ERMSG, IERM1, IERV1\rule[-5pt]{0pt}{8pt}}\\
DBSOL & \parbox[t]{2.7in}{\hyphenpenalty10000 \raggedright
DBSOL, ERFIN, ERMSG, IERM1, IERV1\rule[-5pt]{0pt}{8pt}}\\
SBACC & \parbox[t]{2.7in}{\hyphenpenalty10000 \raggedright
ERFIN, ERMSG, IERM1, IERV1, SBACC, SHTCC, SNRM2\rule[-5pt]{0pt}{8pt}}\\
SBSOL & \parbox[t]{2.7in}{\hyphenpenalty10000 \raggedright
ERFIN, ERMSG, IERM1, IERV1, SBSOL}\\
\end{tabular}

\begcode

\medskip\
\lstset{language=[77]Fortran,showstringspaces=false}
\lstset{xleftmargin=.8in}

\centerline{\bf \large DRDBACC}\vspace{10pt}
\lstinputlisting{\codeloc{dbacc}}
\newpage
\vspace{30pt}\centerline{\bf \large ODDBACC}\vspace{10pt}
\lstset{language={}}
\lstinputlisting{\outputloc{dbacc}}

\end{document}
