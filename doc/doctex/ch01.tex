\documentclass[twoside]{MATH77}
\usepackage{multicol}
\usepackage[fleqn,reqno,centertags]{amsmath}
\usepackage{url}
\begin{document}
\intro{0}

\begmath 1.0 MATH77 and {\em mathc90}

\silentfootnote{$^\copyright$1997 Calif. Inst. of Technology, \thisyear \ Math \`a la Carte, Inc.}

\silentfootnote{$^\copyright$1997 Calif. Inst. of Technology, \thisyear \ Math
  \`a la Carte, Inc.}

This manual and the software described herein constitute
the \base_site MATH77 and {\em mathc90} libraries. MATH77 is a library of
Fortran~77 subprograms implementing algorithms useful in numerical
computation. MATH77 contains over 550 user-callable entries.

{\em mathc90} is an ANSI C language version of most of the MATH77 library.
For usage of {\em mathc90} see Appendices C and D. The remainder of this
introductory chapter applies primarily to MATH77 (Fortran).

The Table of Contents provides a directory of the library by topics. The
Index lists each user-callable subprogram name, with its argument
list, and a reference to the chapter in which it is described. Appendix A
lists for each subprogram name all of the program files from MATH77 needed
for that subprogram. Appendix B lists all SUBROUTINE, FUNCTION, ENTRY, and
COMMON names used in the library. This will be useful if one wishes to avoid
using the same names.

A set of demonstration drivers that illustrate the use of library
subprograms accompanies the library.  These can be used to test that the
library is properly installed on a new system.  An individual driver may
be useful to a user as a starting point for using a library subprogram.
Listings of about half the demonstration drivers are included in this
manual.

These libraries were developed at the Jet Propulsion Laboratory 
from 1977--1997 and have been licensed to Math \`a la Carte by the
California Institute of Technology.

\subsection{Purpose and Scope}

The purpose of high-quality mathematical subprogram libraries is to make
scientific and engineering computing more reliable and economical, and to
reduce the length of time from problem conception until a solution is
obtained.  The MATH77 and {\em mathc90} libraries are particularly useful
in that there are no language portability difficulties to inhibit
transport of the libraries to all Fortran~77 and ANSI C environments.

Computers ranging from microcomputers to supercomputers are used for
scientific and engineering computing.  Most of these systems support both ANSI
Fortran~77 and ANSI C, and thus allow use of the MATH77 and/or {\em mathc90}
libraries.  For persons programming in other languages, most systems provide
support for making calls to Fortran~77 or ANSI C libraries.

The Fortran~77 language was initially issued as an ANSI standard in 1978,
\cite{Fortran77}, and was reaffirmed by ANSI in 1988.  It was also a U.S.
Federal standard (FIPS PUB 69, Sept.~1980) and an international (ISO)
standard.  The ANSI X3J3 committee then developed Fortran~90,
\cite{Fortran90}, and Fortran~95, \cite{Fortran95}, which became the
current Fortran standard.  Fortran~90 was designed to include all of
Fortran~77, although this is no longer true of Fortran~95.  There are a
very few programs in MATH77 that do not satisfy the Fortran~95 standard,
but these should still compile on such compilers.  Although the later
versions of Fortran are a substantial improvement over Fortran~77, we
continue to use Fortran~77, as it is only in this language that we have a
way to generate the C library automatically.

The first ANSI standard for C was issued in 1989 \cite{C90}, and ANSI C is
also a U.S. Federal standard (FIPS PUB 160, March 1991).

The MATH77 library of mathematical subprograms has the following
attributes:
\begin{itemize}
\item[1.] Most of the subprograms in MATH77 do not require any
modifications to function as described in this document on any computer
system supporting the full Fortran~77 language.  The only program file
requiring attention when moving the library to different machines is
AMACH.  This file contains machine-dependent constants that are accessed
by other library subprograms, and can also be accessed by user programs,
by referencing D1MACH, R1MACH, and I1MACH.  See Chapter~19.1 for
instructions on converting the file AMACH to different systems.

\item[2.] All subprograms in MATH77 are coded in conformity to the Fortran~77
standard. This property has been checked by use of a processor for standards
checking and by use of the standards checking option on various compilers.
\end{itemize}

\subsection{Access to the MATH77 and {\em mathc90} Libraries}

Access to the libraries is via the URL, \url{http://mathalacarte.com}.  One
can freely browse what is available, but if you wish to download anything you
must register.

Items available include mangled ({\em i.e.}\ compiler readable only) and clean
source code for the library codes, clean source for the demonstration drivers,
as well as the users' manual suitable for viewing or printing with a PDF,
PostScript, or dvi processor.  \if j\outtyp The hard copy version of the
users' manual can be obtained, for a nominal internal JPL crosscharge, from
the Documentation Department of the Computer Store, (4--8600, T--1704A).  \fi

\subsubsection{Files containing the users' manual}

Individual chapters and appendices of the users' manual are processed under
Linux using te\TeX.

Special processors can be used on the {\tt .pdf}, or {\tt .ps}, {\tt .dvi}
files to view the manual as it appears on the printed page.  There are also
processors that convert {\tt .dvi} files to files that can be printed (such as
{\tt .ps} files); the {\tt .ps} and {\tt .pdf} files can usually be printed
directly.  Most machines have public domain versions of the processors
mentioned here.

\if j\outtyp
\subsubsection{Other Libraries}

Other libraries, such as LINPACK, EISPACK, and LAPACK, are available
starting from \url{http://math.jpl.nasa.gov/more-ftp.html}.  They can also
be obtained by anonymous FTP from {\tt math.jpl.nasa.gov}.  The IP address
is presently 137.78.227.106, but this may change without notice.  If you
have trouble contacting the FTP server by using the name, make sure your
{\tt hosts} file is up to date.  Once connected, use the user-id {\tt
anonymous} or {\tt ftp}, and use your electronic mail address for the
password.  If you have no electronic mail address, just use your name,
with hyphens or underscores instead of spaces.

FTP supports {\em ascii} mode set by the command {\tt ascii} or {\tt
TYPE~A}, and {\em image} mode set by the command {\tt bin} or {\tt
TYPE~I}. Use {\em image} mode for files with suffixes {\tt .dvi,
.arj, .a01,..., .tar,.tar.gz, .zip,} or {\tt .Z,} and otherwise use {\em
ascii} mode.

The mathematics libraries are stored in the directory {\tt
/dist/mathlibs}.

With anonymous FTP, users can put files into the {\tt incoming} directory.
This facility allows users to supply programs that show errors in, or
illustrate difficulty in using, mathematics library components.  Please
contact Fred Krogh before depositing a file in the {\tt incoming}
directory.

Links to many other sources of mathematical and statistical software are
maintained at \url{http://math.jpl.nasa.gov/other-sw.html}.

\subsubsection{Usage on JPL Supercomputers}

On the JPL Cray J90 system ({\tt galaxy.jpl.nasa.gov}), the complete
libraries are provided with the names {\tt MATH77} and {\tt mathc90}.  In
the MATH77 version, all library subprogram arguments and function results
that are specified in this manual as DOUBLE PRECISION will instead be of
type REAL.  This version should be used with application code that is
compiled using the {\tt -dp} option.  DOUBLE PRECISION on a Cray (i.e.,
16-byte floating-point numbers) is not efficient, almost never necessary,
and is thus not recommended.  If it is deemed necessary, contact the
Supercomputing consultants by email at \url{consult@galaxy} to obtain a
version of the Fortran library compiled for Cray DOUBLE PRECISION.

On the Caltech/JPL Exemplar system (\url{neptune.caltech.edu}), the
complete libraries are provided with the names {\tt MATH77} and {\tt
mathc90}.  In order to link these libraries into your compiled code,
follow the instructions found in: \url{
http://www.cacr.caltech.edu/local_docs/exemplar/mini-guide.html} or
contact the consultants by email at \url{hp-support@cacr.caltech.edu}.
\fi

\subsection{Conventions Followed in the Code and Documentation}

A number of conventions are followed in the library code and its
descriptions. All of the library subprograms, demonstration drivers,
and test drivers exist as ANSI Fortran~77 source code.

Subprograms that produce printed output use the PRINT or WRITE(*,...)
statement to write to the standard system output unit. Exceptions are the
message writing subroutines of Chapter~19.3 which can write to arbitrary
user-specified Fortran I/O units. Error message writing is handled through
subroutines described in Chapters 19.2. and~19.3.

For most library subprograms we follow the convention introduced in the
BLAS \cite{Lawson:1979:BLA} and LINPACK \cite{Dongarra:1979:LUG}, by
which the initial letter of the name of a SUBROUTINE is C, D, I, S, or Z,
to indicate the principal type of data with which the subprogram is
concerned, and the initial letter of the name of a FUNCTION is C, D, I, or
S, to indicate the type of the returned result.  These letters denote,
respectively, COMPLEX (or CHARACTER), DOUBLE PRECISION, INTEGER, REAL, or
double-precision complex.  The Fortran~77 standard does not directly
support a double-precision complex type, so subprograms oriented toward
this type of data use pairs of DOUBLE PRECISION numbers.  This
representation is compatible with the representation used by most
compilers that extend the Fortran~77 standard to provide a double
precision complex data type, and is compatible with the representation
specified by the later Fortran standards.

When a DOUBLE PRECISION or COMPLEX valued FUNCTION from this library is
used, it is essential to have the FUNCTION name appear in a DOUBLE PRECISION
or COMPLEX type statement.

Each subprogram description consists of six sections:
\begin{tabbing}
\hspace{.3in}\=A.\ \ \=Purpose\\
\>B.\>Usage\\
\>C.\>Examples and Remarks\\
\>D.\>Functional Description\\
\>E.\>Error Procedures and Restrictions\\
\>F.\>Supporting Information
\end{tabbing}

The contents of each of these sections and special notational conventions
used in the descriptions are specified below under these six headings.

\introalt
\subsection{Purpose}

A brief statement of the area of application of the subprogram is given here.

\subsection{Usage}

A detailed explanation of how to use the subprogram is given in this
section. Typical subsections are as follows:

\subsubsection{Program Prototype}

Specification statements, variables that must be initialized, calling
sequences, and any other statements likely to be required are given here. It
should not be difficult to use the subprogram if one follows this subsection
line-by-line.

In giving dimension information, a statement of the form

{\bf REAL} {\bf A}(IDIMA, $\geq $ N){\bf , B}($\geq 5 \times \text{J} + 20)$

is used to indicate that A() is a real two-dimensional array with a first
dimension that must equal IDIMA and a second dimension that can be assigned
any value greater than or equal to the value assigned to N, and that B()
is a real one dimensional array with dimension $\geq 5 \times \text{J} + 20.$

The calling sequence for any entry is always enclosed in a box. For FUNCTION
subprograms, the calling sequence is always given in the form Y = FNAME(...).
The reader should recall that Fortran syntax also permits a FUNCTION name
to be used directly in an expression, as for instance Y = 2.0 * FNAME(...)
+ SQRT(X).

\subsubsection{Argument Definitions}

A detailed explanation of the parameters in the calling sequence is given
here.  Any parameter that is an array name is followed by ``()" to call
attention to this fact.  The intent attributes $in$, $out$, and $inout$,
that are described in the Fortran~90 standard \cite{Fortran90}, are
listed for each subprogram argument.  We also use intent attributes $work$
or $scratch$, not part of the Fortran~90 standard, to describe parameters
for which the using program must provide space, but need not provide
initial values, and in which no meaningful results are returned.

\subsubsection{Modifications}

One or more {\bf Modifications} subsections may be present to describe such
things as DOUBLE PRECISION versions of a subprogram or to describe
significantly distinct options in the usage of a subprogram.

\subsection{Examples and Remarks}

This section discusses a sample (``demo") program, with
output, illustrating the use of the subprogram.  Any listings and actual
output are at the end of the chapter.

In designing the demonstration programs, the example problems have
purposely been kept simple. The reader should keep in mind that the
subprograms described frequently have features and modes of usage
that are not illustrated in the demonstration program. Thus the reader is
encouraged to read the entire subprogram description and not judge
the range of applicability of a subprogram on the basis of the
demonstration programs alone.

This section may also contain remarks that will help one in using the
subprogram.

\subsection{Functional Description}

This section describes what the subprogram does. It also gives
information on the methods used, and, if appropriate and available, gives
timing, accuracy data and references.

\subsection{Error Procedures and Restrictions}

Information on how errors are treated, and any pertinent information
concerning restrictions in the use of the subprogram are given here. Some
subprograms return a flag indicating the success or failure mode.
Some subprograms call the error message processing subroutines described in
Chapters 19.2 and~19.3. These subroutines provide a means for the user to
alter the action of error message processing.

\subsection{Supporting Information}

This section gives names of all program files needed in order to use
the described subprograms. Those responsible for the
development of the subprograms are also identified.

\if j\outtyp
\intro{4}
\subsection{Subprogram Library Task Organization}

The Computational Mathematics Activity responsible for the quality and scope
of the JPL MATH77 has been disbanded.  The library is supported by Math \`a la
Carte at \url{fkrogh@mathalacarte.com}.  The main development of this library
was cone by Fred Krogh, Charles Lawson, and W.\ Van Snyder.
\fi

\bibliography{math77}
\bibliographystyle{math77}

\end{multicols}
\end{document}
