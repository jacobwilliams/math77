\documentclass[twoside]{MATH77}
\usepackage{multicol}
\usepackage[fleqn,reqno,centertags]{amsmath}
\begin{document}
\hyphenation{SINTOP}
\begmath 13.2 Numerical Evaluation of Integrals Over Several Dimensions

\silentfootnote{$^\copyright$1997 Calif. Inst. of Technology, \thisyear \ Math \`a la Carte, Inc.}

\subsection{Purpose}

This collection of subprograms estimates the value of the integral $I_n$,
where%
\begin{gather*}
\{I_0\}_f=\emptyset \vspace{2pt} \\
I_k=\int \rule[-15pt]{0pt}{34pt}_{\displaystyle
\hspace{-9pt}a_k(x_{k+1},...,x_n,
\{I_{k^{\prime}}\}_{a_k})}^%
{\displaystyle \hspace{-2pt}b_k(x_{k+1},...,x_n,\{I_{k^{\prime}}\}_{b_k})}
\hspace{-1.3in}f_k(x_k,...,x_n,\{I_{k-1}\}_{f_k})\,dx_k,\ \
1\leq k\leq n-1\\
I_n=\int_{\displaystyle a_n}^{\displaystyle b_n}f_n(x_n,
\{I_{n-1}\}_{f_n})\,dx_n,
\end{gather*}
$a_n$ and $b_n$ are constants, $a_1$, ..., $a_{n-1}$, $b_1$, ..., $b_{n-1}$
and $f_1$, ..., $f_n$ are functions provided by the user. The notation
$\{I_{k-1}\}_g$ means a set of zero or more integrals of dimension less
than $k$, used in the calculation of $g$, and $k^{\prime }$ is less than
the dimensionality of the integral enclosing $I_k(k^{\prime }$ may be
greater than $k)$. Usually, $f_k$ is $I_{k-1}$ for $k>1$, and
$\{I_{k^{\prime}}\}_{a_k}$ and $\{I_{k^{\prime}}\}_{b_k}$ are empty.

Using $n=2$ for illustration, this formulation of the multiple integration
problem includes the simple case of%
\begin{equation*}
I_2=\int_{\displaystyle a_2}^{\displaystyle b_2}\int_{\displaystyle %
a_1(x_2)}^{\displaystyle b_1(x_2)}f_1(x_1,x_2)\,dx_1\,dx_2,
\end{equation*}
but it also allows for improved computational efficiency if the integrand
can be factored as%
\begin{equation*}
I_2=\int_{\displaystyle a_2}^{\displaystyle b_2}f_2(x_2)\int_{\displaystyle %
a_1(x_2)}^{\displaystyle b_1(x_2)}f_1(x_1,x_2)\,dx_1\,dx_2,
\end{equation*}
and in addition it allows for the more general case of%
\begin{equation*}
I_2=\int_{\displaystyle a_2}^{\displaystyle %
b_2}f_2(x_2,I_1^{(a)}(x_2),I_1^{(b)}(x_2),...)\,dx_1\,dx_2,
\end{equation*}
where $I_1^{(a)}(x_2)$, $I_1^{(b)}(x_2)$, ... are integrals over one
dimension. This includes the case in which, for example, $I_1^{(a)}(x_2)$ is
a limit for $I_1^{(b)}(x_2).$

\subsection{Usage}

Described below under B.1 through B.5 are:

\begin{tabular*}{3.3in}{@{}l@{~~}l}
B.1 & \hspace{-20pt} Program Prototype, Single Precision\dotfill
\pageref{PPSP}\\
\quad B.1.a & The Calling Routine\dotfill \pageref{Calling}\\
\quad B.1.b & Argument Definitions\dotfill \pageref{ArgDef}\\
\quad B.1.c & The User-supplied Subroutine SINTF to\rule{.34in}{0pt}\\
 & Calculate Limits and Integrands\dotfill \pageref{SINTF}\\
\quad B.1.d & Argument Definitions for SINTF\dotfill \pageref{ArgSINTF}\\
\quad B.1.e & Actions to be Accomplished by SINTF\dotfill \pageref{ActSINTF}\\
\quad B.1.f & Passing Extra Information into SINTF\dotfill \pageref{ExtSINTF}\\
\end{tabular*}

\begin{tabular*}{3.3in}{@{}l@{~~}l}
B.2 & \hspace{-20pt} Program Prototype, Single Precision,\\
 & \hspace{-20pt} Reverse Communication\dotfill \pageref{PPRC}\\
\quad B.2.a & The Calling Routine\dotfill \pageref{CallingRC}\\
\quad B.2.b & Argument Definitions\dotfill \pageref{ArgDefRC}\\
B.3 & \hspace{-20pt} Methods to Request Unusual Usage
Through\rule{.3in}{0pt}\\
 & \hspace{-20pt} the Arguments IOPT and WORK\dotfill \pageref{UnusualUse}\\
B.4 & \hspace{-20pt} Changing the Selection of Some Options\\
 & \hspace{-20pt} During the Computation\dotfill \pageref{ChangeSel}\\
B.5 & \hspace{-20pt} Modifications for Double Precision
Usage\dotfill \pageref{DPuse}\\
\end{tabular*}

\subsubsection{Program Prototype, Single Precision.\label{PPSP}}

\paragraph{The Calling Routine\label{Calling}}

\begin{description}
\item[INTEGER]  \ {\bf NDIMI, NWORK, IOPT}$(\geq k)$\newline
[$k$ depends on options used ($\geq 2$).]

\item[REAL]  \ {\bf ANSWER, WORK}(NWORK)
\end{description}

Assign values to NDIMI and NWORK.

Assign values to IOPT() and elements of WORK() referenced by options (see
Section B.3). For simplest usage, set

\hspace{.2in}IOPT$(2)=0$

\begin{center}
\fbox{\begin{tabular}{@{\bf }c}
CALL SINTM (NDIMI, ANSWER,\\
WORK, NWORK, IOPT)\\
\end{tabular}}
\end{center}

If the computation is successful (see the description of values of IOPT(1)
below) the result is in ANSWER and the estimate of the error in the result
is in WORK(1).

\paragraph{Argument Definitions\label{ArgDef}}

\begin{description}
\item[NDIMI] \  [in] Number of dimensions of integration, $n$ in Section A.

\item[ANSWER] \  [out] Estimate of the integral.

\item[WORK()] \  [inout] On completion, WORK(1) contains an estimate of the
upper bound of the magnitude of the difference between ANSWER and the true
value of the integral. During the integration, WORK(1:NDIMI)
contains the abscissas $x_1$ through $x_{NDIMI}$, WORK(NDIMI+1:2$\times $NDIMI)
contains the lower limits $a_1$ through $a_{NDIMI}$,
and WORK(2$\times $NDIMI+1:3$\times $NDIMI) contains the
upper limits $b_1$ through $b_{NDIMI}$.
WORK(NDIMI+1:NWORK) may be referenced by the option vector IOPT() as
described below and in Section B.3, and to pass parameters to SINTF as
described in Section B.1.f.

\item[NWORK] \  [in] Specifies the amount of space allocated for the array
WORK(). NWORK must be $\geq 220\times \text{NDIMI} - 217.$

\pagebreak
\centerline{{\bf Table of Options for SINTM}\hspace{.5in}}
\ {\bf Option}\newline
{\bf Number\hspace{.3in}Brief Description}\vspace{-3pt}
\begin{itemize}
\item[0]  No more options.

\item[1]  No effect. Reserved for future use. Do not use option~1.

\item[2]  Select level of diagnostic output.

\item[3]  Specify error tolerances.

\item[4]  Specify an a posteriori estimate of the absolute error in the
calculated value of the integrand.

\item[5]  Specify an a priori estimate of the relative error expected in the
calculated values of integrands.

\item[6]  Use reverse communication.

\item[7]  Specify minimum index of quadrature formula.

\item[8]  No effect. The parameter may provide information to SINTF.

\item[9]  Specify maximum number of integrand evaluations allowed.

\item[10]  Return number of integrand evaluations used.

\item[11]  Specify location of singularity or discontinuity in integrand.

\item[12]  Specify absolute errors in the limits.

\item[13]  Specify location in IOPT of notification of nonstandard dimension
changes.
\end{itemize}\vspace{-3pt}

\rule{2.9 in}{1 pt}\vspace{3pt}

\item[IOPT()] \  [inout] Used to return status information to the user, to
allow the selection of options, and to pass parameters to SINTF as described
in Section B.1.f. IOPT(1) returns a status indicator with the possible
values:

\begin{description}
\item[\rm $-$NDIMI] \  Normal termination with either the absolute or relative
error tolerance criteria satisfied (see option~3 below and in Section
B.3).

\item[\rm $-$NDIMI$-1$] \  Normal termination with neither the absolute nor
relative error tolerance criteria satisfied, but the tolerance relative to
the locally achievable precision is satisfied. This is the normal status
value when no tolerances are specified. (See option~3 below and in
Section B.3).

\item[\rm $-$NDIMI$-2$] \  Normal termination with none of the error tolerance
criteria satisfied. (See option~3 below and in Section B.3).

\item[\rm $-$NDIMI$-3$] \  Error termination, NWORK is too small.

\item[\rm $-$NDIMI$-4$] \  Bad value for an element of IOPT().

\item[\rm $-$NDIMI$-5$] \  Too many function values needed (see option~9 below and
in Section B.3).

\item[\rm $-$NDIMI$-$k$-5$] \  Error termination. The $k^{th}$ integrand
apparently contains a non-integrable singularity. The approximate abscissa
of the singularity in the $k^{th}$ dimension is in ANSWER, and the abscissas
of exterior dimensions are in WORK$(k$+1:NDIMI). WORK(1)
contains a large number.

\item[\rm $-$NDIMI$-$NDIMI$-$k$-5$] \  When applying the usage that allows
changing the dimensionality of the next integral to be computed in a
nonstandard way (Section B.1.e), the specified dimensionality $k$ of an
inner integral is not less than the dimensionality of the outer integral.
\end{description}

The remainder of IOPT() may be used to select options and to pass parameters
to SINTF. If no options are selected, set IOPT$(2)=0.$
\end{description}

See Section B.3 for a detailed description of options and the table
on the left for an overview.

\paragraph{The User-Supplied Subroutine SINTF to Calculate Integrands and
Limits\label{SINTF}}

SINTM requires that the user provide values of the integrand and values of
the lower and upper limits, and allows the user to make functional
transformations of inner integrals before they are used as integrands of
outer integrals. In the case of simple usage these values are provided by a
user-supplied subroutine of the form:

{\tt \begin{tabbing}
SUBROUTINE SINTF(ANSWER, WORK, IFLAG)\\
REAL \ ANSWER, WORK(*)\\
INTEGER \ IFLAG\\
\rm .\ .\ .
\end{tabbing}}

\paragraph{Argument Definitions for SINTF\label{ArgSINTF}}

\begin{description}
\item[ANSWER] \  [inout] Usage depends on IFLAG.

\item[WORK({\rm 1:NDIMI})]  \ [inout] Storage for the currently needed
values of $x_1$,\ ..., $x_{NDIMI}$. When IFLAG $\neq 0$, $x_1$ is not
needed and WORK(1) has a different usage described below.

\item[WORK({\rm NDIMI+1:2$\times $NDIMI})]  \ [inout] Storage for
lower integration limits, $a_1$, ..., $a_{NDIMI}.$

\item[WORK({\rm 2$\times $NDIMI+1:3$\times $NDIMI)}]  \ [inout]
Storage for upper integration limits, $b_1$, ..., $b_{NDIMI}.$

\item[WORK({\rm $220\times \text{NDIMI}-216:\text{NWORK}$})] \ [inout]
May be used to pass extra information into SINTF, See Section~B.1.f.

\item[IFLAG] \ [inout] On input, IFLAG indicates the action to be taken by
SINTF.  The value of IFLAG may be changed by SINTF.  IFLAG may be used to
pass extra information into SINTF, See Section~B.1.f.

\end{description}

Most mathematical software that requires user defined function
information allows the user to pass in the name of a subprogram for
evaluating the function.  The approach used here has the advantage of not
requiring the user to declare his subprogram in an external statement
(It's not unusual for this to be forgotten.), and of giving a meaningful
diagnostic when no such routine is provided.  If one is solving different
problems with different programs, one can use different file names for the
different function subprograms, all of which would use the same entry
name.  The linker must then be told which of these function subprograms
should be used in forming the executable file.  If one wants to solve
several different problems in the same program, one should code the
separate cases in one subprogram and select the one desired by passing a
``case'' variable into the routine.  This can either be done as part of
IFLAG as mentioned above, or can be done through the use of named
common.  One can also use reverse communication and call subprograms with
different names for the different cases.

\paragraph{Actions to be Accomplished by SINTF\label{ActSINTF}}

{\bf IFLAG} = 0\ \ The Inner Integrand\newline
When IFLAG\ $= 0$, SINTF must compute the inner integrand, $f_1$, as a
function of $x_1$, ..., $x_{NDIMI}$, and return the result in ANSWER.

{\bf IFLAG} $< 0$\ \ After Computing $I_{-IFLAG}$\newline
When IFLAG $< 0$, say IFLAG\ $= -i$, the current value of the integral
$I_i$ is provided in ANSWER, and the estimated error $E_i$ in this value
is provided in WORK(1). Frequently, one will have $f_{i+1} = I_i$, and
$a_i$ and $b_i$ will not depend on integrals, in which case
$\varepsilon_{i+1} = E_i = $ WORK(1), and SINTF can simply RETURN, taking
no action.

In more complicated cases, SINTF must compute the integrand $f_{i+1}$ as a
function of $x_{i+1}$, ..., $x_{NDIMI}$, $I_i$, and perhaps other
integrals as described in the next paragraph. The value of $f_{i+1}$ must
be in ANSWER on return, and the error $\varepsilon _{i+1}$ in $f_{i+1}$
must be in WORK(1).

If $f_{i+1}$ depends on integral(s) not yet computed then SINTF must
remember $I_i$ and $E_i$ for subsequent use, and set IFLAG to the number of
dimensions of the next integral to be evaluated; IFLAG must be less than the
number of dimensions of the enclosing integral. If option~13 is selected
then K13 is used for this communication instead of IFLAG. IFLAG is the first
element of the IOPT vector passed to SINTM (see Section B.1.f). It is the
responsibility of SINTF to remember the state of computation of the several
integrals upon which $f_{i+1}$ depends. That is, to know when to change
IFLAG (or K13), when not to change it, and when to evaluate $f_{i+1}.$

To estimate errors and control the selection of evaluation points during
calculation of $I_{i+1}$, SINTM needs an estimate of the error in
$f_{i+1}$. In the simple case when $f_{i+1}=I_i$, and $a_i$ and $b_i$ do
not depend on integrals, one has $\varepsilon _{i+1}=E_i=$ WORK(1) and no
special action is required. In general, however, one must compute%
\begin{equation*}
\text{WORK(1) }=\sum_{\{\,j\mid I_i^{(j)}\in \{I_i\}_f\,\}}|\partial
f_{i+1}/\partial I_i^{(j)}|\ E_i^{(j)}
\end{equation*}
Where $E_i^{(j)}$ is the error in $I_i^{(j)}$. If some arguments of
$f_{i+1}$, say $y_j$, cannot be precisely calculated and represented, then
one must add $\Sigma \ |\partial f_{i+1}/\partial y_j|\ E(y_j)$, where
E($y_j)$ is the error in the calculation of $y_j$, onto the above sum.

{\bf IFLAG} $> 0$\ \ Before Computing $I_{IFLAG}$\newline
When IFLAG $>0$, say IFLAG $=i$, SINTF must compute $a_i$ and $b_i$ as
functions of $x_{i+1}$, ..., $x_{NDIMI}$, storing $a_i$ in WORK(NDIMI +
IFLAG) and $b_i$ in WORK(2$\times $NDIMI + IFLAG). SINTF will be called
only once with IFLAG = NDIMI, since the outer limits $a_{NDIMI}$ and
$b_{NDIMI}$ must be constants, and thus need to be set only once. These
may be stored into WORK() either by the user's main program before the
initial call to SINTM or else by SINTF when it is called with IFLAG =
NDIMI. Similarly, if the limits $a_k$ and $b_k$ of $I_k$ are constant,
they may be stored into WORK() by the user's main program before the
initial call to SINTM, or on any call to SINTF for which IFLAG $\geq k.$

Suppose a limit $a_i$ or $b_i$ depends on variables of integration $\{x_m |\
i < k \leq m \leq $ NDIMI$\}$. Then $a_i$ or $b_i$ may be calculated any
time that $i \leq $ IFLAG\ $< k$, but for maximum efficiency should be
calculated when IFLAG\ $= k-1$. Similarly a subexpression of $f_i$ that
depends on $\{x_m |\ i < k \leq m \leq $ NDIMI$\}$ should be calculated when
IFLAG $= k$, and used when IFLAG $= -i.$

For $1 \leq i <$ NDIMI the integral $I_i$, having the limits $a_i $ and $b_i$,
may subsequently be used as an argument in computing $f_{i+1}$.  Let $q$ be
the maximum of $|\partial f_{i+1}/\partial I_i|$ over all integrals on which
$f_{i+1}$ depends. If $q \neq 1$, then during an entry at which $a_i$ and
$b_i$ are being computed, SINTF must also compute $q$, or an upper bound for
$q$, and store this value in WORK(1).  This value is used internally to decide
how much accuracy is needed for the coming integration.  Note that if
$f_{i+1}$ is of the form $q(x_{i+1},\ \ldots,\ x_{NDIMI}) I_i$, then the $q$
computed here can be saved and reused for computing the values of ANSWER and
WORK(1) when IFLAG = $-(i+1)$.

If it is known that the integrand, $f_i$, has a single singularity
affecting integration from $a_i$ to $b_i$, SINTF should transmit
information on the type and location of this singularity to SINTM
during each entry that computes $a_i$ and $b_i$. To do this, SINTF
should store the location (on the $x_i$ axis) of the singularity into
WORK($|$K11$|$), where K11 is transmitted to SINTM, either by a call
to SINTOP with Option~11, or by storing K11 into IFLAG.  If the
singularity is at one of the limits, then $\text{NDIMI}< |\text{K11}|
\leq 3\times \text{NDIMI}$ is allowed.  Otherwise, one should have
$|\text{K11}| > 220\times \text{NDIMI}-217$. The sign of K11 affects
the internal transformations made to cope with the singularity as
described in Section~B.3 of Chapter~13.1.  If singularities are
present in more than one dimension of the multiple integration, one
must use a different value of $|$K11$|$ for each different
singularity location.

SINTM provides for the case that $f_{i+1}$ might depend on several
integrals, some of which have fewer than $i$ dimensions. When it is
necessary to evaluate an integral of dimension less than $i$, set IFLAG (or
K13 if option~13 has been selected) to the number of dimensions of the
integral to be evaluated, and set the appropriate limits into WORK() as
described above.

To illustrate nonstandard changes in dimensionality, suppose that the
integrand of a three dimensional integral depends on a function of several
one- and two-dimensional integrals, and suppose one chooses to evaluate the
one-dimensional integrals first. When SINTM first requests SINTF to provide
the limits of the inner (two-dimensional) integral (IFLAG $= 2)$, change
IFLAG to~1. Upon completion of each but the last of the one dimensional
integrals (IFLAG $= -1)$, set IFLAG to~1. Upon completion of the last of the
one-dimensional integrals, and upon completion of all but the last
two-dimensional integral, set IFLAG to~2. SINTF must keep track of it's
state using SAVE variables. That is, SINTF is responsible for knowing which
integrand to evaluate when IFLAG $= 0$, which transformation to apply when
IFLAG $< 0$, and which limits to supply when IFLAG $> 0$. For this
example, an error will be signaled with IOPT$(1)=-14=-\text{NDIMI}-
\text{NDIMI}-3-5$ if one sets IFLAG~$\geq $~3.

If either of the limits $a_i$ or $b_i$ are imprecisely known or imprecisely
representable $(e.g.$, they depend on integrals, or there is significant
cancellation in their evaluation), then for maximum reliability the errors
in the limits should be made known to SINTM by invoking SINTOP and selecting
option~12. See Section~B.3.

\paragraph{Passing Extra Information into SINTF\label{ExtSINTF}}

The argument WORK of SINTF is the vector WORK passed from the user's calling
routine to SINTM.  WORK() may be used to provide information for options as
described in Section B.3, and to pass floating point information from the
user's calling routine into SINTF.

The argument IFLAG of SINTF is the first element of the IOPT() vector
passed from the user's calling routine to SINTM. Elements of IOPT()
(after the first) that are not used for options as described in
Section B.3 may be used to pass integer valued information into
SINTF. In addition, the parameter of option~8 described in Section
B.3 may be examined by SINTF. If IFLAG is used in this way, it must
be declared

{\bf INTEGER} \ {\bf IFLAG}(*)

and IFLAG(1) must be examined to determine the action.

\subsubsection{Program Prototype, Single Precision, Reverse Communication.
\label{PPRC}}

\paragraph{The Calling Routine\label{CallingRC}}

\begin{description}
\item[INTEGER] \  {\bf NDIMI, NWORK, IOPT}$(\geq 3)$

\item[REAL] \  {\bf ANSWER, WORK}(NWORK)
\end{description}

Assign values to NDIMI, NWORK and IOPT() and elements of WORK() referenced
by options. Constant limits may be stored in the appropriate positions in
WORK() as described in Section~B.1.b. Option~6 must be selected (see
Section~B.3). For simple usage
\begin{tabbing}
\hspace{.2in}\=IOPT$(2) = 6$\\
\>IOPT$(3) = 0$
\end{tabbing}
\begin{center}
\fbox{\begin{tabular}{@{\bf }c}
CALL SINTM (NDIMI, ANSWER,\\
WORK, NWORK, IOPT) \\
\end{tabular}}
\end{center}
\hspace{.2in}DO
$$
\fbox{{\bf CALL SINTMA (ANSWER, WORK, IOPT)}}
$$
\begin{tabbing}
\hspace{.2in}\=\ \ \ \ \=IF (IOPT(1) .GT.\ 0) THEN\\

\>\>\ \ \ \ \=$\{$Calculate the limits as described in\\
\>\>\>\ \ \ \ \=Section B.1.d when IFLAG $>0.\}$\\

\>\>ELSE IF (IOPT(1) .LT. 0) THEN\\

\>\>\>IF (IOPT(1) + NDIMI .LE. 0) EXIT\\

\>\>\>$\{$Transform the integral as described in\\
\>\>\>\>Section B.1.d when IFLAG $<0.\}$\\

\>\>ELSE\\

\>\>\>ANSWER = value of innermost integrand,\\
\>\>\>\ \ \ \ $f_1$, at (WORK(1), ..., WORK\ (NDIMI)).\\

\>\>END IF\\

\>END DO\\

\>$\{$Integration is complete$.\}$
\end{tabbing}

Multi-dimensional quadrature can use significant amounts of computer time.\
The usage described above can be modified to reduce the execution time
slightly at a cost of the user's code becoming more complicated. To do
this, replace the single line

\hspace{.2in}ANSWER $= ...$\newline
by
\begin{tabbing}
\hspace{.2in}\=IOPT$(1) = 0$\\

\>DO WHILE (IOPT(1) .EQ.\ 0)\\

\>\ \ \ \ \=ANSWER = Value of the innermost\\
\>\>\ \ \ \ \=integrand, $f_1$, at (WORK(1), ...,\\
\>\>\>WORK(NDIMI))
\end{tabbing}
$$
\quad \quad \fbox{{\bf CALL SINTA (ANSWER, WORK, IOPT)}}
$$
\begin{tabbing}
\hspace{.2in}\=END DO\\

\>IF (IOPT(1) .GT.\ 0) THEN\\

\>\ \ \ \ \=$\{$Values of IOPT(1) produced by SINTMA and\\
\>\>\ \ \ \ \=SINTA have different meanings. Calculate\\
\>\>\>IOPT(1) as described in Section B.1.b.$\}$\\

\>\>IOPT$(1) = -$(IOPT(1) + NDIMI)\\

\>\>EXIT\ \  \=(from DO --- END DO in which this\\
\>\>\>\ code is embedded)\\

\>END IF
\end{tabbing}

This modification reduces by one the number of subroutine calls for every
evaluation of the innermost integrand.

\paragraph{Argument Definitions\label{ArgDefRC}}

\begin{description}
\item[NDIMI, WORK(), NWORK, IOPT()]  \ Used as described in Section
B.1.b, except if $-$NDIMI $<$ IOPT(1) $\leq $ NDIMI, IOPT(1) is used as
described for IFLAG in Section B.1.e.

\item[ANSWER]  \ [inout] On completion, ANSWER contains an estimate of the
integral. During integration, ANSWER provides a value of the integrand to
SINTMA.
\end{description}

\subsubsection{Methods to Request Unusual Usage Through the Arguments IOPT
and WORK\label{UnusualUse}}

All options other than option~13 have the same qualitative effect as when
calculating an integral over one dimension, as described in Section B.3
of Chapter~13.1, but some options have separate effects in separate
dimensions.  Direct calls to SINTOP should be made as described in
Section~B.4 of Chapter~13.1. Options different from those in 13.1 are:

\begin{itemize}
\item[2]  (Argument K2) K2 is an NDIMI decimal digit integer, where the
low order digit selects the level of diagnostic output during integration
over $x_1$, the next digit selects the level of diagnostic output during
integration over $x_2$, etc. The meaning of each digit is the same as the
meaning of K2 described in Section~B.3 of Chapter~13.1.

\item[9]  (Argument K9) K9 specifies only the maximum number of evaluations
of the inner integrand $(f_1).$

\item[11] (Argument K11) WORK($|$K11$|$) specifies the location of a
singularity or discontinuity in the outer dimension.  Singularities or
discontinuities in inner dimensions may be specified as described in
Section~B.1.e.  In either case, a different value of K11 must be used for
each different abscissa of a singularity or discontinuity.  However if
several dimensions happen to have singularities or discontinuities at the
same abscissa, the same value of K11 can be used for all of them.  The
sign of K11 affects the internal transformations made to cope with the
singularity as described in Section~B.3 of Chapter~13.1.

\item[12] (Argument K12) The errors in the lower and upper limits are
stored in WORK(K12) and WORK(K12+1) respectively prior to a
call to SINTOP.  The error in the lower limit $a_i$ is defined by $ \Sigma
\ |\partial a_i/\partial x_j|\ \epsilon \ x_j+\Sigma \ |\partial
a_i/\partial y_j|\ E(y_j)+\Sigma \|\partial a_i/\partial I_j|\ E_j$, where
$ x_j$ are variables of integration of outer integrals, $\epsilon $ is the
round-off level for the appropriate precision, $y_j$ are arguments of
$a_i$ that are neither variables of integration of outer integrals nor
integrals, E($y_j)$ is the error in $y_j$, $I_j$ are integrals that are
arguments of $ a_i$, and $E_j$ is the error in $I_j$.  The error in the
integral due to error in the limit is the error in the limit times the
integrand evaluated near the limit.  The error in the upper limit is
calculated similarly.  If this option is not selected, or K12 $=0$, then
the limits will be assumed to be exact.

\item[13]  (Argument K13) In the IOPT vector passed to SINTM, K13 may be
used to notify SINTM of nonstandard changes in the dimensionality of
integration. If option~13 is not selected, IOPT(1) may be used for this
purpose. But notice that reporting singularities might also use IOPT(1).
\end{itemize}

\subsubsection{Changing the Selection of Some Options During the Computation
\label{ChangeSel}}

The method of changing the selection of options 1, 2, 6, 7, 9, 12 and~13
during the integration is described in Chapter~13.1, Section~B.4.

\subsubsection{Modifications for Double Precision Usage\label{DPuse}}

For double precision usage, change all REAL type statements to DOUBLE
PRECISION and change the prefix of all subroutine names from SINT to DINT.

\subsection{Examples and Remarks}

See DRSINTMF and ODSINTMF, or DRSINTMR and ODSINTMR for an example of the
use of SINTM to compute%
\begin{equation*}
\int_0^\pi \int_0^y\frac{x\cos y}{x^2+y^2}\,dx\,dy=0.
\end{equation*}
The difference between DRSINTMF and DRSINTMR is that DRSINTMF uses forward
communication, while DRSINTMR uses reverse communication.

DRSINTMF and DRSINTMR demonstrate the use of functional transformation of
the inner integrand to reduce the cost of calculating the integral. The
factor $\cos y$ in the integrand does not depend on the inner variable of
integration. In this context, $f_1 = x\ / (x^2 + y^2)$, and $f_2 = I_1 \cos
\ y$. Special care is taken when $y$ is near zero, as the denominator of
$f_1 $ might underflow.

The above integral is really the product of two one-dimensional integrals
(let $z = x/y)$. This situation is not uncommon. Since the cost of
estimating multi-dimensional integrals by nested estimation over one
dimension is exponential in the number of dimensions, the possibility of
this situation should always be considered.

\subsection{Functional Description}

The integral over several dimensions is computed by repeated integration
over one dimension, as shown in Section~A. The integral over one dimension
is estimated using the subprograms described in Chapter~13.1. See Section~D
of Chapter~13.1 for a functional description.

Extensive test results are given in \cite{Krogh:1978:PTR}.  Since there
are no other multi-dimensional quadrature subprograms extant that provide
the functionality of SINTM and DINTM, it is not practical to carry out
extensive comparative tests.  Since SINT1 and DINT1 are, however, more
reliable and require fewer function values than other one-dimensional
quadrature routines, it is to be expected that SINTM and DINTM are more
reliable and require fewer function values than other subprograms that
evaluate multi-dimensional integrals by repeated integration over one
dimension.

\bibliography{math77}
\bibliographystyle{math77}

\subsection{Error Procedures and Restrictions}

Error messages are printed using the extended error message processor
described in Chapter~19.3. If an error does not result in a ``stop,'' error
signals are returned to the user by values of the status flag in IOPT(1).\
Printing, when enabled by Option~2, is executed in subroutines SINTO for
single precision and DINTO for double precision. Both of these programs also
use the message processor described in Chapter~19.3. One can change the
action on errors and parameters affecting the output of messages, by calling
the message/error routine MESS before calling this routine.

\subsection{Supporting Information}

The source language for these subroutines is ANSI Fortran 77.

Common blocks referenced: SINTC and SINTEC in the single precision
versions, and DINTC and DINTEC in the double precision versions.
SINTC and DINTC are written using several COMMON statements, for
maintenance purposes. This usage conforms to the ANSI Fortran-77
standard, but at least one compiler interprets it improperly. If you
experience inscrutable errors, try re-writing the COMMON statements
for SINTC (or DINTC) into a single statement.

\begin{tabular}{@{\bf}l@{\hspace{5pt}}l}
\bf Entry & \hspace{.35in} {\bf Required Files}\vspace{2pt} \\
DINTM & \parbox[t]{2.7in}{\hyphenpenalty10000 \raggedright
AMACH, DCOPY, DINTA, DINTDL, DINTDU, DINTF, DINTM, DINTMA, DINTNS,
DINTO, DINTOP, DINTSM, DMESS, MESS\rule[-5pt]{0pt}{8pt}}\\
DINTMA & \parbox[t]{2.7in}{\hyphenpenalty10000 \raggedright
AMACH, DCOPY, DINTA, DINTDL, DINTDU, DINTF, DINTMA, DINTNS, DINTO,
DINTSM, DMESS, MESS\rule[-5pt]{0pt}{8pt}}\\
SINTM & \parbox[t]{2.7in}{\hyphenpenalty10000 \raggedright
AMACH, MESS, SCOPY, SINTA, SINTDL, SINTDU, SINTF, SINTM, SINTMA,
SINTNS, SINTO, SINTOP, SINTSM, SMESS\rule[-5pt]{0pt}{8pt}}\\
SINTMA & \parbox[t]{2.7in}{\hyphenpenalty10000 \raggedright
AMACH, MESS, SCOPY, SINTA, SINTDL, SINTDU, SINTF, SINTMA, SINTNS,
SINTO, SINTSM, SMESS}\\
\end{tabular}

Designed by Fred T.\ Krogh and W.\ Van Snyder, JPL, 1977. Programmed by W.\
Van Snyder, 1977. Revised by W. Van Snyder, 1986, 1988.

Error/Message handling revised by F. T. Krogh, March~1992.


Two demonstration drivers are shown below, with the output they produced
when run on an IBM PC/AT with an 80287 floating point coprocessor. The first
demonstrates forward communication usage; the second demonstrates reverse
communication usage.

\begcodenp
\lstset{language=[77]Fortran,showstringspaces=false}
\lstset{xleftmargin=.8in}

\centerline{\bf \large DRSINTMF}\vspace{10pt}
\lstinputlisting{\codeloc{sintmf}}

\vspace{20pt}\centerline{\bf \large ODSINTMF}\vspace{-5pt}
\lstset{language={}}
\lstinputlisting{\outputloc{sintmf}}
\bigskip\

\centerline{\bf \large DRSINTMR}\vspace{10pt}
\lstinputlisting{\codeloc{sintmr}}

\vspace{30pt}\centerline{\bf \large ODSINTMR}\vspace{-5pt}
\lstset{language={}}
\lstinputlisting{\outputloc{sintmr}}
\end{document}
