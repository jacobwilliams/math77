\documentclass[twoside]{MATH77}
\usepackage{multicol}
\usepackage[fleqn,reqno,centertags]{amsmath}
\usepackage{url}
\begin{document}

\begmath 10.0 Overview of Fourier Transforms and Spectral Analysis

\silentfootnote{$^\copyright$1997 Calif. Inst. of Technology, \thisyear \ Math \`a la Carte, Inc.}

\subsection{Introduction}

The subroutines in this chapter compute discrete Fourier transforms,
using the fast Fourier transform (FFT). Discrete Fourier transforms
can be used in turn to approximate Fourier coefficients or evaluate
truncated Fourier series, to approximate power spectra, to compute
convolutions and lagged products (as might be required in computing
correlations and in filtering), and in several other applications
where the speed of the FFT has been found of value, see, $e.g.$,
\cite{Cooley:1966:AFF}.

The purpose of this introductory section is to outline what is
available, to help the reader in the selection of the appropriate
subroutine, to outline how the discrete transform can be used for the
computation of convolutions and lagged products, to examine the
errors introduced in using the discrete transform as an approximation
to other transforms, and to suggest a computational procedure to
those with doubts on how to proceed.

Lower case English letters are used for functions of $t$, the
independent variable, and Greek letters for Fourier transform
functions, that are functions of $\omega $, with units of
radians/(units of $t)$.

\subsection{Subroutines Available}

The one dimensional discrete transform pairs available are indicated
below.  In ``SRFT1/DRFT1'' (for example) SRFT1 is the name for the
single precision version and DRFT1 is the name for the double
precision version. In all cases N $=2^M$ where M is a nonnegative
integer and W $=e^{2\pi i/N}=\cos 2\pi /{N} +i\ \sin 2\pi /N.$

SRFT1/{DRFT1} One dimensional real transform\vspace{-6pt}
\begin{equation}\label{O1}
x_j=\sum_{k=0}^{N-1}\xi _kW^{jk},\quad j=0,1,...,N-1\vspace{-6pt}
\end{equation}
\vspace{-15pt}
\begin{equation}\label{O2}
\xi _k=\frac 1N\sum_{j=0}^{N-1}x_jW^{-jk},\quad k=0,1...,N-1
\end{equation}
STCST/DTCST Trigonometric transform\vspace{-6pt}%
\begin{multline}\label{O3}
y_j=\frac 12\alpha _0+\,\sum_{k=1}^{(N/2)-1}\left[
\alpha _k\cos \frac{2\pi jk}N+\beta _k\sin \frac{2\pi jk}N\right]\\
+\frac 12\alpha _{N/2}(-1)^j,\quad j=0,1,...,N-1
\end{multline}
\vspace{-20pt}
\begin{equation}
\begin{split}\label{O4}
\alpha _k&=\frac 2N\sum_{j=0}^{N-1}y_j\cos \frac{2\pi jk}N,\quad
k=0,1,...,\frac N2\\
\beta _k&=\frac 2N\sum_{j=1}^{N-1}y_j\sin \frac{2\pi jk}N,\quad
k=1,2,...,\frac N2-1
\end{split}
\end{equation}

STCST/DTCST Cosine transform%
\begin{multline}\label{O5}
y_j=\frac 12\alpha _0+\sum_{k=1}^{N-1}\alpha _k\cos
\frac{\pi jk}N+\frac 12\alpha _N(-1)^j,\\
j=0,1,...,N
\end{multline}
\vspace{-15pt}
\begin{multline}\label{O6}
\alpha _k=\frac 2N\left[ \frac 12y_0+\sum_{j=1}^{N-1}y_j\cos
\frac{\pi jk} N+\frac 12y_N(-1)^k\right]\\
k=0,1,...,N
\end{multline}
\vspace{-15pt}

STCST/DTCST Sine transform
\begin{equation}\label{O7}
y_j=\sum_{k=1}^{N-1}\beta _k\sin \frac{\pi jk}N,\quad
j=1,2,...,N-1
\end{equation}
\vspace{-10pt}
\begin{equation}\label{O8}
\beta _k=\frac 2N\sum_{j=1}^{N-1}y_j\sin \frac{\pi jk}N,\quad
k=1,2,...,N-1
\end{equation}

SCFT/DCFT Complex transform
\begin{equation}\label{O9}
z_j=\sum_{k=0}^{N-1}\zeta _kW^{jk},\quad j=0,1,...,N-1
\end{equation}
\vspace{-10pt}
\begin{equation}\label{O10}
\zeta _k=\frac 1N \sum_{j=0}^{N-1}z_jW^{-jk},\quad k=0,1,...,N-1
\end{equation}

All variables in the above equations are real, except $W$, $\xi $,
$z$, and $ \zeta $ which are complex. For each of the transform pairs
given, either equation can be derived from the other -- there are no
approximations involved.

Taking real and imaginary parts of Eq.\,(2) and making comparisons with Eq.
(4), it is clear that if $x_j=y_j$, then 2 $\Re \xi _k=\alpha _k$ and
2 $\Im \xi _k=-\beta _k$. Thus the trigonometric transform and the real
transform are closely related. Since SRFT1 is slightly more efficient and
shorter than STCST, it is recommended unless one has a distinct preference
for the trigonometric form. If one has data that are even (or odd) then
one can save a factor of two in both storage and computation by using the
cosine (or sine) transform in STCST. One could use SCFT for real data by
setting $\Im z_j=0$, but since this requires twice the storage and twice
the work as SRFT1, SCFT is recommended only for complex data.

Subroutines STCST, SCFT, and SRFT (similar to SRFT1) can be used for
data in more than one dimension. As above, there is a connection
between the real transform and the trigonometric transform, but the
relations connecting the two are not as simple. Indexing of the
coefficients in SRFT is more complicated, and thus we recommend STCST
for multi-dimensional real data unless one prefers the form of the
solution provided by SRFT.

\subsection{Discrete Convolutions and Correlations}

Here we consider computing sums of the form
\begin{equation}\label{O11}
c_n=\sum_{j=0}^{N-1}a_jb_{n\pm j},\quad n=0,1,...,N-1
\end{equation}
Initially it is assumed that $a_j$ and $b_j$ are periodic with period N.
(That is, if $j$ is not in the range $0\leq j\leq {N-1}$, it is replaced by
its value mod N.) Thus $\alpha _k$ and $\beta _k$ can be defined by%
\begin{equation*}
a_j=\sum_{k=0}^{N-1}\alpha _kW^{jk},\quad \text{and}\quad
b_j=\sum_{k=0}^{N-1}\beta _kW^{jk}.
\end{equation*}
Substitution in Eq.\,(11) yields
\begin{equation}\label{O12}
\begin{split}
 c_n&=\sum_{j=0}^{N-1}\left[ \sum_{k^{\prime }=0}^{N-1}\alpha
_{k^{\prime }}W^{jk^{\prime }}\right] \left[ \sum_{k=0}^{N-1}\beta
_kW^{(n\pm j)k}\right] \vspace{4pt} \\ \displaystyle\phantom{c_n}%
&=\sum_{k^{\prime }=0}^{N-1}\sum_{k=0}^{N-1}\alpha _{k^{\prime }}\beta
_kW^{nk}\sum_{j=0}^{N-1}W^{j(k^{\prime }\pm k)}.
\end{split}
\end{equation}
From the easily verified fact that
\begin{equation}\label{O13}
\sum_{j=0}^{N-1}W^{j(k^{\prime }\pm k)}=
\begin{cases}
N & \text{if }k^{\prime }\equiv \mp k\text{ mod }N \\
0 & \text{if }k^{\prime }\not \equiv \mp k\text{ mod }N
\end{cases}
\end{equation}
and since $\alpha _k$ is periodic with period N there follows from Eq.\,(12)
\begin{equation}\label{O14}
\begin{array}{ll}
\displaystyle c_n=N\sum_{k=0}^{N-1}\alpha _{\mp k}\beta _kW^{nk},&%
n=0,1,...,N-1\rule[-20pt]{0pt}{20pt}\\
\displaystyle \phantom{c_n}=N\sum_{k=0}^{N-1}\gamma _kW^{nk},&%
\text{where }\gamma _k=\alpha _{\mp k}\beta _k.
\end{array}
\end{equation}

The $c_n$ can be computed most efficiently by the indirect route of
using the FFT to compute the $\alpha ^{\prime }$s and $\beta ^{\prime
}$s from the $a^{\prime }$s and $b^{\prime }s$, then computing
$\gamma _k$ as defined in Eq.\,(14), and finally using the FFT to
compute the $c^{\prime }$s from the $\gamma ^{\prime }s$. For real
data, SRFT1 is recommended for computing the transforms. Note that
with SRFT1 $\alpha _k$, $\beta _k$, and $\gamma _k$ are only computed
for $k=0,1,...,N/2$; for $k=(N/2)+1,...,N-1$, one must use the fact
that for real data $\alpha _{N-k}=\overline{\alpha }_k\
(\overline{z}=$ conjugate of $z).$

Ordinarily one has $a_j$ and $b_j$ defined for $j=0,1,...,J-1$,
assumed to be~0 for other values of $j$, and desires $c_n$ for
$n=0,1,...,L-1$, where $L\leq J$. The above procedure can be used to
get the first $L$ values of $c_n $ if one sets $N=$ smallest power of
$2>J+L$ and sets $a_j$ and $b_j=0$ for $j=J,J+1,...,N-1.$ (If $J+L$
is equal to or just slightly greater than a power of~2, it pays to
reduce $L$ so that $J+L$ is one less than a power of 2, and to
compute $c_n$ directly from Eq.\,(11) for values of $n=new\ L,...,$
desired $L-1$.) Direct computation of the $c_n$ for $n=0,1,...,L-1$
from Eq.  (11) requires $L(2J-L+1)/2$ multiplies and adds. Using
SRFT1 and the procedure above requires approximately $(9/4)N\ \log
_2N$ multiplies, $(33/8)N(\log _2N+1)$ adds, and a little additional
overhead. If $a_j=b_j$ then the counts using SRFT1 can be multiplied
by $2/3$. The fastest procedure clearly depends on the values of
$J,L$, and $N$. For $J<64$, or for values of $L$ small relative to
$J$, the direct method is fastest, see \cite{Cooley:1966:AFF}.

Note that no assumption need be made about $a_j$ and $b_j$; the only
errors introduced in the calculation of $c_n$ using SRFT1 are
round-off errors.

\subsection{Estimating Power Spectra}

To within a constant factor (different normalizations are used), an
estimate of the power at the $k^{th}$ frequency (defined in E below)
is given by $|\xi _k|^2=(\Re \xi _k)^2+(\Im \xi _k)^2$, where the
$\xi _k$ are obtained from SRFT1, Eq.\,(2) above. This estimate
suffers from the same type of errors discussed below for the case of
computing Fourier integrals.

\subsection{Replacement of Continuous Transforms with Discrete Ones}

To simplify the material that follows we consider only the case of
complex functions. Results for real data follow immediately by
considering real and imaginary parts of the variables and equations
below. Proofs for the transform pairs are given in many books on
Fourier transforms, although different scalings for $\varphi $ and
$\omega $ are used.  We have used definitions that maximize symmetry
while matching up with the form of the discrete transform provided by
the FFT.

For continuous data we have the following transform pairs.

Fourier Integral $\left( \int_{-\infty }^\infty |f(t)|\,dt\text{\quad
exists}\right)$%
\begin{equation}\label{O15}
f(t)=\int_{-\infty }^\infty \varphi (\omega )e^{2\pi i\omega
t}d\omega
\end{equation}
\vspace{-10pt}
\begin{equation}\label{O16}
\varphi (\omega )=\int_{-\infty }^\infty f(t)e^{-2\pi i\omega
t}dt
\end{equation}
Fourier Series $\left( s(t)=s(t+T)\right) $%
\begin{equation}\label{O17}
s(t)=\sum_{k=-\infty }^\infty \sigma _ke^{2\pi ikt/T}
\end{equation}
\vspace{-10pt}
\begin{equation}\label{O18}
\sigma _k=\frac 1T\int_{-T/2}^{T/2}s(t)e^{-2\pi ikt/T}dt
\end{equation}
The discrete transform Eq.\,(10) can be used to approximate Eq.\,(16)
if the infinite limits are replaced by finite limits and $f$ is
sampled at equally spaced points between the two limits. As an
example let the infinite limits be replaced by $-T/2$ and T$/2$, and
let
\begin{equation}\label{O19}
z_j=
\begin{cases}
\displaystyle f(-\frac T2+j\Delta t),&j=1,2,...,N-1\\
\displaystyle \frac 12\left[ f(-\frac T2)+f(\frac T2)\right] ,& j=0,
\end{cases}
\end{equation}
\vspace{-10pt}
\begin{equation}\label{O20}
\Delta t=\frac TN.
\end{equation}
This value of $z_0$ gives significantly better results than simply
setting $z_0=f(-T/2)$. With the assumption that the contribution to
the integral in Eq.\,(16) is negligible for $|t|>T/2$, the trapezoidal
rule gives
\begin{equation}\label{O21}
\begin{split}
\varphi (\omega )&\approx \Delta t\sum_{j=0}^{N-1}z_je^{-\pi
i\omega (-T+2j\Delta t)} \\
&\approx \frac TNe^{\pi i\omega T}\sum_{j=0}^{N-1}z_je^{-2\pi ij\omega T/N}.
\end{split}
\end{equation}
In order that Eq.\,(21) have the form of Eq.\,(10), the solution is
obtained for $\omega =\omega _k$, where
\begin{equation}\label{O22}
\omega _k=\frac kT,\quad k=0,1,...N-1.
\end{equation}
Then
\begin{equation}\label{O23}
\varphi (\omega _k)\approx \frac TNe^{\pi
ik}\sum_{j=0}^{N-1}z_jW^{-jk}\quad (W=e^{2\pi i/N}).\hspace{-10pt}
\end{equation}
Thus $\varphi (\omega _k)\approx Te^{\pi ik}\zeta _k$, $\zeta _k$
defined as in (10). The factor $e^{\pi ik}(=(-1)^k)$ is due to
shifting the lower limit of $-T/2$ on the integral to a lower limit
of~0 on the summation.

Note that the $k^{th}$ frequency, $2 \pi \omega _k$, depends only on
T, the length of the interval over which $f$ is sampled. Thus, from
Eq.\,(22), $2 \pi \omega _k \text{ radians/(units of }t) =k/T$
cycles/(units of $t)$. From Eqs.\,(20) and Eq.\,(21) it follows that
the largest frequency for which a result is obtained is $\omega
_{N-1}=(N-1)/(N\Delta T)$ cycles.  For real data, our approximation
is such that $\varphi (\omega _{N-k})$ is the conjugate of $\varphi
(\omega _{-k})$, and thus in this case the largest effective
frequency is \begin{equation}\label{O24} \omega _{N/2}=\frac
1{2\Delta t}\ \text{cycles}/\text{(units of } t\text{).}
\end{equation} This frequency is commonly called the {\em Nyquist
frequency.} Note that this frequency depends only on the sampling
interval $\Delta t.$

All that has been said above for Eq.\,(16) applies almost word for
word to Eq.\,(18), except that Eq.\,(18) does not require replacing
infinite limits by finite ones. Because of the factor $1/T$ appearing
in Eq.\,(18), in place of $\varphi (\omega _k)\approx Te^{\pi
ik}\zeta _k$, we have $\sigma _k\approx e^{\pi ik}\zeta _k.$

\subsection{Errors Introduced by Using the Discrete Transform}

The discrete Fourier transform when used as above is a crude
approximation to a continuous transform. Its primary virtue is the
speed with which it can be computed using the FFT ({\bf F}ast {\bf
F}ourier {\bf T}ransform). Many procedures have been suggested for
computing continuous transforms and power spectra, some of which
permit the use of the FFT and some which do not.  References in
Section H below give a sampling of what has been suggested.

Any computational procedure involves making assumptions about either
$f(t)$ or $\varphi (\omega )$ (or both), and any assumption about one
implies something about the other. Answers to questions such as the
following help one in understanding the implications of a
computational procedure.

(Q1) How are the true and computed transform related?

(Q2) What is the error in the computed transform?

(Q3) What assumptions (if any) are made or implied concerning the transform?

(Q4) What assumptions are made about the function at points where its value
is not used?

(Q5) At the points where the value of the function is used, how is the
function related to a function that would give the true transform at
selected values of $\omega ?$

These questions are considered below for the case of the discrete
transform. We begin by giving some results required in the analysis,
then consider the effect of limiting the sampling of $f$ to a finite
range, the effect of discrete sampling, a combination of the two, and
finally the effect on the discrete transform of filling in with zeros.

\subsubsection{Convolution Theorems}

For Fourier integrals we have
\begin{equation}\label{O25}
\varphi (\omega )=\varphi _1(\omega )\varphi _2(\omega )\qquad
\text{if and only if}
\end{equation}
\vspace{-10pt}
\begin{equation}\label{O26}
f(t)=\int_{-\infty }^\infty f_1(\tau )f_2(t-\tau )\,d\tau .
\end{equation}
This result can be derived from Eqs.\,(15) and (16), and is also true
if $\varphi $ and $f$ are interchanged throughout Eqs.\,(25) and (26).

In the case of Fourier series there are two convolution theorems. We
work with the three transform pairs $(a,\alpha )$, $(b,\beta )$, and
$(c,\gamma )$. If $c(t)=a(t)b(t)$, then
\begin{equation}\label{O27}
\gamma _n=\frac 1T\int_{-T/2}^{T/2}\sum_{k=-\infty }^\infty
\sum_{k^{\prime }=-\infty }^\infty \alpha _k\beta _{k^{\prime }}e^{2\pi
it(k+k^{\prime }-n)/T}dt.
\end{equation}
Interchanging integration and summation, the integral is~0 if $k^{\prime
}\neq n-k$, and is T if $k^{\prime }=n-k$. Thus
\begin{equation}\label{O28}
\gamma _n=\sum_{k=-\infty }^\infty \alpha _k\beta _{n-k}.
\end{equation}
The second convolution theorem starts with
\begin{multline}\label{O29}
c(t)=\frac 1T\int_{-T/2}^{T/2}a(\frac{1+\tau }2)b(\frac{1-\tau
}2)\,d\tau \vspace{4pt} \\
\hspace{-15pt}=\frac 1T\int_{-T/2}^{T/2}
\sum_{k=-\infty }^\infty \sum_{k^{\prime }=-\infty
}^\infty \!\!\!\alpha _k\beta _{k^{\prime }}e^{2\pi i[t(k+k^{\prime })+\tau
(k-k^{\prime })]/2T}d\tau .
\end{multline}
Moving the integral inside the summations, we get a nonzero result only
for $k^{\prime }=k$ in which case the integral is $\exp (2\pi itk/T)$.
Thus%
\begin{equation*}
c(t)=\sum_{k=-\infty }^\infty \alpha _k\beta _ke^{2\pi itk/T},\quad
\text{and so}
\end{equation*}
\begin{equation}\label{O30}
\gamma _k=\alpha _k\beta _k.
\end{equation}
The convolution theorem for the discrete Fourier transform has already been
given in Eqs.\,(11) and (14).

\subsubsection{Fourier Transform of a Step Function}

For Fourier integrals, let
\begin{equation}\label{O31}
r_T(t)=
\begin{cases}
1 & -\frac T2<t<\frac T2\\
0 & |t|>\frac T2
\end{cases}
\end{equation}
\begin{equation}\label{O32}
\begin{split}
\rho _T(\omega )&=\int_{-\infty }^\infty r_T(t)e^{-2\pi i\omega
t}dt\vspace{4pt} \\
\phantom{\rho _T(\omega )}&=\int_{-T/2}^{T/2}e^{-2\pi i\omega t}dt=
\frac{\sin \pi \omega T}{\pi \omega }.
\end{split}
\end{equation}
Similarly, one has the transform pair $(\frac 1{\pi t}\sin \pi t\Omega $, $r_\Omega
(\omega )).$

For Fourier series, we are interested in the case
\begin{equation}\label{O33}
\hat \rho _k^{(K)}=
\begin{cases}
1 & -K\leq k<K \\
0 & k\geq K\text{ or }k<-K
\end{cases}
\end{equation}
\vspace{-10pt}
\begin{equation}\label{O34}
\begin{split}
\displaystyle\hat r^{(K)}&=\sum_{k=-\infty }^\infty \hat \rho _k^{(K)}e^{2\pi
itk/T}=\sum_{k=-K}^{K-1}e^{2\pi itk/T}\vspace{4pt} \\
\displaystyle &=\dfrac{2i(\sin 2\pi Kt/T)}{e^{2\pi it/T}-1}\vspace{4pt}\\
\displaystyle &=\dfrac{(1+e^{-2\pi it/T})(\sin 2\pi Kt/T)}{\sin 2\pi t/T}.
\end{split}
\end{equation}
For the discrete Fourier transform, consider
\begin{equation}\label{O35}
\tilde r_{j+kN}^{(J)}=
\begin{cases}
1 & 0\leq j\leq J-1 \\
0 & J\leq j\leq N-1
\end{cases}
k=0,\pm 1,\pm 2,...
\end{equation}
\begin{equation*}
\tilde \rho _k^{(J)}=\frac 1N\sum_{j=0}^{N-1}\tilde r_j^{(J)}W^{-jk}=\frac
1N\sum_{j=0}^{J-1}W^{-jk}
\end{equation*}
\begin{equation}\label{O36}
\tilde \rho _k^{(J)}=
\begin{cases}
J/N & k=0,\pm N,\pm 2N,...\\
\dfrac{1-W^{-Jk}}{N(1-W^{-k})} & \text{otherwise.}
\end{cases}
\end{equation}

\subsubsection{Errors Due to a Finite Range}

Consider approximating $\varphi (\omega )$ in Eq.\,(16) with
\begin{equation}\label{O37}
\hat \varphi (\omega )=\int_{-T/2}^{T/2}f(t)e^{-2\pi i\omega t}dt.
\end{equation}
This is equivalent to answering (Q4) with $f(t)=0$ for $|t|>T/2.$

To answer (Q1), rewrite the above equation as follows
\begin{equation}\label{O38}
\hat \varphi (\omega )=\int_{-\infty }^\infty r_T(t)f(t)e^{-2\pi
i\omega t}dt,
\end{equation}
and using Eqs.\,(25) and (26) (with $f$ and $\varphi $ interchanged)
\begin{equation}\label{O39}
\hat \varphi (\omega )=\int_{-\infty }^\infty \varphi (\omega )
\frac{\sin \pi T(\omega -w)}{\pi (\omega -w)}\,dw.
\end{equation}
Thus the effect of a finite range is a loss of resolution due to the
smearing as indicated in Eq.\,(39).

One answer to (Q2) is to subtract both sides of Eq.\,(39) from $\varphi
(\omega )$. A more useful result is obtained by subtracting Eq.\,(37)
from Eq.\,(16).
\begin{equation}\label{O40}
\varphi (\omega )-\hat \varphi (\omega )=\int_{T/2}^\infty \left[
f(t)e^{-2\pi i\omega t}+f(-t)e^{2\pi i\omega t}\right] \,dt.
\end{equation}
Thus a bound on the error is given by
\begin{equation}\label{O41}
|\varphi (\omega )-\hat \varphi (\omega )|\leq \int_{T/2}^\infty
\left[ |f(t)|+|f(-t)|\right] \,dt.
\end{equation}
If $\int_{-\infty }^\infty |f^{(j)}(t)|\,dt$ exists for $j=0,1,...,J+1$,
integration by parts of the two terms in the integral of Eq.\,(40) yields $(k$ an
integer)
\begin{multline}\label{O42}
\varphi (k/T)-\hat \varphi (k/T)=\hfill\vspace{4pt} \\
\hspace{-10pt}(-1)^k\sum_{j=0}^J\left[ \frac{-iT}{2\pi k}\right]
^{(j+1)}\left[ f^{(j)}(T/2)-f^{(j)}(-T/2)\right] +R,
\end{multline}
where R is a remainder term that goes to~0 as $k\rightarrow \infty $.
Except for the form of the remainder, the same result is obtained in the
same way from the negative of the right side of Eq.\,(37), where $\hat
\varphi $ is defined. Thus for large $\omega $, $\hat \varphi (\omega )$ is
mainly error unless derivatives of $f$ at $-T/2$ are very nearly equal to
those at T$/2$. If the $j^{th}$ derivatives of $f$ at the endpoints are not
equal, then $\hat \varphi (\omega )$ decreases no faster than $\omega
^{-(j+1)}$ for large $\omega $. We find below that nearly equal derivatives
at the ends of the interval are also important when estimating $\hat
\varphi $ using the FFT.

To answer (Q5) consider%
\begin{equation*}
\begin{split}
\varphi (k/T)&=\int_{-\infty }^\infty f(t)e^{-2\pi ikt/T}dt\\
&=\sum_{j=-\infty}^\infty \int_{-T/2+jT}^{T/2+jT}f(t)e^{-2\pi ikt/T}dt\\
&=\int_{-T/2}^{T/2}\sum_{j=-\infty }^\infty f(t+jT)e^{-2\pi ik(t+jT)/T}dt.
\end{split}
\end{equation*}
And since $e^{-2\pi ijk}=1$, we have
\begin{equation}\label{O43}
\begin{split}
\varphi (k/T)&=\int_{-T/2}^{T/2}\hat f(t)e^{-2\pi ikt/T}dt, \quad
\text{where}\vspace{4pt} \\
\hat f(t)&=\sum_{j=-\infty }^\infty f(t+jT)
\end{split}
\end{equation}
If $f(t)$ is replaced by $\hat f(t)$ as defined above, then $\hat \varphi
(k/T)=\varphi (k/T).$

\subsubsection{Errors Due to Discrete Sampling}

Given $f(j\Delta t)$, $j=0$, $\pm 1$, $\pm 2$, ..., we have (proceeding much
as was done in obtaining Eq.\,(42))%
\begin{equation*}
\begin{split}
f(j\Delta t)&=\int_{-\infty }^\infty \varphi (\omega )e^{2\pi
i\omega j\Delta t}d\omega\\
&=\sum_{k=-\infty }^\infty \int_{(k-1/2)/\Delta
t}^{(k+1/2)/\Delta t}\varphi (\omega )e^{2\pi i\omega j\Delta t}d\omega
\end{split}
\end{equation*}
\begin{equation}\label{O44}
\begin{split}
f(j\Delta t)&=\int_{-1/2\Delta t}^{1/2\Delta t}\tilde \varphi
(\omega )e^{2\pi i\omega j\Delta t}d\omega ,\quad \text{where}\\
\tilde \varphi (\omega )&=\sum_{k=-\infty }^\infty \varphi (\omega
+\frac k{\Delta t}).
\end{split}
\end{equation}
Clearly $\tilde \varphi (\omega )$ is periodic with period $1/\Delta t$,
thus from Eqs.\,(18) and (17)
\begin{equation}\label{O45}
\tilde \varphi (\omega )=\Delta t\sum_{j=-\infty }^\infty f(j\Delta
t)e^{-2\pi i\omega j\Delta t}.
\end{equation}
If no assumptions are made about $f$ or $\varphi $, then given just
$f(j\Delta t)$, $j=0$, $\pm 1$, ... values of $\varphi $ for
frequencies that differ by a multiple of $1/\Delta t$ cycles are
irrevocably mixed. This phenomenon is commonly called {\em aliasing.}

When sampling at discrete points the usual assumption made is that $f$ is
band-limited. That is, $\varphi (\omega )=0$ for $|\omega |>1/(2\Delta t)$.
Thus the computed transform is equal to $\hat \varphi $ as given in Eq.
(45), and questions (Q1) and (Q2) are answered by the expression for $\hat
\varphi$ as given in Eq.\,(44).

To answer (Q4) and (Q5), note that if $\tilde f$ is a function whose true
transform is 0 for $|\omega |>1/(2\Delta t)$ and is $\varphi (\omega )$
otherwise, then
\begin{equation}\label{O46}
\tilde f(t)=\int_{-\infty }^\infty r_{1/\Delta t}(\omega )\varphi
(\omega )e^{2\pi i\omega t}d\omega ,
\end{equation}
and using Eqs.\,(25), (26), (31), and (32)
\begin{equation}\label{O47}
\tilde f(t)=\int_{-\infty }^\infty f(\tau )\frac{\sin (\pi
(t-\tau )/\Delta t)}{\pi (t-\tau )}\,d\tau .
\end{equation}
Note the symmetries between $f$ and $\varphi $ in Eqs.\,(38) and (46); (39) and
(47); and (43) and (44). Relations symmetric to Eqs.\,(40) -- (42) are easily
obtained, but we do not bother to do so here.

\subsubsection{Errors Due to Discrete Sampling on a Finite Interval}

Here we consider errors that result from evaluating the integral in Eq.\,(18)
using the discrete Fourier transform. The results also apply to the integral
in Eq.\,(37), and thus errors examined here combined with those due to a
finite range give the total error due to replacing a Fourier integral with a
discrete Fourier transform.

The aliasing problem is revealed much as it was obtained in Eq.\,(44). Let
$\Delta t=T/N$, and from Eq.\,(17)%
\begin{equation*}
\begin{split}
s(j\Delta t)&=\!\sum_{k=-\infty }^\infty \!\sigma _ke^{2\pi
ijk/N}=\!\sum_{m=-\infty }^\infty \!\!\sum_{k=mN}^{(m+1)N-1}\!\!\sigma
_kW^{jk}\\
&=\sum_{k=0}^{N-1}\sum_{m=-\infty }^\infty \sigma _{k+mN}W^{jk},
\end{split}
\end{equation*}
\begin{equation}\label{O48}
s(j\Delta t)=\sum_{k=0}^{N-1}\tilde \sigma _kW^{jk},\text{ where }%
\tilde \sigma _k=\sum_{m=-\infty }^\infty \sigma _{k+mN}.
\end{equation}
And, as for Eq.\,(45), there results from Eqs.\,(9) and (10)
\begin{equation}\label{O49}
\tilde \sigma _k=\frac 1N\sum_{j=0}^{N-1}s(j\Delta t)W^{-jk}.
\end{equation}
Since $\tilde \sigma _{k-N}=\tilde \sigma _k$, and $W^{j(k-N)}=W^{jk},$ Eq.
(48) can be rewritten to make more apparent the assumption that $s$ is
band-limited. Subtracting N from all indices $k\geq N/2$, there results
\begin{equation}\label{O50}
s(j\Delta t)=\sum_{k=-N/2}^{(N/2)-1}\tilde \sigma _kW^{jk}.
\end{equation}
As before, if we assume that $s(t)$ is band-limited, questions (Q1) and (Q2)
are answered by the expression for $\tilde \sigma $ given in Eq.\,(48).

To answer (Q4) and (Q5) as in Eq.\,(47) we use Eqs.\,(29), (30), (33), and
(34) to obtain from
\begin{equation}\label{O51}
\tilde s(t)=\sum_{k=-\infty }^\infty \sigma _k\hat \rho
_k^{(N/2)}e^{2\pi ikt/T},
\end{equation}
\vspace{-10pt}
\begin{multline}\label{O52}
\tilde s(t)=\frac 1T\int_{-T/2}^{T/2}\Biggl [s(\frac{t+\tau }%
2)\left( 1+e^{-\pi i(t-\tau )/T}\right) \vspace{4pt} \\
\times \frac{\sin (\pi N(t-\tau )/2T)}{\sin (\pi (t-\tau )/T)}\Biggr ]%
\,d\tau .
\end{multline}
Another answer to (Q2) can be obtained using the Euler-Maclaurin
formula.  Let $S_k(t)=s(t)e^{-2\pi ikt/T}$, define $S_{k,j}^{(\mu )}$
to be the $\mu ^{th}$ derivative of $S_k$ evaluated at $t=-\frac
T2+j\,\Delta t$, and $S_{k,j}=S_{k,j}^{(0)}$. Then the
Euler-Maclaurin formula applied to Eq.\,(18) gives
\begin{equation}\label{O53}
\hspace{-10pt}
\sigma _k=\frac{\Delta t}T\left \{\frac 12\left[
S_{k,N}+S_{k,0})\right] +\sum_{j=1}^{N-1}S_{k,j} + C_k^{(r)} \right \} +
R_k^{(r)},
\end{equation}\vspace{-5pt}
where
\begin{equation*}
C_k^{(r)}=\sum_{\nu =1}^r\frac{B_{2\nu }(\Delta
t)^{2\nu -1}}{2\nu !}\left[ S_{k,N}^{(2\nu -1)}-
S_{k,0}^{(2\nu -1)}\right] ,
\end{equation*}
$B_{2\nu }$ is the $2\nu ^{th}$ Bernoulli number, and $R_k^{(r)}$ is
a remainder term. The terms in Eq.\,(53) that contain S are what one
would use following the procedure given in Section E above. The terms
involving a $(2\nu -1)^{st}$ derivative of S are correction terms,
which if not used, give an indication (which can be misleading) of
the error. To examine the errors as they depend on the derivatives of
$s$, we write
\begin{equation}\label{O54}
\hspace{-5pt}S_k^{(\mu )}(t)=\sum_{j=1}^\mu \binom
\mu js^{(j)}(t)\left[ \frac{-2\pi ik}T\right] ^{\mu -j}e^{-2\pi ikt/T},
\end{equation}
\vspace{-10pt}
\begin{equation}\label{O55}
\hspace{-5pt}S_k^{(\mu )}(\pm T/2)=\sum_{j=1}^\mu \binom
 \mu j\left[ \frac{-2\pi ik}T\right] ^{\mu -j}(-1)^ks^{(j)}(\pm T/2).
\end{equation}
Substitution into $C_k^{(r)}$ and a little algebraic manipulation gives the
following result.
\begin{equation}\label{O56}
\hspace{-10pt}\begin{array}{l}\displaystyle
\frac{\Delta t}TC_k^{(r)}\!=\!\frac{(-1)^k}%
N\sum_{j=0}^{2r-1}b_j^{(r)}\!\left[ \frac{iT}{2\pi k}\right] ^j\!\!\left[
s^{(j)}(\frac T2)-s^{(j)}(-\frac T2)\right],\\
~
\end{array}% Weird stuff with array needed to avoid overlap on label.
\end{equation}
where
\begin{equation}\label{O57}
b_j^{(r)}=\sum_{\nu =\lfloor j/2\rfloor +1}^r\binom{2\nu
-1}j\frac{B_{2\nu }(-2\pi ik/N)^{2\nu -1}}{(2\nu )!},
\end{equation}
and $\lfloor j/2\rfloor $ is the integer part of $j/2$. Note the similarities
between Eqs.\,(42) and (56). The degree of continuity in the periodic
extension of the function sampled on $[-T/2$, T$/2]$ is extremely important
in determining how well the discrete Fourier transform will approximate
either of the other Fourier transforms.

\subsubsection{Filling in with Zeros}

The requirement that N be a power of~2 imposed by the FFT routines in
this chapter may be inconvenient. It is sometimes suggested that if
one has $N^{\prime }$ function values, that the remaining
$N-N^{\prime }$ values be set to~0, where N is the first power of
$2\geq N^{\prime }$. Let $z_j$ denote the true values of the function
for $j=0,1,...,N-1$, and let $\zeta _k $ and $\hat \zeta _k$ denote
the true and computed coefficients for the discrete Fourier transform
of $z$. Then clearly
\begin{equation}\label{O58}
\zeta _k-\hat \zeta _k=\frac 1N\sum_{j=N^{\prime
}}^{N-1}z_jW^{-jk}.
\end{equation}
And, using Eqs.\,(11), (14), (35), and (36)
\begin{equation}\label{O59}
\hat \zeta _k=\sum_{\nu =0}^{N-1}\zeta _\nu \frac{%
1-W^{-N^{\prime }(k-\nu )}}{N(1-W^{-(k-\nu )})}
\end{equation}
where for $\nu =k$ the multiplier of $\zeta _\nu $ is $N^{\prime }/N.$

It is perhaps more instructive to consider the connection between $\hat
\zeta _k$ and $\tilde \zeta _k$, where
\begin{equation}\label{O60}
\tilde \zeta _k=\frac 1{N^{\prime }}\sum_{j=0}^{N^{\prime
}-1}z_je^{-2\pi ijk/N^{\prime }}.
\end{equation}
By extending the definitions of $\hat \zeta _k$, and $\tilde \zeta _k$ to
noninteger $k$ in the obvious way, one obtains
\begin{equation}\label{O61}
\begin{split}
\hat \zeta _{kN/N^{\prime }}&=\frac 1N\sum_{j=0}^{N^{\prime
}-1}z_je^{-2\pi ijk/N^{\prime }}=\frac{N^{\prime }}N\tilde \zeta _k
\text{,\quad and}\\
\tilde \zeta _{kN^{\prime }/N}&=\frac 1{N^{\prime }}
\sum_{j=0}^{N^{\prime }-1}z_je^{-2\pi ijk/N}=\frac N{N^{\prime }}
\hat \zeta _k
\end{split}
\end{equation}
Thus $\frac N{N^{\prime }}\hat \zeta _k$ can be thought of as a way
of computing $\tilde \zeta $ with a smaller than usual $\Delta \omega
$. In particular, if $N=2N^{\prime }$, $\hat \zeta _{2k}=\tilde \zeta
_k/2.$ It is sometimes suggested that $N^{\prime }$ extra zeros be
added even when $N^{\prime }$ is a power of~2. By doing this one
can use the FFT to get the autocorrelation from Eqs.\,(11) and (14),
and the power spectrum, which can be defined as the Fourier transform
of the autocorrelation function, is obtained automatically. An
alternative method for those who like this approach is to compute
$\tilde \zeta _k/2$ $(N^{\prime }$ a power of~2) to get $\hat \zeta
_{2k}$, and obtain $\hat \zeta _{2k+1}$ by setting it equal to the
$k^{th}$discrete Fourier coefficient of $z^{\prime }$ where
$z_j^{\prime }=z_je^{\pi ij/N^{\prime }},\ j=0,1,...,N^{\prime }-1.$

If one zeros $\zeta _k$ for $k=N^{\prime }$, $N^{\prime }+1$, ..., N, then
instead of defining a trigonometric polynomial of degree $N-1$ that passes
through $z_j$, $j=0,1,...,N-1,$ the $\zeta ^{\prime }s$ define a
trigonometric polynomial of degree $N^{\prime }-1$, that fit the $z_j$ in a
least-squares sense. Multiplication by the Lanczos sigma factors, see below,
should usually give better smoothing characteristics for nonperiodic data.

\subsection{Recommendations}

The FFT gives good results if one is sampling a periodic function over one
period. Results are less satisfactory for other cases. What we suggest here
should usually give an improvement over the FFT, however, in many cases one
can undoubtedly do ever better.

It is assumed that trends in the data $(e.g.$, a linear trend) have been
removed, and that appropriate action has been taken for wild points or gaps
in the data. By a trend, we mean a smooth function S, so that the
(estimated) average value of $|S(t)-z(t)|$ is as small as possible for $t$
outside the sampling interval. In some cases one may want to add the Fourier
transform of S to the computed transform of $z$.

For functions that are not periodic it makes little sense to attempt a
representation in terms of a discrete set of frequencies. In Eq.\,(61) the
value of the discrete transform for noninteger values is defined. What we
propose here amounts to computing $\zeta _k$, which is an average of the
values of the discrete transform for nearby noninteger values of $k$.
Starting from Eq.\,(21), consider
\begin{align}
\varphi _a(\omega )=&\frac 1{2a}\int_{\omega -a}^{\omega
+a}\varphi (\alpha )\,d\alpha \vspace{4pt}\notag \\
=&\frac TN\sum_{j=0}^{N-1}z_j\frac
1{2a}\int_{\omega -a}^{\omega +a}e^{\pi i\alpha T(N-2j)/N}d\alpha
\vspace{4pt}\notag \\
=\frac TNe^{\pi i\omega T}&\sum_{j=0}^{N-1}\left[
\frac{\sin [\pi aT(N-2j)/N]}{\pi aT(N-2j)/N}\right]
z_je^{-2\pi ij\omega T/N}.\label{O62}
\end{align}
With $\omega =\omega _k$, Eq.\,(62) is the same as Eq.\,(23) except
for the multiplier of $z_j$. The choice $a=1/T$ is attractive for
several reasons:  it is the smallest value of $a$ that gives a
multiplier of~0 for $z_0$(and for $z_N$ if it were used), and thus
helps to minimize the effect of a discontinuity in the periodic
extension of $z_j$; the value of $\varphi _{1/T}(\omega _k)$ is the
average value of $\varphi $ from $\omega _{k-1}$ to $\omega
_{k+1}$, so not much resolution is lost; finally, if Eq.\,(39) is
integrated from $\omega -a$ to $\omega +a$ as was done in Eq.\,(21),
one finds that the choice $a=1/T$ minimizes the effect of $\varphi
(\hat \omega )$ on $\hat \varphi _k(\omega )$ for values of $\hat
\omega $ remote from $\omega $. Thus, we define
\begin{equation}\label{O63}
\sigma _j=\begin{cases}
\dfrac{\sin \pi j/N}{\pi j/N}\vspace{4pt} & j\neq 0\\
1 & j=0
\end{cases}
\end{equation}
and the multiplier of $z_j$ in Eq.\,(62) is $\sigma _{N-2j}$. And since
$\sigma _{-N+2j}=\sigma _{N-2j}$, $z_{N-j}$ has the same multiplier.

The $\sigma _j$ defined in Eq.\,(63) are the Lanczos sigma factors, see
\cite{Lanczos:1956:AA} for a different motivation in their derivation, and
for instructive examples.  If one wants a greater smoothing, rather than
increasing $a$, we recommend averaging the average.  Thus if the same
procedure is applied to Eq.\,(62) as was applied to Eq.\,(21) one finds
that $\sigma _{N-2j}$ is replaced by $\sigma _{N-2j}^2$.  This process can
of course be repeated, depending on how much resolution one is willing to
give up.  The $\sin \pi j/N$ are readily available from the sine table
required by the FFT subroutines as illustrated in the example for SRFT1.

The choice of $a = 1/T$ is not appropriate if one has filled out with
zeros because $N^{\prime}$ values are given and $N^{\prime}$ is not a
power of~2.  We assume the given values of $z$ are stored in $z_j$,
$j = N^{\prime}/2$, ..., $[(N+N^{\prime})/2]$, and that both ends
have been zero filled. Then one should set $a =N/(N^{\prime}T).$

The $\sigma $ factors are also useful for smoothing. Given real data,
$x$, one can replace $x_j$ by the average value of the trigonometric
polynomial interpolating the data on the interval $x_{j-1}\leq x\leq
x_{j+1}$ using the $\sigma $ factors. Simply compute the $\zeta _k$
using the FFT, multiply $\zeta _k$ by $\sigma _{2k}$, $k=1$, 2,
..., N$/2$, and then compute the inverse transform. This application
is illustrated in the example for SRFT.

When the number of data points is not a power of two, one may have an
interest in using a mixed radix algorithm such as that in
\cite{Singleton:1969:AAC}, or one could use the codes given here together
with the technique described in \cite{Bailey:1991:TFF}.  Before making the
decision to go with something more complicated than padding the data
with~0's, (or when padding the data with~0's), one should understand the
connections described from Eq.\,(60) to just below Eq.\,(61).
\nocite{Hamming:1962:NMS}
\nocite{Jenkins:1968:SAA}
\nocite{Havie:1973:REI}
\nocite{Abramovici:1973:TAC}
\nocite{Lyness:1971:TCF}
\nocite{VandeVooren:1968:OCI}
\nocite{Stetter:1966:NAF}
\nocite{Clendenin:1966:AMN}
\nocite{VanLoan:1992:CFF}

\bibliography{math77}
\bibliographystyle{math77}

For information on the web, see:\newline
{\bf http://theory.lcs.mit.edu/$^\sim$fftw/fft-links.html}

Fred T. Krogh, JPL, November~1974. Minor additions and corrections,
October~1991, October~1993, and April~1998.
\end{multicols}
\end{document}
