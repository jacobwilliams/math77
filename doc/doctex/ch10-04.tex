\documentclass[twoside]{MATH77}
\usepackage{multicol}
\usepackage[fleqn,reqno,centertags]{amsmath}
\begin{document}
\begmath 10.4 Multi-Dimensional Real Fourier Transform

\silentfootnote{$^\copyright$1997 Calif. Inst. of Technology, \thisyear \ Math \`a la Carte, Inc.}

\subsection{Purpose}

This subroutine computes Real Fourier Transforms for real data in up to six
dimensions using the fast Fourier transform. In ND dimensions, the relations
between the values $x$ and the Fourier coefficients $\xi $ have the form%
\begin{multline*}
x(j_1,j_2,...,j_{ND})=\sum_{k_1=0}^{N_1-1}\cdots\sum_{k_{ND}=0}^{N_{ND}-1}%
\xi (k_1,k_2,...,k_{ND})\\
\times W_1^{j_1k_1}\cdots W_{ND}^{j_{ND}k_{ND}},
\end{multline*}
where $N_\ell=2^{\text{M}(\ell)}$, $W_\ell=e^{2\pi i/N_\ell}$, 0 $\leq $ $j_\ell
\leq N_\ell-1$, $x$ is real and $\xi $ is complex.

\subsection{Usage}

\subsubsection{Program Prototype, Single Precision}

\begin{description}
\item[\bf REAL]  \ {\bf A}$(N_1$, $N_2$, ..., $\geq N_{ND})$\quad $%
[N_k=2^{\text{M}(k)}]$

\item[\bf REAL]  \ {\bf S}$(\geq \max (\nu _1$, $\nu _2$, ..., $\nu
_{ND})-1)$\  $[\nu _k=2^{\text{M}(k)}-^2]$

\item[\bf INTEGER]  \ {\bf M}$(\geq $ ND){\bf , ND, MS}

\item[\bf CHARACTER]  \ {\bf MODE}
\end{description}
On the initial call set MS to~0 to indicate the array S() does not yet
contain a sine table. Assign values to A(), MODE, M(), and ND.
$$
\fbox{{\bf CALL SRFT(A, MODE, M, ND, MS, S)}}
$$
On return A() will contain computed results. S() will contain the sine table
used in computing the Fourier transform. MS may have been changed.

\subsubsection{Argument Definitions}

\begin{description}
\item[A()]  [inout] If the argument MODE selects Analysis, A() contains
values $x$ on entry and the Fourier coefficients $\xi $ on exit. If MODE
selects Synthesis, A() contains the Fourier coefficients $\xi $ on entry and
the values $x$ on exit. The Functional Description below describes the way $x
$ and $\xi $ are stored in A().

\item[MODE]  [in] The character variable MODE selects Analysis or Synthesis.

'A' or 'a' selects Analysis, transforming $x$'s to $\xi ^{\prime }s.
$

'S' or 's' selects Synthesis, transforming $\xi $'s to $x^{\prime
}s.$

\item[M()]  [in] Defines $N_k=2^{\text{M}(k)}$, the number of real data points in
the $k^{th}$ dimension. Require $0\leq \text{M}(k)\leq 31$ for all $k$%
, and M($1)=0$ only if M($k)\equiv 0$ for all $k$. No action is taken with
respect to dimensions for which M($k)=0.$

\item[ND]  [in] Number of dimensions. Require 1 $\leq $ ND $\leq $ 6.

\item[MS]  [inout] Gives the state of the sine table in $S()$.  Let
$\text{MS}_{in}\text{ and MS}_{out}$ denote the values of MS on entry
and return respectively. If the sine table has not previously been
computed, set $\text{MS}_{in} = 0$ or $-$1 before the call. Otherwise
the value of $\text{MS}_{out}$ from the previous call using the same
S() array can be used as $\text{MS}_{in}$ for the current call.

Certain error conditions described in Section E cause the subroutine
to set $\text{MS}_{out} = -2$ and return.  Otherwise, with $\max _i
\{\text{M}(i)\} > 0$, the subroutine sets $\text{MS}_{out} = \max
($M(1), M(2), ..., M(ND), $\text{MS}_{in}).$

If $\text{MS}_{out} > \max (2, \text{MS}_{in}),$ the subroutine sets
NT = $2^{\text{MS}_{out}-2}$ and fills S() with NT $-$ 1 sine values.

If $\text{MS}_{in}=-1$, the subroutine returns after the above
actions, not transforming the data in A().  This is intended to allow
the use of the sine table for data alteration before a subsequent Fourier
transform, as discussed in Section G of Chapter~16.0.

\item[S()]  [inout] When the sine table has been computed, S($j)=\sin \pi
j/(2\times {\textstyle NT})$, $j=1$, 2, ..., NT $-$ 1, see MS above.
\end{description}

\subsubsection{Modifications for Double Precision}

Change SRFT to DRFT and the REAL type statements to DOUBLE PRECISION.

\subsection{Examples and Remarks}

A ``smooth'' function that approximates%
\begin{equation*}
A(i,j)=
\begin{cases}
0 & \text{if}\ |9-i|+|9-j|>4 \\
1 & \text{if}\ |9-i|+|9-j|\leq 4
\end{cases}
\ 1\leq i,\ j\leq 16.
\end{equation*}
is desired. The example at the end of the chapter does this by computing
the two dimensional transform of A, applying sigma factors (see Section
G of Chapter~16.0), and then transforming back. Results are printed only
for 1 $\leq $ $i,\ j\leq 9$ since, to within round-off limitations, A($%
9+m,9-n)={\textstyle A}(9-m,9-n)$, $1\leq 9\pm m\leq 16$ and $1\leq $ $9\pm
n\leq 16.$

\subsection{Functional Description}

The multi-dimensional real Fourier transform is done by changing it to a
problem in complex variables, doing a multi-dimensional complex Fourier
transform, and then adjusting the results to obtain the solution for the
original problem.

Taking complex conjugates in Eq.\,(1) and using the fact that $x$ is real, it
can be verified that%
\begin{equation*}
\xi (N_1-k_1,\;N_2-k_2,...,N_d-k_d)=\overline{\xi }(k_1,k_2,...,k_d),
\end{equation*}
where $N_i-k_i$ is interpreted modulo $N_i$. (Thus $N_i-k_i=0$ if $k_i$ is
0.) Using the above, it is possible to pack the nonredundant $\xi ^{\prime
}s $ in the same space as is required for $x$.

Storage of $\xi $ in A() for the case $d=2$ is illustrated in Table~1. The
rows and columns in the table correspond to rows and columns in the array
A(). Only the subscripts $k_1$, $k_2$ of the $\xi $'s are given.
The symbols $J_1$ and $J_2$ are used as abbreviations for $N_1/2$ and $N_2/2$
respectively. The coefficients with subscripts (0,0), $(J_1,0)$, $(0,J_2)$,
and $(J_1,J_2)$ are real and occupy single array elements as shown. For the
other coefficients, the subscript $``k_1,k_2"$ identifies the location of
the real part of $\xi (k_1$, $k_2)$, and an ``I'' immediately below such a
subscript gives the location of the imaginary part of $\xi (k_1$,\ $k_2).$

The first column (with the second subscript ignored) gives the storage
scheme for the case $d=1$. For $d>2$, and with $\nu =$ the smallest value of
$i(>1)$ for which $k_i\neq 0$ and $k_i\neq N_\nu /2$, the storage scheme
generalizes as follows.

If $1 < k_1 < N_1$ (A$(2k_1+1$, $k_2+1$, ..., $k_d+1),$ A$(2k_1+2$,
$k_2+1$, ..., $k_d+1))$ contains $\xi (k_1,k_2,...,k_d)$.  Else
(A(1, $k_2+1$, ..., $k_d+1),$ A(2, $k_2+1$, ..., $k_d+1))$ contains
\begin{alignat*}{2}
&\xi (0,k_2,...,k_d) &\ &1\leq k_\nu <N_\nu /2\\
&\xi (N_1/2,k_2,...,k_d)&\ \qquad &N_\nu /2<k_\nu <N_\nu
\end{alignat*}
and when $k_i$ is either 0 or $N_i/2$ for all i,
(A(1, $k_2+1$, ..., $k_d+1),$ A(2, $k_2+1$, ..., $k_d+1))$ contains $(\xi
(0,k_2,...,k_d),~\xi (N_1/2,k_2,...,k_d)).$ Note that in this last case both
$\xi $'s are real.

More details can be found in \cite{Krogh:1970:RFT}.

\bibliography{math77}
\bibliographystyle{math77}

\subsection{Error Procedures and Restrictions}

Require $0 \leq \text{M}(k)\leq 31$ and $1 \leq \text{ND} \leq $ 6.  MODE
must contain one of the allowed values.  If any of these conditions are
violated the subroutine will issue an error message using the error
processing procedures of Chapter~19.2 with a severity level of~2 to cause
execution to stop.  A return is made with $\text{MS}=-2$ instead of
stopping if the statement ``CALL\ ERMSET($-$1)'' is executed before
calling this subroutine.

If the sine table does not appear to have valid data, an error message is
printed, and the sine table and then the transform are computed.

\subsection{Supporting Information}

The source language is ANSI Fortran~77.

\begin{tabular}{@{\bf}l@{\hspace{5pt}}l}
\bf Entry & \hspace{.35in} {\bf Required Files}\vspace{2pt} \\
DRFT & \parbox[t]{2.7in}{\hyphenpenalty10000 \raggedright
 DFFT, DRFT, ERFIN, ERMSG, IERM1, IERV1\rule[-5pt]{0pt}{8pt}}\\
SRFT & \parbox[t]{2.7in}{\hyphenpenalty10000 \raggedright
 ERFIN, ERMSG, IERM1, IERV1, SFFT, SRFT}\\
\end{tabular}

Subroutine designed and written by: Fred T. Krogh, JPL, October~1969,
revised January~1988.


\end{multicols}

\begin{table} \centering
  \begin{tabular}{cccccccc}
     0,0 & 0,1 & $\cdots$ & 0,$J_2-1$ & 0,$J_2$ & $J_1$,$J_2+1$ & $\cdots$ & $J_1$,$N_2 -1$ \\
     $J_1$, 0 & I & $\cdots$ & I & $J_1$,$J_2$ & I & $\cdots$ & I \\
      1,0 & 1,1 & $\cdots$ & 1,$J_2-1$ & 1,$J_2$ & 1,$J_2+1$ & $\cdots$ & 1,$N_2 -1$ \\
     I & I & $\cdots$ & I & I & I & $\cdots$ & I \\
    $\cdots$ & & & & $\cdots$ & & & $\cdots$ \\
     $J_1 -1$,0 & $J_1-1$,1 & $\cdots$ & $J_1-1$,$J_2-1$ & $J_1-1$, $J_2$ & $J_1-1$,$J_2+1$ & $\cdots$ & $J_1-1$,$N_2 -1$ \\
    I & I & $\cdots$ & I & I & I & $\cdots$ & I \\
  \end{tabular}
  \caption{Storage of $\xi$ in A() for Two Dimensions}
\end{table}

\begcodenm

\lstset{language=[77]Fortran,showstringspaces=false}
\lstset{xleftmargin=.8in}

\centerline{\bf \large DRSRFT}\vspace{10pt}
\lstinputlisting{\codeloc{srft}}

\vspace{30pt}\centerline{\bf \large ODSRFT}\vspace{10pt}
\lstset{language={}}
\lstinputlisting{\outputloc{srft}}
\end{document}
