\documentclass[twoside]{MATH77}
\usepackage[\graphtype]{mfpic}
\usepackage{multicol}
\usepackage[fleqn,reqno,centertags]{amsmath}
\begin{document}
\opengraphsfile{pl02-17}
\begmath 2.17 Fresnel Integrals

\silentfootnote{$^\copyright$1997 Calif. Inst. of Technology, \thisyear \ Math \`a la Carte, Inc.}

\subsection{Purpose}

The subprograms described in this chapter compute the Fresnel Integrals%
\begin{equation*}
\hspace{-13pt}C(x)=\int_0^x\!\!\cos \left( \frac \pi 2t^2\right) dt\quad
\text{and}\quad S(x)=\int_0^x\!\!\sin \left( \frac \pi 2t^2\right) dt,
\end{equation*}
and the associated functions%
\begin{align*}
\hspace{-13pt}f(x)&=\left[ \frac 12-S(x)\right] \cos \left(
\frac \pi 2x^2\right) -\left[\frac 12-C(x)\right] \sin \left(
\frac \pi 2x^2\right)\\
\hspace{-13pt}g(x)&=\left[ \frac 12-C(x)\right] \cos \left(
\frac \pi 2x^2\right) +\left[\frac 12-S(x)\right] \sin \left(
\frac \pi 2x^2\right)
\end{align*}
as defined by Equations 7.3.1,~7.3.2, 7.3.5 and~7.3.6 in \cite{ams55}.

\subsection{Usage}

\subsubsection{Program Prototype, Single Precision}

To compute $C(x)$ use

{\bf REAL \ SFRENC, X, Y}
$$
\fbox{{\bf Y = SFRENC(X)}}
$$
To compute $S(x)$ use

{\bf REAL \ SFRENS, X, Y}
$$
\fbox{{\bf Y = SFRENS(X)}}
$$
To compute $f(x)$ use

{\bf REAL \ SFRENF, X, Y}
$$
\fbox{{\bf Y = SFRENF(X)}}
$$
To compute $g(x)$ use

{\bf REAL \ SFRENG, X, Y}
$$
\fbox{{\bf Y = SFRENG(X)}}
$$

\subsubsection{Argument Definitions}

\begin{description}
\item[X]  \ [in] The value at which the function is to be evaluated.
\end{description}

\subsubsection{Modifications for Double Precision}

Change the REAL type statements to double precision, and change the initial
letter of the subprogram names from S to D. It is important that the
subprogram names be explicitly typed.
\vspace{10pt}

\hspace{5pt}\mbox{\input pl02-17 }

\subsection{Examples and Remarks}

See DRSFRENL and ODSFRENL for an example of the usage of this subprogram.

There are no restrictions on the range of applicability of these functions.
The accuracy of the trigonometric functions decreases, however, for large $%
|x|$. Thus evaluation of $C(x)$, $S(x)$, $f(-|x|)$ or $g(-|x|)$ for large $%
|x|$ will be less accurate than evaluation of $f(|x|)$ and $g(|x|)$ for the
same value of $x$. When formulating an application, one should when possible
use $C(x)$ and $S(x)$ when $|x|\leq 1.6$ (to achieve maximum efficiency). To
achieve maximum accuracy and efficiency use $f(x)$ and $g(x)$ when $x>1.6$,
and avoid using $x<-1.6.$

\subsection{Functional Description}

\begin{table*}
\begin{center}
\begin{tabular}{cl*{6}{r}}
& \multicolumn{1}{c}{\bf Argument} & \multicolumn{1}{c}{\bf Mean} &
\multicolumn{1}{c}{\bf Max} & \multicolumn{1}{c}{\bf Mean} &
\multicolumn{1}{c}{\bf Max} & \multicolumn{1}{c}{\bf Mean} &
\multicolumn{1}{c}{\bf Max}\\
{\bf Function} & \multicolumn{1}{c}{\bf Interval} &
\multicolumn{1}{c}{\bf ULP} & \multicolumn{1}{c}{\bf ULP} &
\multicolumn{1}{c}{\bf REL} & \multicolumn{1}{c}{\bf REL} &
\multicolumn{1}{c}{\bf ABS} & \multicolumn{1}{c}{\bf ABS}\\
$C(x)$ & [0..1.2] & 0.57 $\rho $ & 2.18 $\rho $ & 0.40 $\rho $ &
1.29 $\rho $ & 0.19 $\rho $ & 0.83 $\rho $\\
& (1.2..1.6] & 0.70 $\rho $ & 2.52 $\rho $ & 0.48 $\rho $ &
1.55 $\rho $ & 0.19 $\rho $ & 0.63 $\rho $\\
S($x)$ & [0..1.2] & 0.74 $\rho $ & 2.42 $\rho $ & 0.52 $\rho $ &
1.39 $\rho $ & 0.09 $\rho $ & 0.65 $\rho $\\
& (1.2..1.6] & 0.75 $\rho $ & 2.20 $\rho $ & 0.55 $\rho $ &
1.55 $\rho $ & 0.38 $\rho $ & 1.10 $\rho $\\
$f(x)$ & (1.6..1.9] & 0.51 $\rho $ & 1.50 $\rho $ & 0.36 $\rho $ &
1.10 $\rho $ & 0.06 $\rho $ & 0.19 $\rho $\\
& (1.9..2.4] & 0.30 $\rho $ & 0.96 $\rho $ & 0.26 $\rho $ &
0.77 $\rho $ & 0.04 $\rho $ & 0.12 $\rho $\\
& (2.4..6.0] & 0.43 $\rho $ & 1.15 $\rho $ & 0.29 $\rho $ &
0.80 $\rho $ & 0.02 $\rho $ & 0.07 $\rho $\\
& (6.0..50.0] & 0.45 $\rho $ & 1.05 $\rho $ & 0.31 $\rho $ &
0.71 $\rho $ & 0.00 $\rho $ & 0.03 $\rho $\\
& (50..1000] & 0.39 $\rho $ & 1.02 $\rho $ & 0.27 $\rho $ &
0.72 $\rho $ & 0.00 $\rho $ & 4E$-$3 $\rho $\\
$g(x)$ & (1.6..1.9] & 0.53 $\rho $ & 1.92 $\rho $ & 0.38 $\rho $ &
1.11 $\rho $ & 0.01 $\rho $ & 0.02 $\rho $\\
& (1.9..2.4] & 1.06 $\rho $ & 3.43 $\rho $ & 0.75 $\rho $ &
1.93 $\rho $ & 0.01 $\rho $ & 0.02 $\rho $\\
& (2.4..6.0] & 1.51 $\rho $ & 4.04 $\rho $ & 1.01 $\rho $ &
2.61 $\rho $ & 0.00 $\rho $ & 0.01 $\rho $\\
& (6.0..50.0] & 1.40 $\rho $ & 3.62 $\rho $ & 0.97 $\rho $ &
2.11 $\rho $ & 0.00 $\rho $ & 3E$-$4 $\rho $\\
& (50..1000] & 1.09 $\rho $ & 2.40 $\rho $ & 0.75 $\rho $ &
1.50 $\rho $ & 0.00 $\rho $ & 2E$-$7 $\rho $
\end{tabular}
\end{center}
\end{table*}

The computer approximations for these functions use Chebyshev rational
approximations developed by W. J. Cody, described in \cite{Cody:1968:CAF}. Cody
provides approximations for $C(x)$ and $S(x)$ for $|x| \leq 1.6$, and for $%
f(x)$ and $g(x)$ for $x > 1.6$. The approximations for $f(x)$ and $g(x)$ for
$x > 2.4$ have the same asymptotic form as the functions. We compute $f(x)$
and $g(x)$ from $S(x)$ and $C(x)$ when $|x| \leq 1.6$, and vice versa for $x
> 1.6$. For $x < 0$ we use $C(-x) = -C(x)$, $S(-x) = -S(x)$, $g(-x) = \cos
(\pi /2\ x^2) + \sin (\pi /2\ x^2) - g(x)$ and $f(-x) = \cos (\pi /2\ x^2) -
\sin (\pi /2\ x^2) - f(x)$.

The approximations and programming were checked by comparing the double
precision functions to an extended precision computation of $w(z)$, the
Fadeeva function described in Chapter~2.16. Testing consisted of dividing
several regions of the argument range into~200 equal-sized intervals, and
selecting a point randomly in each interval. To test $f(x)$ and $g(x)$ when $%
50 < x < 1000$ we divided the range $10^{-3} < 1/x < .02$ into~200 equal
subranges. In each interval we report the error in units of the last
position of the test value in the column headed ULP, the error relative to
the true value in the column headed REL, and the absolute error in the
column headed ABS. The quantity $\rho $ is the round off level, that is, the
difference between~1.0 and the next representable number, which is provided
by D1MACH(4) (Chapter~19.1). For IEEE arithmetic, $\rho \approx 2.22$E$-$16 in
double precision. The results are summarized above.

Cody's testing of the approximations, as described in \cite{Cody:1968:CAF},
indicates a relative accuracy in the approximations of~15 to~18 digits, so
one should not expect to achieve more accuracy simply by carrying out the
calculations using more precision, as, for example, by using double
precision on a Cray computer.

The errors in $f(x)$ and $g(x)$ decrease as $x$ increases in the range $6 <
x \leq 1000$. The approximations for $f(x)$ and $g(x)$ have the same
asymptotic form as the functions when $x > 2.4$, and therefore they become
more accurate as $x$ increases. For IEEE format double precision arithmetic,
the approximation for $f(x)$ is identical to the asymptotic expansion when $%
x > 29$, and the approximation for $g(x)$ is identical to the asymptotic
expansion when $x > 14.$

We verified correct programming of $f(x)$ and $g(x)$ for $|x| \leq 1.6$, and
for $C(x)$ and $S(x)$ for $x > 1.6$, by comparing results to values in table
7.7 in \cite{ams55}. Extensive accuracy testing would simply have validated
the trigonometric function routines.

\bibliography{math77}
\bibliographystyle{math77}

\subsection{Error Procedures and Restrictions}

There are no restrictions on the argument range for these functions; they do
not announce any errors.

\subsection{Supporting Information}

The source language is ANSI Fortran~77.

\begin{tabular}{@{\bf}l@{\hspace{5pt}}c}
\bf Entry & \hspace{.27in} {\bf Required Files}\vspace{2pt} \\
DFRENC & \hspace{.2in} AMACH, DFRENL\\
DFRENF & \hspace{.2in} AMACH, DFRENL\\
DFRENG & \hspace{.2in} AMACH, DFRENL\\
DFRENS & \hspace{.2in} AMACH, DFRENL\\
SFRENC & \hspace{.2in} AMACH, SFRENL\\
SFRENF & \hspace{.2in} AMACH, SFRENL\\
SFRENG & \hspace{.2in} AMACH, SFRENL\\
SFRENS & \hspace{.2in} AMACH, SFRENL\\
\end{tabular}

Subprograms designed and developed by W. V. Snyder, JPL, 1992.


\begcodenp

\enlargethispage*{10pt}
\lstset{language=[77]Fortran,showstringspaces=false}
\lstset{xleftmargin=.8in}
\centerline{\bf \large DRSFRENL}\vspace{-5pt}
\lstinputlisting{\codeloc{sfrenl}}

\vspace{0pt}\centerline{\bf \large ODSFRENL}\vspace{2pt}
\lstset{language={}}
\lstinputlisting{\outputloc{sfrenl}}

\closegraphsfile
\end{document}
