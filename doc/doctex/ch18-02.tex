\documentclass[twoside]{MATH77}
\usepackage{multicol}
\usepackage[fleqn,reqno,centertags]{amsmath}
\begin{document}
\begmath 18.2  Sorting Data of Arbitrary Structure in Memory

\silentfootnote{$^\copyright$1997 Calif. Inst. of Technology, \thisyear \ Math \`a la Carte, Inc.}

\subsection{Purpose}

Sort data having an organization or structure not supported by one of the
subprograms in Chapter~18.1, for example, data having more than one key to
determine the sorted order. The subprogram INSORT in Chapter~18.3 has
similar functionality to GSORTP and is more efficient if the data are
initially partly ordered, or the ordering criterion is expensive.

\subsection{Usage}

\subsubsection{Program Prototype}

\begin{description}

\item[INTEGER] \ {\bf N, IP}($\geq |\text{N}|$){\bf , COMPAR}

\item[EXTERNAL] \ {\bf COMPAR}

\end{description}

Assign values to N and data elements indexed by~1 through N. Require
N $\geq $ 1.
$$
\fbox{\bf CALL GSORTP (COMPAR, N, IP)}
$$
Following the call to GSORTP the contents of IP(1) through IP(N) are such
that the $\text{J}^{th}$ element of the sorted sequence is the IP(J$)^{th}$ element
of the original sequence.

\subsubsection{Argument Definitions}

\begin{description}

\item[COMPAR] \ [in] An INTEGER FUNCTION subprogram that defines the relative
order of elements of the data. COMPAR is invoked as COMPAR(I, J), and is
expected to return $-$1 (or any negative integer) if the $\text{I}^{th}$
element of the original data is to precede the $\text{J}^{th}$ element in
the sorted sequence, +1 (or any positive integer)
if the $\text{I}^{th}$ element is to follow the $\text{J}^{th}$ element,
and zero if the order is immaterial. GSORTP does not have access to the
data. It is the caller's responsibility to make the data known to COMPAR.
Since COMPAR is a dummy procedure, it may have any name. Its name must
appear in an EXTERNAL statement in the calling program unit.

\item[N] \ [in] $|$N$|$ is the number of elements to sort, and the upper
 bound of subscripts to use to access IP.  If N $>$ 0 then IP(I) is
 initialized to I, for $1 \leq I \leq N$.  Actual arguments for COMPAR
 are always elements of IP.

\item[IP()] \ [out] An array to contain the definition of the sorted
sequence. IP(1:N) are set so the $\text{J}^{th}$ element of the sorted sequence
is the IP(J$)^{th}$ element of the original sequence.
\end{description}

\subsection{Examples and Remarks}

Program DRGSORTP illustrates the use of GSORTP to sort 1000 randomly
generated real numbers. The output should consist of the single line

\hspace{.2in}GSORTP succeeded

\subparagraph{Stability}

A sorting method is said to be $stable$ if the original relative order of
equal elements is preserved. This subroutine uses the quicksort algorithm,
which is not inherently stable. To impose stability, return COMPAR =
I $-$ J if the $\text{I}^{th}$ and $\text{J}^{th}$ elements are equal.

\subsection{Functional Description}

See Section~18.1.D.

\subsection{Error Procedures and Restrictions}

See Section~18.1.E.

\subsection{Supporting Information}

The source language for these subroutines is ANSI Fortran 77.

\begin{tabular}{@{\bf}l@{\hspace{5pt}}l}
\bf Entry & \hspace{.2in} {\bf Required Files}\vspace{2pt}\\
GSORTP & \hspace{.35in} GSORTP\\\end{tabular}

Designed and coded by W. V. Snyder, JPL~1988.


\begcodenp
\medskip\

\lstset{language=[77]Fortran,showstringspaces=false}
\lstset{xleftmargin=.8in}

\centerline{\bf \large DRGSORTP}\vspace{10pt}
\lstinputlisting{\codeloc{gsortp}}
\end{document}
