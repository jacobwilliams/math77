\documentclass[twoside]{MATH77}
\usepackage{multicol}
\usepackage[fleqn,reqno,centertags]{amsmath}
\begin{document}
\hyphenation{SWPWRI}
\begmath 17.1 Computation Using Derivative Arrays or
\hbox{Univariate Taylor Series}

\silentfootnote{$^\copyright$1997 Calif. Inst. of Technology, \thisyear \ Math \`a la Carte, Inc.}

\subsection{Purpose}

This set of subroutines performs computations using arrays of length N+1,
where each array contains the value of a function, $f(t)$, and its first N
derivatives with respect to $t$, evaluated at some point, $t_0$. Such an
array can alternatively be regarded as a scaled representation of the first
N + 1 coefficients of the Taylor series of $f(t)$ expanded at $t_0$, since
the coefficient of $(t-t_0)^k$ in such a Taylor series is $f^{(k)}(t)/k!$
evaluated at $t_0.$

This package provides a way of computing values of the first N derivatives
of a univariate function that is defined by a sequence of computational
steps involving arithmetic and elementary functions, without the need to
derive and code expressions for the derivatives.

\subsection{Usage}

We shall use $t$ as the generic name of the single independent variable with
respect to which all derivatives are defined. We shall use the term
W-variable to denote an (N+1)-tuple, consisting of a function value and
values of the function's first N derivatives with respect to $t$, evaluated
at some point. Note that the representation of $t$ as a W-variable evaluated
at $t_0$ is the (N+1)-tuple, $(t_0$, 1, 0, 0, ...,~0).

This package consists of one subroutine, SWSET, for assigning a value to a
W-variable, 22~subroutines for doing arithmetic operations and computing
elementary functions using W-variables, and three supplementary subroutines,
SWCHN, SWRCHN, and SPASCL. These will be described in the following sections:

\begin{tabular*}{3.3in}{@{}l@{~~}l}
B.1 & SWSET\dotfill \pageref{B1}\\
B.2 & Arithmetic and elementary functions\dotfill \pageref{B2}\\
B.3 & SWCHN\dotfill \pageref{B3}\\
B.4 & SWRCHN\dotfill \pageref{B4}\\
B.5 & SPASCL\dotfill \pageref{B5}\\
B.6 & Modifications for double-precision usage\dotfill \pageref{B6}\\
 & \rule{3in}{0pt}~
\end{tabular*}\vspace{-5pt}

\subsubsection{SWSET, Assigning a value to a W-variable\label{B1}}

\paragraph{Program Prototype, Single Precision}

\begin{description}
\item[INTEGER]  \ {\bf N}

\item[REAL]  \ {\bf VAL, DERIV, W}($\geq $N+1)
\end{description}

Assign values to N, VAL, and DERIV
$$
\fbox{{\bf CALL SWSET(N, VAL, DERIV, W)}}
$$
Computed quantities are returned in W().

\paragraph{Argument Definitions}

\begin{description}
\item[N]  \ [in] Highest order derivative to be considered. Require 0 $\leq $
N $\leq $ NMAX. See Section E for the definition of NMAX.

\item[VAL]  \ [in] Value to be assigned to the W-variable, W().

\item[DERIV]  \ [in] Value to be assigned as the first derivative of the
W-variable, W().

\item[W()]  \ [out] Array of length at least N + 1 in which this subroutine
will place N + 1 values to represent a W-variable as follows: (VAL, DERIV,
0.0, ..., 0.0). The user should set DERIV = 1.0 to define this variable as
the variable, $t$, with respect to which all differentiation is done. The
user can set DERIV = 0.0 to define this W-variable to be a constant, $i.e$.
a variable not depending on $t.$
\end{description}

\subsubsection{Arithmetic and elementary functions using W-variables\label{B2}}

In describing the following subroutines, X() and Y() denote input
W-variables and Z() denotes an output W-variable. The variables A and I are
input variables that are constant relative to $t.$

In most of these subroutines the output array Z() must occupy storage
locations distinct from any of the input data. Exceptions to this rule are
SWSUM, SWDIF, SWPRO, SWSUM1, SWDIF1, and SWPRO1. The subroutine SWQUO, which
computes $x/y \rightarrow z$, permits $x$ and $z$ to occupy the same
storage, but $y$ and $z$ must occupy distinct storage.

\paragraph{Program Prototype, Single Precision}

\begin{description}
\item[INTEGER]  \ {\bf N, I}

\item[REAL]  \ {\bf A, X}($\geq $N+1){\bf , Y}($\geq $N+1){\bf , Z}($\geq $%
N+1)
\end{description}

Assign values to N, I, A, X(), and Y(), as appropriate.

\begin{center}
{\bf Two-argument operations with both arguments depending on $t.$}

\fbox{\begin{tabular}{@{\bf \ }lr}
CALL SWSUM(N, X, Y, Z) & $x+y \rightarrow z$\\
CALL SWDIF(N, X, Y, Z) & $x-y \rightarrow z$\\
CALL SWPRO(N, X, Y, Z) & $x\times y \rightarrow z$\\
CALL SWQUO(N, X, Y, Z) & $ x/y \rightarrow z$\\
CALL SWATN2(N, X, Y, Z) &  atan2($x,y) \rightarrow z$\\
\multicolumn{2}{c}{where for atan2: \ $-\pi < z \leq \pi $ and
$\tan (z) = x/y$}\\
\end{tabular}}
\end{center}

\pagebreak

\begin{center}
{\bf Two-argument operations with only one argument depending on $t.$}

\fbox{\begin{tabular}{@{\bf \ }lr}
CALL SWSUM1(N, A, Y, Z) & $ a+y \rightarrow z$\\
CALL SWDIF1(N, A, Y, Z) & $ a-y \rightarrow z$\\
CALL SWPRO1(N, A, Y, Z) & $ a\times y \rightarrow z$\\
CALL SWQUO1(N, A, Y, Z) & $ a/y \rightarrow z$\\
CALL SWPWRI(N, I, Y, Z) & $ y^i \rightarrow z$\\
\multicolumn{2}{c}{(See following note.)}\\
\end{tabular}}
\end{center}

Note: I may be positive, negative, or zero. If I = 0, SWPWRI sets Z(1) = 1.0
and all derivative values of $z$ to~0.0, regardless of the given value of $y$%
. It is an error to have $y = 0.0$ when I $< 0.$

\begin{center}
{\bf One-argument operations with the argument depending on $t.$}

\fbox{\begin{tabular}{@{\bf \ }lr}
CALL SWSQRT(N, X, Z) &  $\sqrt x \rightarrow z$\\
CALL SWEXP(N, X, Z) & $\exp (x) \rightarrow z$\\
CALL SWLOG(N, X, Z) & $\log (x) \rightarrow z$\\
CALL SWSIN(N, X, Z) & $\sin (x) \rightarrow z$\\
CALL SWCOS(N, X, Z) & $\cos (x) \rightarrow z$\\
CALL SWTAN(N, X, Z) & $\tan (x) \rightarrow z$\\
CALL SWASIN(N, X, Z) & $\sin^{-1}(x) \rightarrow z$\\
CALL SWACOS(N, X, Z) & $\cos^{-1}(x) \rightarrow z$\\
CALL SWATAN(N, X, Z) & $\tan^{-1}(x) \rightarrow z$\\
CALL SWSINH(N, X, Z) & $\sinh(x) \rightarrow z$\\
CALL SWCOSH(N, X, Z) & $\cosh(x) \rightarrow z$\\
CALL SWTANH(N, X, Z) & $\tanh(x) \rightarrow z$\\
\end{tabular}}
\end{center}

\paragraph{Argument Definitions}

\begin{description}
\item[N]  \ [in] Highest order derivative to be considered. Require 0 $\leq $
N $\leq $ NMAX. See Section E for the definition of NMAX.

\item[A]  \ [in] Floating point value that is independent of $t.$

\item[I]  \ [in] Integer value that is independent of $t.$

\item[X()]  \ [in] An input W-variable.

\item[Y()]  \ [in] An input W-variable.

\item[Z()]  \ [out] An output W-variable.
\end{description}

\subsubsection{SWCHN, application of the chain rule\label{B3}}

This subroutine is called by all of the elementary function subroutines. The
user will not need to call it directly unless the user is developing a new
function subroutine that is not conveniently representable in terms of the
available functions.

\paragraph{Program Prototype, Single Precision}

\begin{description}
\item[INTEGER]  \ {\bf N}

\item[REAL]  \ {\bf X}($\geq $N+1){\bf , F}($\geq $N+1)
\end{description}

Assign values to N, X(), and F().
$$
\fbox{{\bf CALL SWCHN(N, X, F)}}
$$
Computed quantities are returned in F().

\paragraph{Argument Definitions}

\begin{description}
\item[N]  \ [in] Highest order derivative to be considered. Require 0 $\leq $
N $\leq $ NMAX. See Section E for the definition of NMAX.

\item[X()]  \ [in] On entry, X() must contain $x(t_0)$ and derivatives
through order N of $x(t)$ with respect to $t$. X() will be unchanged by this
subroutine.

\item[F()]  \ [inout] On entry, F() must contain $f(x_0)$ and derivatives
through order N of $f(x)$ with respect to $x$, evaluated at $x_0=x(t_0)$. On
return F() will contain $f(x(t_0))$ and derivatives through order N of $%
f(x(t))$ with respect to $t$, evaluated at $t_0$. Note that the contents of
F(1) will remain unchanged.
\end{description}

\subsubsection{SWRCHN, application of the reverse chain rule\label{B4}}

This subroutine reverses the action of SWCHN, in the sense that if one were
to make the two calls

CALL SWCHN(N, X, F)

CALL SWRCHN(N, X, F)

the first would transform the contents of the array F(), and the second
would (if X(2) is nonzero) transform the contents of F() back to the same
values (to within computational errors) as before the first call. See
Section C for application of SWRCHN to series reversion.

\paragraph{Program Prototype, Single Precision}

\begin{description}
\item[INTEGER]  \ {\bf N}

\item[REAL]  \ {\bf X}($\geq $N+1){\bf , F}($\geq $N+1)
\end{description}

Assign values to N, X(), and F().
$$
\fbox{{\bf CALL SWRCHN(N, X, F)}}
$$
Computed quantities are returned in F().

\paragraph{Argument Definitions}

\begin{description}
\item[N]  \ [in] Highest order derivative to be considered. Require 0 $\leq $
N $\leq $ NMAX. See Section E for the definition of NMAX.

\item[X()]  \ [in] On entry, X() must contain $x(t_0)$ and derivatives
through order N of $x(t)$ with respect to $t$. X(2), representing $dx/dt$
evaluated at $t_0$, must be nonzero. The contents of X(1) will not be used
by this subroutine. X() will be unchanged by this subroutine.

\item[F()]  \ [inout] On entry, F() must contain $f(x(t_0))$ and derivatives
through order N of $f(x(t))$ with respect to $t$, evaluated at $t=t_0$. On
return F() will contain $f(x_0)$ and derivatives through order N of $f(x)$
with respect to $x$, evaluated at $x_0=x(t_0)$. Note that the contents of
F(1) will remain unchanged.
\end{description}

\subsubsection{SPASCL, computation of the ``Pascal triangle" of binomial
coefficients\label{B5}}

\paragraph{Program Prototype, Single Precision}

\begin{description}
\item[INTEGER]  \ {\bf N}

\item[REAL]  \ {\bf C}($\geq $ 1 + (N*(N+1))/2)
\end{description}

Assign a value to N.
$$
\fbox{{\bf CALL SPASCL(N, C)}}
$$
Computed quantities are returned in C().

\paragraph{Argument Definitions}

\begin{description}
\item[N]  \ [in] Highest order derivative to be considered. Require 0 $\leq $
N $\leq $ NMAX. See Section E for the definition of NMAX.

\item[C()]  \ [out] On return will contain values constituting the ``Pascal
triangle'' of binomial coefficients. For example, if N = 4, the Pascal
triangle is
\end{description}

\begin{center}
\begin{tabular}{c}
1\\
1\ \ \ 1\\
1\ \ \ 2\ \ \ 1\\
1\ \ \ 3\ \ \ 3\ \ \ 1\\
1\ \ \ 4\ \ \ 6\ \ \ 4\ \ \ 1
\end{tabular}
\end{center}

SPASCL omits the first element of each row except the first, and thus
returns the 11~values: 1, 1, 2, 1, 3, 3, 1, 4, 6, 4,~1.

\subsubsection{Modifications for Double Precision\label{B6}}

For double-precision usage, replace the initial S in the name of each
subroutine with D, and replace the REAL declarations by DOUBLE PRECISION.

\subsection{Examples and Remarks}

As a demonstration problem, consider the following assignments:

\begin{tabbing}
\hspace{.2in}\=$t$ = 2\\
\>$z_1$ = log(sqrt($t$))\\
\>$z_2$ = exp(2 * $z_1$)\\
\>$d$ = $z_2$ - $t$
\end{tabbing}

This should result in $z_2 = t$ and $d = 0$. These quantities and their
derivatives through order~3 with respect to $t$ are computed by the program
DRDCOMP. The results are shown in ODDCOMP. Note that $d$ and all of its
derivatives are zero to machine accuracy, as they should be.

The running time for this package is approximately proportional to $\text{N}%
^3$. To save time when developing new code using this package, one may run
with N = 0 until one is satisfied that the function evaluation is as
desired, and then increase N to activate the derivative computation.

\subparagraph{Implicit functions and series reversion}

Let $[x;t]$ denote a W-variable containing $x$ and all of its derivatives
through order N with respect to $t$, evaluated at $t_0$. Let $x_0 = x(t_0)$.
Suppose one would like to have $[t;x]$. This would represent the inverse
function that is defined implicitly by $[x;t]$. When the components of these
arrays are interpreted as scaled coefficients of Taylor series the
transformation from $[x;t]$ to $[t;x]$ is known as series reversion.

The subroutine SWRCHN can be used to produce $[t;x]$. Note, from the
specifications in Section B.4, that given $[x;t]$ and $[f;t]$, SWRCHN
transforms $[f;t]$ to $[f;x]$. Formally replacing the symbol $f$ by $t$, we
see that given $[x;t]$ and $[t;t]$, SWRCHN transforms $[t;t]$ to $[t;x]$,
thus producing the desired object. Note that the W-variable, $[t;t]$, needed
as part of the input, is just $(t_0$, 1, 0, ..., 0).

\subsection{Functional Description}

The ``W" in the names of subroutines in this package refers to R.  E.
Wengert, who, in \cite{Wengert:1964:ASA}, presented the ideas on which the
package is based.  Subsequent authors have incorporated this approach into
more automated systems \cite{Griewank:1991:ADA} that would probably be
easier to use than the present package, but such systems present various
hurdles regarding acquisition and installation.  The present
implementation provides the basic capabilities in a highly portable form.

The fundamental idea is simply to provide, for each basic mathematical
operation, code that not only performs the operation but propagates
derivative values up to the desired order. For example, to support the sine
function with derivatives through second degree, we note that if

\begin{tabbing}
\hspace{.4in}\=$z = \sin y,$\\
then\\
\>$z'= y' \cos y,$\\
and\\
\>$z''= y'' cos y - {y'}^2 \sin y,$
\end{tabbing}

where the primes denote derivatives with respect to an independent variable,
$t$. Clearly these formulas permit one to write a subroutine that can accept $%
y$, $y^{\prime}$, and $y^{\prime \prime}$ as input, and
produce $z$, $z^{\prime}$, and $z^{\prime \prime}$
as output.

This approach has been systematized for arbitrary N by appropriate use of
the chain rule of differentiation. Our subroutine SWCHN implements the chain
rule as

\begin{tabbing}
\hspace{.2in}\=do L = 0, N $-$ 1\\
\>\ \ \ \ call SWPRO(L, F(N+1$-$L), X(2), F(N+1$-$L))\\
\>enddo
\end{tabbing}

Since SWCHN is called by all of the elementary function subroutines, this
loop essentially determines the dependence on N of the running time of the
whole package. SWPRO is an $O(\text{N}^2)$ process and SWCHN is an $O(\text{N%
}^3)$ process. To improve efficiency we have coded the cases of N = 0, 1, 2,
3, and~4 as special cases in SWPRO. Multiplication counts in SWPRO and SWCHN
are as follows:

\begin{tabular}{@{\ \ }r@{\ \ }r@{\ \ }r@{\ \ }r@{\ \ }r@{\ \ }r@{\ \ }r@{\ \ }r@{\ \ }r@{\ \ }r@{\ \ }r@{\ \ }r@{\ \ }}
N $\Rightarrow$ & 0 & 1 & 2 & 3 & 4 & 5 & 6 & 7 & 8 & 9 & 10\\
SWPRO & 1 & 3 & 7 & 12 & 19 & 27 & 37 & 48 & 61 & 75 & 91\\
SWCHN & 0 & 1 & 4 & 11 & 23 & 42 & 69 & 106 & 154 & 215 & 290
\end{tabular}
\nocite{Lawson:1971:CDU}

\bibliography{math77}
\bibliographystyle{math77}

\subsection{Error Procedures and Restrictions}

\subsubsection{Upper limit on the derivative order, N}

Subroutines SWATAN, SWSUM, and SWPRO in this package contain internal arrays
whose dimensions depend on NMAX. NMAX is a PARAMETER in each of these
subroutines, nominally set to~10. This limits the derivative order argument,
N, in all of the subroutines to~10. The user would need to increase NMAX in
these three subroutines if higher order derivatives are needed.

\subsubsection{Invalid arguments for derivative computation}

The user will likely be accustomed to avoiding sending invalid arguments to
the elementary functions, such as a negative argument to the square root. In
computing derivatives there are some additional singularities to avoid. Note
that the derivative is infinite at zero for the square root, and at $\pm 1$
for arcsin and arccosine.

\subsubsection{Error handling}

Following is a list of error conditions the package detects and for which
error messages are issued. These errors are fatal in the sense that the
requested operation cannot be done; however, the default action is to return
after issuing an error message. The user can use the MATH77 library
subroutine, ERMSET of Chapter~19.2, to alter this action to cause a STOP
if desired. Error conditions not on this list, $e.g.$, negative argument
in log, will be handled by the usual host system error handler.

\begin{tabular}{@{}l@{\ \ }l}
\bf Error No.\\
\bf \& Program & \multicolumn{1}{c}{\bf Explanation}\\
\bf 1 SWASIN & Infinite derivative when arg = $-$1 or +1\\
\bf 1 SWACOS & Infinite derivative when arg = $-$1 or +1\\
\bf 2 SWSQRT & Infinite derivative when arg = 0\\
\bf 3 SWQUO1 & Zero divisor\\
\bf 4 SWPWRI & Y**I is infinite when Y = 0 and I $< 0$\\
\bf 5 SWPRO & Require dimension NMAX $\geq $ N\\
\bf 6 SWQUO & Require dimension NMAX $\geq $ N\\
\bf 7 SWQUO & Zero divisor.\\
\bf 8 SPASCL & Require N $\geq 0$\\
\bf 9 SWRCHN & Require X($2) \neq 0.$
\end{tabular}

\subsection{Supporting Information}

The source language is ANSI Fortran~77.

All of the double precision entry points require files:

DWCOMP, ERFIN, ERMSG, IERM1, IERV1.

For single precision, file SWCOMP is required in place of file DWCOMP.
\vspace{-5pt}
\begin{center}
\begin{tabular}{llll}
\multicolumn{4}{c}{\bf Entries} \\
DPASCL & DWACOS & DWASIN & DWATAN\\
DWATN2 & DWCHN & DWCOS & DWCOSH\\
DWDIF & DWDIF1 & DWEXP & DWLOG\\
DWPRO & DWPRO1 & DWPWRI & DWQUO\\
DWQUO1 & DWRCHN & DWSET & DWSIN\\
DWSINH & DWSQRT & DWSUM & DWSUM1\\
DWTAN & DWTANH & SPASCL & SWACOS\\
SWASIN & SWATAN & SWATN2 & SWCHN\\
SWCOS & SWCOSH & SWDIF & SWDIF1\\
SWEXP & SWLOG & SWPRO & SWPRO1\\
SWPWRI & SWQUO & SWQUO1 & SWRCHN\\
SWSET & SWSIN & SWSINH & SWSQRT\\
SWSUM & SWSUM1 & SWTAN & SWTANH\\
\end{tabular}
\end{center}

Designed by C. L. Lawson, JPL, 1971. Adapted to Fortran~77 for the JPL
MATH77 library, Aug.~1987.


\begcodenp
\lstset{language=[77]Fortran,showstringspaces=false}
\lstset{xleftmargin=.8in}

\centerline{\bf \large DRSWCOMP}\vspace{10pt}
\lstinputlisting{\codeloc{swcomp}}

\vspace{30pt}\centerline{\bf \large ODSWCOMP}\vspace{10pt}
\lstset{language={}}
\lstinputlisting{\outputloc{swcomp}}
\end{document}
