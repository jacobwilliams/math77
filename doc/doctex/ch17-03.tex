\documentclass[twoside]{MATH77}
\usepackage{multicol}
\usepackage[fleqn,reqno,centertags]{amsmath}
\begin{document}
\begmath 17.3  Double Precision Complex Computation

\silentfootnote{$^\copyright$1997 Calif. Inst. of Technology, \thisyear \ Math \`a la Carte, Inc.}

\subsection{Purpose}

These subprograms do the following computations for double precision complex
data: sum, difference, product, quotient, square root, and absolute value.
The Fortran~77 standard does not support a double precision complex data
type. Using the convention of storing the real and imaginary parts of a
complex number as an adjacent pair of double precision numbers, this set of
subprograms provides the above operations while staying within the standard.

\subsection{Usage}

\subsubsection{Program Prototype, Double Precision}

\begin{description}

\item[DOUBLE PRECISION] \ {\bf  A}(2){\bf , B}(2){\bf , RESULT}(2){\bf ,
ABSVAL, DZABS}

\end{description}

Assign values to A(), or to A() and B(), and access the appropriate
subprogram. The results will be stored in RESULT() or ABSVAL.
\begin{center}
\fbox{\begin{tabular}{@{\bf \ }ll@{\ }}
CALL ZSUM(A, B, RESULT) & Result $= a + b$\\
CALL ZDIF(A, B, RESULT) & Result $= a - b$\\
CALL ZPRO(A, B, RESULT) & Result $= a \times b$\\
CALL ZQUO(A, B, RESULT) & Result $= a\ /\ b$\\
CALL ZSQRTX(A, RESULT) & Result $= \sqrt a$\\
ABSVAL = DZABS(A) & Absval $= |a|$\\
\end{tabular}}
\end{center}
\subsubsection{Argument Definitions}

\begin{description}

\item[A()] \ [in] Contains the double precision complex value, $a$, with the real part in A(1) and
the imaginary part in A(2).

\item[B()] \ [in] Contains the double precision complex value, $b$, with the real part in B(1)
and the imaginary part in B(2).

\item[RESULT()] \ [out] On return contains the double precision complex result with the real part
in RESULT(1) and the imaginary part in RESULT(2). It is allowable for the
array RESULT() to occupy the same storage locations as the arrays A()
and/or B(). For example the statement, CALL ZPRO(Z, Z, Z), is
permissible.

In the case of the complex square root, which is double valued, ZSQRTX
returns the root with nonnegative real part. If the real part of the root is
zero it returns the root whose imaginary part is nonnegative. The other root
is always the negative of the returned root.

\item[DZABS] \ [out] Returns the double precision absolute value of the complex
number represented by the pair, (A(1),A(2)).

\end{description}

\subsection{Examples and Remarks}

The program, DRZCOMP, with its output, ODZCOMP, illustrates the use of these
subprograms. The example begins with three complex constants, $a$, $b$, and $u$%
. It computes $v = u+a$, $w = v\times b$, and $z = \sqrt w$. It then inverts this
sequence of computations by computing $w_2 = z\times z$, $v_2 = w_2/b$, and $u_2 =
v_2-a$. Mathematically this should result in $w_2 = w$, $v_2 =
v$, and $u_2 = u$. We test the last of these relations by computing TEST\ $=
u_2-u$.  Additionally TEST2 is a measure of the error in an
application of DZABS.  Both TEST and TEST2 are seen to be acceptably
small given that the computation was done with double precision IEEE
arithmetic.

\subsection{Functional Description}

\subsubsection{Method}

To compute $(u,v)$ as the square root of $(x,y)$ the basic algorithm is
\begin{gather*}
r = \sqrt{x^2 + y^2}\\
u = \sqrt{(r+|x|)/2}\\
v = |y|/(2u)\\
\text{if } x < 0 \text{ swap }u\text{ and }v\\
\text{if } y < 0 \text{ set }v = -v
\end{gather*}
Subroutine ZSQRTX contains special treatment for the cases of $x = 0$ or $y =
0$, and uses scaling to avoid unnecessary overflow that could result from
computing $x^2$ or $y^2.$

Subprograms ZQUO and DZABS use scaling to avoid unnecessary overflow.
In the five subroutines the implementations permit the output array to
occupy the same storage locations as the input arrays.

\subsubsection{Accuracy tests}

These subprograms have been tested using arguments on the four principal
axes and interior to the eight octants. Results were consistent with the
machine precision.

\subsection{Error Procedures and Restrictions}

If B($1) = 0$ and B($2) = 0$ in ZQUO, the subroutine will execute a division
by zero. We assume the host system will produce a runtime error diagnostic
in this case.

If the function DZABS is used its name must be typed as double precision in
the referencing program.

\subsection{Supporting Information}

The source language is ANSI Fortran~77.

Designed by C. L. Lawson, JPL, May~1986.

Programmed by C. L. Lawson and S. Y. Chiu, JPL, May~1986, Feb.~1987,
Nov.~1987.


Each of the subprograms, DZABS, ZDIF, ZPRO, ZQUO, ZSQRTX, and ZSUM is
contained in a program file of the same name.

\begcode

\medskip\

\lstset{language=[77]Fortran,showstringspaces=false}
\lstset{xleftmargin=.8in}

\centerline{\bf \large DRZCOMP}\vspace{10pt}
\lstinputlisting{\codeloc{zcomp}}

\vspace{30pt}\centerline{\bf \large ODZCOMP}\vspace{10pt}
\lstset{language={}}
\lstinputlisting{\outputloc{zcomp}}
\end{document}
