\documentclass[twoside]{MATH77}
\usepackage{multicol}
\usepackage[fleqn,reqno,centertags]{amsmath}
\begin{document}
\begmath 4.1  Square Nonsingular Systems of Equations

\silentfootnote{$^\copyright$1997 Calif. Inst. of Technology, \thisyear \ Math \`a la Carte, Inc.}

\subsection{Purpose}

For a square nonsingular matrix, $A$, of order N, subroutines are provided
for solving systems of equations involving the matrix $A$ or $A^t (A^h$ in the
complex case), and for computing the inverse, determinant, or (reciprocal)
condition number of $A$. Versions are provided for REAL, DOUBLE PRECISION and
COMPLEX data types.

\subsection{Usage}

The individual subroutines are described as follows:
\begin{description}
\item[B.1 SGEFS] Factor and solve.

\item[B.2 SGEFSC] Factor, solve, and reciprocal condition number.

\item[B.3 SGEFA] Factor.

\item[B.4 SGECO] Factor and reciprocal condition number.

\item[B.5 SGESLD] Solve $A{\bf x} = {\bf c}.$

\item[B.6 SGESLT] Solve $A^t{\bf x} = {\bf c}$. $(A^h{\bf x} = {\bf c}$ for the complex case.)

\item[B.7 SGED] Determinant.

\item[B.8 SGEI] Inverse.

\item[B.9] Modifications for double-precision and complex.
\end{description}

Subroutine SGEFA is the fundamental subroutine of this set. It does the
forward pass of Gaussian elimination with partial pivoting, replacing the
matrix $A$, given in the array A(,), with two triangular matrices representing
its LU factorization and storing a record in IPVT() of the row interchanges
done during the factorization. The other subroutines of this set either call
SGEFA to accomplish this factorization, or else must be used after the
factorization has already been computed by SGEFA.

This code is all derived from LINPACK \cite{Dongarra:1979:LUG}, but the
organization of the code into separate subroutines differs from LINPACK.

\subsubsection{Usage to factor and solve}

Use SGEFS to factor $A$ and solve the system, $AX = B$, where $A$ is an
N$\times $N square nonsingular matrix and $B$ is an N$\times $NB matrix.
The N$\times $NB
solution matrix, $X$, will overwrite the given matrix, $B$, and the factors
of $A$ will overwrite A. SGEFS calls SGEFA to factor $A$, and calls SGESLD, NB
times, to compute the columns of $X$.

\paragraph{Program Prototype, Single Precision}

\begin{description}

\item[INTEGER] \ {\bf LDA, N, LDB, NB, IPVT}($\geq $N){\bf , INFO}

\item[REAL] \ {\bf A}(LDA, $\geq $N){\bf , B}(LDB, $\geq $NB)

\end{description}

Assign values to A(,), LDA, N, B(,), LDB, and NB.

\begin{center}
\fbox{\begin{tabular}{@{\bf }c}
CALL SGEFS (A, LDA, N, B,\\
LDB, NB, IPVT, INFO)\\
\end{tabular}}
\end{center}

Computed quantities are returned in A(,), B(,), IPVT(), and INFO.

\paragraph{Argument Definitions}

\begin{description}

\item[A(,)] \ [inout] On entry contains the N$\times $N matrix $A$. On return contains
the LU factorization of $A$ as computed by SGEFA.

\item[LDA] \ [in] Leading dimensioning parameter for the array A(,). Require LDA $
\geq $ N.

\item[N] \ [in] The order of the matrix $A$ and number of rows in the matrices $B$
and $X$.

\item[B(,)] \ [inout] On entry contains the N$\times $NB matrix, $B$. On return
contains the N$\times $NB solution matrix, $X$. The array B(,) can be
declared as singly subscripted if NB = 1. No solution will be computed if
the returned value of INFO is nonzero.

\item[LDB] \ [in] Leading dimensioning parameter for the array B(,). Require LDB $
\geq $ N.

\item[NB] \ [in] Number of columns in the matrices $B$ and $X$. If NB $< 1$, the
matrix, $A$, will be factored but no reference will be made to the array B(,).

\item[IPVT()] \ [out] Integer array of length at least N. On return will contain a
record of the row interchanges done during factorization of $A$.

\item[INFO] \ [out] Set to zero if all diagonal elements in the $U$ matrix of the LU
factorization are nonzero. If nonzero, INFO is the index of the first
diagonal element of $U$ that is zero. In this latter case the solution $X$
will not be computed.

\end{description}

\subsubsection{Usage to factor, solve, and compute reciprocal condition number}

Use SGEFSC to factor $A$ and solve the system, $AX = B$, where $A$ is an
N$\times $N nonsingular matrix and $B$ is an N$\times $NB matrix. The
N$\times $NB solution matrix, $X$, will overwrite the given matrix, B, and
the factors of $A$ will overwrite A. In addition SGEFSC sets RCOND to the
reciprocal condition number of $A$.

SGEFSC calls SGECO to factor $A$ and compute RCOND, and then calls SGESLD, NB
times, to compute the columns of $X$.

\paragraph{Program Prototype, Single Precision}

\begin{description}

\item[INTEGER] \ {\bf LDA, N, LDB, NB, IPVT}($\geq $N)

\item[REAL] \ {\bf A}(LDA, $\geq $N){\bf , B}(LDB, $\geq $NB){\bf , RCOND,
Z}($\geq $N)

\end{description}

Assign values to A(,), LDA, N, B(,), LDB, and NB.

\begin{center}
\fbox{\begin{tabular}{@{\bf }c}
CALL SGEFSC(A, LDA, N, B, LDB,\\
NB, IPVT, RCOND, Z)\\
\end{tabular}}
\end{center}

Computed quantities are returned in A(,), B(,), IPVT(), RCOND, and Z().

\paragraph{Argument Definitions}

The first seven arguments for SGEFSC have the same meaning as for SGEFS. The
final two arguments are defined as follows:
\begin{description}
\item[RCOND] \ [out] An estimate for the reciprocal condition number of $A$.
Will satisfy $0.0 \leq \text{RCOND} \leq 1.0$. A zero value indicates a
singular matrix. Larger values indicate a better conditioned matrix.

\item[Z()] \ [scratch] An array of length N used as working space in SGEFSC. The
contents on return will generally not be of interest to the user. It will be
an N-vector, ${\bf z}$, that satisfies
\begin{equation*}
\|A{\bf z}\| = RCOND \cdot \|A\| \cdot \|{\bf z}\|
\end{equation*}
If $A$ is nearly singular then ${\bf z}$ will be an approximate null vector.
\end{description}

\subsubsection{Usage to factor only}

Use SGEFA to factor a given N$\times $N matrix, $A$, obtaining matrices, $L$
and $U$, satisfying $LU =A$, where $U$ is upper triangular and $L$ is a
permutation of a lower triangular matrix. $U$ and a representation
of $L^{-1}$ will be returned in the array A(,), and a record of the
permutations will be returned in IPVT().

\paragraph{Program Prototype, Single Precision}

\begin{description}

\item[INTEGER] \ {\bf LDA, N, IPVT}($\geq $N){\bf , INFO}

\item[REAL] \ {\bf A}(LDA, $\geq $N)

\end{description}

Assign values to A(,), LDA, and N.
$$
\fbox{\bf CALL SGEFA(A, LDA, N, IPVT, INFO)}
$$
Computed quantities are returned in A(,), IPVT(), and INFO.

\paragraph{Argument Definitions}

The arguments have the same meaning as the arguments of the same names for
the subroutine SGEFS described above in Section~B.1.

\subsubsection{Usage to factor and compute reciprocal condition number}

SGECO calls SGEFA to factor the given matrix, $A$, and then computes RCOND as
an estimate of the reciprocal condition number of $A$.

\paragraph{Program Prototype, Single Precision}

\begin{description}

\item[INTEGER] \ {\bf LDA, N, IPVT}($\geq $N)

\item[REAL] \ {\bf A}(LDA, $\geq $N){\bf , RCOND, Z}($\geq $N)

\end{description}

Assign values to A(,), LDA, and N.
$$
\fbox{\bf CALL SGECO(A, LDA, N, IPVT, RCOND, Z)}
$$
Computed quantities are returned in A(,), IPVT(), RCOND, and Z().

\paragraph{Argument Definitions}

\begin{description}
\item[A(,), LDA, N, IPVT()] \ See description in Section~B.1 above.

\item[RCOND, Z()] \ See description in Section~B.2 above.
\end{description}

\subsubsection{Usage to solve the direct system, $A{\bf x} = {\bf c}$}

Given A(,) and IPVT() containing factorization results produced by SGEFA for
a matrix, $A$, use SGESLD to solve the direct system, $A{\bf x} = {\bf c}.$

\paragraph{Program Prototype, Single Precision}

\begin{description}

\item[INTEGER] \ {\bf LDA, N, IPVT}($\geq $N)

\item[REAL] \ {\bf A}(LDA, $\geq $N){\bf , C}($\geq $N)

\end{description}

Values must initially be present in A(,), LDA, N, IPVT(), and C().
$$
\fbox{\bf CALL SGESLD(A, LDA, N, IPVT, C)}
$$
Computed quantities are returned in C().

\paragraph{Argument Definitions}

\begin{description}
\item[A(,)] \ [in] On entry contains the N$\times $N set of numbers representing the
LU decomposition of a matrix $A$ as computed by SGEFA. The contents of A(,) are
not altered by this subroutine.

\item[LDA] \ [in] Leading dimensioning parameter for the array A(,). Require LDA $
\geq\ $N.

\item[N] \ [in] The order of the original matrix $A$.

\item[IPVT()] \ [in] Integer array of length at least N. On entry contains a record
of the row interchanges done during factorization of $A$.

\item[C()] \ [inout] On entry must contain the right-side N-vector for the
problem, $A{\bf x} = {\bf c}$. On return, if $A$ is nonsingular, C() will contain the
solution N-vector, ${\bf x}$. See Section~E for discussion of singularity.
\end{description}

\subsubsection{Usage to solve the transposed system, $A^t{\bf x} = {\bf c}$}

Given A(,) and IPVT() containing factorization results produced by SGEFA for
a matrix, $A$, use SGESLT to solve the transposed system, $A^t{\bf x} = {\bf c}$. In the
complex case, $i.e$. using CGEFA rather than SGEFA, the problem solved is $%
A^h{\bf x} = {\bf c}$, where $A^h$ denotes the conjugate transpose of $A$.

\paragraph{Program Prototype, Single Precision}

\begin{description}

\item[INTEGER] \ {\bf LDA, N, IPVT}($\geq $N)

\item[REAL] \ {\bf A}(LDA, $\geq $N){\bf , C}($\geq $N)

\end{description}

Values must initially be present in A(,), LDA, N, IPVT(), and C().
$$
\fbox{\bf CALL SGESLT(A, LDA, N, IPVT, C)}
$$
Computed quantities are returned in C().

\paragraph{Argument Definitions}

\begin{description}
\item[A(,), LDA, N, IPVT()] \ [in] See description in Section~B.5 above.

\item[C()] \ [inout] On entry must contain the right-side N-vector for the
problem, $A^t{\bf x} = {\bf c}$. On return, if $A$ is nonsingular, C() will contain the
solution N-vector, ${\bf x}$. See Section~E for discussion of singularity.
\end{description}

\subsubsection{Usage to compute the determinant of $A$}

Given A(,) and IPVT() containing factorization results produced by SGEFA for
a matrix, $A$, use SGED to compute the determinant of $A$.

\paragraph{Program Prototype, Single Precision}

\begin{description}

\item[INTEGER] \ {\bf LDA, N, IPVT}($\geq $N)

\item[REAL] \ {\bf A}(LDA, $\geq $N){\bf , DET}(2)

\end{description}

Values must initially be present in A(,), LDA, N, and IPVT().
$$
\fbox{\bf CALL SGED(A, LDA, N, IPVT, DET)}
$$
Computed quantities are returned in DET(1) and DET(2).

\paragraph{Argument Definitions}

\begin{description}
\item[A(,), LDA, N, IPVT()] \ [in] See description in Section~B.5 above.

\item[DET()] \ [out] DET(1) and DET(2) will be set to values representing the
determinant of $A$ in the form:
\begin{equation*}
\text{Determinant = DET}(1) \times  10^{\text{DET}(2)}
\end{equation*}
DET(2) will contain an integer value, which may be positive, negative, or
zero.

If the determinant is zero, both DET(1) and DET(2) will be zero.

If the determinant is nonzero, DET(1) and DET(2) will be some choice among
the many pairs of values that could be used to represent the value of the
determinant. The algorithm tends to produce a representation with DET(2) = 0
if the magnitude of the determinant is not extremely large or small,
however it is not designed to select this representation in all possible
cases. See Section~D for further explanation.
\end{description}

\subsubsection{Usage to compute the inverse matrix of A}

Given A(,) and IPVT() containing factorization results produced by SGEFA for
a matrix, $A$, use SGEI to compute the inverse matrix of $A$.

\paragraph{Program Prototype, Single Precision}

\begin{description}

\item[INTEGER] \ {\bf LDA, N, IPVT}($\geq $N)

\item[REAL] \ {\bf A}(LDA, $\geq $N){\bf , WORK}($\geq $N)

\end{description}

Values must initially be present in A(,), LDA, N, and IPVT().
$$
\fbox{\bf CALL SGEI(A, LDA, N, IPVT, WORK)}
$$
Computed quantities are returned in A(,).

\paragraph{Argument Definitions}

\begin{description}
\item[A(,)] \ [inout] On entry contains an N$\times $N set of numbers representing
the LU decomposition of a matrix $A$ as computed by SGEFA. On return, if $A$ is
nonsingular, contains the inverse matrix of $A$. See Section~E for
discussion of singularity.

\item[LDA, N, IPVT()] [in] See description in Section~B.5 above.

\item[WORK()] \ [scratch] An array of at least N locations used as internal work
space.
\end{description}

\subsubsection{Modifications for Double Precision or Complex}

For double-precision usage change the initial letter of each subroutine name
from S to D, and change the REAL declarations to DOUBLE PRECISION.

For complex usage change the initial letter of each subroutine name from S
to C, and change the REAL declarations to COMPLEX, with the exception that
the argument, RCOND, must remain REAL.

Note that the COMPLEX subroutine CGESLT solves $A^h{\bf x} = {\bf c}$, where $A^h$
denotes the conjugate transpose of $A$.

\subsection{Examples and Remarks}

Program DRSGEFSC illustrates the use of subroutine SGEFSC to solve a system
of linear equations and compute the reciprocal condition number of the
matrix of the system. Output is shown in ODSGEFSC. The data for this problem
were chosen so the exact solution has components 2, $-$5, and~3.

\subparagraph{Avoiding computation of the inverse}

If $A$ is a nonsingular matrix, the relations $AX = B$ and $X = A^{%
-1}B$ are mathematically equivalent. Thus, given $A$ and $B$,
where $B$ and thus $X$ may be either a matrix or a vector, one could either
solve the system $AX = B$ for $X$, or one could compute $A^{-1}$
and then multiply $A^{-1}$ times $B$ to obtain $X$. The
former approach will be faster as it requires fewer arithmetic operations.
It will also generally be slightly more accurate. Thus wherever an inverse
matrix appears in an expression to be computed, it is preferable to
formulate the computation in terms of solving systems rather than computing
inverses, unless the inverse matrix is needed for another reason.

\subsection{Functional Description}

This code is all derived from LINPACK \cite{Dongarra:1979:LUG}.  Reference
\cite{Dongarra:1979:LUG} gives complete descriptions of the algorithms
implemented.

SGEFA and SGECO are the same in name, argument list, and functionality in
Fortran~77 as the LINPACK subroutines of the same name. SGESLD and SGESLT
provide the two distinct functionalities provided by the LINPACK subroutine,
SGESL. SGED and SGEI provide the two distinct functionalities provided by
the LINPACK subroutine, SGEDI. SGEFS and SGEFSC are shell subroutines added
to package certain convenient sequences of calls to the other subroutines.

\subsubsection{Factorization}

The fundamental subroutine of this set is SGEFA. SGEFA performs the forward
pass of Gaussian elimination with partial pivoting. Partial pivoting means
row interchanges are used to bring the element of largest magnitude, at or
below the diagonal position in the pivot column, to the diagonal position at
each stage of the elimination process. A record of the row interchanges is
kept in IPVT(). This algorithm has very good numerical stability and
efficiency, the count of arithmetic operations being $n^3/3 + O(n^2)$
additions and multiplications.

The result of this process is a factorization of the given matrix, $A$, of the
form
\begin{equation*}
A = LU = P_1K_1P_2K_2 \cdots P_{n-1}K_{n-1}U
\end{equation*}
where $U$ is an upper triangular matrix, each $P_i$ is a permutation matrix
representing the permutation of the index $i$ with an index $\geq i$, and
each $K_i$ differs from the identity only by having nonzero elements below
the diagonal in column $i$. SGEFA stores a convenient representation of this
factorization in the arrays A(,) and IPVT(). Each of the subroutines SGESLD,
SGESLT, SGED, and SGEI is designed to use this factorization as the starting
point for its computation.

\subsubsection{Scaling}

Even with partial pivoting, the accuracy of the solution can be adversely
affected if the scaling is extremely bad. There is no universally applicable
algorithmic criterion for determining a ``good" scaling, and consequently
the LINPACK subroutines do not introduce any scaling. The user should assure
that the scaling is reasonable for his or her application. One reasonable
approach is to scale the rows and columns of $A$ so that, if possible, the
absolute size of the uncertainty in each element of the matrix, $A$, is nearly
the same for all elements.

\subsubsection{Solving equations using SGESLD or SGESLT}

SGESLD completes the solution of the problem, $A{\bf x} = {\bf b}$. Given the
factorization, $A = LU$, this is done by first solving $L{\bf y} = {\bf b}$, and then
solving $U{\bf x} = {\bf y}.$

SGESLT completes the solution of the problem, $A^t{\bf x} = {\bf b}$. Given the
factorization, $A = LU$, this is done by first solving $U^t{\bf y} = {\bf b}$, and then
solving $L^t{\bf x} = {\bf y}.$

Each of these two subroutines has an operation count of $n^2 + O(n)$
additions and multiplications.

\subsubsection{Computing $A^{-1}$}

SGEI completes the computation of $A^{-1}$. Given the
factorization, $A = LU$, $A^{-1}$ can be expressed as $A^{%
-1} = U^{-1}L^{-1}$. SGEI first
computes the triangular matrix $U^{-1}$, overwriting $U$. It
then does the two steps of forming $L^{-1}$ and multiplying $%
U^{-1}$ times $L^{-1}$ in an interleaved manner
to conserve storage. The algorithm uses $n^2$ locations in A(,) and $n$
locations in WORK(). Operations required are $(2/3)n^3 + O(n^2)$ additions
and multiplications.

\subsubsection{Estimating the reciprocal condition number}

The condition number of an $n\times n$ nonsingular matrix, $A$, is defined
as $\kappa = \|A\|\times \|A^{-1}\|$, a quantity that is never less than
unity.  This is the largest factor by which the relative error in a
vector, ${\bf y}$, can be amplified as a result of multiplication by $A$
or by $A^{-1}$.  Roughly speaking, if $\kappa = 10^k$ and the components
of the vector ${\bf b}$ in the problem, $A{\bf x} = {\bf b}$, are known to
$d$ decimal digits, and the components of $A$ are known to more than $d$
decimal digits, and the problem is solved using precision greater than $d$
decimal digits, then the solution will be known to about $d-k$ decimal
digits.  In particular, if $k \geq d$, the solution may have no reliable
digits at all.  (See \cite{Dongarra:1979:LUG} for a more precise
discussion of the relation between $\kappa $ and the accuracy of ${\bf
x}$.)

Any method of computing $\kappa $ beginning with $A$ appears to require at
least $n^3$ additions and multiplications. The authors of LINPACK, in
collaboration with J. H. Wilkinson, developed an algorithm for estimating $%
\kappa $ (actually $\kappa ^{-1})$ that requires only $3n^2 +
O(n)$ additions and multiplications following the factorization produced by
SGEFA.

Subroutine SGECO first computes $\|A\|$ as a maximum of the absolute sum
norms of the columns of $A$.  It then calls SGEFA to factorize $A$, and
finally executes the LINPACK algorithm for condition number estimation,
returning RCOND as its estimate of $\kappa ^{-1}$.  Let $C = RCOND^{%
-1}$.  The hope is that $C/\kappa $ will always be close to unity.  It can
be shown that with exact arithmetic one would always have $C/\kappa \leq
1$, but no theoretical lower bound is known.  In a test with 1250~cases
reported in \cite{Dongarra:1979:LUG}, the lowest value that $C/\kappa $
attained was~0.062, {\em i.e.}\ the condition number was slightly more
than 16 times what was estimated.

\subsubsection{Computing the determinant of $A$}

SGED completes the computation of the determinant of $A$. In the
factorization, $A = LU$, we have $Det(L) = \pm 1$ and $U$ is triangular.
Thus SGED computes $Det(A)$ by forming the product of the diagonal elements of
$U$, and attaching a sign determined by analysis of the permutation record
in IPVT(). This process may be represented as
\begin{align*}
d_0 &= 1\\
d_i &= \pm u_{i,i}d_{i-1},\quad i = 1,\ \ldots,\ n
\end{align*}
determinant $= d_n$

For a matrix of large order it is not uncommon for the determinant to be of
extremely large or small magnitude. Consequently the LINPACK approach
computes and stores the determinant as a pair of numbers, permitting a very
large exponent range.

SGED differs, however, from the LINPACK subroutine SGEDI in the two-number
representation selected. If all of the quantities, $|u_{i,i}|$ and $|d_i|$,
in the above equations lie between the square roots of the underflow and
overflow limits of the host computer system, the returned value of DET(2)
will be zero. Thus in many problems of moderate order and having moderate
scaling, DET(1) by itself will be the determinant value. Besides increasing
the likelihood of DET(2) being zero, this modification reduces significantly
the number of multiplications used for scaling.

\subsubsection{Other types of matrices and problems handled in
LINPACK}

The subroutines described here represent only a small part of the full
LINPACK collection. It is thus desirable to have LINPACK as well as the MATH
77 library. LINPACK provides functionalities analogous to those described
here for the following special types of matrices:

\begin{tabular}{@{}l@{}l}
General band & Hermitian indefinite\\
Positive definite & Hermitian indefinite packed\\
Positive definite packed & Triangular\\
Positive definite band & General tridiagonal\\
Symmetric indefinite & Positive definite tridiagonal\\
Symmetric indefinite packed & \
\end{tabular}

LINPACK also provides subroutines for least-squares problems.

\bibliography{math77}
\bibliographystyle{math77}

\subsection{Error Procedures and Restrictions}

The LU factorization can be completed for a singular matrix, however the
subroutines of this set are not capable of solving equations using a
singular matrix. The inverse matrix does not exist for a singular matrix.

A singular matrix is indicated by the returned value of INFO being nonzero
or RCOND being zero. In either of these cases the subroutines SGESLD,
SGESLT, and SGEI cannot compute their usual outputs. If any of these latter
three subroutines encounters a zero diagonal element in the $U$ matrix, it
will issue an error message using the subroutines of Chapter~19.2
with an error level of zero and return with incomplete results.

These subroutines all require N $\geq 1$ and LDA $\geq $ N. Subroutines SGEFS
and SGEFSC also require LDB $\geq $ N. These conditions are not checked and
their violation causes unpredictable effects.

\subsection{Supporting Information}

The source language is ANSI Fortran~77.

LINPACK was developed as an NSF funded project from~1974 through~1976. This
library of high-quality public-domain portable Fortran linear algebra
software is widely used in the U.S. and throughout the world.

The subroutines described here were adapted from LINPACK to Fortran~77 for
the JPL MATH77 library by C. L. Lawson and S. Y. Chiu, JPL, August~1987.


\begin{tabular}{@{\bf}l@{\hspace{5pt}}l}
\bf Entry & \hspace{.35in} {\bf Required Files}\vspace{2pt} \\
CGECO & \parbox[t]{2.7in}{\hyphenpenalty10000 \raggedright
CAXPY, CDOTC, CGECO, CGEFA, CSCAL, CSSCAL, ICAMAX, SCASUM\rule[-5pt]{0pt}{8pt}}\\
CGED & \parbox[t]{2.7in}{\hyphenpenalty10000 \raggedright AMACH, CGED\rule[-5pt]{0pt}{8pt}}\\
CGEFA & \parbox[t]{2.7in}{\hyphenpenalty10000 \raggedright
CAXPY, CGEFA, CSCAL, ICAMAX\rule[-5pt]{0pt}{8pt}}\\
CGEFS & \parbox[t]{2.7in}{\hyphenpenalty10000 \raggedright
CAXPY, CGEFA, CGEFS, CGESLD, CSCAL, ERFIN, ERMSG, ICAMAX\rule[-5pt]{0pt}{8pt}}\\
CGEFSC & \parbox[t]{2.7in}{\hyphenpenalty10000 \raggedright CAXPY, CDOTC,
CGECO, CGEFA, CGEFSC, CGESLD, CSCAL, CSSCAL, ERFIN, ERMSG, ICAMAX, SCASUM\rule[-5pt]{0pt}{8pt}}\\
CGEI & \parbox[t]{2.7in}{\hyphenpenalty10000 \raggedright
CAXPY, CGEI, CSCAL, CSWAP, ERFIN, ERMSG\rule[-5pt]{0pt}{8pt}}\\
CGESLD & \parbox[t]{2.7in}{\hyphenpenalty10000 \raggedright
CAXPY, CGESLD, ERFIN, ERMSG\rule[-5pt]{0pt}{8pt}}\\
CGESLT & \parbox[t]{2.7in}{\hyphenpenalty10000 \raggedright
CDOTC, CGESLT, ERFIN, ERMSG\rule[-5pt]{0pt}{8pt}}\\
DGECO & \parbox[t]{2.7in}{\hyphenpenalty10000 \raggedright
DASUM, DAXPY, DDOT, DGECO, DGEFA, DSCAL, IDAMAX\rule[-5pt]{0pt}{8pt}}\\
DGED & \parbox[t]{2.7in}{\hyphenpenalty10000 \raggedright AMACH, DGED\rule[-5pt]{0pt}{8pt}}\\
DGEFA & \parbox[t]{2.7in}{\hyphenpenalty10000 \raggedright
DAXPY, DGEFA, DSCAL, IDAMAX\rule[-5pt]{0pt}{8pt}}\\
DGEFS & \parbox[t]{2.7in}{\hyphenpenalty10000 \raggedright
DAXPY, DGEFA, DGEFS, DGESLD, DSCAL, ERFIN, ERMSG, IDAMAX}\\
\ & \ \\\end{tabular}

\begin{tabular}{@{\bf}l@{\hspace{5pt}}l}
\bf Entry & \hspace{.35in} {\bf Required Files}\vspace{2pt} \\
DGEFSC & \parbox[t]{2.7in}{\hyphenpenalty10000 \raggedright DASUM, DAXPY,
DDOT, DGECO, DGEFA, DGEFSC, DGESLD, DSCAL, ERFIN, ERMSG, IDAMAX\rule[-5pt]{0pt}{8pt}}\\
DGEI & \parbox[t]{2.7in}{\hyphenpenalty10000 \raggedright
DAXPY, DGEI, DSCAL, DSWAP, ERFIN, ERMSG\rule[-5pt]{0pt}{8pt}}\\
DGESLD & \parbox[t]{2.7in}{\hyphenpenalty10000 \raggedright
DAXPY, DGESLD, ERFIN, ERMSG\rule[-5pt]{0pt}{8pt}}\\
DGESLT & \parbox[t]{2.7in}{\hyphenpenalty10000 \raggedright
DDOT, DGESLT, ERFIN, ERMSG\rule[-5pt]{0pt}{8pt}}\\
SGECO & \parbox[t]{2.7in}{\hyphenpenalty10000 \raggedright
ISAMAX, SASUM, SAXPY, SDOT, SGECO, SGEFA, SSCAL\rule[-5pt]{0pt}{8pt}}\\
SGED & \parbox[t]{2.7in}{\hyphenpenalty10000 \raggedright AMACH, SGED\rule[-5pt]{0pt}{8pt}}\\
SGEFA & \parbox[t]{2.7in}{\hyphenpenalty10000 \raggedright
ISAMAX, SAXPY, SGEFA, SSCAL\rule[-5pt]{0pt}{8pt}}\\
SGEFS & \parbox[t]{2.7in}{\hyphenpenalty10000 \raggedright
ERFIN, ERMSG, ISAMAX, SAXPY, SGEFA, SGEFS, SGESLD, SSCAL\rule[-5pt]{0pt}{8pt}}\\
SGEFSC & \parbox[t]{2.7in}{\hyphenpenalty10000 \raggedright ERFIN, ERMSG,
ISAMAX, SASUM, SAXPY, SDOT, SGECO, SGEFA, SGEFSC, SGESLD, SSCAL\rule[-5pt]{0pt}{8pt}}\\
SGEI & \parbox[t]{2.7in}{\hyphenpenalty10000 \raggedright
ERFIN, ERMSG, SAXPY, SGEI, SSCAL, SSWAP\rule[-5pt]{0pt}{8pt}}\\
SGESLD & \parbox[t]{2.7in}{\hyphenpenalty10000 \raggedright
ERFIN, ERMSG, SAXPY, SGESLD\rule[-5pt]{0pt}{8pt}}\\
SGESLT & \parbox[t]{2.7in}{\hyphenpenalty10000 \raggedright
ERFIN, ERMSG, SDOT, SGESLT\rule[-5pt]{0pt}{8pt}}\\\end{tabular}

\begcodenp
\lstset{language=[77]Fortran,showstringspaces=false}
\lstset{xleftmargin=.8in}

\centerline{\bf \large DRSGEFSC}\vspace{0pt}
\lstinputlisting{\codeloc{sgefsc}}

\vspace{10pt}\centerline{\bf \large ODSGEFSC}\vspace{-10pt}
\lstset{language={}}
\lstinputlisting{\outputloc{sgefsc}}
\end{document}
