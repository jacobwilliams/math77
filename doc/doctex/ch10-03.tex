\documentclass[twoside]{MATH77}
\usepackage{multicol}
\usepackage[fleqn,reqno,centertags]{amsmath}
\begin{document}
\begmath 10.3 Complex Fourier Transform

\silentfootnote{$^\copyright$1997 Calif. Inst. of Technology, \thisyear \ Math \`a la Carte, Inc.}

\subsection{Purpose}

This subroutine computes Fourier transforms for complex data in up to
6~dimensions using the fast Fourier transform. The relation between values $z$
and Fourier coefficients $\zeta $ is defined by%
\begin{multline*}
 z(j_1,j_2,\ldots,j_{ND})= \sum_{k_1=0}^{N_1-1} \cdots\\
\sum_{k_{ND}=0}^{N_{ND}-1}
\zeta (k_1,k_2,...,k_{ND})W_1^{j_1k_1} \cdots
W_{ND}^{j_{ND}k_{ND}},\text{ and}
\end{multline*}\vspace{-20pt}
\begin{multline*}
\zeta(k_1,k_2,\ldots,k_{ND})=\frac 1N_1 \cdots \frac 1N_{ND}
\sum_{j_1=0}^{N_1-1} \cdots\\
\sum_{j_{ND}=0}^{N_{ND}-1}
 z(j_1,\ldots,j_{ND}) W^{-j_1k_1} \cdots W^{-j_{ND}k_{ND}}
\end{multline*}
where $N_\ell =2^{\text{M}(\ell)}$, $W_\ell =e^{2\pi i/N_\ell}$, $%
0\leq j_\ell,\ k_\ell \leq N_\ell -1$, and $z$ and $\zeta $ are complex.

\subsection{Usage}

\subsubsection{Program Prototype, Single Precision}

\begin{description}
\item[\bf COMPLEX]  \ {\bf A}$(N_1$, $N_2$, ..., $\geq N_{ND})$
\ \ $[N_k=2^{\text{M}(k)}]$
\item[\bf REAL]  \ {\bf S}$(\geq \max (\nu _1${\bf , $\nu _2$, ..., $\nu
_{ND})-1)$} $[\nu _k=2^{\text{M}(k)-2}]$
\item[\bf INTEGER]  \ {\bf M}$(\geq {\textstyle ND})${\bf , ND, MS}
\item[\bf CHARACTER]  \ {\bf MODE$*(\geq {\textstyle ND})$}
\end{description}
On the initial call set MS to~0 to indicate the array S($)$ does not yet
contain a sine table. Assign values to A(), MODE, M, and ND.
$$
\fbox{{\bf CALL SCFT(A, MODE, M, ND, MS, S)}}
$$
A() will contain computed results. S($)$ will contain the sine table used in
computing the Fourier transform. MS may have been changed.

\subsubsection{Argument Definitions}

\begin{description}
\item[A()]  [inout] If the argument MODE selects analysis in all dimensions,
A() contains values $z$ on entry, and Fourier coefficients $\zeta $ on exit.
If MODE selects analysis in all dimensions, A() contains Fourier
coefficients $\zeta $ on entry, and values $z$ on exit. When A() contains $z$%
, A($j_1+1$, $j_2+1$, ..., $j_{ND}+1)=z(j_1$, $j_2$, ..., $j_{ND})$, and
when A() contains $\zeta $, A($k_1+1$, $k_2+1$, ..., $k_{ND}+1)=\zeta (k_1$,
$k_2$, ..., $k_{ND})$, 0\ $\leq j_i,k_i\leq 2^{\text{M}(i)}-1$, $i=0$, 1, ..., ND.

\item[MODE]  [in] The character MODE$(k{:}k)$ selects Analysis or Synthesis
in the $k^{th}$ dimension. 'A' or 'a' selects Analysis, transforming $%
z$'s to $\zeta $'s. 'S' or 's' selects Synthesis,
transforming $\zeta $'s to $z^{\prime }s.$

\item[M($)$]  [in] Defines $N_k=2^{\text{M}(k)}$, the number of complex data points
in the $k^{th}$ dimension. Require 0$\ \leq \text{M}(k)\leq 30$ for
all $k$. No action is taken in dimensions for which M($k)=0.$

\item[ND]  [in] Number of dimensions. Require 1 $\leq $ ND $\leq $ 6.

\item[MS]  [inout] Gives the state of the sine table in $S()$.  Let
$\text{MS}_{in}\text{ and MS}_{out}$ denote the values of MS on entry
and return respectively. If the sine table has not previously been
computed, set $\text{MS}_{in} = 0$ or $-$1 before the call. Otherwise
the value of $\text{MS}_{out}$ from the previous call using the same
S() array can be used as $\text{MS}_{in}$ for the current call.

Certain error conditions described in Section E cause the subroutine
to set $\text{MS}_{out} = -2$ and return.  Otherwise, with $\max _i
\{\text{M}(i)\} > 0$, the subroutine sets $\text{MS}_{out} = \max ($M(1),
M(2), ..., M(ND), $\text{MS}_{in}).$

If $\text{MS}_{out} > \max (2, \text{MS}_{in}),$ the subroutine sets
NT = $2^{\text{MS}_{out}-2}$ and fills S() with NT $-$ 1 sine values.

If $\text{MS}_{in}=-1$, the subroutine returns after the above
actions, not transforming the data in A().  This is intended to allow
the use of the sine table for data alteration before a subsequent Fourier
transform, as discussed in Section G of Chapter~16.0.

\item[S($)$]  [inout] When the sine table has been computed, S($j)=\sin \pi
j/(2\times {\textstyle NT})$, $j=1$, 2, ..., NT $-$ 1, see MS above.
\end{description}
\subsubsection{Modifications for Double Precision}

Change SCFT to DCFT and the REAL type statement to DOUBLE PRECISION. If it
is available, one can change the COMPLEX type statement to DOUBLE PRECISION
COMPLEX. For portability, (or out of necessity) one can change the COMPLEX
statement to DOUBLE PRECISION and change the first dimension to be twice as
big. The data should then be stored in A with the imaginary parts of the
complex numbers following immediately after the real parts.  This
representation is compatible with the representation used in Fortran
90, and with most compilers that extend Fortran~77 to provide a double
precision complex type.

\subsection{Examples and Remarks}

Estimate the spectral composition of%
$$
f(t)=[\sin 2\pi (t+0.1)+4\cos 2\pi (\sqrt{2}\,t+0.3)]+0i
$$
where we make the same assumptions and use the same $\Delta t$ and N as in
the example for SRFT1. Differences between the results given here and those
obtained for SRFT1 are due to the use of sigma factors in SRFT1. (It is more
efficient to use SRFT1 when $f$ is a real function. A real function was used
here to show the effect of the sigma factors.) Note that the peaks are
slightly sharper here than they are for SRFT1, but that as one leaves the
peaks the coefficients do not tend to zero nearly as rapidly as when the
sigma factors are used. The program to do these calculations and the results
are given at the end of this chapter.

\subsection{Functional Description}

The multi-dimensional complex transform involves calling SFFT to compute
one-dimensional complex transforms with respect to each dimension. For ND $=
1$, the formulas are given by Eqs.\,(9) and (10) in Chapter~16.0. For ${%
\textstyle ND} \geq 1$, the formula for $z$ given $\zeta $ is given in
Purpose above.  More details can be found in \cite{Krogh:1970:CFT}.

\bibliography{math77}
\bibliographystyle{math77}

\subsection{Error Procedures and Restrictions}

Require 1 $\leq $ ND $\leq $ 6 and 0 $\leq $ M($k)\leq 30$ for all $k$.
MODE must have one of its allowed values. If any of these conditions are
violated, the subroutine will issue an error message using the error
processing procedures of Chapter~19.2 with severity level~2
to cause execution to stop. A return is made with $\text{MS}=-2$
instead of stopping if the statement ``CALL\ ERMSET($-$1)'' is executed
before calling this subroutine.

If the sine table does not appear to have valid data, an error message is
printed, and the sine table and then the transform are computed.

\subsection{Supporting Information}

The source language is ANSI Fortran~77.

\begin{tabular}{@{\bf}l@{\hspace{5pt}}l}
\bf Entry & \hspace{.35in} {\bf Required Files}\vspace{2pt} \\
DCFT & \parbox[t]{2.7in}{\hyphenpenalty10000 \raggedright
 DCFT, DFFT, ERFIN, ERMSG, IERM1, IERV1\rule[-5pt]{0pt}{8pt}}\\
SCFT & \parbox[t]{2.7in}{\hyphenpenalty10000 \raggedright
 ERFIN, ERMSG, IERM1, IERV1, SCFT, SFFT}\\
\end{tabular}

Subroutine designed and written by: Fred T. Krogh, JPL, October~1969,
revised January~1988.


\begcodenp

\lstset{language=[77]Fortran,showstringspaces=false}
\lstset{xleftmargin=.8in}

\centerline{\bf \large DRSCFT}\vspace{0pt}
\lstinputlisting{\codeloc{scft}}

\vspace{5pt}\centerline{\bf \large ODSCFT}\vspace{5pt}
\lstset{language={}}
\lstinputlisting{\outputloc{scft}}
\end{document}
