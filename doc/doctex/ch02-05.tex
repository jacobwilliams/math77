\documentclass[twoside]{MATH77}
\usepackage[\graphtype]{mfpic}
\usepackage{multicol}
\usepackage[fleqn,reqno,centertags]{amsmath}
\begin{document}
\opengraphsfile{pl02-05}
\hyphenation{ODSBESJN}
\begmath 2.5 Bessel Functions of General Orders $J_\nu $ and $Y_\nu $

\silentfootnote{$^\copyright$1997 Calif. Inst. of Technology, \thisyear \ Math \`a la Carte, Inc.}

\subsection{Purpose}

These subroutines compute a sequence of values $J_\nu (x)$ or $Y_\nu (x)$
for $\nu = \alpha $, $\alpha + 1$, ..., $\alpha +\text{NUM} - 1$. $J_\nu $
and $Y_\nu $ are Bessel functions of the first and second kinds,
respectively, as described in \cite{ams55:bes}. $J_\nu $ and $Y_\nu $ are a
pair of linearly independent solutions of the differential equation
\begin{equation*}
x^2\frac{d^2w}{dx^2} + x\frac{dw}{dx} + (x^2 - \nu ^2)w = 0
\end{equation*}
$Y_\nu $ is also sometimes called the Neumann function and denoted by $N_\nu
.$

\subsection{Usage}

\subsubsection{Program Prototype, Single Precision}

\begin{description}
\item[REAL]  \ {\bf X, ALPHA, BJ}($\geq $NUM){\bf ,BY}($\geq $NUM)

\item[INTEGER]  \ {\bf NUM}
\end{description}

Assign values to X, ALPHA, and NUM. To evaluate J Bessel functions:
$$
\fbox{{\bf CALL SBESJN (X, ALPHA, NUM, BJ)}}
$$
To evaluate Y Bessel functions:
$$
\fbox{{\bf CALL SBESYN (X, ALPHA, NUM, BY)}}
$$
The results are stored in BJ() or BY(), respectively.

\subsubsection{Argument Definitions}

\begin{description}
\item[X]  \ [in] Argument for function evaluation. Require X $\geq 0$ for
the J function and X $>0$ for the Y function. Require X $<(16\rho
)^{-1}$ for both functions, where $\rho $ denotes the machine
precision.

\item[ALPHA]  \ [in] Lowest order, $\nu $, for which $J_\nu (x)$ or $Y_\nu
(x)$ is to be computed. Require ALPHA $\geq 0$. For sufficiently large $\nu $%
, depending on $x$, positive values of $J_\nu (x)$ will be smaller than the
computer's underflow limit and the magnitude of $Y_\nu (x)$ will exceed the
overflow limit. SBESYN issues an error message before overflow occurs.

\item[NUM]  \ [in] Number of values of $\nu $ for which $J_\nu (x)$ or $%
Y_\nu (x)$ is to be computed. Require NUM $\geq 1.$

\item[BJ()]  \ [out] Array in which SBESJN will store results. BJ($%
i)=J_{\alpha +i-1}(x)$ for $i=1$, 2, ..., NUM.

\item[BY()]  \ [out] Array in which SBESYN will store results. BY($%
i)=Y_{\alpha +i-1}(x)$ for $i=1$, 2, ..., NUM.
\end{description}

\subsubsection{Modifications for Double Precision}

For double precision usage, change the REAL statement to DOUBLE PRECISION
and change the subroutine names to DBESJN and DBESYN, respectively.

\subsection{Examples and Remarks}

These Bessel functions satisfy the Wronskian identity (\cite{ams55:bes},
Eq.\,9.1.16)
\begin{equation*}
z(\nu ,x) = \frac{x\pi }{2} \left[J_{\nu +1}(x) Y_\nu (x) - J_\nu (x) Y_{\nu
+1}(x)\right] - 1 = 0
\end{equation*}
The program DRSBESJN evaluates this expression for a few values of $\nu $
and $x$. The results are shown in ODSBESJN.

\subsection{Functional Description}

\subsubsection{Properties of J and Y}

In the region $x \geq \nu $, both J and Y are oscillatory and are bounded in
magnitude by one. For fixed $\nu \geq 0$ and increasing $x$ these functions
have asymptotic behavior described by (\cite{ams55:bes}, Eqs.\,9.2.1~--~9.2.2)
\begin{align}
\label{O1}J_\nu (x) &\sim [2/(\pi x)]^{1/2} \cos (x-(\nu + 0.5)\pi /2)
\hspace{-1in}\\
\label{O2}Y_\nu (x) &\sim [2/(\pi x)]^{1/2} \sin (x - (\nu + 0.5)\pi /2)
\hspace{-1in}
\end{align}
In the region $\nu \geq x$, $J_\nu (x)$ is positive and bounded and
approaches zero as $\nu $ increases with fixed $x > 0$, while $Y_\nu (x)$ is
negative and unbounded and approaches $-\infty $ as $\nu $ increases with
fixed $x > 0$. For fixed $x > 0$ and increasing $\nu $, these functions have
asymptotic behavior described by (\cite{ams55:bes}, Eqs.\,9.3.1~--~9.3.2).
\begin{align}
\label{O3}J_\nu (x) &\sim (2\pi \nu )^{-1/2}(ex/(2\nu ))^\nu\\
\label{O4}Y_\nu (x) &\sim -(2/(\pi \nu ))^{\frac{1}{2}}(ex/(2\nu))^{-\nu}
\end{align}
where $e = 2.718\cdots .$

Both J and Y satisfy the recursion (\cite{ams55:bes}, Eq.\,9.1.27)
\begin{equation}
\label{O5}f_{\nu +1}(x) - (2\nu /x) f_\nu (x) + f_{\nu - 1}(x) = 0
\end{equation}
For $\nu > x$ this recursion is stable in the forward direction for Y and in
the backward direction for J. For $x > \nu $ the recursion is stable in
either direction for both J and Y.

%\vspace{10pt}
% \hspace{.3in}\mbox{\input pl02-05a }\vspace{12pt}
%\centerline{$x$}
%\centerline{Figure 1. ~ $J_{\nu}(x)$}
%\vspace{15pt}

%\vspace{10pt}

% \hspace{.3in}\mbox{\input pl02-05b }\vspace{12pt}
%\centerline{$x$}
%\centerline{Figure 2. ~ $Y_{\nu}(x)$}\vspace{10pt}
%
\subsubsection{Machine dependent quantities}

Let $\rho $ denote the machine precision, $i.e.$, R1MACH(3) or
D1MACH(3) of Chapter~19.1. Let $\Omega $ denote the overflow limit, $i.e.$,
R1MACH(2) or D1MACH(2). Define
\begin{equation*}
\mathit{XPQ} = 1.1293(-\log _{10}(\rho /4)) - 0.59
\end{equation*}
The asymptotic series used in these subroutines is valid for $x \geq
\mathit{XPQ}$ and $0 \leq \nu \leq 2.$

Let $\nu ^*(x)$ denote the value of $\nu $ for which Eq.\,(4) reaches the overflow
limit, $\Omega $, for a given value of $x$. It happens that $\nu ^*(x)$ is very
close to the value of $\nu $ for which Eq.\,(3) reaches the underflow limit on the
same machine. The figure below shows plots of $\nu ^*(x)$ for some
computer systems currently in use at JPL.
\vspace{5pt}

\mbox{\input pl02-05c }

%\centerline{$x$}
%\centerline{Figure 3. ~Underflow limit for $J_{\nu}$; overflow limit
%for $Y_{\nu}$}\vspace{5pt}

\subsubsection{Computation of $J_\nu (x)$}

Given $x$, $\alpha $, and NUM, define $\beta = \alpha +\text{NUM} - 1$.
Thus, $\beta $ is the largest requested order.

For $x = 0$ the result is 1 if $\nu = 0$, and 0 if $\nu > 0.$

For $0 < x \leq 0.1$ the Taylor series in $x$ is used
(\cite{ams55:bes},Eq.\,9.1.10).  For $0.1 < x \leq \max (\beta ,
\mathit{XPQ})$ forward recursion on $\nu $ is used to determine a starting
point for backward recursion.  The execution time in this region increases
linearly with $\beta $ and can be substantial for large $\beta .$

For $\max (\beta , \mathit{XPQ}) < x < (16\rho )^{-1}$ the subroutine
evaluates the asymptotic series in $x$ (\cite{ams55:bes}, Eqs.\,9.2.5,
9.2.9, and~9.2.10) for two values of $\nu $ in the range [0,~2], and then
uses forward recursion.  The execution time in this region increases
linearly with $\beta $ and decreases with increasing $x.$

If $x > (16\rho )^{-1}$ an error message is issued
because the phase of the sine and cosine functions will not be known with
any accuracy.

\subsubsection{Computation of $Y_\nu (x)$}

If $x = 0$ an error message is issued since the result would be $-\infty $.
The output values are set to $-\Omega /2.$

For $0 < x \leq \rho $ and $\nu = 0$, the result is $(2/\pi ) (\gamma +\ln
(x/2))$ (\cite{ams55:bes}, Eq.\,9.1.13), where $\gamma $ denotes Euler's
constant, $0.57721\cdots$.  For $0 < x \leq \rho $ and $\nu > 0$, the result is
$-\pi ^{-1} \Gamma (\nu ) (x/2)^{-\nu }$ (\cite{ams55:bes}, Eq.\,9.1.9).

For $\rho < x < \mathit{XPQ}$ the subroutine first computes values of J. From
these values it computes Y for two values of $\nu $ in [0,~2], and then uses
forward recursion on $\nu $ to obtain the requested values.

For $\mathit{XPQ} \leq x \leq (16\rho )^{-1}$ the subroutine evaluates the
asymptotic series in $x$ for two values of $\nu $ in [0,~2], and then uses
forward recursion.

If $x > (16\rho )^{-1}$ an error message is issued as noted
previously for J.

\subsubsection{Accuracy tests}

The subroutines SBESJN and SBESYN were tested on an
IBM compatible PC using IEEE arithmetic by comparison with
the corresponding double precision subroutines.  Tables 1 and
2 give a summary of the errors found in these tests.  Each
number in a rectangular cell is the maximum value of the
error observed at 2592 points tested in the indicated range.
Each number in a triangular cell is the maximum over 1296
points.  The underflow limit for $J_{\nu}$, and the overflow
limit for $Y_{\nu}$, actually extend down the $\nu $ axis
(see Figure~3).  Where the function underflows or overflows,
fewer samples are used.

\begin{quote}Table 1. Maximum errors found in indicated regions for SBESJN.
Relative error is shown above the diagonal and absolute error below. Error
is shown as a multiple of the machine precision, $\approx 1.19
\times 10^{-7}$ for these tests.
\end{quote}\vspace{10pt}

\setlength{\unitlength}{1pt}
\begin{picture}(218, 160)(-36, -20)
\put(-30, 75){$\nu$}
\multiput(0, 0)(0, 25){7}{\line(1, 0){180}}
\put(0, 0){\line(0, 1){150}}
\put(30, 0){\line(0, 1){125}}
\multiput(60, 0)(30, 0){5}{\line(0, 1){150}}
\put(-2, -10){0}
\put(28, -10){2}
\put(58, -10){5}
\put(85, -10){10}
\put(115, -10){20}
\put(144, -10){50}
\put(170, -10){100}
\put(-24, -10){\makebox(20, 20)[r]{0}}
\put(-24, 15){\makebox(20, 20)[r]{2}}
\put(-24, 40){\makebox(20, 20)[r]{5}}
\put(-24, 65){\makebox(20, 20)[r]{10}}
\put(-24, 90){\makebox(20, 20)[r]{20}}
\put(-24, 115){\makebox(20, 20)[r]{50}}
\put(-24, 140){\makebox(20, 20)[r]{100}}
\put(0, 0){\line (6, 5){180}}
\put(90, -20){$x$}
\put(0, 125){\makebox(60, 25){\small OVERFLOW}}
\put(0, 100){\makebox(30, 25){17}}
\put(0, 75){\makebox(30, 25){27}}
\put(0, 50) {\makebox(30, 25){10}}
\put(0, 25) {\makebox(30, 25){4}}
\put(0, 0)    {\makebox(30, 25){~\raisebox{5pt}{2} \hfill
\raisebox{-5pt}{1}~}}
\put(30, 100){\makebox(30, 25){17}}
\put(30, 75){\makebox(30, 25){10}}
\put(30, 50) {\makebox(30, 25){7}}
\put(30, 25) {\makebox(30, 25){~\raisebox{5pt}{3} \hfill
\raisebox{-5pt}{1}~}}
\put(30, 0)    {\makebox(30, 25){1}}
\put(60, 125){\makebox(30, 25){18}}
\put(60, 100){\makebox(30, 25){24}}
\put(60, 75){\makebox(30, 25){10}}
\put(60, 50){\makebox(30, 25){~\raisebox{5pt}{5} \hfill
\raisebox{-5pt}{2}~}}
\put(60, 25) {\makebox(30, 25){1}}
\put(60, 0) {\makebox(30, 25){1}}
\put(90, 125){\makebox(30, 25){30}}
\put(90, 100){\makebox(30, 25){26}}
\put(90, 75){\makebox(30, 25){~\raisebox{5pt}{9} \hfill
\raisebox{-5pt}{4}~}}
\put(90, 50){\makebox(30, 25){2}}
\put(90, 25) {\makebox(30, 25){1}}
\put(90, 0) {\makebox(30, 25){1}}
\put(120, 125){\makebox(30, 25){44}}
\put(120, 100){\makebox(30, 25){~\raisebox{5pt}{21} \hfill
\raisebox{-5pt}{27}~}}
\put(120, 75){\makebox(30, 25){4}}
\put(120, 50){\makebox(30, 25){2}}
\put(120, 25) {\makebox(30, 25){2}}
\put(120, 0) {\makebox(30, 25){2}}
\put(150, 125){\makebox(30, 25){~\raisebox{5pt}{39} \hfill
\raisebox{-5pt}{96}~}}
\put(150, 100){\makebox(30, 25){16}}
\put(150, 75){\makebox(30, 25){3}}
\put(150, 50){\makebox(30, 25){3}}
\put(150, 25) {\makebox(30, 25){3}}
\put(150, 0) {\makebox(30, 25){3}}
\end{picture}

As a test of the double precision subroutines, and an additional test of the
single precision subroutines, the expression $z(\nu ,x)$ defined in Section
C was evaluated at 40 points. Nine values are shown in Table 3 from these
tests of SBESJN and SBESYN and in Table 4 from the tests of DBESJN and
DBESYN.

These subroutines are designed for use with arithmetic precision to about $%
10^{-20}$. The auxiliary subroutine DBESPQ has no inherent accuracy
limitations.

\begin{quote}Table 2. Maximum errors found in indicated regions for SBESYN.
Relative error is shown above the diagonal and absolute error below. Error
is shown as a multiple of the machine precision, $\approx 1.19
\times 10^{-7}$ for these tests.
\end{quote}\vspace{10pt}

\begin{picture}(208, 160)(-36, -20)
\put(-30, 75){$\nu$}
\multiput(0, 0)(0, 25){7}{\line(1, 0){180}}
\put(0, 0){\line(0, 1){150}}
\put(30, 0){\line(0, 1){125}}
\multiput(60, 0)(30, 0){5}{\line(0, 1){150}}
\put(-2, -10){0}
\put(28, -10){2}
\put(58, -10){5}
\put(85, -10){10}
\put(115, -10){20}
\put(144, -10){50}
\put(170, -10){100}
\put(-24, -10){\makebox(20, 20)[r]{0}}
\put(-24, 15){\makebox(20, 20)[r]{2}}
\put(-24, 40){\makebox(20, 20)[r]{5}}
\put(-24, 65){\makebox(20, 20)[r]{10}}
\put(-24, 90){\makebox(20, 20)[r]{20}}
\put(-24, 115){\makebox(20, 20)[r]{50}}
\put(-24, 140){\makebox(20, 20)[r]{100}}
\put(0, 0){\line (6, 5){180}}
\put(90, -20){$x$}
\put(0, 125){\makebox(60, 25){\small OVERFLOW}}
\put(0, 100){\makebox(30, 25){44}}
\put(0, 75){\makebox(30, 25){24}}
\put(0, 50) {\makebox(30, 25){8}}
\put(0, 25) {\makebox(30, 25){9}}
\put(0, 0)    {\makebox(30, 25){~\raisebox{5pt}{12} \hfill
\raisebox{-5pt}{28}~}}
\put(30, 100){\makebox(30, 25){84}}
\put(30, 75){\makebox(30, 25){25}}
\put(30, 50) {\makebox(30, 25){9}}
\put(30, 25) {\makebox(30, 25){~\raisebox{5pt}{9} \hfill
\raisebox{-5pt}{14}~}}
\put(30, 0)    {\makebox(30, 25){15}}
\put(60, 125){\makebox(30, 25){162}}
\put(60, 100){\makebox(30, 25){128}}
\put(60, 75){\makebox(30, 25){27}}
\put(60, 50){\makebox(30, 25){~\raisebox{5pt}{9} \hfill
\raisebox{-5pt}{9}~}}
\put(60, 25) {\makebox(30, 25){15}}
\put(60, 0) {\makebox(30, 25){10}}
\put(90, 125){\makebox(30, 25){286}}
\put(90, 100){\makebox(30, 25){143}}
\put(90, 75){\makebox(30, 25){~\raisebox{5pt}{29} \hfill
\raisebox{-5pt}{4}~}}
\put(90, 50){\makebox(30, 25){2}}
\put(90, 25) {\makebox(30, 25){1}}
\put(90, 0) {\makebox(30, 25){1}}
\put(120, 125){\makebox(30, 25){585}}
\put(120, 100){\makebox(30, 25){~\raisebox{7pt}{153}
\hspace{-4pt}\raisebox{-5pt}{19}\hspace{2pt}~}}
\put(120, 75){\makebox(30, 25){3}}
\put(120, 50){\makebox(30, 25){2}}
\put(120, 25) {\makebox(30, 25){2}}
\put(120, 0) {\makebox(30, 25){2}}
\put(150, 125){\makebox(30, 25){~\raisebox{7pt}{659}
\hspace{-4pt}\raisebox{-5pt}{68}\hspace{2pt}~}}
\put(150, 100){\makebox(30, 25){21}}
\put(150, 75){\makebox(30, 25){3}}
\put(150, 50){\makebox(30, 25){3}}
\put(150, 25) {\makebox(30, 25){3}}
\put(150, 0) {\makebox(30, 25){3}}
\end{picture}

\begin{center}
\centerline{Table 3. Single precision Wronskian test.}
\centerline{Tabulated value is $z(\nu ,x)/\text{R1MACH}(3)$}
\centerline{where R1MACH(3) $\approx 5.96 \times 10^{-8}.$}\vspace{-5pt}
\begin{tabular}{r|rrr}
\multicolumn{1}{c}{$\nu $} & $x \Rightarrow $ 5.1 & 15.3 & 30.6\\\hline
30.6 & 6.2 & 24.6 & 3.5\\
15.3 & 5.9 & 5.9 & 0.2\\
5.1 & 1.3 & 0.7 & 3.5\\
\end{tabular}\vspace{10pt}

\centerline{Table 4. Double precision Wronskian test.}
\centerline{Tabulated value is $z(\nu ,x)/\text{D1MACH}(3)$}
\centerline{where D1MACH(3) $\approx 1.11 \times 10^{-16}.$}\vspace{-5pt}
\begin{tabular}{r|rrr}
\multicolumn{1}{c}{$\nu $} & $x \Rightarrow $ 5.1 & 15.3 & 30.6\\\hline
30.6 & 32.2 & 17.2 & 16.0\\
15.3 & 3.9 & 5.8 & 0.4\\
5.1 & 1.0 & 1.8 & 0.1
\end{tabular}
\end{center}
\nocite{Olver:1972}
\nocite{Amos:1977:CSI}
\bibliography{math77}
\bibliographystyle{math77}

\subsection{Error Procedures and Restrictions}

These subroutines require $x \geq 0$, ALPHA $\geq 0$, and NUM $\geq 1$.
Violation of any of these conditions causes an error message and an
immediate return.

The subroutines attempt to anticipate and avoid overflow conditions.
Intermediate overflows are avoided by dynamic rescaling. If a final value of
Y would be beyond the overflow limit the value is set to $-\Omega /2$ and
an error message is issued. It is assumed that the host
system will set underflows to zero. No messages are issued for underflow.

If $x > (16\rho )^{-1}$ an error message is issued
since no accuracy can be obtained.

Subroutines SBESYN and DBESYN each contain an internal array AJ() to hold
values of $J_\nu (x)$ needed to compute $Y_\nu (x)$. The size requirement of
this array varies with the machine precision and is about $3(-\log _{10}
\rho ) + 3$. For example, for precisions of $10^{-10}$, $10^{-20}$,
and $10^{-30}$ the required size is 33, 63, and~95. The array is nominally
dimensioned~95 to handle all anticipated computers. An error message will be
issued in the unlikely event that a larger dimension is needed.

Error messages are issued by the error message processor of
Chapter~19.2.

The user should be aware that these subroutines require a substantial amount
of execution time, generally increasing linearly with the sum, ALPHA$+\text{%
NUM}.$

\subsection{Supporting Information}

The source language for these subroutines is ANSI Fortran 77.

Original subroutines SBJNU, SBYNU, DBJNU, DBYNU, BESJ, and BESY were
designed and programmed by W. V. Snyder and E. W. Ng, JPL, 1973, with
modifications by S.\ Singletary in 1974. The present subroutines are
modifications of the earlier subroutines to improve portability and
accuracy, avoid overflows, and conform to Fortran~77. These
subroutines were produced in 1984 by C. L. Lawson and S. Y. Chiu in
consultation with Snyder and Ng.


\begin{tabular}{@{\bf}l@{\hspace{5pt}}l}
\bf Entry & \hspace{.35in} {\bf Required Files}\vspace{2pt} \\
DBESJN & \parbox[t]{2.7in}{\hyphenpenalty10000 \raggedright
AMACH, DBESJN, DBESPQ, DERM1, DERV1, DGAMMA, ERFIN, ERMSG,
 IERV1\rule[-5pt]{0pt}{8pt}}\\
DBESYN & \parbox[t]{2.7in}{\hyphenpenalty10000 \raggedright
AMACH, DBESPQ, DBESYN, DERM1, DERV1, DGAMMA, DLGAMA, ERFIN,
ERMOR, ERMSG, IERV1\rule[-5pt]{0pt}{8pt}}\\
SBESJN & \parbox[t]{2.7in}{\hyphenpenalty10000 \raggedright
AMACH, ERFIN, ERMSG, IERV1, SBESJN, SBESPQ, SERM1, SERV1,
SGAMMA\rule[-5pt]{0pt}{8pt}}\\
SBESYN & \parbox[t]{2.7in}{\hyphenpenalty10000 \raggedright
AMACH, ERFIN, ERMOR, ERMSG, IERV1, SBESPQ, SBESYN, SERM1,
SERV1, SGAMMA, SLGAMA\rule[-5pt]{0pt}{8pt}}\\
\end{tabular}

\begcodenp
\lstset{language=[77]Fortran,showstringspaces=false}
\lstset{xleftmargin=.8in}

\centerline{\bf \large DRSBESJN}\vspace{10pt}
\lstinputlisting{\codeloc{sbesjn}}

\vspace{30pt}\centerline{\bf \large ODSBESJN}\vspace{10pt}
\lstset{language={}}
\lstinputlisting{\outputloc{sbesjn}}
\closegraphsfile
\end{document}
