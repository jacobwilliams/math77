\documentclass[twoside]{MATH77}
\usepackage{multicol}
\usepackage[fleqn,reqno,centertags]{amsmath}
\begin{document}
\begmath 5.4 Eigenvalues and Eigenvectors of an Unsymmetric Matrix

\silentfootnote{$^\copyright$1997 Calif. Inst. of Technology, \thisyear \ Math \`a la Carte, Inc.}

\subsection{Purpose}

Compute all eigenvalues and right eigenvectors of a real N $\times $ N unsymmetric
matrix $A$. Some or all of the eigenvalues and eigenvectors may be complex.

\subsection{Usage}

\subsubsection{Program Prototype, Single Precision}

\begin{description}
\item[REAL]  \ {\bf A}(LDA,$\geq $N)\ [LDA$\geq $N]{\bf , VR}($\geq $N){\bf %
, VI}($\geq $ N){\bf , VEC}(LDA,$\geq $N), WORK($\geq $N)

\item[INTEGER]  \ LDA, N, IFLAG($\geq $N)
\end{description}

Assign values to A(,), LDA, and N.

\begin{center}
\fbox{\begin{tabular}{@{\bf }c}
CALL SEVVUN(A, LDA, N, VR, VI,\\
VEC, IFLAG, WORK)\\
\end{tabular}}
\end{center}

Results are returned in VR(), VI(), VEC(,), and IFLAG(1). The contents of
A(,), IFLAG(), and WORK() will be modified.

\subsubsection{Argument Definitions}

\begin{description}
\item[A(,), LDA, N]  \ A(,) is [inout], LDA and N are [in]. On entry A(,) must
contain the N $\times $ N matrix $A$ whose eigenvalues and eigenvectors are to be
computed. The integer LDA is the dimension of the first subscript of the
arrays A(,) and VEC. Require LDA $\geq $ N. On return the contents
of A(,) will be modified.

\item[VR(), VI()]  \ [out] The subroutine will store the $J^{th}$ eigenvalue
in VR($J$) and VI($J$), $J$ = 1, ..., N. The real part is stored in VR($J$), and the
imaginary part in VI($J$). If the $J^{th}$ eigenvalue is real VI($J$) will be
zero. The eigenvalues will be sorted so that VR(1) $\leq $ VR(2) $\leq $ ...
$\leq$ VR(N), and if VR($J$) = VR($J$+1) for some $J$ then $|$VI$(J)|\leq
|$VI$(J+1)|.$

Complex eigenvalues will occur in conjugate pairs. Such pairs will be stored
in adjacent locations with the eigenvalue having positive imaginary part
preceding its conjugate partner.

\item[VEC(,)]  \ [out] The eigenvectors will be stored in this array. If the $%
J^{th}$ eigenvalue is real then the $J^{th}$ eigenvector will be real and
will be stored in column $J$ of VEC(,). It will be normalized to have unit
Euclidean length.

If the $J^{th}$ and $(J+1)^{st}$ eigenvalues are a complex conjugate pair,
then the $J^{th}$ eigenvector will be complex, say ${\bf u}+i{\bf v}$, and
the $(J+1)^{st}$ eigenvector will be its complex conjugate vector, ${\bf u}-i%
{\bf v}$. The subroutine will store ${\bf u}$ in column $J$ of VEC(,) and will
store ${\bf v}$ in column $J+1$ of VEC(,). The eigenvector ${\bf u}+i{\bf v}$
will be normalized to have unit unitary norm and real first component, $i.e
$. the first component of ${\bf v}$ will be zero.

\item[IFLAG()]  \ [out, scratch] The N-array IFLAG() will be used as INTEGER
working space. In addition, the first location, IFLAG(1), will be used to
pass information back to the user as follows:

\begin{itemize}
\item[= 1]  If successful and all eigenvalues are real.

\item[= 2]  If successful and some eigenvalues are complex.
\end{itemize}

See Section E for use of IFLAG(1) in error conditions.

\item[WORK()] \ [scratch] Working space.
\end{description}

\subsubsection{Modifications for Double Precision}

Change SEVVUN to DEVVUN, and the REAL type statement to DOUBLE PRECISION.

\subsection{Examples and Remarks}

The following unsymmetric matrix $A$ is
given on page~84 of~\cite{Gregory:1969:ACM}.
\begin{equation*}
A=\left[
\begin{array}{rrr}
8 & -1 & -5 \\
-4 & 4 & -2 \\
18 & -5 & -7
\end{array}
\right] .
\end{equation*}

The eigenvalues are 1, $2+4i$, and $2-4i$. The (unnormalized) right
eigenvectors are column vectors with the following triples of
elements:  (1, 2, 1), (1, $1+i$, $1-i$), and (1, $1-i$, $1+i$).  This
example illustrates the way SEVVUN returns complex eigenvalues and
eigenvectors in storage.

The demonstration program DRSEVVUN below applies SEVVUN to compute
eigenvalues and eigenvectors for the above matrices. The results are
in the file ODSEVVUN. Before the call to SEVVUN, the matrix is saved in
order to compute the relative residual matrix $D$ defined as%
\begin{equation*}
D=\left( AW-W\Lambda \right) /\gamma ,
\end{equation*}
where $A$ denotes the current test matrix, $W$ is the matrix whose columns are
the computed eigenvectors of $A$, $\Lambda $ is the diagonal matrix of
eigenvalues, and $\gamma $ is the maximum-row-sum norm of $A$. The
(possibly complex) matrix $D$ is packed into the array D(,) and printed.

\subsection{Functional Description.}

Given an N $\times $ N real unsymmetric matrix $A$ there exists an N $\times $ N nonsingular
matrix $C$ such that the matrix%
\begin{equation*}
U=C^{-1}AC
\end{equation*}
is N $\times $ N upper triangular. The matrices $C$ and $U$ may be complex. The
diagonal elements of $U$ are called the eigenvalues of $A$. This set of N
numbers is uniquely determined by $A$ although $C$ and $U$ are not unique. Note
that $\lambda $ is an eigenvalue of $A$ if and only if $A-\lambda I$ is
singular.

A nonzero vector ${\bf w}$ is a right eigenvector of $A$ associated with an
eigenvalue $\lambda $ if%
\begin{equation*}
A{\bf w}={\bf w}\lambda
\end{equation*}
If $A$ has N distinct eigenvalues then it will also have N linearly
independent eigenvectors. If the eigenvalues of $A$ are not all distinct then
an eigenvalue $\lambda $ of multiplicity $\mu $ may have any number of
linearly independent eigenvectors from~1 to $\mu $. If some multiple
eigenvalue of $A$ has fewer linearly independent eigenvectors than its
multiplicity the matrix is called defective.

If a set of computed eigenvalues returned by SEVVUN are equal or nearly equal,
it is not uncommon for the associated computed eigenvectors to be
linearly dependent, or nearly so.  This subroutine cannot be used
to distinguish between defective and nondefective matrices.

The subroutine SEVVUN was developed using the
subroutines BALANC, ELMHES, ELTRAN, HQR2, and BALBAK from the EISPACK
package of eigenvalue-eigenvector subroutines, \cite{Smith:1974:MER}. The Fortran
subroutines in EISPACK are based directly on the earlier set of Algol
procedures described in \cite{Wilkinson:1971:HAC}.

Subroutine SEVVUN first calls SEVBH which consists of the two EISPACK
subroutines BALANC and ELMHES.

BALANC applies similarity permutations to isolate eigenvalues available by
inspection, if any. Then, applies diagonal similarity scaling to balance the
size of the matrix elements%
\begin{equation*}
B=D^{-1}P^TAPD.
\end{equation*}
ELMHES reduces $B$ to upper Hessenberg form using stabilized elementary
transformations%
\begin{equation*}
H=G^{-1}BG.
\end{equation*}
The remainder of SEVVUN consists of a minor modification of three
EISPACK subroutines: ELTRN, HQR2, and BALBAK.

ELTRN computes explicitly the matrix $G$ that was stored in factored form by
SEVBH.

HQR2 applies the QR algorithm to $H$. This is an iterative process which
reduces $H$ to a real nearly-upper-triangular matrix $R$%
\begin{equation*}
R=Q^THQ.
\end{equation*}
The transformations applied to $H$ are also applied to $G$ forming%
\begin{equation*}
K=GQ.
\end{equation*}
The matrix $R$ has a mixture of single elements and $2\times 2$
blocks on its diagonal and is otherwise upper triangular.

The eigenvalues of $A$ are the single diagonal elements of $R$ along with the
eigenvalues of the $2 \times 2$ blocks on the diagonal of $R$. These latter
eigenvalues are computed by direct formulas.

The eigenvectors of $R$, say ${\bf z}_1$, ..., ${\bf z}_N$ are each computed
by a single back substitution process without any iteration. These
eigenvectors are transformed to eigenvectors of $B$, say ${\bf s}_1$, ..., $%
{\bf s}_N$ by computing%
\begin{equation*}
{\bf s}_j=K{\bf z}_j,\quad j=1,...,N.
\end{equation*}
BALBAK transforms the vectors $s_j$ to eigenvectors of $A$, say $w_j$, $j=1$,
..., N by computing%
\begin{equation*}
{\bf w}_j=PD{\bf s}_j,\quad j=1,...,N.
\end{equation*}
SEVVUN normalizes each eigenvector ${\bf w}_j$ to have unit unitary norm
and a real first component, and then reorders the eigenvalues along with
their associated eigenvectors, to achieve the ordering described in Section B.

\bibliography{math77}
\bibliographystyle{math77}

\subsection{Error Procedures and Restrictions}

If N $\leq $ 0 or if there is convergence failure in the QR algorithm the
error processing subroutine ERMSG of Chapter~19.2 will be called with
an error level of 0 to print an error message.  Upon return, IFLAG(1) = 3
or~4 to indicate N $\leq $ 0 or convergence failure, respectively.
In these error conditions all computed eigenvalues and eigenvectors should
be regarded as invalid.

If a set of computed eigenvalues are equal or nearly equal, the set of
associated computed eigenvectors will frequently not have as large a
numerical rank as would be possible for the given matrix.

\subsection{Supporting Information}

The source language is ANSI Fortran~77.

The EISPACK package of Fortran subroutines was acquired at JPL
from Argonne National Laboratories where it was developed with financial
support from the AEC and the NSF. The subroutine SEVVUN was written by F. T.
Krogh, JPL, October~1991.


\begin{tabular}{@{\bf}l@{\hspace{5pt}}l}
\bf Entry & \hspace{.35in} {\bf Required Files}\vspace{2pt} \\
DEVVUN & \parbox[t]{2.7in}{\hyphenpenalty10000 \raggedright
AMACH, DEVBH, DEVVUN, DNRM2, DSCAL, ERFIN, ERMSG\rule[-5pt]{0pt}{8pt}}\\
SEVVUN & \parbox[t]{2.7in}{\hyphenpenalty10000 \raggedright
AMACH, ERFIN, ERMSG, SEVBH, SEVVUN, SNRM2, SSCAL}\\
\end{tabular}

\begcode

\medskip\
\lstset{language=[77]Fortran,showstringspaces=false}
\lstset{xleftmargin=.8in}

\centerline{\bf \large DRSEVVUN}\vspace{10pt}
\lstinputlisting{\codeloc{sevvun}}

\vspace{30pt}\centerline{\bf \large ODSEVVUN}\vspace{10pt}
\lstset{language={}}
\lstinputlisting{\outputloc{sevvun}}
\end{document}
