\documentclass[twoside]{MATH77}
\usepackage[\graphtype]{mfpic}
\usepackage{multicol}
\usepackage[fleqn,reqno,centertags]{amsmath}
\begin{document}
\opengraphsfile{pl02-10}
\begmath 2.10 Exponential Integrals Ei and $E_1$

\silentfootnote{$^\copyright$1997 Calif. Inst. of Technology, \thisyear \ Math \`a la Carte, Inc.}

\subsection{Purpose}

These subroutines compute the exponential integrals Ei and $E_1$, defined
by%
\begin{equation*}
\text{Ei}(x)=\int_{-\infty }^x\frac{e^t}t\,dt,\quad \text{and\quad }%
E_1(x)=\int_x^\infty \frac{e^{-t}}t\,dt.
\end{equation*}

These functions are related by the equation
\begin{equation*}
\text{Ei}(x) = -E_1(-x)
\end{equation*}
The functions Ei$(x)$ for $x > 0$ and $E_1(x)$ for $x < 0$ are defined as
Cauchy principal value integrals. These functions thus have well-defined
finite values for all real $x$ except $x = 0$ where Ei$(0) = -\infty $ and $%
E_1(0) = +\infty .$

For additional properties of these functions see \cite{ams55:exp-int}.

\subsection{Usage}

\subsubsection{Program Prototype, Single Precision}

\begin{description}
\item[REAL]  \ {\bf X, Y, EI, SE1}
\end{description}

Assign a value to X and obtain the value of Ei or $E_1$ respectively by
use of the statements,
$$
\fbox{{\bf Y =SEI(X)}} \hspace{.4in} \fbox{{\bf Y =SE1(X)}}
$$

\subsubsection{Argument Definitions}

\begin{description}
\item[X]  \ [in] Argument of function. Require X $\neq 0.$
\end{description}

\subsubsection{Modifications for Double Precision}

For double precision usage change the REAL statement to DOUBLE PRECISION and
change the function names to DEI and DE1 respectively.

\subsection{Examples and Remarks}

See the program DRSEI and the output ODSEI for an example of the use of SEI
and SE1 to tabulate values of Ei and $E_1.$

\subsection{Functional Description}

As $x$ varies from $-\infty $ to~0, $E_1(x)$ varies monotonically from $%
-\infty $ to $+\infty $. There is a single real root
at~$-$0.37250~74107~81366~63446.

As $x$ varies from~0 to~$+\infty $, $E_1(x)$ varies monotonically from $%
+\infty $ to zero.

$E_1(x)$ is asymptotic to $x^{-1}e^{-x}$ as $x$ approaches $+\infty $ or $%
-\infty $, and to $-\ln |x|$\ as $x$ approaches zero.
\vspace{10pt}

\hspace{5pt}\mbox{\input pl02-10 }

Let $\mu $ and $\lambda $ be defined so that $e^\mu $ is the overflow limit
and $e^{-\lambda }$ is the underflow limit for the machine arithmetic.
Define $\alpha = \mu + \ln \mu $ and $\beta = \lambda - \ln \lambda $. Then $%
E_1(x)$ would overflow for $x < -\alpha $ and underflow for $x > \beta .$

This algorithm, due to L. W. Fullerton, with minor changes by Lawson and
Chiu, partitions the interval $[-\alpha ,\beta ]$ into eight subintervals.
On each subinterval a polynomial approximation is used.

The polynomial degrees and the numbers $\alpha $ and $\beta $ are determined
on the first entry to the subprogram by use of the System Parameters
subprograms (see Chapter~19.1). The subprograms adapt to any precision up to
about~31 decimal places.

\subparagraph{Accuracy tests}

Subprogram SE1 was tested on an IBM compatible PC using IEEE
arithmetic by comparison with DE1 at~50,000
points between~$-$80 and 80. The relative precision of the IEEE single
precision arithmetic is $\rho = 2^{-23} \approx 1.19 \times 10^{-7}$. The
test results may be summarized as follows:

\begin{tabular}{l@{}r@{{.}}l@{{, }}r@{{.}}lr}
\multicolumn{5}{c}{\bf Argument Interval} &
\multicolumn{1}{c}{\bf Max. Rel. Error}\\
\rule{.2in}{0pt} [ & $-$80&00  &$-$1&20] & $2.5\rho $\rule{.4in}{0pt} \\
\rule{.2in}{0pt} [ &  $-$1&20  &$-$1&00] & $4.6\rho $\rule{.4in}{0pt} \\
\rule{.2in}{0pt} [ &  $-$1&00  &$-$0&41] & $0.9\rho $\rule{.4in}{0pt} \\
\rule{.2in}{0pt} [ &  $-$0&41 &$-$0&30] & (see just below)~~\\
\rule{.2in}{0pt} [ &  $-$0&30 &  80&00] & $0.8\rho $\rule{.4in}{0pt}
\end{tabular}

The relative error in the interval [$-$0.41, $-$0.30] is large due to the
root near $-$0.3725. However, $|E_1(x)|$ is bounded by~0.31 and the absolute
error has a satisfactorily small bound of $0.22\rho $ in this interval.

\bibliography{math77}
\bibliographystyle{math77}

\subsection{Error Procedures and Restrictions}

In the following cases the function value would be beyond the representable
range. The subprograms will issue an error message and return a value as
follows ($\Omega $ is the overflow limit):
\begin{center}
\begin{tabular}{lr|lr}
\multicolumn{1}{c}{X} & \multicolumn{1}{c}{SEI(X)\rule{10pt}{0pt}} &
\multicolumn{1}{|c}{\rule{10pt}{0pt}X} & \multicolumn{1}{c}{SE1(X)}\\
$< -\beta $            & 0 \rule{20pt}{0pt}                    &
\rule{10pt}{0pt}$< -\alpha $           & $-\Omega $\rule{13pt}{0pt}\\
$= \phantom{-}0.$      & $-\Omega $ \rule{20pt}{0pt}           &
\rule{10pt}{0pt}$= \phantom{-}0.$      & $\Omega $\rule{13pt}{0pt}\\
$> \phantom{-}\alpha $ & $\phantom{-}\Omega $ \rule{20pt}{0pt} &
\rule{10pt}{0pt}$> \phantom{-} \beta $ & 0\rule{13pt}{0pt}\\
\end{tabular}
\end{center}
Error messages are processed using the subroutines of Chapter~19.2 with
an error level of zero.

\subsection{Supporting Information}

The source language is Fortran~77.

Based on code designed and programmed by L.\ W.\ Fullerton, Los Alamos
National Lab., 1977. Modified by C. L.\ Lawson and S.\ Y. Chiu, JPL, 1983.


\begin{tabular}{@{\bf}l@{\hspace{5pt}}l}
\bf Entry & \hspace{.35in} {\bf Required Files}\vspace{2pt} \\
DE1 & \parbox[t]{2.7in}{\hyphenpenalty10000 \raggedright
AMACH, DCSEVL, DEI, DERM1, DERV1, DINITS, ERFIN, ERMSG, IERM1,
IERV1\rule[-5pt]{0pt}{8pt}}\\
DEI & \parbox[t]{2.7in}{\hyphenpenalty10000 \raggedright
AMACH, DCSEVL, DEI, DERM1, DERV1, DINITS, ERFIN, ERMSG, IERM1,
IERV1\rule[-5pt]{0pt}{8pt}}\\
SE1 & \parbox[t]{2.7in}{\hyphenpenalty10000 \raggedright
AMACH, ERFIN, ERMSG, IERM1, IERV1, SCSEVL, SEI, SERM1, SERV1,
SINITS\rule[-5pt]{0pt}{8pt}}\\
SEI & \parbox[t]{2.7in}{\hyphenpenalty10000 \raggedright
AMACH, ERFIN, ERMSG, IERM1, IERV1, SCSEVL, SEI, SERM1, SERV1
SINITS\rule[-5pt]{0pt}{8pt}}\\
\end{tabular}

\begcode

\bigskip

\lstset{language=[77]Fortran,showstringspaces=false}
\lstset{xleftmargin=.8in}

\centerline{\bf \large DRSEI}\vspace{10pt}
\lstinputlisting{\codeloc{sei}}

\newpage

\vspace{30pt}\centerline{\bf \large ODSEI}\vspace{10pt}
\lstset{language={}}
\lstinputlisting{\outputloc{sei}}
\closegraphsfile
\end{document}
