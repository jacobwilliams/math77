\documentclass[twoside]{MATH77}
\usepackage{multicol}
\usepackage[fleqn,reqno,centertags]{amsmath}
\begin{document}
\hyphenation{DRDHERQL DRSHERQL ODSHERQL}
\begmath 5.2 Eigenvalues and Eigenvectors of a Hermitian Complex Matrix

\silentfootnote{$^\copyright$1997 Calif. Inst. of Technology, \thisyear \ Math \`a la Carte, Inc.}

\subsection{Purpose}

Compute the N eigenvalues and right eigenvectors of an N $\times $ N
complex Hermitian matrix $A$. A complex matrix is Hermitian if its
diagonal elements are real and its off-diagonal pairs $a_{i,j}$ and
$a_{j,i}$ are complex conjugates of each other. Such a matrix will
have real eigenvalues. The eigenvectors will in general be complex
but can be chosen so that the matrix $V$ of N eigenvectors is
unitary, $i.e.$, the conjugate transpose of $V$ is the inverse of
$V$.

\subsection{Usage}

\subsubsection{Program Prototype}

\begin{description}
\item[REAL]  \ {\bf AR}(LDA,$\geq $N){\bf , AI}(LDA,$\geq $N) [LDA$\geq $N]
{\bf ,\newline
EVAL}($\geq $N)

\item[REAL]  \ {\bf VR}(LDA,$\geq $N){\bf , VI}(LDA,$\geq $N){\bf , WORK}($%
\geq $3N)

\item[INTEGER]  \ {\bf LDA, N, IERR}
\end{description}

Assign values to AR(,), AI(,), LDA, and N.

\begin{center}
\fbox{\begin{tabular}{@{\bf }c}
CALL SHERQL(AR, AI, LDA, N,\\
EVAL, VR, VI, WORK, IERR)\\
\end{tabular}}
\end{center}

Results are returned in EVAL(), VR(,), VI(,), and IERR.

\subsubsection{Argument Definitions}

\begin{description}
\item[AR(,), AI(,)]  \ [inout] On entry the locations on and below the
diagonal of these arrays must contain the lower-triangular elements of the
N $\times $ N complex Hermitian matrix $A$ with the real part in AR(,) and
the imaginary part in AI(,). On return AR(,) and AI(,) contain information
about the unitary transformations used in the reduction of $A$ in their lower
triangle. The strict upper triangles of AR(,) and AI(,), and the diagonal of
AR(,) are unaltered.

\item[LDA]  \ [in] Dimension of the first subscript of the arrays AR(,),
AI(,), VR(,), and VI(,). Require LDA $\geq $ N.

\item[N]  \ [in] Order of the complex Hermitian matrix $A$. N $\geq 1.$

\item[EVAL()]  \ [out] Array in which the N real eigenvalues of $A$ will be
stored by the subroutine. The eigenvalues will be sorted with the
algebraically smallest eigenvalues first.

\item[VR(,), VI(,)]  \ [out] On return contains the real and imaginary parts
of the eigenvectors with column $k$ corresponding to the eigenvalue in EVAL$%
(k)$. These N eigenvectors will be mutually orthogonal and will have a
unit unitary norm.

\item[WORK()]  \ [scratch] An array of at least 3N locations used as
temporary space.

\item[IERR]  \ [out] On exit this is set to~0 if the QL algorithm converges,
otherwise see Section E.
\end{description}

\subsubsection{Modifications for Double Precision}

Change SHERQL to DHERQL, and the REAL type statement to DOUBLE PRECISION.

\subsection{Examples and Remarks}

Consider the following complex Hermitian matrix:
\begin{equation*}
A=\left[
\begin{array}{cccc}
\phantom{-}25 & -3-4i & -8+6i & \phantom{-}0 \\
-3+4i & \phantom{-}25 & \phantom{-}0 & -8-6i \\
-8-6i & \phantom{-}0 & \phantom{-}25 & \phantom{-}3-4i \\
\phantom{-}0 & -8+6i & \phantom{-}3+4i & \phantom{-}25
\end{array}
\right]
\end{equation*}
unit unitary norm associated with these eigenvalues are column
vectors with the following quadruples of elements:
($0.5i$, $0.4+0.3i$, $-0.3+0.4i$, 0.5),
($-0.5i$, $0.4+0.3i$, $0.3-0.4i$, 0.5),
($-0.5i$, $-0.4-0.3i$, $-0.3+0.4i$, 0.5),  and
($0.5i$, $-0.4-0.3i$, $0.3-0.4i$, 0.5), respectively.

The code in DRSHERQL, given below, computes the eigenvalues and eigenvectors
of this matrix. Output from this program is given in the file ODSHERQL.

Before the call to SHERQL, the matrix is saved in order to compute the
relative residual matrix $D$ defined as%
\begin{equation*}
C=\left( AW-W\Lambda \right) /\gamma
\end{equation*}
where $W$ is the matrix whose columns are the computed eigenvectors
of $A$, $\Lambda $ is the diagonal matrix of eigenvalues, and $\gamma $ is the
maximum-row-sum norm of $A$.

Recall that if ${\bf v}$ is an eigenvector, then so is $\alpha{\bf
v}$ for any nonzero complex scalar $\alpha$.  More generally, if an
eigenvalue, $\lambda$, of a complex Hermitian matrix occurs with
multiplicity $k$, there will be an associated  $k$-dimensional
complex subspace in which every vector is an eigenvector for
$\lambda$.  This subroutine will return eigenvectors constituting an
orthogonal basis for such an eigenspace.

\subsection{Functional Description}

Householder complex unitary similarity transformations are used to transform
the matrix $A$ to a Hermitian tridiagonal matrix. Additional unitary
similarity transformations are used to transform the matrix to a real
tridiagonal matrix. From this point, this subroutine uses the same method as
the subroutine SSYMQR of Chapter~5.1. As was the case there, all routines
are minor modifications of EISPACK routines, \cite{Smith:1974:MER}.

\bibliography{math77}
\bibliographystyle{math77}

\subsection{Error Procedures and Restrictions}

If the QL algorithm fails to converge in 30~iterations on the $J^{th}$
eigenvalue the subroutine sets IERR = $J$. In this case $J-1$ eigenvalues
are computed correctly but the eigenvalues are not ordered, and the
eigenvectors are not computed. If N $\leq $ 0 on entry, IERR is set to $-$1.
In either case an error message is printed using IERM1 of Chapter 19.2
with an error level of 0, before the return.

\subsection{Supporting Information}

The source language is ANSI Fortran~77.

\begin{tabular}{@{\bf}l@{\hspace{5pt}}l}
\bf Entry & \hspace{.35in} {\bf Required Files}\vspace{2pt} \\
DHERQL & \parbox[t]{2.7in}{\hyphenpenalty10000 \raggedright
DHERQL, DIMQL, ERFIN, ERMSG, IERM1, IERV1\rule[-5pt]{0pt}{8pt}}\\
SHERQL & \parbox[t]{2.7in}{\hyphenpenalty10000 \raggedright
ERFIN, ERMSG, IERM1, IERV1, SHERQL, SIMQL}\\
\end{tabular}

Converted by: F. T. Krogh, JPL, October~1991.


\begcodenp

\lstset{language=[77]Fortran,showstringspaces=false}
\lstset{xleftmargin=.8in}

\centerline{\bf \large DRSHERQL}\vspace{10pt}
\lstinputlisting{\codeloc{sherql}}

\vspace{30pt}\centerline{\bf \large ODSHERQL}\vspace{10pt}
\lstset{language={}}
\lstinputlisting{\outputloc{sherql}}
\end{document}
