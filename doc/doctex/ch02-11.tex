\documentclass[twoside]{MATH77}
\usepackage{multicol}
\usepackage[fleqn,reqno,centertags]{amsmath}
\begin{document}
\begmath 2.11 Finite Legendre Series

\silentfootnote{$^\copyright$1997 Calif. Inst. of Technology, \thisyear \ Math \`a la Carte, Inc.}

\subsection{Purpose}

This subroutine computes the value of a finite sum of Legendre polynomials,
\begin{equation*}
y=\sum_{j=0}^{\text{N}}a_jP_j(x)
\end{equation*}
for a specified summation limit, N, argument, $x$, and sequence of
coefficients, $a_j$. The Legendre polynomials are defined in \cite{ams55:or-poly}.

\subsection{Usage}

\subsubsection{Program Prototype, Single Precision}

\begin{description}
\item[INTEGER]  \ {\bf N}

\item[REAL]  \ {\bf X, Y, A}($0:m\geq $ N)
\end{description}

Assign values to X, N, and A(0), A(1), ... A(N).
$$
\fbox{{\bf CALL SLESUM (S, N, A, Y)}}
$$
The sum will be stored in Y.

\subsubsection{Argument Definitions}

\begin{description}
\item[X]  \ [in] Argument of the polynomials.

\item[N]  \ [in] Highest degree of polynomials in sum.

\item[A()]  \ [in] The coefficients must be given in A(J), J = 0, ..., N.

\item[Y]  \ [out] Computed value of the sum.
\end{description}

\subsubsection{Modifications for Double Precision}

For double precision usage, change the REAL statement to DOUBLE PRECISION
and change the subroutine name from SLESUM to DLESUM.

\subsection{Examples and Remarks}

See DRSLESUM and ODSLESUM for an example of the usage of SLESUM. DRSLESUM
evaluates the following identity, the coefficients of which were obtained
from Table~22.9, page~798, of~\cite{ams55:or-poly}.
\begin{equation*}
z = y - w = 0,
\end{equation*}
where
\begin{multline*}
y = 0.07P_0(x) + 0.27P_1(x) + 0.20P_2(x)\\
+ 0.28P_3(x) + 0.08P_4(x) + 0.08P_5(x),
\end{multline*}
and
\begin{equation*}
w = 0.35x^4 + 0.63x^5.
\end{equation*}

\subsection{Functional Description}

The sum is evaluated by the following algorithm:
\begin{gather*}
b_{N+2}=0,\quad b_{N+1}=0,\\
b_k=\frac{2k+1}{k+1}b_{k+1}x-\frac{k+1}{k+2}b_{k+2}+a_k,\ \ k=N,...,0,\\
y=b_0.
\end{gather*}
For an error analysis applying to this algorithm see \cite{Ng:1968:DSS} and
\cite{Ng:1971:RAC}. The first four Legendre polynomials are \begin{gather*}
P_0(x) = 1,\quad P_1(x) = x,\\
P_2(x) = 1.5x^2 - 0.5,\quad P_3(x) = 2.5x^3 - 1.5x.
\end{gather*}
For $k \geq 2$ the Legendre polynomials satisfy the recurrence
\begin{equation*}
kP_k(x) = (2k-1)xP_{k-1}(x) - (k-1)P_{k-2}(x).
\end{equation*}

The Legendre polynomials are orthogonal relative to integration over the
interval [$-$1,~1] and are normally used only with an argument, $x$, in this
interval.

\bibliography{math77}
\bibliographystyle{math77}

\subsection{Error Procedures and Restrictions}

The subroutine will return Y = 0 if N $< 0$. It is recommended that $x$
satisfy $|x| \leq 1.$

\subsection{Supporting Information}

The source language is ANSI Fortran

\begin{tabular}{@{\bf}l@{\hspace{5pt}}l}
\bf Entry & \hspace{.2in} {\bf Required Files}\vspace{2pt} \\
DLESUM & \hspace{.35in}DLESUM\rule[-5pt]{0pt}{8pt}\\
SLESUM & \hspace{.35in}SLESUM\\
\end{tabular}

Based on a 1974 program by E. W. Ng, JPL. Present version by C. L. Lawson
and S. Y. Chiu, JPL, 1983.


\begcodenp

\medskip

\lstset{language=[77]Fortran,showstringspaces=false}
\lstset{xleftmargin=.8in}

\centerline{\bf \large DRSLESUM}\vspace{10pt}
\lstinputlisting{\codeloc{slesum}}

\vspace{30pt}\centerline{\bf \large ODSLESUM}\vspace{10pt}
\lstset{language={}}
\lstinputlisting{\outputloc{slesum}}

\end{document}
