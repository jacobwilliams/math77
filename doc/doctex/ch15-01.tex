\documentclass[twoside]{MATH77}
\usepackage{multicol}
\usepackage[fleqn,reqno,centertags]{amsmath}
\begin{document}
\hyphenation{ISTATS}
\begmath 15.1 Basic Statistics and Histogram

\silentfootnote{$^\copyright$1997 Calif. Inst. of Technology, \thisyear \ Math \`a la Carte, Inc.}

\subsection{Purpose}

Computes the count, minimum, maximum, mean, standard deviation, and histogram
for a one-dimensional array of numbers. Can be called multiple times to update
these statistics to include additional data. Subroutines are provided to
display the results on a screen or printer. Versions are provided for REAL,
DOUBLE PRECISION, and INTEGER data.

\subsection{Usage}

\subsubsection{Computation of statistics, REAL data}

\paragraph{Program Prototype}

\begin{description}
\item[INTEGER]  \ {\bf NX, IHIST}($\geq $NCELLS){\bf , NCELLS}

\item[REAL]  \ {\bf XTAB}($\geq $NX){\bf , STATS}(5){\bf , X1, X2}
\end{description}

Assign values to XTAB(), NX, NCELLS, X1, X2, and on the first call of
a set, set STATS(1) = 0.0.

\begin{center}
\fbox{%
\begin{tabular}{@{\bf }c@{}}
CALL SSTAT1(XTAB, NX, STATS,\\
IHIST, NCELLS, X1, X2)\\
\end{tabular}}
\end{center}

New or updated quantities are returned in STATS() and IHIST().

\paragraph{Argument Definitions}

\begin{description}
\item[XTAB()]  \ [in] Array of NX values of $x$ whose statistics are to be
computed.

\item[NX]  \ [in] Number of values given in XTAB(). If NX $\leq $ 0, the
subroutine returns taking no action.

\item[STATS()]  \ [inout] Array of length~5 into which statistics are or
will be stored. On entry STATS(1) must be zero or positive.

If STATS$(1)=0$, the subroutine assumes there are no prior results in
STATS() and IHIST(). Results will be computed just for the data given in
XTAB().

If STATS$(1)>0$, the subroutine assumes there are prior results in STATS()
and IHIST(). Results will be updated to include the data given in XTAB().

The statistics stored in STATS() are:

\begin{tabbing}
\hspace{.5in}\= STATS(1) \== $count$\\
\>STATS(2) \>= $minimum$\\
\>STATS(3) \>= $maximum$\\
\>STATS(4) \>= $mean$\\
\>STATS(5) \>= $standard\ deviation$
\end{tabbing}

\item[IHIST()]  \ [inout] Integer array of length at least NCELLS, to hold
histogram results. Interpretation of contents on entry depends on the value
of STATS(1).

Used to store counts of occurrences of $x$-values in the different
classification intervals. Values less than X1 are counted in IHIST(1).
Values greater than X2 are counted in IHIST(NCELLS). The interval from X1 to
X2 is divided into NCELLS $-$ 2 equal-length subintervals, for which the
counts are stored in IHIST(2 : NCELLS $-$ 1).  Each of these subintervals
includes its left endpoint but not its right endpoint, with the exception that
the subinterval counted in IHIST(NCELLS $-$ 1) includes both of its endpoints.

\item[NCELLS]  \ [in] Number of elements to be used in the array IHIST().
Require NCELLS $\geq 3.$

\item[X1, X2]  \ [in] Lower and upper boundaries, respectively, defining the
range of $x$ values to be classified into NCELLS $-$ 2 equal intervals. Require
X1 $<$ X2.
\end{description}

\subsubsection{Display of statistics, REAL data}

\paragraph{Program Prototype}

All arguments should be declared as for SSTAT1 and should contain values
resulting from a previous call to SSTAT1.

\begin{center}
\fbox{%
\begin{tabular}{@{\bf }c@{}}
CALL SSTAT2( STATS, IHIST,\\
NCELLS, X1, X2)\\
\end{tabular}}
\end{center}

This subroutine writes to the standard system output the
contents of STATS() with labels, and a representation of the
histogram specified by the contents of IHIST(), NCELLS, X1, and X2.
The output is produced using statements PRINT and WRITE(*,~...).

\subsubsection{Computation of statistics, INTEGER data, REAL results}

\paragraph{Program Prototype}

\begin{description}
\item[INTEGER]  \ {\bf NI, ILOW, NCELLS}{\bf , ITAB}($\geq $NI)

\item[INTEGER]  \ {\bf ISTATS}(3){\bf , IHIST}($\geq $NCELLS)

\item[REAL]  \ {\bf XSTATS}(2)
\end{description}

Assign values to ITAB(), NI, ILOW, NCELLS, and on the first call of
a set, set ISTATS(1) = 0.

\begin{center}
\fbox{%
\begin{tabular}{@{\bf }c@{}}
CALL ISSTA1(ITAB, NI, ISTATS, XSTATS,\\
IHIST, ILOW, NCELLS)\\
\end{tabular}}
\end{center}

New or updated quantities are returned in ISTATS(), XSTATS(), and IHIST().

\paragraph{Argument Definitions}

\begin{description}
\item[ITAB()]  \ [in] Array of NI integer values whose statistics are to be
computed.

\item[NI]  \ [in] Number of values given in ITAB(). If NI $\leq $ 0, the
subroutine returns taking no action.

\item[ISTATS()]  \ [inout] Array of length~3 into which statistics are or
will be stored. On entry ISTATS(1) must be zero or positive.

If ISTATS$(1)=0$, the subroutine assumes there are no prior results in
ISTATS(), XSTATS(), and IHIST(). Results will be computed just for the data
given in ITAB().

If ISTATS$(1)>0$, the subroutine assumes there are prior results in
ISTATS(), XSTATS(), and IHIST(). Results will be updated to include the data
given in ITAB().

The statistics stored in ISTATS() are:

\begin{tabbing}
\hspace{.5in}\=ISTATS(1) \== $count$\\
\>ISTATS(2)\>= $minimum$\\
\>ISTATS(3)\>= $maximum$
\end{tabbing}

\item[XSTATS()]  \ [inout] Array of length~2 into which statistics are or
will be stored. Interpretation of contents on entry depends on the value of
ISTATS(1). The statistics stored in XSTATS() are:

\begin{tabbing}
\hspace{.5in}\=XSTATS(1) \== $mean$\\
\>XSTATS(2)\>= $standard\ deviation$
\end{tabbing}

\item[IHIST()]  \ [inout] Integer array of length at least NCELLS, to hold
histogram results. Interpretation of contents on entry depends on the value
of ISTATS(1).

Used to store counts of occurrences of integer data according to their
values. Values less than ILOW are counted in IHIST(1). Values greater than
ILOW + NCELLS $-$ 3 are counted in IHIST(NCELLS). The occurrences of the
NCELLS $-$ 2 consecutive integer values ILOW, ILOW $+1$, ..., ILOW + NCELLS $%
-$ 3, are counted in the cells IHIST(2 : NCELLS $-$ 1),
respectively.

\item[ILOW]  \ [in] Specifies the integer value whose occurrences are to be
counted in IHIST(2).

\item[NCELLS]  \ [in] Number of elements to be used in the array IHIST().
Require NCELLS $\geq 3.$
\end{description}

\subsubsection{Display of statistics, INTEGER data, REAL results}

\paragraph{Program Prototype}

All arguments should be declared as for ISSTA1 and should contain values
resulting from a previous call to ISSTA1.

\begin{center}
\fbox{%
\begin{tabular}{@{\bf }c@{}}
CALL ISSTA2(ISTATS, XSTATS,\\
IHIST, ILOW, NCELLS)\\
\end{tabular}}
\end{center}

This subroutine writes to the standard system output the
contents of ISTATS() and XSTATS() with labels, and a representation
of the histogram specified by the contents of IHIST(), ILOW, and
NCELLS.  The output is produced using statements PRINT and WRITE(*,~...).

\subsubsection{Modifications for DOUBLE PRECISION data and results}

For double-precision usage change the names SSTAT1, SSTAT2, ISSTA1
and ISSTA2, to DSTAT1, DSTAT2, IDSTA1, and IDSTA2, and change the
REAL declarations to DOUBLE PRECISION.

\subsection{Examples and Remarks}

See demonstration drivers and sample output in Chapter~3.3 for examples of
the use of these subroutines.

Note that to compute statistics for a set of $n$ REAL numbers one could call
SSTAT1 once giving the whole set of $n$ numbers, or call SSTAT1 $n$ times
giving just one new datum at each call, or use any strategy of grouping
the data intermediate between these two extremes. Giving all data on one
call will be most efficient and will introduce slightly less round off
error. The SQRT function is referenced once on each call.

In ISSTA1 the histogram records a separate count for each distinct integer
value from ILOW through ILOW + NCELLS $-$ 3. If this is a finer resolution
than one wants, the data should be converted from INTEGER to REAL or DOUBLE
PRECISION, and one should use SSTAT1 or DSTAT1 to let each counting cell
cover a range of values of the data.

\subsection{Functional Description}

\subsubsection{Method}

The sample mean, $\mu _n$, and standard deviation, $\sigma _n$, of a set of
numbers, $x_i$, $i=1$, ..., $n$, are defined by%
\begin{equation*}
\mu _n=\frac 1n\sum_{i=1}^nx_i,\quad \text{and\quad }\sigma _n^2=\frac
1{n-1}\sum_{i=1}^n\left( x_i-\mu _n\right) ^2.
\end{equation*}
To discuss recursive computation of these quantities it is useful to define
the auxiliary quantity%
\begin{equation*}
S_n=\sum_{i=1}^n\left( x_i-\mu _n\right) ^2.
\end{equation*}
The quantities $\mu _n$ and $S_n$ satisfy the recursive relations
\begin{align}
\label{O1}\mu _n&=\mu _{n-1}+\frac{\delta _n}n,\quad \text{and}\\
\label{O2}S_n&=S_{n-1}+\frac{n-1}n\delta _n^2,\quad \text{where}\\
\delta _n&=x_n-\mu _{n-1}.\notag
\end{align}
Formulas (1) and~(2) apply for $n\geq 2$ after initiating the sequences by setting
$\mu _1=x_1$ and $S_1=0.$

Formulas (1) and~(2) are satisfactory from a numerical accuracy point of view,
however, since $S_n$ is about the magnitude of the square of $\sigma _n$
there is a remote possibility that $S_n$ could be outside the exponent range
of a computer's floating-point arithmetic even when $\sigma _n$ is within
range.

The range of data that can be processed without encountering this problem
can be substantially increased by introducing scaling. Let $c_n$, $n=2$, 3,
..., be an as yet unspecified sequence of positive numbers and define $%
\delta _n^{^{\prime }}=\delta _n/c_n$ and $S_n^{^{\prime }}=S_n/c_n^2$. Then
from Eq.\,(2) we may write
\begin{equation}
\label{O3}S_n^{\prime }=\left[ \frac{c_{n-1}}{c_n}\right] ^2S_{n-1}^{\prime
}+\frac{n-1}n\delta _n^{\prime 2}
\end{equation}
Let $k$ denote the first index for which $\delta _i\neq 0$. No scaling is
needed for $i<k$, but formally we may set $c_i=1$ for $i<k$. Define $%
c_k=|\delta _k|$. A satisfactory scaling would be provided by
setting $c_i=\max (c_{i-1}$, $|\delta _i|)$ for $i>k$. This would
assure that $|\delta _i^{^{\prime }}|\leq 1$ and $1/2\leq
S_i^{^{\prime }}\leq i$ for all $i\geq k$. Note, however, that when $%
c_i=c_{i-1}$ it is not necessary to compute and apply the factor $(c_{i-1}/c_i)
$. To reduce substantially the number of times the scale factor is changed
we choose to leave it unchanged, $i.e.$, set $c_i=c_{i-1}$, until some $%
\delta _i$ satisfies $|\delta _i|>64c_{i-1}$. At that point the
scale factor will be changed to $c_i=|\delta _i|.$

The standard deviation is recovered from $S_n^{^{\prime }}$ by%
\begin{equation*}
\sigma _n=c_n\left[ \frac{S_n^{\prime }}{n-1}\right] ^{1/2}.
\end{equation*}
Subroutines SSTAT1 and DSTAT1 use Eqs.\,(1) and (3) and the scaling technique
just described. Subroutine ISSTA1 uses the simpler Eqs.\,(1) and (2) since
the integer data cannot reach such extremes of magnitude.

\subsubsection{Accuracy tests}

These subroutines have been tested for correctness against simpler
algorithms. SSTAT1 and DSTAT1 have been tested successfully using very large
and very small data values that would have led to failure on overflow or
underflow without the scaling.

\subsection{Error Procedures and Restrictions}

For useful processing one must have NX $\geq 1$, STATS$(1) \geq 0$, and
NCELLS\ $\geq 3$ on entry to SSTAT1 or DSTAT1, and NI $\geq 1$, ISTATS$(1)
\geq 0$, and NCELLS $\geq 3$ on entry to ISSTA1.

If NX $\leq $ 0 or NI $\leq $ 0, the entered subroutine will return
immediately, doing no processing. If STATS$(1) < 0$, ISTATS$(1) < 0$, or
NCELLS $< 3$, the results will be unpredictable.

When calling any of these subroutines to update previous statistics with new
data, the contents of all arguments must remain unchanged from a previous
call except for XTAB() and NX in SSTAT1 or DSTAT1, and ITAB() and NI in
ISSTA1.

\subsection{Supporting Information}

The source language is ANSI Fortran~77.

\begin{tabular}{@{\bf}l@{\hspace{5pt}}l}
\bf Entry & \hspace{.2in} {\bf Required Files}\vspace{2pt}\\
DSTAT1 & \hspace{.35in} DSTAT1\rule[-5pt]{0pt}{8pt}\\
DSTAT2 & \hspace{.35in} DPRPL, DSTAT2\rule[-5pt]{0pt}{8pt}\\
IDSTA1 & \hspace{.35in} IDSTA1\rule[-5pt]{0pt}{8pt}\\
IDSTA2 & \hspace{.35in} DPRPL, IDSTA2\rule[-5pt]{0pt}{8pt}\\
ISSTA1 & \hspace{.35in} ISSTA1\rule[-5pt]{0pt}{8pt}\\
ISSTA2 & \hspace{.35in} ISSTA2, SPRPL\rule[-5pt]{0pt}{8pt}\\
SSTAT1 & \hspace{.35in} SSTAT1\rule[-5pt]{0pt}{8pt}\\
SSTAT2 & \hspace{.35in} SPRPL, SSTAT2\\
\end{tabular}

Designed by C. L. Lawson and F. T. Krogh, JPL, Apr.~1987. Programmed by C.
L. Lawson and S. Y. Chiu, JPL, Apr.~1987. Changed Nov.~1988 (MATH77
Release~2.2) to use SPRPL and DPRPL instead of PRPL.

\end{multicols}
\end{document}
