\documentclass[twoside]{MATH77}
\usepackage{multicol}
\usepackage[fleqn,reqno,centertags]{amsmath}
\begin{document}
\begmath 19.1 System Parameters

\silentfootnote{$^\copyright$1997 Calif. Inst. of Technology, \thisyear \ Math \`a la Carte, Inc.}

\subsection{Purpose}

These subprograms provide values of various system parameters that
are needed in library subprograms and typically have different values
on different computer systems.  ``F. Supporting Information'' below
has instructions for customizing these programs for your system.

\subsection{Usage}

\subsubsection{Program Prototype}

\begin{description}
\item[REAL]  \ {\bf R1MACH, S}

\item[DOUBLE PRECISION]  \ {\bf D1MACH, D}

\item[INTEGER]  \ {\bf I1MACH, I, J}
\end{description}

Set J in the range 1 $\leq $ J $\leq $ 5 for R1MACH or D1MACH or 1 $\leq $ J
$\leq $ 16 for I1MACH. Then, use the appropriate one of the following
statements:
$$
\fbox{{\bf S = R1MACH(J)}}
$$
$$
\fbox{{\bf D = D1MACH(J)}}
$$
$$
\fbox{{\bf I = I1MACH(J)}}
$$
The results, S, D, or I are set as described in Section D.

\subsubsection{Argument Definitions}

\begin{description}
\item[J]  \ [in] Integer argument selecting the desired system parameter as
described in Section D.
\end{description}

\subsection{Examples and Remarks}

The program DRMACH lists all the values obtainable from R1MACH, D1MACH, and
I1MACH on the host computer system. Output is shown from several different
host systems.

\subsection{Functional Description}

For the purpose of this package a model of the Fortran~77 INTEGER, REAL, and
DOUBLE PRECISION number sets is characterized by a total of nine fundamental
parameters.

The model of Fortran~77 numbers of type INTEGER is parameterized by two
numbers, $a$ and $s$, where $a$ denotes the base (radix) of the number
system and $s$ denotes the maximum number of base $a$ digits available to
represent a Fortran integer. Thus the integers range from $-(a^s -1)$ to $%
a^s -1.$

The model of Fortran~77 numbers of type REAL is characterized by four
parameters, $b$, $t$, $emin$, and $emax$, where $b$ is the base of the
fraction part, $t$ is the number of base $b$ digits in the fraction part, $%
emin $ is the minimum exponent and $emax$ is the maximum exponent. The
magnitude of a floating-point number is thus of the form%
\begin{equation*}
b^e\left( c_1b^{-1}+c_2b^{-2}+...+c_tb^{-t}\right) ,
\end{equation*}
where, $emin\leq e\leq emax$, and the digits $c_i$ satisfy $0\leq c_i\leq
b-1.$ A nonzero floating-point number is normalized if and only if the digit
$c_1$ is nonzero. We shall consider only normalized floating-point numbers,
although numeric processors based on the IEEE standard also support a
range of unnormalized numbers.

Fortran~77 numbers of type DOUBLE PRECISION are modeled in the same form as
REAL numbers and are assumed to have the same base, $b$, but generally
different values of $t$, $emin$, and $emax$.

For some computer systems no setting of these parameters will make the model
system coincide exactly with the actual computer's number set. In such cases
the model parameters are selected so the model system will be as large a
subset of the actual number set as possible. In particular, for Cray
X/MP and Y/MP (but not T3D) systems the parameters $t$ and $emax$ are set
smaller and $emin$ is set larger than one might expect from the structure
of floating-point numbers on these systems. The reasons for this are
described in \cite{Schryer:1981:ATO}.

The values returned by this package are either prestored constants or are
computed at compile-time from prestored constants by use of expressions in
PARAMETER statements. Thus correct values must be determined and edited into
the package whenever this package is moved to a new computer system. Correct
values for many systems are present as comments in the source code.

On some systems, the Fortran compiler may not be able to compute
numbers in the full range of the model, in PARAMETER statements (this
disease usually strikes computation of the overflow limit).  On such
systems it may be necessary to restrict the model, $e.g.$ by reducing
$emax$.

{\bf Specification of Values Returned}
\begin{itemize}
\item[\bf J]  {\hspace{.5in} {\bf I1MACH (J)}}
\item[1]  Standard input unit number.
\item[2]  Standard output unit number.
\item[3]  Standard punch unit number.
\item[4]  Standard error message unit number.
\item[5]  Number of bits per Fortran integer storage unit.
\item[6]  Number of characters per Fortran integer storage unit.
\item[7]  $a$, the base for integers.
\item[8]  $s$, the number of base $a$ digits in an integer.
\item[9]  $a^s-1$, the largest integer magnitude.
\item[10]  $b$, the base for floating-point numbers. Assumed the same for
REAL and DOUBLE PRECISION arithmetic.
\item[11]  $t$, the number of base $b$ digits for REAL arithmetic.
\item[12]  $emin$, minimum exponent for REAL arithmetic.
\item[13]  $emax$, maximum exponent for REAL arithmetic.
\item[14]  $t$, the number of base $b$ digits for DOUBLE PRECISION
arithmetic.
\item[15]  $emin$, minimum exponent for DOUBLE PRECISION arithmetic.
\item[16]  $emax$, maximum exponent for DOUBLE PRECISION arithmetic.
\end{itemize}
\begin{itemize}
\item[\bf J]  {\hspace{.5in} {\bf R1MACH(J)}}
\item[1]  $b^{emin-1}$ smallest positive normalized REAL number,
(underflow limit).
\item[2]  $b^{emax}(1-b^{-t})$, largest REAL number, (overflow limit).
\item[3]  $b^{-t}$, smallest relative difference between two successive
nonzero REAL numbers. This is also the difference between~1.0 and the next
smaller REAL number.
\item[4]  $b^{-(t-1)}$ largest relative difference between two successive
nonzero REAL numbers. This is also the difference between~1.0 and the next
larger REAL number.
\item[5]  $\log _{10}b$, useful in certain conversions between base $b$ and
base~10.
\end{itemize}
The values returned by D1MACH are as described above for R1MACH with REAL
replaced by DOUBLE PRECISION.

\subparagraph{Historical perspective and relations to other languages}

The specifications of R1MACH, D1MACH, and I1MACH and the original
implementation were developed at the AT\&T Bell Laboratories, Murray Hill,
New Jersey, in the~1970's to support the development of portable
mathematical software, and specifically the PORT library,
\cite{Fox:1978:PMS}, which is a proprietary AT\&T Bell Laboratories
product.  These three subprograms were published as a subset of
Algorithm~528 in TOMS,~\cite{Fox:1978:AFP}, and are not proprietary.  The
MATH77 version has the same specification but is substantially different
in its implementation from the original versions.

The attributes associated with J = 1, 2, 3, 4, and~6 in I1MACH are less
relevant in Fortran~77 in the~90's than they were in Fortran~66 in the~70's.
In particular, only the DNLxxx and SNLxxx subroutines of
Chapter~9.3 access I1MACH(2).

Other MATH77 library subprograms use PRINT or WRITE(*,...) for printing.

Languages developed more recently than Fortran~77, such as Ada, ANSI C, and
Fortran~90, provide methods within the language to obtain certain
environmental parameters. Consider, for example, the underflow and overflow
limits, and precision for floating-point arithmetic. Using the present
package, these can be obtained for DOUBLE PRECISION arithmetic by
referencing D1MACH(1), D1MACH(2), and D1MACH(4), respectively, and for REAL
arithmetic by referencing R1MACH(1), R1MACH(2), and R1MACH(4). In Fortran~90
these parameters can be obtained by referencing the generic inquiry
functions TINY(X), HUGE(X), and EPSILON(X), where X may be any DOUBLE
PRECISION entity to obtain the values for DOUBLE PRECISION arithmetic, and
any REAL entity to obtain the values for REAL arithmetic. In ANSI C these
values for arithmetic of type $double$ are given by the macro names DBL\_MIN,
DBL\_MAX, and DBL\_EPSILON, for type $float$ there are FLT\_MIN, FLT\_MAX,
and FLT\_EPSILON, and for type $long\ double$ LDBL\_MIN, LDBL\_MAX, and
LDBL\_EPSILON.  All of these are defined in the standard
header file $float.h$.

\bibliography{math77}
\bibliographystyle{math77}

\subsection{Error Procedures and Restrictions}

If the argument is outside the range 1\ $\leq $ J $\leq $ 16 for I1MACH or
outside 1 $\leq $ J $\leq $ 5 for R1MACH or D1MACH, an error message is
printed and execution is terminated.

This package contains a partial protection against the inadvertent use of
the wrong version of one of these subroutines; say using the PC version on a
VAX. On the first call to any one of these subprograms, tests are done to
verify that two of the stored parameter values are not grossly wrong for the
current environment. These tests depend on assumptions about hardware,
compilers, and linkers that may be invalidated by technological changes.
Subroutines AMTEST and AMSUB1 are used to support these tests and are not
intended for any other usage.

\subsection{Supporting Information}

The source language is ANSI Fortran~77. All the program units are
grouped into a single file, AMACH.FOR. The filename may be different
on different systems, $e.g.$, ``amach.f" on UNIX systems.

One can either customize AMACH for a new system by commenting out
lines defining the parameters required for the system as it is
currently configured, and uncommenting lines required for the desired
system, or one can use the program {\tt m77con} described in
Chapter~19.4.  This requires making up a small control file {\tt
m77job}, and compiling, linking, and running {\tt m77con}, which is
self contained.  To make up {\tt m77job} for a VAX running UNIX, the
control file would contain the following.

{\tt SET SYS = VAX\newline
FILE amach.f}

Running {\tt m77con} with this control file and amach in the current
directory will generate a file {\tt amach.f} for the VAX.  If one
wants the extension ``{\tt .for}'' change the {\tt h.f} to {\tt
h.for}.  If one wants a machine other than the VAX, choose a value
for SYS (without any parenthetical remark) from the following table.
{\tt SYS=IEEE} covers any machine that uses the IEEE binary standard
for floating point arithmetic.  If your machine is not included in
this list, either pick a machine with the same parameters for the
floating point arithmetic as for a machine on this list, or enter
parameters for your machine as a new option into AMACH.

\begin{tabbing}
{\tt SYS = IEEE}\\
{\tt SYS = AMDAHL}\\
{\tt SYS = APOLLO\_10000}\\
{\tt SYS = BUR1700}\\
{\tt SYS = BUR5700}\\
{\tt SYS = BUR67\_7700}\\
{\tt SYS = CDC60\_7000}\\
{\tt SYS = CONVEXC\_1}\\
{\tt SYS = CRAY1}\\
{\tt SYS = CRAY1\_SD}\hspace{30pt}\=(Sngl prec.arith. used for dble.)\\
{\tt SYS = CRAY1\_64}\>(64 bit integers)\\
{\tt SYS = CRAY1\_SD\_64}\>(64 bit int, SP used for DP)\\
{\tt SYS = CRAY\_J90}\\
{\tt SYS = CRAY\_J90\_SD}\>(Sngl prec. used for dble.)\\
{\tt SYS = DG\_S2000}\\
{\tt SYS = HARRIS220}\\
{\tt SYS = HON600\_6000}\\
{\tt SYS = HON\_DPS\_8\_70}\\
{\tt SYS = HP700Q}\>(Q Precision on HP700 series)\\
{\tt SYS = IBM360\_370}\\
{\tt SYS = INTERDATA\_8\_32}\\
{\tt SYS = PDP10\_KA}\\
{\tt SYS = PDP10\_KB}\\
{\tt SYS = PDP11}\\
{\tt SYS = PRIME50}\\
{\tt SYS = SEQ\_BAL\_8000}\\
{\tt SYS = UNIVAC}\\
{\tt SYS = VAX}\\
{\tt SYS = VAX\_G}\\
{\tt SYS = ALPHA\_D3}
\end{tabbing}

Designed and programmed by P.A. Fox, A.D. Hall, and N.L. Schryer, AT\&T Bell
Laboratories, 1978. Adapted to the JPL MATH77 library, 1984 and~1987.

\begin{tabular}{@{\bf}l@{\hspace{5pt}}l}
\bf Entry & \hspace{.2in} {\bf Required Files}\vspace{2pt} \\
D1MACH & \hspace{.35in} AMACH\\
I1MACH & \hspace{.35in} AMACH\\
R1MACH & \hspace{.35in} AMACH\\\end{tabular}


\begcode

\medskip
\lstset{language=[77]Fortran,showstringspaces=false}
\lstset{xleftmargin=.8in}

\centerline{\bf \large DRMACH}\vspace{0pt}
\lstinputlisting{\codeloc{mach}}

\centerline{{\bf \large Results from Various Machines\hspace{1in}}}

\begin{lstlisting}{}

              MACHINE CONSTANTS for IEEE Arithmetic
              -----------------------------------------
 J      I1MACH(J)       R1MACH(J)             D1MACH(J)

 1              5    0.11754944E-37    0.222507385850720100-307
 2              6    0.34028235E+39    0.179769313486231600+309
 3              7    0.59604645E-07    0.111022302462515700E-15
 4              6    0.11920929E-06    0.222044604925031300E-15
 5             32    0.30103001        0.301029995663981200
 6              4
 7              2
 8             31
 9     2147483647
10              2
11             24
12           -125
13            128
14             53
15          -1021
16           1024
\end{lstlisting}

\begin{lstlisting}{}


              MACHINE CONSTANTS for VAX
              -----------------------------------------
 J      I1MACH(J)       R1MACH(J)             D1MACH(J)
 1              5    0.29387359E-38    0.293873587705571877E-38
 2              6    0.17014117E+39    0.170141183460469229E+39
 3              7    0.59604645E-07    0.138777878078144568E-16
 4              6    0.11920929E-06    0.277555756156289135E-16
 5             32    0.30103001        0.301029995663981198
 6              4
 7              2
 8             31
 9     2147483647
10              2
11             24
12           -127
13            127
14             56
15           -127
16            127
\end{lstlisting}

\newpage
\begin{lstlisting}{}

              MACHINE CONSTANTS for CRAY J90
              -----------------------------------------
 J      I1MACH(J)       R1MACH(J)             D1MACH(J)
 1              5    0.73344155-2465   0.733441547021938866-2465
 2              6    0.13634352+2466   0.136343516952426991+2466
 3            102    0.71054274E-14    0.504870979341447555E-28
 4              6    0.14210855E-13    0.100974195868289511E-27
 5             64    0.30103000        0.301029995663981195
 6              8
 7              2
 8             46
 9 70368744177663
10              2
11             47
12          -8188
13           8189
14             94
15          -8188
16           8189



              MACHINE CONSTANTS for UNISYS 1100
              -----------------------------------------
 J      I1MACH(J)       R1MACH(J)             D1MACH(J)
 1              5     .14693679-038     .278134232313400172-308
 2              6     .17014118+039     .898846567431157951+308
 3              7     .74505806-008     .867361737988403547-018
 4              6     .14901161-007     .173472347597680709-017
 5             36     .30103000         .301029995663981194
 6              4
 7              2
 8             35
 9    34359738367
10              2
11             27
12           -128
13            127
14             60
15          -1024
16           1023
\end{lstlisting}
\end{document}
