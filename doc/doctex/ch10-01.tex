\documentclass[twoside]{MATH77}
\usepackage{multicol}
\usepackage[fleqn,reqno,centertags]{amsmath}
\begin{document}
\begmath 10.1 One-Dimensional Real Fourier Transform

\silentfootnote{$^\copyright$1997 Calif. Inst. of Technology, \thisyear \ Math \`a la Carte, Inc.}

\subsection{Purpose}

This subroutine computes discrete Fourier transforms for real data using the
fast Fourier transform. The relations between the real values $x$ and the
complex Fourier coefficients $\xi $ are as follows:

For Fourier analysis:
\begin{equation}\label{fa}
\xi _k=\frac 1N\sum_{j=0}^{N-1}x_jW^{-jk},\quad k=0,1...,N-1,
\end{equation}
For Fourier synthesis:
\begin{equation}
\label{fs}x_j=\sum_{k=0}^{N-1}\xi _kW^{jk},\quad j=0,1,...,N-1,
\end{equation}
where $N=2^{\text{M}}$ and $W=e^{2\pi i/N}=\cos (2\pi /N)+i\ \sin (2\pi /N)$, and $%
i^2=-1$. Note that the sign on the exponent of $W$ is negative for analysis
and positive for synthesis.

\subsection{Usage}

\subsubsection{Program Prototype, Single Precision}

\begin{description}
\item[\bf INTEGER]  \ {\bf M, MS}
\item[\bf REAL]  \ {\bf A}($\geq 2^{\text{M}}$)\bf
 {, S}($\geq 2^{\text{M}-2}-1$)
\item[\bf CHARACTER]  \ {\bf MODE}
\end{description}
On the initial call set MS = 0 to indicate the array S() does not yet
contain a sine table. Assign values to A(), MODE, and M.%
$$
\fbox{{\bf CALL SRFT1(A, MODE, M, MS, S)}}
$$
On return A() will contain the computed results. S() will contain the sine
table used in computing the Fourier transform. MS may have been changed.

\subsubsection{Argument Definitions}

\begin{description}
\item[A()]  [inout] If the argument MODE selects Analysis, A() must contain $%
x$ on input and will contain $\xi $ on output. If MODE selects Synthesis,
A() must contain $\xi $ on input and will contain $x$ on output. When A()
contains $x$'s, $A(k+1)$ contains $x_k$, $k=0$, 1, ..., $N-1$. When
A() contains $\xi $'s, the real coefficients $\xi _0$ and $\xi
_{N/2}$ are in A(1) and A(2), respectively and the complex coefficients, $%
\xi _k$, for $k=1$, ..., $(N/2)-1$, are in the paired locations $(A(2k+1)$, $%
A(2k+2))$. The remaining complex coefficients satisfy $\xi _{N-k} =
\overline{\xi}_k$, for $k$ = 1, ..., $\frac N2-1$.

\item[MODE]  [in] MODE\ =\ $^{\prime }\text{A}^{\prime }$
or $^{\prime }\text{a}^{\prime }$ selects Analysis.
MODE\ =\ $^{\prime }\text{S}^{\prime }$ or $^{\prime }\text{s}^{\prime }$
selects Synthesis.

\item[M]  [in] The number of data points is $N=2^{\text{M}}$. If $\text{M}=0$,
the subroutine returns taking no action. Require $0\leq \text{M}\leq 31.$

\item[MS]  [inout] Gives the state of the sine table in S().  Let
$\text{MS}_{in}\text{ and MS}_{out}$ denote the values of MS on entry
and return respectively. If the sine table has not previously been
computed, set $\text{MS}_{in} = 0$ or $-$1 before the call. Otherwise
the value of $\text{MS}_{out}$ from the previous call using the same
S() array can be used as $\text{MS}_{in}$ for the current call.

Certain error conditions described in Section E cause the subroutine
to set $\text{MS}_{out} = -2$ and return.  Otherwise, with M $>$ 0, the
subroutine sets $\text{MS}_{out} = \max (\text{M}, \text{MS}_{in}).$

If $\text{MS}_{out} > \max (2, \text{MS}_{in}),$ the subroutine sets
NT = $2^{\text{MS}_{out}-2}$ and fills S() with NT $-$ 1 sine values.

If $\text{MS}_{in}=-1$, the subroutine returns after the above
actions, not transforming the data in A().  This is intended to allow
the use of the sine table for data alteration before a subsequent Fourier
transform, as discussed in Section~G of Chapter~16.0, and as
illustrated in the example of Section~C.

\item[S()]  [inout] When the sine table has been computed, $\text{S}(j)=
\sin \pi j/(2\times NT)$, $j=1,2,...,NT-1$, see MS above.
\end{description}

\subsubsection{Modifications for Double Precision}

Change SRFT1 to DRFT1, and the REAL type statement to DOUBLE PRECISION.

\subsection{Examples and Remarks}

Program DRSRFT1 computes an estimate of the spectral composition of
\begin{equation}
\label{e1}f(t)=[\sin 2\pi (t+.01)+4\cos 2\pi (\sqrt{2}\,t+.3)]
\end{equation}
with output in ODSRFT1. It is assumed $t$ has units of seconds, $f$ has
discrete frequencies with mutual separation of at least 0.4~cycles/second,
and that there is no interest in frequencies $>2$ cycles/second. To resolve
different frequencies we select $\Delta \omega =0.4/10=.04$ cycles/second,
which according to Eq.~(20) of Chapter~16.0 implies a sampling period of $%
T=1/\Delta \omega =25$ seconds. According to Eq.~(24) of Chapter~16.0, $\Delta
t\leq \frac 12\omega _{N/2}=\frac 14$~second, if any information about the
highest frequency is to be obtained. Thus, $N=T/\Delta t\geq 100$. Since
modest accuracy is sufficient, we give results for $N=128$ in the program
below. Lanczos sigma factors are used as described in Section G of
Chapter~16.0. The complex transform has peaks for $k=26$ and for $k=36$.
These values of $k$ correspond to frequencies of 1 and~1.4~cycles/second.
Results for this problem, but without the use of Sigma factors are
given in Chapter~16.3.

\subsection{Functional Description}

From Eq.~(\ref{fa}), the fact that $x_j$ is real, and $W^N=e^{2\pi i}=1$, it
follows that $\xi _{N-k}=\overline{\xi }_k$. Thus $\xi _{N/2}$ is real, and
clearly $\xi _0$ is also real. Thus only half the $\xi $'s need be
stored if $\xi _0$ and $\xi _{N/2}$ are counted as one. The $\xi $'s
are stored in the same space used for the $x$'s by taking advantage
of this symmetry.

The primitive FFT routine SFFT (or DFFT) is used to compute $\zeta $ from $x$%
, or $x$ from $\zeta $ where $\zeta $ is defined by
\begin{equation}\label{e4}
\zeta _k=\frac 2N\sum_{j=0}^{\frac
N2-1}(x_{2j}+ix_{2j+1})W^{-2jk},k=0,1,...,\frac N2-1.
\end{equation}
Using Eq.\ (\ref{e4}), one finds that $\zeta _k+\overline{\zeta }_{(N/2)-k}$
depends only on $x_{2j}$ and $\zeta _k-\overline{\zeta }_{(N/2)-k}$ only on $%
x_{2j+1}$. From this and Eq.\ (\ref{fa}) there follows immediately%
\begin{equation}\label{e5}
\begin{split}
\xi_k  &=\frac 14 \left[(\zeta _k+\overline{\zeta }_{(N/2)-k})\right.\\
 -i&\left.W^{-k}(\zeta_k-\overline{\zeta}_{(N/2)-k})\right] ,
 \ \ 0<k<N/4\\
 \overline{\xi}_{(N/2)-k} &= \frac 14 \left[(\zeta _k+\overline{\zeta }_{(N/2)
 -k})\right.\\
 +i&\left.W^{-k}(\zeta _k-\overline{\zeta }_{(N/2)-k})\right] ,
 \ \ 0<k<N/4\\
  \xi_0 &=\frac 12 [\Re\zeta _0+\Im\zeta _0]\\
  \xi_{N/2} &=\frac 12 [\Re\zeta _0-\Im\zeta _0]\\
  \xi _{N/4} &=\frac 12 \overline {\zeta}_{N/4}
\end{split}
\end{equation}
%\begin{eqnarray}
%    \nonumber \xi_k  & = &  \frac 14
%        \left[(\zeta _k+\overline{\zeta }_{(N/2)-k}) \hspace{1in}\right. \\
% \nonumber &  &  \left. \hspace{-.35in} \mbox{}-iW^{-k}
%                        (\zeta _k-\overline{\zeta }_{(N/2)-k})\right] ,\  0<k<N/4\\
% \nonumber \overline{\xi}_{(N/2)-k} & = &
%                     \frac 14 \left[(\zeta _k+\overline{\zeta }_{(N/2)-k})\right. \\
%   & & \left. \hspace{-.35in} \mbox{}+iW^{-k}(\zeta _k-\overline{\zeta }_{(N/2)-k})\right] ,
%     \  0<k<N/4\ \ \ \ \\
%  \nonumber \xi_0 & = & \frac 12 [\Re\zeta _0+\Im\zeta _0] \\
%   \nonumber \xi_{N/2} & = & \frac 12 [\Re\zeta _0-\Im\zeta _0] \\
%  \nonumber      \xi _{N/4} & = & \frac 12 \overline {\zeta}_{N/4} \\
%  \nonumber
%\end{eqnarray}
Thus to compute $\xi $ given $x$, $\zeta /4$ is computed
using Eq.~(\ref{e4}) (divided by~4) and the $\xi $'s are computed
from Eq.~(\ref{e5}). To compute $x$ given $\xi $, the above process is reversed.
More details can be found in \cite{Krogh:1970:RF1}.

\bibliography{math77}
\bibliographystyle{math77}

\subsection{Error Procedures and Restrictions}

M must satisfy 0 $\leq \text{M} \leq 31$ and MODE must have one of its
allowed values.  If these conditions are violated an error message will be
issued using the error processing procedures of Chapter~19.2 with a
severity level of~2 to stop execution.  A return will be made with $%
\text{MS}=-2$ instead of stopping if the statement ``CALL ERMSET($-1$)''
is executed before calling this subroutine.

If the sine table does not appear to have valid data, an error message is
printed, and the sine table and then the transform are computed.

\subsection{Supporting Information}

The source language is ANSI Fortran~77.

\begin{tabular}{@{\bf}l@{\hspace{5pt}}l}
\bf Entry & \hspace{.35in} {\bf Required Files}\vspace{2pt} \\
DRFT1 & \parbox[t]{2.7in}{\hyphenpenalty10000 \raggedright
 DFFT, DRFT1, ERFIN, ERMSG, IERM1, IERV1\rule[-5pt]{0pt}{8pt}}\\
SRFT1 & \parbox[t]{2.7in}{\hyphenpenalty10000 \raggedright
 ERFIN, ERMSG, IERM1, IERV1, SFFT, SRFT1}\\
\end{tabular}

Subroutine designed and written by: Fred T. Krogh, JPL Section~373, August
1969; Revised January~1988.


\begcodenp

\lstset{language=[77]Fortran,showstringspaces=false}
\lstset{xleftmargin=.8in}

\centerline{\bf \large DRSRFT1}\vspace{10pt}
\lstinputlisting{\codeloc{srft1}}

\vspace{30pt}\centerline{\bf \large ODSRFT1}\vspace{10pt}
\lstset{language={}}
\lstinputlisting{\outputloc{srft1}}
\end{document}
