\documentclass[twoside]{MATH77}
\usepackage{multicol}
\usepackage[fleqn,reqno,centertags]{amsmath}
\begin{document}
\begmath 15.4 Cumulative Distribution Function for Poisson
\hbox{Probability Distribution}

\silentfootnote{$^\copyright$1997 Calif. Inst. of Technology, \thisyear \ Math \`a la Carte, Inc.}

\subsection{Purpose}

The procedure described in this chapter computes the Cumulative Distribution
Function (CDF) of the Poisson probability distribution. The CDF is sometimes
called the lower tail. The lower tail,
or CDF, $Q(n|\lambda )$, and the upper tail, $P(n|\lambda )$ for the Poisson
probability distribution with parameter $\lambda $ and argument $n$ are
defined by%
\begin{gather*}
\Pr \{x<n|\lambda \}=Q(n|\lambda )=e^{-\lambda }\sum_{j=0}^{
n-1}\frac{\lambda ^j}{j!},\phantom{=1-Q(n|\lambda )}\\
\Pr \{x\geq n|\lambda \}=P(n|\lambda )=e^{-\lambda }
\sum_{j=n}^\infty \frac{\lambda ^j}{j!}=1-Q(n|\lambda ).
\end{gather*}
\subsection{Usage}

\subsubsection{Program Prototype, Single Precision}

\begin{description}
\item[REAL]  \ {\bf LAMDA, P, Q}

\item[INTEGER]  \ {\bf N, IERR}
\end{description}

Assign values $n$ to N and $\lambda $ to LAMDA, and obtain P $= P(n|\lambda
) $ and Q $= Q(n|\lambda )$ by using
$$
\fbox{{\bf CALL SCDPOI (N, LAMDA, P, Q, IERR)}}
$$

\subsubsection{Argument Definitions}

\begin{description}
\item[N]  \ [in] Argument $n$ of the functions $P(n|\lambda )$ and $%
Q(n|\lambda ).$ N must be nonnegative.  Must be positive if LAMBDA =
0.

\item[LAMDA]  \ [in] Parameter $\lambda $ of the functions $P(n|\lambda )$
and $Q(n|\lambda ).$ LAMBDA must be nonnegative.  Must be positive if
N = 0.

\item[P]  \ [out] The value of the function $P(n|\lambda ).$

\item[Q]  \ [out] The value of the function $Q(n|\lambda ).$

\item[IERR]  \ [out] A flag that normally is zero to indicate successful
computation. See Section E below for discussion of non-zero values.
\end{description}

\subsubsection{Modifications for Double Precision}

For double precision computation, change the REAL type statement to DOUBLE
PRECISION and change the initial letter of the procedure name to D.

\subsection{Example and Remarks}

See DRDCDPOI and ODDCDPOI for an example of the usage of this subprogram.

The procedures SGAMIK and SGAMIE, described in Chapter~2.19, are used to
control the procedure SGAMI and determine the error estimate it returns.
SGAMI is used as described in Section D below.

\subsection{Functional Description}

\subsubsection{Method}

The identities $P(n|\lambda )=P(a,x)$ and $Q(n|\lambda )=Q(a,x)$, with $a=n$
and $x=\lambda $, where $P(a,x)$ and $Q(a,x)$ are incomplete gamma function
ratios, are used. The procedure SGAMI described in Chapter~2.19 is used to
evaluate $P(a,x)$ and $Q(a,x).$

\subsubsection{Accuracy Tests}

See Section~2.19.D.

\subsection{Error Procedures and Restrictions}

The procedure SGAMI issues error messages, under several conditions, at
level 2 + MSGOFF, where MSGOFF is zero unless specified by a call to SGAMIK
(see Chapter~2.19) at some time before calling SCDPOI. If error termination
is suppressed by setting MSGOFF $< 0$, or by calling ERMSET (see Chapter
19.2), IERR will be set to a non-zero value.

If the desired tolerance could not be achieved, IERR is set to~2.

If SCDPOI is called with both LAMDA and N zero, IERR is set to~3 and P is
set to~3.0.

If SCDPOI is called with one or both of LAMDA and N negative, IERR is set to
4 and P is set to~4.0.

\subsection{Supporting Information}

\begin{tabular}{@{\bf}l@{\hspace{5pt}}l}
\bf Entry & \hspace{.35in} {\bf Required Files}\vspace{2pt} \\
DCDPOI & \parbox[t]{2.7in}{\hyphenpenalty10000 \raggedright
AMACH, DCDPOI, DCSEVL, DERF, DERM1, DERV1, DGAM1, DGAMMA, DINITS, DRCOMP,
DREXP, DRLOG, DXPARG, ERFIN, ERMSG, IERM1, IERV1\rule[-5pt]{0pt}{8pt}}\\
SCDPOI & \parbox[t]{2.7in}{\hyphenpenalty10000 \raggedright
AMACH, ERFIN, ERMSG, IERM1, IERV1, SCDPOI, SCSEVL, SERF, SERM1, SERV1, SGAM1,
SGAMMA, SINITS, SRCOMP, SREXP, SRLOG, SXPARG}\\\end{tabular}

Designed and programmed by W. V. Snyder, JPL, 1993.


\begcodenp
\lstset{language=[77]Fortran,showstringspaces=false}
\lstset{xleftmargin=.8in}

\centerline{\bf \large DRDCDPOI}\vspace{10pt}
\lstinputlisting{\codeloc{dcdpoi}}

\vspace{30pt}\centerline{\bf \large ODDCDPOI}\vspace{10pt}
\lstset{language={}}
\lstinputlisting{\outputloc{dcdpoi}}
\end{document}
