\documentclass[twoside]{MATH77}
\usepackage{multicol}
\usepackage[fleqn,reqno,centertags]{amsmath}
\begin{document}
\begmath 16.2  Character-based Graphics --- Single Print Line

\silentfootnote{$^\copyright$1997 Calif. Inst. of Technology, \thisyear \ Math \`a la Carte, Inc.}

\subsection{Purpose}

This subroutine constructs a character string that may be printed as part of a
single line image by the user's program to give a graphical representation of
data.  It is intended primarily for use as a supplement to ordinary tabular
output of data as an aid in spotting trends, wild points, etc.

\subsection{Usage}

\subsubsection{Program Prototype, Single precision}

\begin{description}

\item[INTEGER] \ {\bf NCHAR}

\item[REAL] \  {\bf Y, Y1, Y2}

\item[LOGICAL] \ {\bf RESET}

\item[CHARACTER*{\rm 1}] \ {\bf SYMBOL}

\item[CHARACTER*$n$] \ {\bf IMAGE} \ $[n \geq $ NCHAR]

\end{description}

Assign values to Y, SYMBOL, NCHAR, Y1, Y2, and RESET.

\begin{center}
\fbox{\begin{tabular}{@{\bf }c}
CALL SPRPL (Y, SYMBOL, IMAGE,\\
NCHAR, Y1, Y2, RESET)\\
\end{tabular}}
\end{center}

On return the string, IMAGE, contains the character SYMBOL positioned as a
function of the value of Y.  The user's program may then print IMAGE as part
of a line containing Y and possibly other information.

\subsubsection{Argument Definitions}

\begin{description}
\item[Y] \ [in]  Data value to be plotted.  Y should be between Y1 and Y2;
otherwise see Section E, Error Procedures.
\item[SYMBOL] \ [in]  A single character to be used as a plot symbol.  The character can
be specified literally in the call statement, as for example:
CALL SPRPL (Y, $^\prime $*$^\prime $, ...)
\item[IMAGE] \ [inout]  Character string in which a plot image is constructed.
\item[NCHAR] \ [in]  Number of character positions in IMAGE to be used in constructing
the plot image.  Require NCHAR $\geq  2.$
\item[Y1,Y2] \ [in]  Numbers that bracket the range of values of Y to be plotted in
IMAGE.  Either Y1 $\leq $ Y2 or Y1 $\geq $ Y2 is acceptable.
\item[RESET] \ [in]  Flag to reset the line image.  If RESET\ = .TRUE., the subroutine
will:

\begin{itemize}
\item[1.] Store NCHAR blank characters into IMAGE, and then store the character '0' in
the zero value position if zero is contained in the interval [$ymin$, $ymax$].  See
Section D.

\vspace{3pt}% Need to get decent column break

\item[2.] Store the character specified by SYMBOL in the Y value position.

\end{itemize}
If RESET = .FALSE., the subroutine will only execute Step~2 above.

\end{description}

\subsubsection{Modifications for Double Precision}

For double-precision usage change the REAL type statement to DOUBLE PRECISION,
and change the subroutine name from SPRPL to DPRPL.

\subsection{Examples and Remarks}

Print a set of ($x$, $y$)-data.  On the right side of the page print a
``strip chart" plot of the data with $x$ increasing downward and $y$
increasing to the right.  See DRSPRPL and ODSPRPL for code and output
illustrating this example.

\subsection{Functional Description}

\begin{itemize}
\item[1.] The subroutine will compute $ymin = \min$(Y1, Y2) and $ymax
= \max$(Y1, Y2).  Then, if $ymin = ymax$ these numbers will be
replaced by $ymin = 0.9 \times ymin$ and $ymax = 1.1 \times ymax$
if $ymin \neq 0.$, or by $ymin = -1$, and $ymax = +1$, if $ymin = 0$.

\item[2.] If zero is not in the interval [$ymin$, $ymax$], then the
scaling will be such that $ymin$ corresponds to the center of the
leftmost character position and $ymax$ corresponds to the center of
the rightmost character position.  If zero is in the interval
[$ymin$, $ymax$], and does not correspond to the first or last
character position in IMAGE(), then scaling will be adjusted so the
value zero corresponds to the center of a character position.  This
adjustment guarantees that values located symmetrically with respect
to zero will be plotted in character positions symmetrically located
with respect to the zero character position.
\end{itemize}

\subsection{Error Procedures and Restrictions}

A Y value outside the stated data range, [Y1, Y2], (plus a small tolerance)
will not be plotted, but the message 'OUT' is placed in either the left or
right end of IMAGE() as appropriate.  This message will be suppressed if
NCHAR $< 6.$
\enlargethispage*{40pt}
\subsection{Supporting Information}

The source language is ANSI Fortran~77.

Based on 1969 code by C.L. Lawson, JPL.  Adapted for MATH77 by C.L. Lawson and
S. Chiu, JPL, 1983.  At MATH77 Release~2.2, Nov.~1988, introduced DPRPL, and changed the
name of the previous PRPL to SPRPL.  Programs that were using PRPL should be
changed to use the name SPRPL.


\begin{tabular}{@{\bf}l@{\hspace{5pt}}l}
\bf Entry & \hspace{.2in} {\bf Required Files}\vspace{2pt} \\
DPRPL & \hspace{.35in} DPRPL\rule[-5pt]{0pt}{8pt}\\
SPRPL & \hspace{.35in} SPRPL\\\end{tabular}

\begcode

\medskip\
\lstset{language=[77]Fortran,showstringspaces=false}
\lstset{xleftmargin=.8in}

\centerline{\bf \large DRSPRPL}\vspace{10pt}
\lstinputlisting{\codeloc{sprpl}}

\vspace{20pt}\centerline{\bf \large ODSPRPL}\vspace{10pt}
\lstset{language={}}
\lstinputlisting{\outputloc{sprpl}}
\end{document}
