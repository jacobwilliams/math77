\documentclass[twoside]{MATH77}
\usepackage{multicol}
\usepackage[fleqn,reqno,centertags]{amsmath}
\begin{document}
\hyphenation{DCKDER}
\begmath 8.3 Check Code for Computing Derivatives

\silentfootnote{$^\copyright$1997 Calif. Inst. of Technology, \thisyear \ Math \`a la Carte, Inc.}

\subsection{Purpose}

Subroutine DCKDER checks the mutual consistency of code for computing values
of a (possibly vector valued) function with code for computing first
derivatives $(e.g.$, gradient vector or Jacobian matrix) of the function. In
particular this is expected to be useful for persons using the software of
Chapters 8.2, 9.2, and~9.3.

\subsection{Usage}

\subsubsection{Program Prototype, Double Precision}
\begin{description}
\item[INTEGER]  \ {\bf MODE, M, N, LDFJAC, IMAX, JMAX}

\item[DOUBLE PRECISION]  {\bf X}($\geq $N){\bf , FVEC}($\geq $M){\bf ,\\ FJAC%
}(LDFJAC,$\geq $N){\bf , TSTMAX,\\ TEST}(LDFJAC,$\geq $N)
\end{description}
\begin{tabbing}
\hspace{.2in}\=Assign values to M, N, and LDFJAC.\\

\>Set X() to the value of the vector ${\bf x}$ at which\\
\>\ \ \ \ \=consistency is to be tested.\\

\>Compute FJAC(,) as the M $\times $ N Jacobian matrix\\
\>\>of first partial derivatives of FVEC with respect\\
\>\>to ${\bf x}$, evaluated at X().\\

\>MODE = 1\\

10\>continue
\end{tabbing}\vspace{-8pt}
\begin{center}
\fbox{\begin{tabular}{@{\bf }c}
CALL DCKDER(MODE, M, N, X,\\
FVEC, FJAC, LDFJAC, TEST,\\
IMAX, JMAX,TSTMAX)\\
\end{tabular}}
\end{center}
\begin{tabbing}
\hspace{.2in}\=if(MODE .eq. 2) then\\

\>\ \ \ \ \=Compute FVEC() as an M-vector of function\\
\>\>\ \ \ \ \=values evaluated at X().\\

\>\>go to 10\\

\>end if\\

\>Here the process is completed. Results are in\\
\>TEST(,), IMAX, JMAX, and TSTMAX.
\end{tabbing}
\subsubsection{Argument Definitions}

\begin{description}
\item[MODE]  \ [inout] On initial entry set MODE = 1. DCKDER will return a
number of times with MODE = 2. The calling code should compute FVEC() as a
function of X() and call DCKDER again, not altering MODE. When DCKDER is
finished, it returns with MODE = 3.

\item[M]  \ [in] Number of terms in FVEC() and number of rows of data in
FJAC(,).

\item[N]  \ [in] Number of terms in X() and number of columns of data in
FJAC(,).

\item[X()]  \ [inout] Initially must contain the vector ${\bf x}$ around
which the testing will be done. Contains perturbed ${\bf x}$ on each return
with MODE = 2. On final return with MODE = 3, contains the original ${\bf x}$
exactly restored.

\item[FVEC()]  \ [in] On entries with MODE = 2, the user stores function
values in FVEC(), $i.e.$, FVEC($i$) $=f_i.$

\item[FJAC(,)]  \ [in] On the initial entry with MODE = 1, the user stores
the Jacobian matrix in FJAC(,), $i.e.$, FJAC$(i,j)=\partial f_i/\partial
x_j.$

\item[LDFJAC]  \ [in] Declared first dimension of the arrays FJAC(,) and
TEST(,). Require LDFJAC $\geq $ M.

\item[TEST(,)]  \ [inout] Array with the same dimensions as FJAC(). On final
return with MODE = 3, TEST$(i,j)$ contains the consistency measure for FJAC$%
(i,j)$ for all $i=1$, ..., M, and $j=1$, ..., N. This quantity is computed
as the signed difference: FJAC$(i,j)$ minus a central finite difference
approximation to $\partial f_i/\partial x_j$. The user does not need to
store anything in TEST(,) before the initial entry with MODE = 1. On
intermediate returns with MODE = 2, TEST(,) contains saved intermediate
quantities, and thus the user must not alter the contents of TEST(,) on
these returns.

\item[IMAX, JMAX, TSTMAX]  \ [inout] On final return with MODE = 3, these
quantities are set so that

TSTMAX $=abs($TEST(IMAX, JMAX$))=\max \{abs(\text{TEST}(i,j)):i=1$, ..., M; $%
j=1$, ..., N$\}$. The user does not need to store anything in these
variables before the initial entry with MODE = 1. On intermediate returns
with MODE = 2, these variables contain saved intermediate quantities, and
thus the user must not alter their contents on these returns.
\end{description}

\subsubsection{Modifications for Single Precision}

For single precision usage change the DOUBLE PRECISION statements to REAL
and change the name DCKDER to SCKDER. It is recommended that one use the
double precision rather than the single precision version of this package
for better reliability, except possibly on computers such as the Cray Y/MP
that have precision of about 14~decimal places in single precision.

\subsection{Examples and Remarks}

The program DRDCKDER illustrates the use of DCKDER. Results are shown in
ODDCKDER. This example was run using double precision IEEE arithmetic which
has precision of $\epsilon \approx 0.22 \times 10^{-15}$. If third derivatives
have about the same magnitude as function values, and the relative error in
function evaluations is about machine precision, then the magnitude of
entries of TEST(,) should be about $\epsilon ^{2/3}$ times the magnitude
of the function values. For example in our sample case we have $|f_1| =
0.0646$ and so we would expect the magnitude of terms in the first row of
TEST(,) to be about $(0.37 \times 10^{-10}) \times 0.0646 \approx 0.24 \times
10^{-11}$, with the stated assumptions on the size of third derivatives and
the error in function evaluations. If the relative error in the computation
of function values is larger than the machine precision, or the magnitudes
of third derivatives are larger than the magnitudes of the function values,
the values in TEST(,) will be larger as discussed in Section D, even when
the code being tested is correct.

\subsection{Functional Description}

Let ${\bf x}$ denote the vector given initially in X(). Assume FJAC(,)
contains the $m \times n$ Jacobian matrix of $\partial f_i / \partial x_j$
evaluated at ${\bf x}$. Let $\epsilon $ denote the
machine precision and $\omega $ denote the underflow limit. These are given
by D1MACH(4) and D1MACH(1) respectively (R1MACH() in single precision)
(See Chapter~19.1).
Define $\alpha = (3\epsilon )^{1/3}$ and $\sigma = \max (10^5\omega
/\alpha $, $\epsilon ^2)$. For each value of $j$ from 1 to $n$, compute $%
h_j = \alpha x_j$ if $|x_j| > \sigma $, or $h_j = \alpha \sigma $ if $0 <
|x_j| \leq \sigma $, or $h_j = \alpha $ if $x_j = 0$. Let ${\bf e}_j$
denote the $n$-vector that is all zeros except for the $j^{th}$ component
which is one.

For each $i$ and $j$ compute%
\begin{equation*}
\hspace{-12pt}\text{TEST}(i,j)=\text{FJAC}(i,j)-
\frac{f_i({\bf x}+h_j{\bf e}_j)-f_i({\bf x}-h_j{\bf e}_j)}{2h_j}
\end{equation*}
The error in this central difference approximation to the derivative $%
\partial f_i/\partial x_j$ is%
\begin{equation*}
h_j^2M_3/6+\delta /h_j
\end{equation*}
where $M_3$ denotes the magnitude of $\partial ^3f_i/\partial ^3x_j$
evaluated at some point on the line segment from ${\bf x}-h_j{\bf e}_j$ to $%
{\bf x}+h_j{\bf e}_j$, and $\delta $ is a bound on the error in computing $%
f_i$. The optimal step length to minimize this error estimate is%
\begin{equation*}
(3\delta /M_3)^{1/3}
\end{equation*}
If we assume $M_3\approx |f_i|$ and $\delta \approx \epsilon |f_i|$, so $%
\delta /M_3\approx \epsilon $, throughout the relevant interval, then the
optimal step length is $(3\epsilon )^{1/3}$ and the error bound is $\frac
12(3\epsilon )^{2/3}|f_i|\approx 1.04\epsilon ^{2/3}|f_i|$. These
formulas, and useful insights into the computational use of finite
differences, are given in Section~8.6 of \cite{Gill:1987:PO}.

\bibliography{math77}
\bibliographystyle{math77}

\subsection{Error Procedures and Restrictions}

Require MODE = 1 or 2 on any entry to DCKDER. If not, DCKDER will call the
error message processor of Chapter~19.2 with an error level of 0 and return
with MODE unchanged.

\subsection{Supporting Information}

The source language is ANSI Fortran~77.

\begin{tabular}{@{\bf}l@{\hspace{5pt}}l}
\bf Entry & \hspace{.35in} {\bf Required Files}\vspace{2pt} \\
DCKDER & \parbox[t]{2.7in}{\hyphenpenalty10000 \raggedright
AMACH, DCKDER, ERFIN, ERMSG, IERM1, IERV1\rule[-5pt]{0pt}{8pt}}\\
SCKDER & \parbox[t]{2.7in}{\hyphenpenalty10000 \raggedright
AMACH, ERFIN, ERMSG, IERM1, IERV1, SCKDER}\\
\end{tabular}

Designed and programmed by C. L. Lawson and F. T. Krogh, JPL, 1991.


\begcode

\medskip\
\lstset{language=[77]Fortran,showstringspaces=false}
\lstset{xleftmargin=.8in}

\centerline{\bf \large DRDCKDER}\vspace{5pt}
\lstinputlisting{\codeloc{dckder}}
\newpage
\vspace{20pt}\centerline{\bf \large ODDCKDER}\vspace{10pt}
\lstset{language={}}
\lstinputlisting{\outputloc{dckder}}
\end{document}
