\documentclass[twoside]{MATH77}
\usepackage{multicol}
\usepackage[fleqn,reqno,centertags]{amsmath}
\begin{document}
\begmath 19.3 Extended Error Message Processor

\silentfootnote{$^\copyright$1997 Calif. Inst. of Technology, \thisyear \ Math \`a la Carte, Inc.}

\subsection{Purpose}

The subroutines described here print error messages and diagnostic messages.
These routines are intended primarily for use by other library routines.

\subsection{Usage}

\subsubsection{Program Prototype, Setting or Getting a Message Processor
Parameter}

\begin{description}
\item[INTEGER]  \ {\bf MACT}($\geq k_1$){\bf , IDAT}($\geq k_2$)

\item[CHARACTER*($\geq k_3$)]  \ {\bf TEXT}($\geq k_4$)
\end{description}

The values for the $k_i$'s depend on the actions desired.  For the
actions described prior to the section on advanced features $k_1 = 2
\times ($the number of actions) + 1, $k_2=1$, and $k_3$ and $k_4$ are~1
unless one of the actions is MEHEAD, in which case $k_3$ should be~4.

Assign values to MACT(). Prior to the section on advanced features,
IDAT(), and TEXT() (with one exception) are not referenced.
$$
\fbox{{\bf CALL MESS(MACT, TEXT, IDAT)}}
$$

\subsubsection{Argument Definitions}

\begin{description}
\item[MACT()]  \ [inout] Used to set or check parameters used by the message
processor. Starting at MACT(1), is a list of integer pairs followed by a
single MERET=51, which terminates the list. If the first integer is $>$ 0,
it is an index specifying the parameter that one wants to set and the
following integer gives the value to be assigned to the parameter. If the
first integer is negative, its absolute value specifies a parameter as
above, and the value of that parameter is returned in the second integer. A
single call can get and/or set as many of these parameters as desired.
Below is a list of parameter names (which we recommend be used for the
clarity of the code), and their values, that are used as the first integer
in an entry. The second integer of the entry is denoted by Kj, where j is
the value of the first integer.

\begin{description}
\item[MESUNI=10]  (0 $\leq $ K10 $\leq $ 99) Set the scratch file unit
number to K10.  If the scratch unit is required and K10 has not been
defined, its default value is~$31-k$, where k is the smallest positive
integer for which unit~$31-k$ is available.  If K10 is set to~0, a
scratch unit is assumed not to be available, and tables with long
lines will be printed with each row on multiple lines.

\item[MEHEAD=11]  (0 $\leq $ K11 $\leq $ 1) Defines the print that
surrounds an error message.  K11=0 gives nothing, and~1 gives
TEXT(1)(1:4) repeated 18~times. If this is not used, one gets
72~\$'s. (To get a blank line use 1 with TEXT(1) = $^{\prime
}{\quad}^{\prime }.)$

\item[MEDDIG=12]  (-50 $\leq $ K12 $\leq $ 50) Set default digits to
print for floating point. If K12 $>$ 0 then K12 significant digits
will be printed, if K12 $<$ 0, then $-$K12 digits will be printed
after the decimal point, and if K12 = 0, the default will be used,
which is the full machine precision. Setting or getting this value
will only work properly if the action is taken by calling SMESS or
DMESS as appropriate; see below. There are separate internal values
for K12 in SMESS and DMESS.

\item[MEMLIN=13]  (39 $\leq $ K13 $\leq $ 500) Set line length for
diagnostic messages to K13.  (Default is~128.)

\item[MEELIN=14]  (39  $\leq $ K14 $\leq $ 500) Set line length for
error messages to K14. (Default is~79.)

\item[MEMUNI=15]  (-99 $\leq $ K15 $\leq $ 99) Set unit number for
diagnostic messages to K15. If K15 = 0 (default), a Fortran PRINT
statement is used.

\item[MEEUNI=16]  (-99 $\leq $ K16 $\leq $ 99) Set unit number for error
messages to K16. If K16 = 0 (default), a Fortran PRINT statement is
used.

\item[MESCRN=17]  (0 $\leq $ K17 $\leq $ 100000000) Set number of lines
to print to standard output before pausing for ``go'' from user.
Default is~0, which never stops.

\item[MEDIAG=18]  (0 $\leq $ K18 $\leq $ 1000000000) Not currently in
use by MATH77 routines.  Described more fully below.

\item[MEMAXE=19]  (0 $\leq $ K19 $\leq $ 1000000000) Set the maximum
error value. When retrieving this value, it is the maximum value seen
for $10000 s+1000 p+i$, where $s$, $p$, and $i$ are the stop and
print levels, and the index on the last error message processed,
respectively.

\item[MESTOP=20]  (0 $\leq $ K20 $\leq $ 8) Set the stop level for error
messages. If an error message has a stop index $>\min $(K20,~8), the
program is stopped after processing the message. The default value is
K20=3.

\item[MEPRNT=21]  (0 $\leq $ K21 $\leq $ 8) Set the print level for
error messages. If an error message has a print index $>$ K21, or the
message is going to stop when finished, information in an error
message is processed, else all the actions including printing are
skipped. (MESTOP controls stopping.) The default value is K21=3.

\item[MERET=51]  End of the list. (Only requires a single integer.)
\end{description}

\item[TEXT()]  \ [in] Only referenced if MEHEAD is one of the actions. See the
description above.

\item[IDAT()]  \ [in] Not referenced by the application discussed here.
\end{description}

\subsubsection{Getting and Setting the Default for Digits to Print for
Floating Point Numbers}

As mentioned above, the setting or retrieving of the number of decimal
digits, MEDDIG above, requires a call to SMESS for single precision, and a
call to DMESS for double precision. Either of these calls can be used to set
any of the other parameters also. These calls require the additional
declaration of a floating point array FDAT(), which must be single precision
if SMESS is called and must be double precision if DMESS is called.
$$
\fbox{{\bf CALL SMESS(MACT, TEXT, IDAT, FDAT)}}
$$
$$
\fbox{{\bf CALL DMESS(MACT, TEXT, IDAT, FDAT)}}
$$
As above, the only argument actually used for this application is MACT().

\subsubsection{Advanced Features}

Most of the features listed below are used someplace in one of the
library routines.  The options that use FDAT require calling SMESS or
DMESS, the other options can call MESS, SMESS, or DMESS.  All
variables passed to these routines are arrays which must have a
declared dimension large enough to process the options being specified.
Internal variables, all of which have a default value of 1, are used to
keep track of locations as follows:

\begin{tabular}{lp{2.3in}}
NTEXT & The next text to be output starts at TEXT(NTEXT).\\
NIDAT & The next output from IDAT starts at IDAT(NIDAT).\\
NFDAT & The next output from FDAT starts at FDAT(NFDAT).\\
NMDAT & The next output from MDAT starts at MDAT(NMDAT), where
        MDAT is defined by actions MEMDA1--MEMDA5 below, and
        NMDAT is set to one at the end of every text output.
\end{tabular}

An action that uses data pointed to by one of the above will cause
the pointer to be incremented to one past the last location used.  An
exception is NMDAT, which when it reaches 5 is not incremented and the
value pointed to is incremented instead.

When an option requires more than a single additional number to
define it, a new letter followed by the index associated with the
option is used to denote each additional number.  The additional
values of MACT available are:

\begin{description}
\item[MEDIAG=18]  (0 $\leq $ K18 $\leq $ 1000000000) Set the diagnostic level
  desired.  Once again, note that this is not used in MATH77.
  The error message routines makes no use of K18.  MESS merely
  serves as a place to set it and to answer inquiries on its
  value.  It is intended to be set by users of library
  software.  Library packages that make use of this number
  are expected to use it as described below.  If K18 = 0 (the
  default), no diagnostic print is being requested.  Else $m$ =
  mod(K18,\ 256) determines whether a package will do
  diagnostic printing.  Associated with a library package is
  a number L that must be a power of 2 $<$ 129, and that
  should be mentioned in the documentation for the package.
  If the bit logical or($m$,L) = L then diagnostic output for
  the routine with the associated value of L is activated.
  The value of L should have been selected by the following
  somewhat vague rules.  Let $\log_2(\text{L}) = 2 i + j,$ where $j$
  is 0 or 1.  Select $i$ = level of the library package, where
  the level is 0 if no other library routine that is likely
  to be used with the package could reasonably be expected to
  want any embedded diagnostics, and otherwise is
  min(4, I+1), where I is the maximum level for any library
  routine that is likely to be used with the package.
  Select $j$ = 0 if the user is relatively unlikely to want
  diagnostics, and $j$ = 1, if this is a routine for which
  considering its level the user is relatively likely to want
  diagnostic output.  The next 8 bits, mod(K18/256, 256), may
  be used by the library routine to select the actual output
  that is to be given.  These bits may be ignored,  but if
  they are used, the lowest order bits should correspond to
  less voluminous output that is more likely to be requested.
  Finally, K18 / ($2^{16}$) may be used to give a count on how
  many times to print the diagnostics that are given.  This
  count may be interpreted by library routines in slightly
  different ways, but when used it should serve to turn off
  all output after a certain limit is reached.  By
  convention, if this is 0 there is no upper bound on the
  count.
\item[METDIG=22]  ($-50$ $\leq $ K22 $\leq $ 50) As for MEDDIG (=12,
  above), except the value here is temporary, lasting until the return,
  or next use of this action.  If 0, the internal value for K12 is used
  instead.
\item[MENTXT=23]  (1 $\leq $ K23 $\leq $ 10000000) Set value of NTEXT to K23.
\item[MEIDAT=24]  (1 $\leq $ K24 $\leq $ 1000000000) Set value of NIDAT to K24.
\item[MEFDAT=25]  (1 $\leq $ K25 $\leq $ 1000000000) Set value of NFDAT to K25.
\item[MEMDAT=26]  (1 $\leq $ K26 $\leq $ 5) Set value of NMDAT to K26.
\item[MEMDA1=27]  (K27) set MDAT(1) to K27.  See description of NMDAT above.
\item[MEMDA2=28]  (K28) set MDAT(2) to K28.
\item[MEMDA3=29]  (K29) set MDAT(3) to K29.
\item[MEMDA4=30]  (K30) set MDAT(4) to K30.
\item[MEMDA5=31]  (K31) set MDAT(5) to K31.
\item[METABS=32]  (1 $\leq $ K32 $\leq $ 100) set spacing for tabs to K32.
  The default value is K32=6.
\item[MEERRS=33] (K33) set the current error counter to K33.  (To retrieve
  this value set a location in MACT() to $-33$, and the count will be returned
  in the following location.)  This count is set to 0 on the first call made
  to MESS.
\item[MECONT=50]  Exit, but no print of current print buffer.  The error or
  diagnostic message is to be continued immediately.
\item[MERET=51] All done with diagnostic or error message.  Complete
  processing and return; or for some error messages stop.
\item[MEEMES=52]  (K52, L52, M52) Start an error message with severity level
  K52, index for the error of L52, and message text starting
  at TEXT(M52).  If M52 is 0, message text starts at
  TEXT(NTEXT), and if M52 $<$ 0, no message text is
  printed as part of this action.  This option assumes that TEXT(1) starts
  with text giving the name of the subprogram or package terminated with a
  `\$B'.  Library routines should set K52 = $10\times s + p$, where $s$ is
  the stop level desired, and $p$ the print level, and should have
  $10 > p \geq s \geq 0.$
  We offer the following guidelines as a yardstick for
  setting the value of $s$.
\begin{description}
\item[= 9]  User has ignored warning that program was going to be stopped.
\item[= 8]  Program has no way to continue.
\item[= 7]  User has given no indication of knowing that functionality of
       results is reduced.  (E.g. not enough space for some result.)
\item[= 6]  Program could continue but with reduced functionality.
\item[= 5]  Results far worse than user expected to want.
\item[= 4]  User has given no indication of knowing that results do not
       meet requested or expected accuracy.
\item[= 3]  Warning is given that program will be stopped without some
       kind of response from the calling program.
\item[= 2]  Program is not delivering requested or expected accuracy.
\item[= 1]  Some kind of problem that user could correct with more care in
       coding or in problem formulation.
\item[= 0]  Message is for information of uncritical nature.
\end{description}
           Print levels might be counted down so that warnings given
           several times are no longer given, or be counted up so
           that a warning is only given after a certain threshold is
           reached.  Levels should be selected with the understanding
           that the default is to print or stop only for levels $>$ 3.
\item[METEXT=53]  Print TEXT, starting at TEXT(NTEXT).  Print ends
           with the last character preceding the first '\$'.  Special
           actions are determined by the character following the '\$'.
           Except as noted, the '\$' and the single character that
           follows are not printed.  In the text below, ``to continue",
           means to continue print of TEXT with the next character
           until the next ``\$".  Except for the one case noted, NTEXT
           is set to point to the second character after the ``\$".
           Possibilities for the character following the ``\$'' are
           (letters must be in upper case):
\begin{description}
\item[B] $\underline{\text{B}}$reak text, but don't start a new line.
\item[E] $\underline{\text{E}}$nd of text and line.
\item[R] Break text, don't change the value of NTEXT.  Thus next
        text $\underline{\text{R}}$epeats the current.
\item[N] Start a $\underline{\text{N}}$ew line, and continue.
\item[I] Print $\underline{\text{I}}$DAT(NIDAT), set NIDAT = NIDAT + 1,
         and continue.
\item[J] As for I above, except use the last integer format
         defined by a ``\$(", see below.
\item[F] Print $\underline{\text{F}}$DAT(NFDAT), set NFDAT = NFDAT + 1,
         and continue.
\item[G] As for F above, except use the last floating format
         defined by a ``\$(", see below.
\item[M] Print $\underline{\text{M}}$DAT(NMDAT), set NMDAT = NMDAT + 1,
         and continue.
\item[H] Marks terminator for column and row $\underline{\text{H}}$eadings;
        see table, vector, and matrix output below.  This causes enough
        blanks to be generated to keep column headings centered over their
        columns.  After the blanks are generated, text is continued
        until the next '\$'.  This is not to be used except inside
        column or row headings.  The last row or column should be
        terminated with a '\$E' or if appropriate, a '\$\#' for a row or
        column label.
\item[(] Starts the definition of a format for integer or floating
        point output.  The format must require no more than 12
        characters for floating point and may not contain a ``P" field
        (e.g. ``(nnEww.ddEe)", where each of the lower case letters
        represents a single digit), and no more than 7 characters for
        integer output.  If the next character following the ``)" that
        ends the format is not a ``\$" then ``\$J" or ``\$G" type output
        is done; see above.  In either case processing of TEXT then
        continues.
\item[T] $\underline{\text{T}}$ab.  See METABS (=32 above).
\item[\#] Used in matrix row or column labels.  This prints the current
        row or column index, respectively, ends the text for the
        current row or column, and resets the text pointer to where
        it started.
\item[\$] a single '\$' is printed; continue output of text.
\item[$-$] Starts a negative number for skipping, see next below.
\item[0-9] A sequence of digits (perhaps preceded by a `$-$' sign defines an
  extra amount to skip the index (ahead or back) for fetching the next thing
  from FDAT or IDAT (whichever is referenced next).  We recommend this be used
  just prior to the \$ item that requires the skip.  (The default is to start
  one past the last thing printed.)
\item[C] Only used by {\tt pmess} which deletes it and the preceding '\$'.
        Used at the end of a line to indicate
        $\underline{\text{C}}$ontinued text.
 \item [{\em blank}] A blank ending up in column 72 is replaced by
        ``\$~'' by pmess, thus avoiding bugs in some Fortran compilers.  If
        the user inputs a ``\$~'', not followed by anything pmess starts a new
        data statement with an incremented array name; if followed by a ``D''
        pmess will arrange for the preceding text to be stored in an array (so
        that not too many continuation lines are required) and outputs a
        comment so the C conversion is done correctly.
\item[other] Don't use this --- the '\$' is ignored, but new features may
        change the action.
\end{description}
\item[ME????=54]  Not used.
\item[METABL=55] (K55, L55, M55, N55) Note this action automatically
  returns when done, further locations in MACT are not examined.  This
  action prints a heading and/or data that follows a heading.  If K55
  is 1, then the heading text starting in TEXT(NTEXT) is printed prior
  to printing the data.  This text should contain embedded ``\$H"'s to
  terminate columns of the heading.  If there is no heading on a
  column, use `` \$H".  Note the leading blank.  If the heading is to
  continue over k columns, begin the text with ``\$H" repeated $k-1$
  times with no other embedded characters.  The very last column must
  be terminated with ``\$E" rather than ``\$H".  After tabular data
  are printed, K55 is incremented by 1, and compared with L55.  If K55
  $>$ L55, K55 is reset to 1, and if the data that was to be printed
  required lines that were too long, data saved in the scratch file is
  printed using the headings for the columns that would not fit on the
  first pass.  Note that only one line of tabular data can be printed
  on one call to this subroutine.\newline M55 gives the number of
  columns of data associated with the heading which is the sum of the
  ``rr'' values.\newline N55 is a vector containing as many entries as
  needed to get the sum of the ``rr'' values equal to M55.  The
  $k^{th}$ integer in N55 defines the printing action for the $k^{th}$
  column of the table.  Let such an integer have a value defined by
  $rr + 100 \times (t + 10 \times (dd + 100 \times ww)),$ i.e.
  $wwddtrr,$ where $0 \leq rr,dd,ww < 100,$ and $0 \leq t < 10.$
\begin{description}
\item[rr] The number of items to print.
\item[t] The type of output.
\begin{description}
\item[1]  Print text starting at TEXT(NTEXT), rr = 01.
\item[2]  Print the value of K55, rr = 01.
\item[3]  Print integers starting at IDAT(NIDAT).
\item[4]  Print starting at FDAT(NFDAT), using an F format.
\item[5]  Print starting at FDAT(NFDAT), using an E format.
\item[6]  Print starting at FDAT(NFDAT), using an G format.
\end{description}
\item[dd]   Number of digits after the decimal point.
\item[ww]   The total number of print positions used by the column,
           including the space used to separate this column from the
           preceding one.  This must be big enough so that the column
           headings will fit without overlap.
\end{description}
\item[MEIVEC=57]  (K57) Print IDAT as a vector with K57 entries.  The vector
           output starts on the current line even if the current line
           contains text.  This is useful for labeling the vector.
           The vector starts at IDAT(NIDAT).
           If K57 $<$ 0,  indexes printed in the labels for the vector
           start at NIDAT, and entries from NIDAT to $-$K57 are
           printed.
\item[MEIMAT=58]  (K58, L58, M58, I58, J58) Print IDAT as a matrix with K58
           declared rows, L58 actual rows, and M58 columns.  If K58 $<$ 0,
           instead of using 1 for the increment between rows, and K58
           for the increment between columns, $-$K58 is used for the
           increment between rows, and 1 is used for the increment
           between columns.  If L58 $<$ 0, the number of actual rows is
           mod($-$L58, 100000), and the starting row index is $-$L58 /
           100000.  Similarly for M58 $<$ 0. TEXT(I58) starts the text for
           printing row labels.  If I58 $<$ 0, no row labels are
           printed.  If I58 = 0, it is as if it pointed to text
           containing ``Row \$E".  Any ``\$" in a row or column label must
           be followed by ``H" or ``E" which terminates the text for the
           label.  In the case of \$H, text for the next label follows
           immediately, in the case of \$E the current row index is
           printed in place of the \$E and the next label uses the same
           text.  J58 is treated similarly to I58, except for column
           labels, and with ``Row \$E" replaced with ``Col \$E".  The
           matrix starts at IDAT(NIDAT), and NIDAT points one past the
           end of the matrix when finished.
\item[MEJVEC=59]  (K59) As for MEIVEC, except use the format set from
           using \$(.
\item[MEJMAT=60]  (K60, L60, M60, I60, J60) As for MEIMAT, except use the
           format set from using \$(.
\item[MEFVEC=61]  (K61) As for MEIVEC, except print FDAT as a vector with
           K61 entries.  The vector starts at FDAT(NFDAT).
\item[MEFMAT=62]  (K62, L62, M62, I62, J62) As for action MEIMAT, but
           print FDAT instead, and use NFDAT in place of NIDAT.
\item[MEGVEC=63]  (K63) As for MEFVEC, except use the format set by using \$(.
\item[MEGMAT=64]  (K64, L64, M64, I64, J64) As for MEIMAT, except use the
           format set by using \$(.
\item[MEIVCI=65]  (K65, L65) As for MEIVEC, except the vector entries have a
           spacing of K65, and there are L65 entries in the vector.
\item[MEJVCI=66]  (K66) As for MEIVCI, except use the format set by
           using \$(.
\item[MEFVCI=67]  (K67, L67) As for MEFVEC, except the vector entries have a
           spacing of K67, and there are L67 entries in the vector.
\item[MEGVCI=68]  (K68) As for MEFVCI, except use the format set by
           using \$(.
\item[MEFSPV=69] (K59)  K59 gives the number of entries in the sparse
  vector, IDAT gives indexes of the entries, and FDAT gives the
  corresponding values.  One may want to make use of MEMDA1, and
  MEMDA2 to save for example the column index and the number of
  entries for output in a text message that precedes the output of the
  vector.  The user will need to write a loop to output a sparse
  matrix, and output of sparse integer vectors in not supported.
\end{description}

\subsubsection{Constructing Message Data with {\tt pmess}}
For complicated messages, making up the Fortran DATA statements,
required can be quite difficult.  Even for simple cases, we recommend
using the stand alone program {\tt pmess} to construct both the DATA
statements, and PARAMETER statements that define variables that can
be used to reference messages.  One can then change input message
text, run {\tt pmess}, and replace the old output in the code with the new
output, without requiring other changes in the code.  {\tt pmess} expects
input from ``standard input", and writes output to the file {\tt tmess}.
Thus for the example given in the section below, with an input file
of {\tt drsmess.err} the program is executed as follows (at least for DOS
and Unix)
$$
\fbox{{\bf \tt pmess $<$drsmess.err}}
$$
and the file {\tt tmess} can be inserted into a Fortran routine.  In order to
follow the declaration order imposed by the Fortran standard, and to
avoid splitting the output generated by {\tt pmess}, we recommend inserting the
output from {\tt pmess} after all other specification statements in the code,
but before other data statements (if any).

{\tt pmess} generates parameters LTXTAA, LTXTAB, \ldots .  These names are
associated with individual messages that might be printed, and are
identified by comments of the form cAA, cAB, \ldots generated by {\tt pmess}.
The values generated for these parameters are the character positions of
the message in the particular character array in which that message is
stored.  Messages are first stored in a character array named MTXTAA, but
if a line consisting of a single `\$' in column 1 is encountered the last
two letters of this name are advanced as for LTXTxx, and further text is
stored in the new character array.  {\tt pmess} generates both the declarations
and the data for these character arrays.

The comments (and just the comments) generated by {\tt pmess} can be used as
inputs to {\tt pmess}.  Thus we recommend including these comments in the code
both as documentation for the messages, and to simplify generating the
data if changes to messages are desired.  If these comments are passed as
input to {\tt pmess}, then the reference to column~1 above should be to
column~5.

{\tt pmess} replaces a blank in column 72 of output character strings with a
`\$', and inserts a blank in the continued string.  This is done because
some editors delete trailing blanks, and some compilers require that the
trailing blanks in continued character data be present.

There are three special cases used in {\tt pmess}, none of which generates
text in the output arrays.  A ``\$C'' at the end of a line indicates that the
text on the following line is a continuation of the text on this line.  At
least for use in a Fortran 77 environment one should not use input lines
longer than 68 characters to allow for the 4 extra spaces required by the
generated comments.  A ``\$~'' at the start of a line causes the code to use a
new array for the following text, and a ``\$~D'' at the start of a line tells
{\tt pmess} to separate the data for the preceding text into separate array
elements.  The latter is necessary if the text would require more than 19
continuation lines.  Special comments for the conversion to C are also output
for this case.

Recall that for error messages, the text must start with the subprogram
name, this name should be terminated with `\$B'.  We suggest that after
the subprogram name, one give the text for the error messages in order.
Then the text for error message $i$, can be located from an array
referenced by the error index.

\subsection{Examples and Remarks}

{\tt pmess} transformed the data in {\tt drsmess.err} to statements included in
the listing for DRSMESS.FOR. The output from running DRSMESS is given
in ODSMESS. This illustrates how error messages can be generated, and
how the length of the output line can be changed from the default
(which is~79) to~40.

Other library routines can be examined to see how various error
messages and diagnostic outputs have been implemented.  Most of the
features described here can be found in the source code for either
DIVA or DIVADB,

\subsection{Functional Description}

The routines mentioned here are called by a number of library routines for
the printing of both error messages and diagnostic information.  The
user can call the routines described here to alter the actions they
take when called by other library routines.

Other approaches to handling error messages can be found in the references
cited below.
\nocite{Fox:1978:PMS}
\nocite{Fox:1978:AFP}
\nocite{IMSL:1989:ML}
\nocite{Jones:1983:XER}

\bibliography{math77}
\bibliographystyle{math77}

\subsection{Error Procedures and Restrictions}

Use of indexes other than those described here, cause MESS to print
the message ``Actions in MESS terminated due to error in usage of
MESS.'' and then to return.  There are a two errors that cause
MESS to execute an unrequested ``STOP.''  Such a STOP will print one
of the two following messages.

``Could not assign scratch unit in MESS.''\newline
``Stopped in MESS {-}{-} Column width too small in a heading.''

\subsection{Supporting Information}

The source language is ANSI Fortran~77.

Algorithm and code due to F. T. Krogh, JPL, November, 1991.  Last
change in March, 2006 to add output of sparse vectors.



\begin{tabular}{@{\bf}l@{\hspace{5pt}}l}
\bf Entry & \hspace{.35in} {\bf Required Files}\vspace{2pt} \\
DMESS & \parbox[t]{2.7in}{\hyphenpenalty10000 \raggedright
AMACH, DMESS, MESS\rule[-5pt]{0pt}{8pt}}\\
MESS & \parbox[t]{2.7in}{\hyphenpenalty10000 \raggedright
AMACH, MESS\rule[-5pt]{0pt}{8pt}}\\
SMESS & \parbox[t]{2.7in}{\hyphenpenalty10000 \raggedright
AMACH, MESS, SMESS}\\
\end{tabular}

\begcode

\bigskip\

\centerline{{\bf \large drsmess.err}}\vspace{10pt}
\lstset{language={}}
\lstset{xleftmargin=.8in}
\lstinputlisting{\extrasloc{drsmess.err}}
\newpage

\lstset{language=[77]Fortran,showstringspaces=false}

\centerline{\bf \large DRSMESS}\vspace{5pt}
\lstinputlisting{\codeloc{smess}}

\vspace{30pt}\centerline{\bf \large ODSMESS}\vspace{10pt}
\lstset{language={}}
\lstinputlisting{\outputloc{smess}}
\end{document}
