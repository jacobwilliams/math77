\documentclass[twoside]{MATH77}
\usepackage{multicol}
\usepackage[fleqn,reqno,centertags]{amsmath}
\begin{document}
\begmath 10.2 Trigonometric, Cosine, and Sine Fourier Transforms

\silentfootnote{$^\copyright$1997 Calif. Inst. of Technology, \thisyear \ Math \`a la Carte, Inc.}

\subsection{Purpose}

This subroutine computes discrete trigonometric, cosine, and sine transforms
in up to six dimensions using the FFT. In the one dimensional case, the
values $y$ and the Fourier coefficients $\alpha $ and $\beta $ are related
by:
\begin{multline}
y_j=\frac 12\alpha _0+\,\sum_{k=1}^{(N/2)-1}\left[ \alpha _k\cos \frac{2\pi
jk}N+\beta _k\sin \frac{2\pi jk}N\right]\\
+\frac 12\alpha _{N/2}(-1)^j,\quad j=0,1,...,N-1\tag{1S}
\end{multline}\vspace{-10pt}
\renewcommand{\theequation}{\arabic{equation}A}
\begin{equation}
\begin{split}
\alpha _k&=\frac 2N\sum_{j=0}^{N-1}y_j\cos \frac{2\pi jk}N,\quad
k=0,1,...,\frac N2\\
\beta _k&=\frac 2N\sum_{j=0}^{N-1}y_j\sin \frac{2\pi jk}N,\quad
k=1,2,...,\frac N2-1
\end{split}
\end{equation}\vspace{-10pt}
\begin{multline}
y_j=\frac 12\alpha _0+\sum_{k=1}^{N-1}\alpha _k\cos \frac{\pi jk}N+\frac
12\alpha _N(-1)^j,\\
j=0,1,...,N\tag{2S}
\end{multline}\vspace{-20pt}
\begin{multline}
\alpha _k=\frac 2N\left[ \frac 12y_0+\sum_{j=1}^{N-1}y_j\cos \frac{\pi jk}%
N+\frac 12y_N(-1)^k\right] ,\\
k=0,1,...,N
\end{multline}\vspace{-20pt}
\begin{equation}
\hspace{-5pt}y_j=\sum_{k=1}^{N-1}\beta _k\sin \frac{\pi jk}N,\quad j=1,2,...,N-1%
\tag{3S}
\end{equation}\vspace{-10pt}
\begin{equation}
\hspace{-5pt}\beta _k=\frac 2N\sum_{j=1}^{N-1}y_j\sin
\frac{\pi jk}N,\quad k=1,2,...,N-1
\end{equation}
where $N =2^{\text{M}(1)}$, ($N_i=2^{\text{M}(i)}$, $i=1$, ..., ND in the
multi-dimensional case.) In the equation labels above, the numeral 1, 2,
or~3 denotes the Trigonometric, Cosine, or Sine transform, respectively,
while the letter S or A denotes Synthesis or Analysis.
\renewcommand{\theequation}{\arabic{equation}}

\subsection{Usage}

\subsubsection{Program Prototype, Single Precision, One-Dimensional Transform}

\begin{description}
\item[\bf INTEGER]  \ {\bf M}(1){\bf , ND, MS}
\item[\bf REAL]  \ {\bf A}$(\geq \mu $)\quad $[\mu =2^{\text{M}(1)}+1$
for a Cosine transform, and $=2^{\text{M}(1)}$ for a Trigonometric transform or
Sine transform.]
\item[\bf REAL]  \ {\bf S}$(\geq \nu {-}1$)\quad $[\nu =2^{\text{M}(1)-2}$ for
a Trigonometric transform and $=2^{\text{M}(1)-1}$ for a Cosine or Sine
transform.]
\item[\bf CHARACTER TCS, MODE]  \
\end{description}

On the initial call set MS to~0 to indicate the array S() does not yet
contain a sine table. Assign values to A(), TCS, MODE, M(), and ND = 1
for a one-dimensional transform.
$$
\fbox{{\bf CALL STCST(A, TCS, MODE, M, ND, MS, S)}}
$$
On return A() will contain computed values. S() will contain the sine
table used in computing the Fourier transform. MS may have been changed.

\subsubsection{Argument Definitions}

\begin{description}
\item[A()]  [inout] If MODE selects analysis, A() contains $y$ on input and $%
\alpha $\ and/or\ $\beta $ on output. If MODE selects synthesis, A()
contains $\alpha $\ and/or\ $\beta $ on input and $y$ on output. Let $N
=2^{\text{M}(1)}.$

When A() contains $y$'s, the element $y_j$ is stored in A($j+1)$.
The range of $j$ is $[0, N-1]$ for the Trigonometric transform, $%
[0, N]$ for the Cosine transform, and $[1, N-1]$ for
the Sine transform.

For the Trigonometric transform, the $\alpha $'s and $\beta $'s
are stored as A($1)=\alpha _0$, A($2)=\alpha _{N/2}$, and ${%
\textstyle A}(2k+1)=\alpha _k$, A($2k+2)=\beta _k$, $k=1$, 2, ..., ($N/2)-1.$

For the Cosine transform, the $\alpha $'s are stored as A($k+1)=\alpha _k$,
$k=0$, 1, ..., $N$.

For the Sine transform, the $\beta $'s are stored as A($k+1)=\beta
_k$, $k=1$, 2, ..., $N-1$. Although the value contained in A(1) is
irrelevant for the Sine transform, the subroutine does access this location,
so A(1) must contain a valid floating-point number on entry and its value
will generally be changed on return.

\item[TCS]  [in] The character variable TCS selects the type of transform to
be done.

$^{\prime }\text{T}^{\prime }$ or $^{\prime }\text{t}^{\prime }$ selects the Trigonometric
transform (Formula 1A or~1S).

$^{\prime }\text{C}^{\prime }$ or $^{\prime }\text{c}^{\prime }$ selects the Cosine
transform (Formula 2A or~2S).

$^{\prime }\text{S}^{\prime }$ or $^{\prime }\text{s}^{\prime }$ selects the Sine
transform (Formula 3A or~3S).

\item[MODE]  [in] The character variable MODE selects Analysis or Synthesis.

$^{\prime }\text{A}^{\prime }$ or $^{\prime }\text{a}^{\prime }$ selects Analysis (Formula
1A, 2A, or~3A).

$^{\prime }\text{S}^{\prime }$ or $^{\prime }\text{s}^{\prime }$ selects Synthesis
(Formula 1S, 2S, or~3S).

\item[M()] [in] Defines $N=2^{\text{M}(1)}$.  The number of real data
points is $N$+1, $N$, and $N{-}$1, for the Cosine, Trigonometric, and Sine
transforms respectively.  Require 0~$\leq \text{M(1)} \leq 31\ (\leq 30$
for the cosine or sine transform).  If M(1) is~0, no action is taken.

\item[ND]  [in] Number of dimensions, =1 for one-dimensional transforms.

\item[MS]  [inout] Gives the state of the sine table in S().  Let
$\text{MS}_{in}\text{ and MS}_{out}$ denote the values of MS on entry
and return respectively. If the sine table has not previously been
computed, set $\text{MS}_{in} = 0$ or $-$1 before the call. Otherwise
the value of $\text{MS}_{out}$ from the previous call using the same
S() array can be used as $\text{MS}_{in}$ for the current call.

Certain error conditions described in Section E cause the subroutine
to set $\text{MS}_{out} = -2$ and return.  Otherwise, with M(1) $>$ 0, the
subroutine sets $\text{MS}_{out} = \max (\text{M}(1), \text{MS}_{in})$
for the Trigonometric transforms and to $\text{MS}_{out} = \max
(\text{M}(1)+1, \text{MS}_{in})$ for the Cosine or Sine transforms.

If $\text{MS}_{out} > \max (2, \text{MS}_{in}),$ the subroutine sets
NT = $2^{\text{MS}_{out}-2}$ and fills S() with NT $-$ 1 sine values.

If $\text{MS}_{in}=-1$, the subroutine returns after the above
actions, not transforming the data in A().  This is intended to allow
the use of the sine table for data alteration before a subsequent Fourier
transform, as discussed in Section G of Chapter~16.0.

\item[S()]  [inout] When the sine table has been computed, S($j)=\sin \pi
j/(2\times \text{NT})$, $j=1$, 2, ..., NT $-$ 1, see MS above.
\end{description}

\subsubsection{Program Prototype, Multi-dimensional Transforms}

{\bf INTEGER} \ {\bf M}($\geq $ ND){\bf , ND, MS}

{\bf REAL} \ {\bf A}($\mu _1$, $\mu _2$, ..., $\geq \mu _{ND})$\quad $[\mu
_k=2^{\text{M}(k)}+1$ for a Cosine transform and $=2^{\text{M}(k)}$ for a
Trigonometric or Sine transform.]

{\bf REAL} \ {\bf S}($\geq \max (\nu _1$, $\nu _2$, ..., $\nu _{ND})-1)$%
\quad $[\nu _k=2^{\text{M}(k)-2}$ for a Trigonometric transform
and $=2^{\text{M}(k)-1}$
for a Cosine or Sine transform.]

{\bf CHARACTER TCS*($\geq $ ND), MODE*($\geq $ ND)}

On the initial call set MS to~0 to indicate the array S() does not yet
contain a sine table. Assign values to A(), TCS, MODE, M(), and ND.
$$
\fbox{{\bf CALL STCST(A, TCS, MODE, M, ND, MS, S)}}
$$
On return A() contains the transformed data and S() contains the sine table
used in computing the Fourier transform. The value of MS may have been
changed.

\subsubsection{Argument Definitions}

\begin{description}
\item[A()]  [inout] Array used for input and output data, see Functional
Description below for specification of the storage of items in A().

\item[TCS]  [in] The character TCS$(k$:$k)$ selects the type of transform to
be done in the $k^{th}$ dimension.

$^{\prime }\text{T}^{\prime }$ or $^{\prime }\text{t}^{\prime }$ selects the Trigonometric
transform (Formula 1A or~1S).

$^{\prime }\text{C}^{\prime }$ or $^{\prime }\text{c}^{\prime }$ selects the Cosine
transform (Formula 2A or~2S).

$^{\prime }\text{S}^{\prime }$ or $^{\prime }\text{s}^{\prime }$ selects the Sine
transform (Formula 3A or~3S).

\item[MODE]  [in] The character MODE$(k$:$k)$ selects Analysis or Synthesis
in the $k^{th}$ dimension.

$^{\prime }\text{A}^{\prime }$ or $^{\prime }\text{a}^{\prime }$ selects Analysis (Formula
1A, 2A, or~3A).

$^{\prime }\text{S}^{\prime }$ or $^{\prime }\text{s}^{\prime }$ selects Synthesis
(Formula 1S, 2S, or~3S).

\item[M()]  [in] In the $k^{th}$ dimension, $k=1$, ..., ND, define $%
N_k=2^{\text{M}(k)}$. The index range for the values $y_j$ in the $k^{th}$
dimension is $[0,N_k-1]$ for the Trigonometric transform, $[0,N_k]$ for the
Cosine transform, and $[1,N_k-1]$ for the Sine transform. Require $0 \leq
\text{M}(k)\leq 31\ (\leq 30$ for the cosine or sine transform).  If
M($k)=0$, no action is taken with respect to dimension $k$.

\item[ND]  [in] Number of dimensions, 1 $\leq $ ND $\leq $ 6.

\item[MS]  [inout]  As for MS in the one dimensional case above, except,
if there is no error and  M($k) >$ 0 for some $k$, the
subroutine sets $\text{MS}_{out} = \max
(\mu _1$, $\mu _2$, ..., $\mu_{ND})$, $\text{MS}_{in}$), where $\mu _k=
\text{M}(k)$ for the Trigonometric transform,
and $=\text{M}(k)+1$ for the Cosine or Sine transform.  NT is defined
in terms of $\text{MS}_{out}$ as described in Section~B.2 above.

\item[S()]  [inout] When the sine table has been computed, S($j)=\sin \pi
j/(2\times \text{NT})$, $j=1$, 2, ..., NT $-$ 1, see MS above.
\end{description}

\subsubsection{Modifications for Double Precision}

Change STCST to DTCST, and the REAL type statements to DOUBLE PRECISION.

\subsection{Examples and Remarks}

Given%
\begin{equation*}
f(t)=\frac{2\sinh t}{\sinh \pi t}
\end{equation*}
obtain an estimate of $\varphi (\omega )=\int_0^Tf(t)\cos \omega t\,dt$ and
compare it with $\varphi (\omega )=\sin (1)/(\cosh \omega +\cos (1))$, the true
solution. Let $T$ and $\Omega $ denote the largest values of $t$ and $\omega $
to be used in the computation. As was done below Eq.\,(18) in Chapter~16.0,
we introduce the approximations%
\begin{multline*}
\varphi (\omega ) = \int_0^Tf(t)\cos \omega t\,dt\\
\approx \frac TN\!\left[ \frac 12f(t_0)+\!\sum_{j=1}^{N-1}f(t_j)\cos
\omega t_j+\frac 12f(t_N)\cos \omega t_N\right]
\end{multline*}
where $t_j=j{T}/N$. With $\omega _k=k\pi /{T}$, we get%
\begin{equation*}\hspace{-15pt}
\varphi (\omega _k)\!\approx \!\frac T2\frac 2N\!\left[ \frac
12f(t_0)\!+\!\!\sum_{j=1}^{N-1}\!f(t_j)\cos \frac{\pi jk}N\!+\!\frac
12f(t_N)(-1)^k\!\right]
\end{equation*}
which except for the factor $T/2$ has the form of Eq.\,(2A) above. Good
results are obtained by balancing the error due to a finite $T$ with the
error due to aliasing $(i.e$. use of a finite $\Omega )$. Thus we want to
select $T$ and $\Omega $ so that%
\begin{equation}
\int_T^\infty |f(t)|\,dt\approx \frac 2{\pi -1}e^{-(\pi -1)T}\text{,
and}
\end{equation}\vspace{-10pt}
\begin{equation}
\int_\Omega^\infty |\varphi (\omega )|\,d\omega \approx 2\sin (1)e^{-\Omega}
\end{equation}
are of the same order of magnitude. Since%
\begin{equation*}
\int_T^\infty |f(t)|\,dt<10^{-4}\int_0^T|f(t)|\,dt\quad \text{for }T\geq 10.
\end{equation*}
there is no point in choosing $T>10$. With $T=10$, $N=64$ gives $\Omega =(%
N \pi /{T})=6.4\pi $. The right hand sides of Equations (4) and (5) are
then $\approx 4.7\times 10^{-10}$ and $3.1\times 10^{-9}$ respectively.

The program at the end of this chapter carries out the above calculations,
but prints results only for every $10^{th}$ value of $k$ to save space.

\subsection{Functional Description}

The one-dimensional transforms computed by this subroutine are given by
equations (1S), (1A), ..., (3S), (3A). The multi-dimensional transform is
accomplished by applying the appropriate one-dimensional transforms in each
dimension. To define the relation between input and output contents of the
A() array in the multi-dimensional case we introduce a 4-argument function,
T(*,*,*,*), in which the first argument can take the values $%
^{\prime }\text{T}^{\prime }$,$^{\prime }\text{C}^{\prime }$,
or $^{\prime }\text{S}^{\prime }$
to denote Trigonometric, Cosine, or Sine, and the second argument can take
the values $^{\prime }\text{A}^{\prime }$ or $^{\prime }\text{S}^{\prime }$
to denote Analysis or Synthesis. The third and fourth arguments are integers.
There is also an implied argument, M, which together with the first
argument contributes to the definition of $N$ and $\mu $.
\begin{equation*}
\hspace{-15pt}N=2^M
\end{equation*}\vspace{-30pt}
\begin{equation*}
\hspace{-15pt}\text{T(}^{\prime }\text{T}^{\prime }{,}^{\prime }
\text{S}^{\prime },j,k)=\begin{cases}
1/2 & k=0\\
\cos (\pi jk/N)& k=2,4,...,N\!-\!2\\
\sin (\pi j(k-1)/N)& k=3,5,...,N\!-\!1\\
(1/2)(-1)^j& k=1
\end{cases}
\end{equation*}\vspace{-10pt}
\begin{equation*}
\hspace{-15pt}\text{T(}^{\prime }\text{T}^{\prime }{,}^{\prime }
\text{A}^{\prime },j,k)\!=\!\frac 2N
\begin{cases}
\cos (\pi jk/N)&\!\!j\!=\!0,2,...,N\!-\!2 \\
\sin (\pi (j\!-\!1)k/N)&\!\!j\!=\!3,5,...,N\!-\!1 \\
(-1)^k,&\!\!j\!=\!1
\end{cases}
\end{equation*}\vspace{-15pt}
\begin{equation*}
\hspace{-15pt}\text{T(}^{\prime }\text{C}^{\prime }{,}^{\prime }
\text{S}^{\prime },j,k)=
\begin{cases}
1/2& k=0 \\
\cos (\pi jk/N)& k=1,2,...,N-1 \\
(1/2)(-1)^j& k=N
\end{cases}
\end{equation*}\vspace{-15pt}
\begin{equation*}
\hspace{-15pt}\text{T(}^{\prime }\text{C}^{\prime }{,}^{\prime }%
\text{A}^{\prime },j,k)=\frac 2N\text{T(}%
^{\prime }\text{C}^{\prime },^{\prime }\text{S}^{\prime },j,k)
\end{equation*}\vspace{-15pt}
\begin{equation*}
\hspace{-15pt}\text{T(}^{\prime }\text{S}^{\prime }{,}^{\prime }%
\text{S}^{\prime },j,k)=\sin \frac{\pi jk}N
\end{equation*}\vspace{-15pt}
\begin{equation*}
\hspace{-15pt}\text{T(}^{\prime }\text{S}^{\prime }{,}^{\prime }%
\text{A}^{\prime },j,k)=\frac 2N\text{T(}^{\prime }\text{S}^{\prime},%
^{\prime }\text{S}^{\prime },j,k)
\end{equation*}\vspace{-15pt}
\begin{equation*}
\hspace{-15pt}\mu _i=
\begin{cases}
2^{\text{M}(i)}& \text{for a Trigonometric or Sine transform} \\
2^{\text{M}(i)}+\!1& \text{for a Cosine transform}
\end{cases}
\end{equation*}

The array A has its contents replaced according to the formula%
\begin{multline*}
\hspace{-5pt}A(j_1,j_2,...,j_{ND})=\sum_{k_1=1}^{\mu _1}\cdots
\sum_{k_{ND}=1}^{\mu_{ND}}A(k_1,k_2,...,k_{ND})\\\hspace{-15pt}
\times \text{T(TCS(1:1), MODE(1:1), }j_1-1,k_1-1)\times \cdots\\
\hspace{-15pt}\times \text{T(TCS(ND:ND), MODE(ND:ND), }j_{ND}-1,k_{ND}-1),
\end{multline*}
where $j_i=1,2,...,\mu _i$ and storage conventions with respect to each
dimension are defined as for the one dimensional transforms.

The computational procedure for the trigonometric transform in the
one-dimensional case is almost identical to the procedure used in SRFT1.
(See the second paragraph below Eq.\,10 in Chapter~16.0 to see why this is
so.)

The procedure for the cosine transform uses the trigonometric transform and
the identity%
\begin{equation*}
\hspace{-0pt}\cos \frac{\pi j(2k+1)}N=\frac{\sin (2\pi j(k+1)/N)-\sin
(2\pi jk/N)}{2\sin (\pi j/N)}.
\end{equation*}
This is used to transform%
\begin{equation*}
y_j=\frac 12\eta _0+\frac 12\eta _N(-1)^j+\sum_{k=1}^{N-1}\eta _k\cos \frac{%
\pi jk}N
\end{equation*}
to%
\begin{multline*}
y_j=\frac 12\eta _0+\frac 12\eta _N(-1)^j+\sum_{k=1}^{(N/2)-1}\eta _{2k}\cos
\frac{2\pi jk}N\\
+\frac 1{2\sin (\pi j/N)}\sum_{k=1}^{(N/2)-1}(\eta
_{2k-1}-\eta _{2k+1})\sin \frac{2\pi jk}N.
\end{multline*}
Let $Y_j$ be the result from the trigonometric transform with $\alpha
_k=\eta _{2k}$, $k=0$, 1, ..., $N/2$ and $\beta _k=\eta _{2k-1}-\eta _{2k+1}$%
, $k=1,2,...,(N/2)-1$. It follows that%
\begin{eqnarray*}
   y_j+y_{N-j}   & = & Y_j+Y_{N-j} \\
   y_j-y_{N-j}  & = & \frac{Y_j-Y_{N-j}}{2\sin (\pi j/N)}
\end{eqnarray*}
and thus one can compute $y_j$ from $Y_j$. The computational
procedure is to compute the $\alpha $'s and $\beta $'s
from the $\eta $'s (the $\alpha $'s require no
computation), use the trigonometric transform to get the $Y$'s, and
the $Y$'s to compute the $y$'s. The inverse transform is
exactly the same except for a factor of 2/$N$ as is clear from
Eqs.\,(2S) and (2A).

The sine transform is obtained in a very similar way to the cosine
transform. Details can be found in \cite{Krogh:1970:SCT}.

\bibliography{math77}
\bibliographystyle{math77}

\subsection{Error Procedures and Restrictions}

Require 1 $\leq $ ND $\leq $ 6, and $0 \leq \text{M}(k)\leq 31$ for the
Trigonometric transform, $0 \leq \text{M}(k)\leq 30$ for the Cosine and
Sine transforms.  Require that TCS and MODE have only the allowed values.
On violation of any of these conditions the subroutine issues an error
message using the error processing procedures of Chapter~19.2 with
severity level $=2$ to cause execution to stop.  A return will be made
with MS $=-2$ instead of stopping if the statement ``CALL\ ERMSET($-1$)''
is executed before calling this subroutine.

If the sine table does not appear to have valid data, an error message is
printed, and the sine table and then the transform are computed.

\subsection{Supporting Information}

The source language is ANSI Fortran~77.

\begin{tabular}{@{\bf}l@{\hspace{5pt}}l}
\bf Entry & \hspace{.35in} {\bf Required Files}\vspace{2pt} \\
DTCST & \parbox[t]{2.7in}{\hyphenpenalty10000 \raggedright
 DFFT, DTCST, ERFIN, ERMSG, IERM1, IERV1\rule[-5pt]{0pt}{8pt}}\\
STCST & \parbox[t]{2.7in}{\hyphenpenalty10000 \raggedright
 ERFIN, ERMSG, IERM1, IERV1, SFFT, STCST}\\
\end{tabular}

Subroutine designed and written by: Fred T. Krogh, JPL, August~1969, revised
January~1988.


\begcode

\medskip\
\lstset{language=[77]Fortran,showstringspaces=false}
\lstset{xleftmargin=.8in}

\centerline{\bf \large DRSTCST}\vspace{10pt}
\lstinputlisting{\codeloc{stcst}}

\vspace{30pt}\centerline{\bf \large ODSTCST}\vspace{10pt}
\lstset{language={}}
\lstinputlisting{\outputloc{stcst}}
\end{document}
