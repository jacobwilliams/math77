\documentclass[twoside]{MATH77}
\usepackage{multicol}
\usepackage[fleqn,reqno,centertags]{amsmath}
\setlength{\textheight}{9in}
\setlength{\textwidth}{6.5in}
\setlength{\oddsidemargin}{0in}
\setlength{\evensidemargin}{0in}
\setlength{\topmargin}{-.5in}
\begin{document}
\begmath 8.1  Find a Zero of a Univariate Function

\silentfootnote{$^\copyright$1997 Calif. Inst. of Technology, \thisyear \ Math
  \`a la Carte, Inc.}

\subsection{Purpose}

Find a zero of a univariate function, $f(x).$

\subsection{Usage}

This subroutine uses reverse communication, $i.e.$, it returns to the
calling program each time it needs to have $f()$ evaluated at a new value of
$x.$

\subsubsection{Program Prototype, Single Precision}
\begin{description}
\item[REAL] \ {\bf X1, F1, X2, F2, TOL}
\item[INTEGER] \ {\bf MODE}
\end{description}
Assign values to X1, X2, and TOL.
\begin{tabbing}
\hspace{.2in}\=MODE = 0\\
\>F2 = The value of $f()$ evaluated at X2.\\
10\>F1 = The value of $f()$ evaluated at X1.
\end{tabbing}\vspace{-2pt}
$$
\fbox{\bf CALL SZERO(X1, F1, X2, F2, MODE, TOL)}
$$
\begin{tabbing}
\hspace{.2in}IF ( MODE .EQ. 1) go to 10
\end{tabbing}
Computed quantities are returned in MODE, X1, F1, X2, and F2.

\subsubsection{Argument Definitions}

\begin{description}

\item[X1, F1, X2, F2] \ [inout] On the initial entry, F1 $= f($X1) and F2
$= f($X2).  If F1 and F2 are of opposite sign, the zero will be found in
the interval spanned by X1 and X2.  If F1$\times \text{F2} > 0$, and X2
$\neq$ X1 then $|\text{X2}-\text{X1}|$ will be used in setting the step
size for the initial search.  If X2 = X1 on entry, then one must also have
F2 = F1, and the search is started as described in Section D.

On returns with MODE = 1, X1 is a new trial abscissa. The calling program
must set F1 $= f($X1) and call SZERO again.

On return with MODE = 2 or 3, X1 is the abscissa at which $|f()|$ had the
smallest value, and F1 $= f($X1). If F1 $\neq 0$, X2 is the nearest abscissa
to X1 at which $f()$ was evaluated and found to have the opposite sign from
F1, and F2 $= f($X2).

\item[MODE] \ [inout] Current state of computation. The user must initially set MODE
= 0, and after that should not change its value, except to control detailed
printing as explained in Section D. On each return, MODE will have one of
the following values:
\begin{itemize}
\item[$= 1$] The calling program is to compute F1 $= f($X1) and call SZERO again.

\item[$= 2$] Normal termination. Error tolerance criterion is satisfied.

\item[$= 3$] Normal termination. Error tolerance criterion is not satisfied.

\item[$= 4$] Apparently $f$ has a discontinuity between X1 and X2. No zero has been
identified.

\item[$= 5$] Set when F1$\times \text{F2} > 0$, and SZERO was unable to find function
values with different signs.

\item[$= 6$] SZERO was called with MODE $> 1$, or called initially with MODE $> 0$.
This causes execution to stop.
\end{itemize}
\item[TOL] \ [in] Error tolerance.
\begin{itemize}
\item[$>0$] the zero is to be isolated to an interval of length less than TOL.

\item[$<0$] an $x$ is desired for which $|f(x)|%
\leq |\text{TOL}|.$

\item[$=0$] the iteration continues until the zero of $f()$ is isolated as
accurately as possible. In this case SZERO will never set MODE $= 3.$
\end{itemize}
\end{description}

\subsubsection{Modifications for Double Precision}

For double-precision usage change the name SZERO to DZERO, and change the
REAL declaration to DOUBLE PRECISION.

\subsection{Examples and Remarks}

The program DRSZERO illustrates the use of SZERO to compute the root of the
function, $f(x) = 2^x - 8$, for which the exact answer is $x =
3$. Output is shown in ODSZERO.

\subsection{Functional Description}

When F1$\times \text{F2} > 0$ at the initial point, iterates are generated
according to the formula $x = x_{\min } + (x_{\min } - x_{\max }) \times
\rho$, where the subscript ``min" is associated with the $(x, f)$ pair
that has the smallest value for $|f|$, and $\rho $ is 8 if $r = f_{\min
}\:/\:(f_{\max } - f_{\min }) \geq 8$, else $\rho = \max (\kappa /4, r)$,
where $\kappa $ is a count of the number of iterations that have been
taken without finding $f$'s with opposite signs.  If X1 and X2 have the
same value initially (and F1 and F2 equal F(X1)), then the next $x$ is a
distance $0.008 + |x_{\min }|/4$ from $x_{\min }$ taken toward~0.  (If
X1 = X2 = 0, the next $x$ is $-$.008.)

Let $x_1$ and $x_2$ denote the first two $x$ values that give $f$ values
with different signs. Let $A < B$ be the two values of $x$ that bracket the
zero as tightly as is known. Thus $A = x_1$ or $A = x_2$ and $B$ is the
other when computing $x_3$. The next point $x_3$ is generated treating $x$
as the linear function $q(f)$ that interpolates the points $(f(x_1)$, $x_1)$
and $(f(x_2)$, $x_2)$, and computing $x_3 = q(0)$, subject to the condition
that $A+\varepsilon  \leq x_3 \leq B-\varepsilon $,
where $\varepsilon  = 0.875$ times the accuracy to which the root has been
requested. (This condition on $x_3$ with updated values for $A$ and $B$
is also applied to future iterates.)

Let $x_4$, $x_5$, ..., $x_m$ denote the abscissae on the following
iterations. Let $a = x_m$, $b = x_{m-1}$, and $c = x_{m%
-2}$. Either $A$ or $B$ (defined as above) will coincide with $a$,
and $B$ will frequently coincide with either $b$ or $c$. Let $p(x)$ be the
quadratic polynomial in $x$ that passes through the values of $f$ evaluated
at $a$, $b$, and $c$. Let $q(f)$ be the quadratic polynomial in $f$ that
passes through the points $(f(a), a)$, $(f(b), b)$, and $(f(c), c).$

Let $C$ = $A$ or $B$, selected so that $C \neq a$. If the sign of $f$ has
changed in the last 4~iterations and $p^{\prime}(a)\times q^{\prime}(f(a))$
and $p^{\prime}(C)\times q^{\prime}(f(C))$ are both in the interval $[1/4$,
4], then $x$ is set to $q(0)$. (Note that if $p$ is replaced by $f$ and $q$
is replaced by $x$, then both products have the value~1.) Otherwise $x$ is
set to $a - (a - C)(\varphi /(1+\varphi ))$, where $%
\varphi $ is selected based on past behavior and on the ratio of $f(a)$ with
values of $f()$ having the same sign as $f(a)$, and different sign from $f(a)
$, evaluated at nearby values of $x$. The method of selecting $\varphi $ is
sufficiently complicated that we simply refer interested readers to the
code. The algorithm is such that $0 < \varphi $, and if the sign of $f()$ does
not change for an extended period, $\varphi $ gets large.

Reference \cite{Alefeld:1995:AEZ} compares a number of algorithms for
finding a zero of a continuous function.  Dr.\ Shi has kindly used the
same test program, ENCL0FX, on DZERO.  With Dr.\ Shi's permission, results
from Table II of that paper are given on the next page with an additional
column added for the results he obtained with DZERO.  With the exception
of DE and M, all codes solved all of the problems.  See \refm{1} for more
details.

\subparagraph{Detailed printing}

Before the initial call with MODE $= 0$, or at any time during the
iterative process, the user may set a counter for detailed output by
calling SZERO with a negative value of MODE.  There is no problem-solving
action on such a call.  A saved counter is set so detailed output will be
written using the message processor described in Chapter~19.3 on the next
$|$MODE$|-1$ normal calls.  To resume normal computation with the detailed
print on, the calling program must set MODE $= 0$ if a new problem
sequence is being started or MODE\ $= 1$ if an iterative sequence is being
continued.  (Detailed print can be turned off by setting MODE $= -1$ as is
implied by the above.)

The detailed output consists of X1, F1, KTYP, DIV, and KS. KTYP is 0 if $\varphi $
above was used on the previous iteration, and KTYP is 1 if $x$ was set to $%
q(0)$. DIV is the name of the program variable corresponding to $\varphi $, and
KS is the number of iterations since there has been a change in the sign of $%
f().$

\bibliography{math77}
\bibliographystyle{math77}

\subsection{Error Procedures and Restrictions}

The user must initially set MODE $= 0$ and must not alter MODE after that
while iterating on the same problem (except as described in Section D for
detailed output.) Entering SZERO with MODE $> 0$ when SZERO has not set MODE
to~1 results in an error condition and an error message will be issued using
the system error handler SMESS (or DMESS). In such a case, if MODE $\neq 6$,
SZERO will set MODE $= 6$ and return, whereas if MODE $= 6$, SZERO will STOP.

SZERO uses the Fortran~77 SAVE statement to save values of internal
variables between the successive calls needed to solve a problem. Thus SZERO
cannot be used to work on more than one problem at a time. In particular,
calling SZERO with MODE $= 0$ will always initialize it for a new problem
and no data will be retained from a previous problem (except for the
internal detailed print counter described in Section D.)

\subsection{Supporting Information}

The source language is ANSI Fortran~77.

\begin{tabular}{@{\bf}l@{\hspace{5pt}}l}
\bf Entry & \hspace{.35in} {\bf Required Files}\vspace{2pt} \\
DZERO & \parbox[t]{2.7in}{\hyphenpenalty10000 \raggedright
AMACH, DMESS, DZERO, MESS\rule[-5pt]{0pt}{8pt}}\\
SZERO & \parbox[t]{2.7in}{\hyphenpenalty10000 \raggedright
AMACH, MESS, SMESS, SZERO}\\
\end{tabular}

Algorithm and code due to F. T. Krogh, JPL, April~1972. Revised to improve
portability, April~1974. Name changed from SFABZ/DFABZ to SZERO/DZERO,
and minor improvements to the algorithm, September~1987.

Revised to allow F1$\times \text{F2} > 0$ on initial entry, and to change error
handling, November~1991.


\begcodenp

\begin{center}

\begin{tabular}{crrrrrrrrrrrrr}
\multicolumn{14}{c}{Total Number of Function Evaluations in
Solving All the Problems Listed in Table I of
\refm{1}\rule[-5pt]{0pt}{10pt}} \\
{\em tol} &
\multicolumn{1}{c}{BR} &

\multicolumn{1}{c}{DE} &
\multicolumn{1}{c}{M} &
\multicolumn{1}{c}{R} &
\multicolumn{1}{c}{LE} &
\multicolumn{1}{c}{2.1} &
\multicolumn{1}{c}{2.2} &
\multicolumn{1}{c}{2.3} &
\multicolumn{1}{c}{2.4} &
\multicolumn{1}{c}{2.5} &
\multicolumn{1}{c}{4.1} &
\multicolumn{1}{c}{4.2} &
\multicolumn{1}{c}{DZERO}\\
$10^{-7}$   & 2804 & 2808 & 2839 & 7630 & 2694 & 3154 & 2950 &
2645 & 2791 & 2687 & 2696 & 2650 & 2100 \\
$10^{-10}$  & 2905 & 2963 & 2992 & 7768 & 2821 & 3338 & 3060 &
2789 & 2922 & 2819 & 2835 & 2786 & 2177 \\
$10^{-15}$  & 2975 & 3196 & 3261 & 8014 & 3061 & 3448 & 3151 &
2948 & 3015 & 2914 & 2908 & 2859 & 2236 \\
0           & 3008 & 2998 & 3146 & 8230 & 3165 & 3509 & 3219 &
3029 & 3060 & 2954 & 2950 & 2884 & 2255
\end{tabular}\\[.5in]
\end{center}

\lstset{language=[77]Fortran,showstringspaces=false}
\lstset{xleftmargin=.8in}

\centerline{\bf \large DRSZERO}\vspace{10pt}
\lstinputlisting{\codeloc{szero}}

\vspace{30pt}\centerline{\bf \large ODSZERO}\vspace{10pt}
\lstset{language={}}
\lstinputlisting{\outputloc{szero}}
\end{document}

