\documentclass[twoside]{MATH77}
\usepackage[\graphtype]{mfpic}
\usepackage{multicol}
\usepackage[fleqn,reqno,centertags]{amsmath}
\begin{document}
\opengraphsfile{pl02-03}
\begmath 2.3 Gamma and Log-Gamma Functions

\silentfootnote{$^\copyright$1997 Calif. Inst. of Technology, \thisyear \ Math \`a la Carte, Inc.}

\subsection{Purpose}

These subprograms compute values of the gamma function and the natural
logarithm of the gamma function, \cite{ams55} and~\cite{Hart:1968:CA:gam}.
The gamma function is defined by%
\begin{equation*}
\Gamma (x)=\int_0^\infty e^{-s}s^{x-1}ds.
\end{equation*}
For integer values of $n\geq 0$, it satisfies the relation
\begin{equation*}
n!=\Gamma (n+1).
\end{equation*}
\subsection{Usage}

\subsubsection{Program Prototype, Single Precision}

\begin{description}
\item[REAL]  \ {\bf SGAMMA, SLGAMA, X, Y}
\end{description}

Assign a value to X and obtain gamma or the natural logarithm of gamma,
respectively, through use of the function statements:
$$
\fbox{{\bf Y =SGAMMA(X)}} \hspace{.5in} \fbox{{\bf Y =SLGAMA(X)}}
$$

\subsubsection{Argument Definition}

\begin{description}
\item[X]  \ [in] Argument at which function evaluation is desired. See
Section E for restrictions on X.
\end{description}

\subsubsection{Modifications for Double Precision}

For double precision usage, change the REAL type statement to DOUBLE
PRECISION and change the function names from SGAMMA to DGAMMA and SLGAMA to
DLGAMA.

\subsubsection{Program Prototype, Complex}

\begin{description}
\item[INTEGER]  \ {\bf MODE}

\item[COMPLEX]  \ {\bf CARG, CVAL}

\item[REAL]  \ {\bf ERREST}
\end{description}

Assign the complex argument in CARG. Set MODE to 0 or 1.

\begin{center}
\fbox{%
\begin{tabular}{@{\bf }c}
CALL CGAM(CARG, CVAL,\\
ERREST, MODE)
\end{tabular}}
\end{center}

The complex result will be in CVAL with an error estimate in ERREST.

\subsubsection{Argument Definitions}

\begin{description}
\item[CARG]  \ [in] Complex argument. The argument must not be zero or a
negative real integer value.

\item[CVAL]  \ [out] Computed complex value of either log-gamma or gamma.
The imaginary part of log-gamma, say $v$, will be standardized to satisfy $%
-\pi <v\leq \pi .$

\item[ERREST]  \ [out] On return ERREST gives an estimate of the absolute
error (for log-gamma) or the relative error (for gamma) of the computed
value.

\item[MODE]  \ [in] Set value to 0 for log-gamma and 1 for gamma.
\end{description}

\subsubsection{Program Prototype, Double Precision Complex}

\begin{description}
\item[INTEGER]  \ {\bf MODE}

\item[DOUBLE PRECISION]  \ {\bf CARG}(2){\bf , CVAL}(2){\bf , ERREST}
\end{description}

Assign the complex argument in CARG(), with real part in CARG(1) and
imaginary part in CARG(2). Set MODE to 0 or 1.

\begin{center}
\fbox{%
\begin{tabular}{@{\bf }c}
CALL ZGAM(CARG, CVAL,\\
ERREST, MODE)
\end{tabular}}
\end{center}

The result will be in CVAL() with an error estimate in ERREST.

\subsubsection{Argument Definitions}

\begin{description}
\item[CARG()]  \ [in] Array of 2 values representing the complex argument,
CARG(1) for the real part and CARG(2) for the imaginary part. The argument
must not be zero or a negative real integer value.

\item[CVAL()]  \ [out] Array of 2 values representing the complex function
value returned: CVAL(1) and CVAL(2) for real and imaginary parts,
respectively. The imaginary part of log-gamma, say $v$, will be standardized
to satisfy $-\pi <v\leq \pi .$

\item[ERREST]  \ [out] On return ERREST gives an estimate of the absolute
error (for log-gamma) or the relative error (for gamma) of the answer.

\item[MODE]  \ [in] Set value to 0 for log-gamma and 1 for gamma.
\end{description}

Although the Fortran 77 standard does not support a double precision complex
type, many vendors do provide for this using a declaration COMPLEX*16 or
DOUBLE COMPLEX. With such a compiler one can probably use ZGAM with its
first two arguments declared as double precision complex.

\subsection{Examples and Remarks}

\subsubsection{Example}

The gamma function satisfies a ``duplication'' identity which may be written
as%
\begin{equation*}
z=\frac{2\sqrt{\pi }\Gamma (x)}{2^x\Gamma (x/2)\Gamma (x/2+\frac 12)}-1=0.
\end{equation*}
The listing of DRSGAMMA and ODSGAMMA gives an example of using these
subprograms to evaluate this identity. The program DRCGAM and output ODCGAM
illustrate the use of CGAM.

\subsubsection{Remarks}

Neither the Fortran 77 standard nor the Fortran 90 standard includes
intrinsic functions for gamma or log-gamma; however, such functions with the
names GAMMA, DGAMMA, LGAMMA, ALGAMA, and DLGAMA are provided as intrinsic
functions in some vendors' Fortran systems. In particular these latter
functions are provided with the UNISYS ASCII Fortran and IBM VS-Fortran, but
not with the VAX-11 Fortran. In the UNISYS and IBM systems GAMMA and LGAMMA
are ``generic" names and thus can be used with either single or double
precision arguments.

In a system having DGAMMA or DLGAMA as intrinsics a reference to one of
these function names will cause the vendor-supplied code to be used. If one
wishes to override this and use the code from this library one must declare
the function name to be EXTERNAL in the referencing program unit.

\subsection{Functional Description}

The gamma function is defined and takes positive values for all real $x>0$,
becoming unbounded both as $x\rightarrow 0$ and as $x\rightarrow +\infty $.
As $x\rightarrow 0$, $\Gamma (x)$ is asymptotic to $1/x$, and as $%
x\rightarrow \infty $, $\Gamma (x)$ is asymptotic to%
\begin{equation*}
h(x)=\left( x/e\right) ^x\sqrt{2\pi /x}.
\end{equation*}
As an estimate of the size of $\Gamma (x)$ for large positive real $x$, note
that for $x>9$, $\Gamma (x)$ satisfies $h(x)<\Gamma (x)<1.01\ h(x)$. The
gamma function is also defined for negative $x$ except at the negative
integers where it has poles.

As a function of a complex variable the gamma function is analytic
throughout the complex plane except for poles at zero and the negative real
integer points. See \cite{ams55} and \cite{Hart:1968:CA:gam} for further
discussion of the gamma function.

The subprograms SGAMMA, DGAMMA, SLGAMA, and DLGAMA are based on subprograms
developed by W.J. Cody, Argonne National Laboratory, which were designed for
$10^{-20}$ precision. These subprograms use rational minmax approximations
on selected subintervals as well as asymptotic formulas and argument
reduction techniques. The degrees of the rational approximations are not
varied between the single precision and double precision subprograms or as
a function of the host computer's precision.
\vspace{10pt}

\hspace{5pt}\mbox{\input pl02-03a }

Subprograms CGAM and ZGAM are based on \cite{Kuki:1972:CGF}, making use of
the Stirling asymptotic series and recursive formulas.  The accuracy
adjusts to the host machine up to about 17 significant decimal digits.
CGAM and ZGAM reference I1MACH(10) to find the radix of the host system's
floating point arithmetic.

\subparagraph{Accuracy tests}

Accuracy tests were run on a Univac 1108 in 1969 and on a Univac 1100 in
1983. The arithmetic precision of these systems is $\rho _1 = 2^{-27}
\approx 0.$ 745E$-$8 for single precision and $\rho _2 = 2^{-60}
\approx 1.15$E$-$18 for double precision.

Subprograms SGAMMA and SLGAMA were tested in 1983 on a Univac 1100 by
comparison with DGAMMA and DLGAMA at 18,000 points. The test results may be
summarized as follows:

\begin{center}
\begin{tabular}{l@{}lr}
& \multicolumn{1}{c}{\bf Argument} & \multicolumn{1}{c}{\bf Max. Rel.}\\
\multicolumn{1}{l}{\bf Function} & \multicolumn{1}{c}{\bf Interval} &
\multicolumn{1}{c}{\bf Error}\\
SGAMMA & \rule{.17in}{0pt}[0., 1.] & $2.5\rho _1$\rule{.12in}{0pt}\\
& \rule{.17in}{0pt}[1., 2.] & $1.4\rho _1$\rule{.12in}{0pt}\\
& \rule{.17in}{0pt}[2., 10.] & $9.2\rho _1$\rule{.12in}{0pt}\\
& \rule{.17in}{0pt}[10., 17.] & $100.0\rho _1$\rule{.12in}{0pt}\\
& \rule{.17in}{0pt}[17., 30.] & $180.0\rho _1$\rule{.12in}{0pt}\\
SLGAMA & \rule{.17in}{0pt}[0.0, 0.5] & $2.7\rho _1$\rule{.12in}{0pt}\\
& \rule{.17in}{0pt}[0.5, 4.0] & $6.5\rho _1$\rule{.12in}{0pt}\\
& \rule{.17in}{0pt}[4.0, 35.0] & $2.8\rho _1$\rule{.12in}{0pt}
\end{tabular}
\end{center}

To test the double precision functions, and as an additional test of the
single precision functions, the function $z(x)$ defined in Section C and
a logarithmic form of $z(x)$ defined as

\hspace{.2in}$w(x) =\text{LGAM}(x) + 0.5\ \ln (2\pi ) + (0.5 - x) \ln(2)$
\vspace{-2pt}

\hspace{.8in}$- \text{LGAM}(0.5x) - \text{LGAM}(0.5x + 0.5) = 0$

were evaluated on a Univac 1100 in 1983 at about 100 points. These tests are
summarized as follows (``Max. Error" is the maximum magnitude of
the test function):

\begin{tabular}{@{}l@{}l@{\ \ }c@{ }r}
\multicolumn{1}{@{}c}{\bf Subprogram} & \multicolumn{1}{c}{\bf Argument} &
\multicolumn{1}{c}{\bf Test} & \multicolumn{1}{c}{\bf Max.}\\
\multicolumn{1}{@{}c}{\bf Used} & \multicolumn{1}{c}{\bf Interval} &
\multicolumn{1}{c}{\bf Function} & \multicolumn{1}{c}{\bf Error}\\
SGAMMA & [$-$20.75, $-$11.25] & $z$ & $114\rho _1$\\
 & [$-$11.25, 11.50] & $z$ & $6\rho _1$\\
 & [11.50, 34.00] & $z$ & $100\rho _1$\\
DGAMMA & [$-$20.75, 43.60] & $z$ & $130\rho _2$\\
 & [43.60, 106.80] & $z$ & $384\rho _2$\\
 & [106.80, 170.00] & $z$ & $1186\rho _2$\\
SLGAMA & [0.0625, 3.1] & $w$ & $3\rho _1$\\
 & [3.1, 11.5] & $w$ & $8\rho _1$\\
 & [11.5, 0.12E37] & $w/\text{SLGAMA}$ & $2.7\rho _1$\\
DLGAMA & [0.0625, 12.0] & $w$ & $110\rho _2$\\
 & [12.0, 0.77D305] & $w/\text{DLGAMA}$ & $2.9\rho _2$
\end{tabular}

Subprogram ZGAM was tested in 1969 on a Univac 1108. The maximum relative
error noted for complex arguments, $x$, satisfying $|x| < 30$, was $345\rho
_2$ for gamma and $93\rho _2$ for log-gamma. The test function, $z$, was
evaluated at a few complex points, $x$, of magnitude near one on a Univac
1100 in 1983. The maximum value of $|z|$ was $30\rho _2.$

\bibliography{math77}
\bibliographystyle{math77}

\subsection{Error Procedures and Restrictions}

Subprograms SGAMMA and DGAMMA accept negative as well as positive arguments.
These subprograms will issue an error message and return the largest machine
number, $\Omega $, if $x$ is zero or a negative integer, or if $x \geq x_g.$

The subprograms SLGAMA and DLGAMA are designed only for positive arguments.
These subprograms will issue an error message and return the value, $\Omega $%
, if $x \leq 0$, or if $x \geq x_\lambda .$

The complex gamma subroutines CGAM and ZGAM issue an error message and
return the complex value $(\Omega $,\ $\Omega )$ if the argument is zero, a
negative real integer, or a large argument that would cause overflow.

Let $\Omega $ denote the machine overflow limit. Let $x_g$ denote the value
of $x$ for which $\Gamma (x) = 0.875\ \Omega $, and let $x_\lambda $ denote
the value of $x$ for which $\ln (\Gamma (x)) = 0.875\ \Omega $. Some
examples of these values follow.

\begin{center}
\begin{tabular}{lrl}
{\bf Computing System}&{\bf $x_g $}\rule{3pt}{0pt}&\ \ \ {\bf $x_\lambda $}\\
VAX 11/780, SP \& DP & 34.8 & 1.8e36\\
UNISYS (Sperry) 1100, SP & 34.5 & 1.8e36\\
IEEE Math processor, SP & 35.0 & 3.6e36\\
IBM Mainframe 3xxx, DP & 57.5 & 3.8e73\\
UNISYS (Sperry) 1100, DP & 171.5 & 1.1e305\\
IEEE Math processor, DP & 171.6 & 2.2e305\\
Cray, SP \& DP & 966.9 & 8.5e2461
\end{tabular}
\end{center}

The values listed above are computed on a first-time flag at run time making
use of the overflow limit obtained from R1MACH(2) or D1MACH(2).

\subsection{Supporting Information}

\begin{tabular}{@{\bf}l@{\hspace{5pt}}l}
\bf Entry & \hspace{.35in} {\bf Required Files}\vspace{2pt} \\
CGAM & \parbox[t]{2.7in}{\hyphenpenalty10000 \raggedright
AMACH, CGAM, ERFIN, ERMSG, SERM1, SERV1\rule[-5pt]{0pt}{8pt}}\\DGAMMA & \parbox[t]{2.7in}{\hyphenpenalty10000 \raggedright
AMACH, DERM1, DERV1, DGAMMA, ERFIN, ERMSG\rule[-5pt]{0pt}{8pt}}\\DLGAMA & \parbox[t]{2.7in}{\hyphenpenalty10000 \raggedright
AMACH, DERM1, DERV1, DGAMMA, DLGAMA, ERFIN, ERMSG\rule[-5pt]{0pt}{8pt}}\\SGAMMA & \parbox[t]{2.7in}{\hyphenpenalty10000 \raggedright
AMACH, ERFIN, ERMSG, SERM1, SERV1, SGAMMA\rule[-5pt]{0pt}{8pt}}\\SLGAMA & \parbox[t]{2.7in}{\hyphenpenalty10000 \raggedright
AMACH, ERFIN, ERMSG, SERM1, SERV1, SGAMMA, SLGAMA\rule[-5pt]{0pt}{8pt}}\\ZGAM & \parbox[t]{2.7in}{\hyphenpenalty10000 \raggedright
AMACH, DERM1, DERV1, ERFIN, ERMSG, ZGAM\rule[-5pt]{0pt}{8pt}}\\\end{tabular}

Subprograms SGAMMA, DGAMMA, SLGAMA, and DLGAMA were developed by W.J. Cody,
Argonne National Lab., 1982, and adapted to the JPL MATH 77 library by C.
Lawson and S. Chiu, JPL, 1983.

Subroutines CGAM and ZGAM were developed by H. Kuki with the name CDLGAM
(\cite{Kuki:1972:CGF} above), adapted for JPL Univac 1108 usage by E.W.~%
Ng, 1969, and adapted to the JPL MATH 77 library by C.~Lawson and S.~%
Chiu, JPL, 1983.


\begcodenp

\lstset{language=[77]Fortran,showstringspaces=false}
\lstset{xleftmargin=.8in}

\centerline{\bf \large DRSGAMMA}\vspace{10pt}
\lstinputlisting{\codeloc{sgamma}}

\vspace{20pt}\centerline{\bf \large ODSGAMMA}\vspace{10pt}
\lstset{language={}}
\lstinputlisting{\outputloc{sgamma}}

\newpage

\lstset{language=[77]Fortran,showstringspaces=false}
\lstset{xleftmargin=.8in}

\centerline{\bf \large DRCGAM}\vspace{10pt}
\lstinputlisting{\codeloc{cgam}}

\vspace{20pt}\centerline{\bf \large ODCGAM}\vspace{10pt}
\lstset{language={}}
\lstinputlisting{\outputloc{cgam}}

\closegraphsfile
\end{document}
