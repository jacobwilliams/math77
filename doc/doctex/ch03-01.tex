\documentclass[twoside]{MATH77}
\usepackage{multicol}
\usepackage[fleqn,reqno,centertags]{amsmath}
\begin{document}
\begmath 3.1 Uniform Random Numbers

\silentfootnote{$^\copyright$1997 Calif. Inst. of Technology, \thisyear \ Math \`a la Carte, Inc.}

\subsection{Purpose}

Generate pseudorandom numbers from the uniform distribution. Capabilities
are also provided for optionally setting and fetching the ``seed" of the
generator.

\subsection{Usage}

\subsubsection{Generating uniform pseudorandom numbers}

Three subprograms are provided for generation of single precision uniform
random numbers:

\begin{description}
\item[X = SRANU()]  \ Returns one random number in [0,~1].

\item[call SRANUA(XTAB, N)]  \ Returns an array of N random numbers in
[0,~1].

\item[call SRANUS(XTAB, N, A, B)]  \ Returns an array of N numbers scaled
as A + B $\times $ U where U is random in [0,~1].
\end{description}

Corresponding double precision subprograms are also provided.

\paragraph{Program Prototype, A Single Random Number, Single Precision}

\begin{description}
\item[REAL]  \ {\bf SRANU, X}
\end{description}
$$
\fbox{{\bf X = SRANU()}}
$$
\subparagraph{Argument Definitions}

\begin{description}
\item[SRANU]  \ [out] The function returns a pseudorandom number from the
uniform distribution on [0.0,~1.0].
\end{description}

\paragraph{Program Prototype, An Array of Random Numbers, Single Precision}

\begin{description}
\item[INTEGER]  \  N

\item[REAL]  \ {\bf XTAB}($\geq $N)
\end{description}

Assign a value to N.
$$
\fbox{{\bf CALL SRANUA(XTAB, N)}}
$$
Computed values will be returned in XTAB().

\subparagraph{Argument Definitions}

\begin{description}
\item[XTAB()]  \ [out] Array into which the subroutine will store N
pseudorandom samples from the uniform distribution on [0.0,~1.0].

\item[N]  \ [in] Number of pseudorandom numbers requested. The subroutine
returns immediately if N $\leq 0.$
\end{description}

\paragraph{Program Prototype, An Array of Scaled Random Numbers, Single
Precision}

\begin{description}
\item[INTEGER]  \ {\bf N}

\item[REAL]  \ {\bf XTAB}($\geq $N){\bf , A, B}
\end{description}

Assign values to N, A, and B.
$$
\fbox{{\bf CALL SRANUS(XTAB, N, A, B)}}
$$
Computed values will be returned in XTAB().

\subparagraph{Argument Definitions}

\begin{description}
\item[XTAB()]  \ [out] Array into which the subroutine will store N numbers
computed as A + B $\times $ U, where for each number, U is a pseudorandom
sample from a uniform distribution on [0.0,~1.0].

\item[N]  \ [in] Number of pseudorandom numbers requested. The subroutine
returns immediately if N $\leq 0.$

\item[A, B] \ [in] Numbers defining the linear transformation (A + B
  $\times$ U) to be applied to the random numbers.
\end{description}

\subsubsection{Modifications for Double Precision}

For double precision usage change the REAL type statements above to DOUBLE
PRECISION and change the initial ``S" of the function and subroutine names
to ``D." Note particularly that if the function name, DRANU, is used it must
be typed DOUBLE PRECISION either explicitly or via an IMPLICIT statement.

\subsubsection{Operations relating to the seed}

The handling of the seed is modeled on the function RANDOM\_SEED which is a
new intrinsic function introduced in Fortran~90. Random number generation
does not require any initialization calls by the user, but initialization
capabilities are provided in case they are wanted.

The seed for random number generation is a set of KSIZE numbers of type
INTEGER. The value of KSIZE depends on the algorithm and implementation used
for generating uniform random numbers.
\begin{description}
\item[call RANSIZ(KSIZE)] \ Returns the value of KSIZE for the current library
implementation.

\item[call RAN1] \ Sets the seed to its default initial value.

\item[call RANPUT(KSEED)] \ Sets the seed to the array of KSIZE values given in
KSEED().

\item[call RANGET(KSEED)] \ Fetches the current seed into the array KSEED().
\end{description}

Besides resetting the seed, RAN1 and RANGET set values in common that have
the effect of reinitializing all of the pseudorandom number generators of
Chapters~3.1, 3.2 and~3.3.

If one needs to produce the same sequence of pseudorandom numbers more than
once within the same run, a suggested approach is to initialize the package
at each point in the computation where the sequence is to be started or
restarted. One could either use RAN1 to initialize the package to its
standard starting seed or use RANPUT to initialize the package to a seed
selected by the user.

The seed returned by RANGET may be the seed associated with the next uniform
number that will be returned by the package, but generally this will not be
the case. Due to buffering within the package this seed may be associated
with a uniform number that will be returned some tens of requests later.

A potential use for the RANGET function would be to assure a different set
of random numbers on a subsequent run. Thus one could use RANGET at the end
of a run and write to a file the seed value returned by RANGET. Then on a
subsequent run one could read this seed from the file and use RANPUT to
initialize the package to this seed value. This would assure a new sequence
of numbers.

\paragraph{Program Prototype, Get the value of KSIZE}

\begin{description}
\item[INTEGER]  \ {\bf KSIZE}
\end{description}
$$
\fbox{{\bf CALL RANSIZ(KSIZE)}}
$$
A value will be returned in KSIZE.

\subparagraph{Argument Definitions}

\begin{description}
\item[KSIZE]  \ [out] The subroutine sets KSIZE to the number of integers
needed to constitute a seed for the current library implementation of a
random number generation algorithm. The user should use this information to
verify that the dimension of the array KSEED() is adequate before calling
RANPUT or RANGET. The preferred algorithm in the MATH77 library has KSIZE =
2. If this is replaced by a different algorithm KSIZE could change.
\end{description}

\paragraph{Program Prototype, Set seed to default value}\vspace{-10pt}
$$
\fbox{{\bf CALL RAN1}}
$$
\subparagraph{Argument Definitions}

This subroutine has no arguments. It causes the seed stored in the random
number generation code to be reset to its default initial value. It also
sets values in common that have the effect of reinitializing all of the
pseudorandom number generators of Chapters~3.1, 3.2, and~3.3.

\paragraph{Program Prototype, Set the seed}

{\bf INTEGER} {\bf KSEED}($\geq $KSIZE)

Assign values to KSEED().
$$
\fbox{{\bf CALL RANPUT(KSEED)}}
$$
\subparagraph{Argument Definitions}

\begin{description}
\item[KSEED()] \ [in] Array of KSIZE integers to be used to set a new seed value. Any
integer values are acceptable. If the given values do not conform to
internal requirements the subroutine will derive usable values from the
given values.
\end{description}

In the preferred MATH77 implementation the internal integer sequence
consists of numbers in the range from~1 to~68719476502. For example, to set
the seed to the value 10987654321 one should set KSEED(1) = 109876 and
KSEED(2) = 54321. In general RANPUT will compute KSEED(1$) \times 10^5 +
\text{KSEED}(2)$ using either single precision or double precision
arithmetic, depending on the ``mode"
described in Section D, and then alter the result, if necessary, to obtain a
seed in the range from~1 to~68719476502.

This subroutine also sets values in common that have the effect of
reinitializing all of the pseudorandom number generators of Chapters~3.1,
3.2, and~3.3.

\paragraph{Program Prototype, Get the seed}

{\bf INTEGER} {\bf KSEED}($\geq $KSIZE)
$$
\fbox{{\bf CALL RANGET(KSEED)}}
$$
Values will be returned in KSEED(). See the discussion at the beginning of
Section B.2 for information on the applicability of this subprogram.

\subparagraph{Argument Definitions}

KSEED() [out] Array into which the subroutine will store the KSIZE integers
constituting the current seed.

\subsection{Examples and Remarks}

DRSRANU demonstrates the use of SRANU to compute uniform random numbers and
uses SSTAT1 and SSTAT2 to compute and print statistics and a histogram based
on a sample of~10000 numbers delivered by SRANU.

The uniform distribution on [0,~1] has mean 0.5 and standard deviation $%
\sqrt{1/12} \approx 0.288675.$

The smallest number that can be produced by SRANU, DRANU, SRANUA, or DRANUA
is approximately $0.15 \times 10^{-10}$. The largest value that can be
produced is approximately $1.0 - 0.15 \times 10^{-10}$. The single precision
subprograms SRANU and SRANUA will return this largest value as exactly 1.0
on many computer systems.

DRDRAN provides a critical test of the correct performance of the
core integer sequence generator on whatever host system it is run.
The seed values are set to cause the generation of the largest and
smallest numbers possible in the underlying integer sequence. This
program is expected to generate exactly the same values in the column
headed ``Integer sequence" on all compiler/computer systems. DRDRAN
also calls RN2 to show the value of MODE (described below in Section
D) being used on the host system. See the output listing, ODDRAN, for
results.

To compute random numbers, uniform in [C, D], one can use the statement

\hspace{.2in}X =C + (D $-$ C) * SRANU()

or to put N such numbers into an array XTAB() one can write

\hspace{.2in}call SRANUS(XTAB, N, C, D $-$ C)

To compute random INTEGER's in the range from I1 through I2, (I1 $<$ I2),
with equal probability, one can write
\begin{tabbing}
\hspace{.2in}\=FAC = real(I2 $-$ I1 + 1)\\
\>K = min(I2, I1 + int(FAC * SRANU() ) )
\end{tabbing}
The min function is used in the above statement because SRANU will, with
very low probability, return the exact value~1.0.

If one needs to compute many random numbers and execution time is critical,
one should note that one call to SRANUA with a sizeable value of N will take
less execution time than N references to SRANU. For even greater efficiency
one could write the random number generation in line, since it only requires
a few declarations and a few executable statements. When and if Fortran~90
compilers come into widespread usage it will probably be more efficient to
use the new intrinsic subroutine RANDOM\_NUMBER, although this subroutine is
not specified to generate the same sequence on different computers.

\subsection{Functional Description}

\subsubsection{The core algorithm for generation of uniform pseudorandom
numbers}

A sequence of integer values, $k_i$, is generated by the equation
\begin{equation}
\label{O1}k_i = ak_{i-1}\text{ mod }\ m
\end{equation}
The rational number, $k_i/m$, is returned as a pseudorandom number from the
uniform distribution on [0,~1].

When $m$ is prime and $a$ is a primitive root of $m$, this integer
sequence has period $m-1$, attaining all integer values in the range [1,
$m-1]$.  According to \cite{Knuth:1981:ACP}, this sequence will have good
equidistribution properties, at least in dimensions up to $d$, if the
numbers $\nu _i$ and $\mu _i$, $i = 2$, ..., $d$, that are functions of
$m$ and $a$, are not exceptionally small.  Finding satisfactory values of
$\nu _i$ and $\mu _i$ is simplified by having $m$ large and $a$ not too
small.  The size of $m$ and $a$ is limited however by the requirement of
computing Eq.\,(1) exactly at a reasonable cost.

We have determined a pair of integers $m$ and $a$ that satisfy all these
requirements.  The values of $\nu _i$ and $\mu _i$, $i = 2$, ..., 6,
attained are excellent compared with any of the 30 $(m, a)$ pairs listed
in Table~1, pp.~102--103 of \cite{Knuth:1981:ACP}, which includes pairs
used in a number of widely distributed random number generation
subprograms.  We use the values $$ \text{MDIV} = m = 6\_87194\_76503 =
2^{36} - 233 $$ and $$ \text{AFAC} = a = 612\_662 \approx 0.58 \times
2^{20} $$ The number $m-1$ is the product of three primes, $p(i)$, listed
here with other relevant number-theoretic values.

\begin{tabular}{rrrr}
$i$ & $p(i)$ & $q(i) = (m-1)/p(i)$ & $a^{q(i)}$ mod $m$\\
1 & 2 & 3\_43597\_38251 & $m-1$\\
2 & 43801 & 15\_68902 & 2\_49653\_21011\\
3 & 784451 & 87602 & 1\_44431\_31136
\end{tabular}

The fact that values in the last column above are not~1 verifies that $a$ is a
primitive root of $m.$

The values of $\mu _i$ and log~base~2 of $\nu _i$ are

\begin{tabular}{r@{\ }c@{\ }r@{\ \ }r@{\ \ }r@{\ \ }r@{\ \ }r}
$(Log_2 \nu _i,\ i=2,6)$ & = & 18.00, & 12.00, & 8.60, & 7.30, & 6.00\\
$(\mu _i$, $i=2,6)$ &      = &  3.00, &  3.05, & 3.39, & 4.55, & 6.01
\end{tabular}

These values may be compared with Table~1, pp.~102--103,
\cite{Knuth:1981:ACP}, that lists the same measures for a number of other
random number generators.

This package contains both a short and a long algorithm to implement Eq.\,(1).
Let XCUR be the program variable containing the current value of $k_i$ of
Eq.\,(1). The short algorithm for advancing XCUR is

\hspace{.2in}XCUR = mod(AFAC * XCUR, MDIV).

The long algorithm, using ideas from \cite{Wichmann:1982:AEP}, is

\begin{tabbing}
\hspace{.2in}\=Q = aint(XCUR/B)\\
\>R = XCUR $-$ Q * B\\
\>XCUR = AFAC * R $-$ C * Q\\
\>do while(XCUR .lt. 0.0)\\
\>\ \ \ \ XCUR = XCUR + MDIV\\
\>end do
\end{tabbing}

where B and C are constants related to MDIV and AFAC by MDIV = B $\times $
AFAC + C. We use B = 112165 and C = 243273. The average number of executions
of the statement XCUR = XCUR + MDIV is 1.09 and the maximum number of
executions is~3.

The largest number that must be handled in the short algorithm is the
product of AFAC with the maximum value of XCUR, $i.e., 612\_662 \times
6\_87194\_76502 = 42\_10181\_19126\_68324 \approx 0.58 \times 2^{56}$. Thus,
the short algorithm requires arithmetic exact to at least 56~bits.

The largest number that must be handled in the long algorithm is the product
of C with the maximum value of aint(XCUR/B), $i.e., 243273 \times 612664 \approx
0.14904 \times 10^{12} \approx 0.54 \times 2^{38}$. Thus the long algorithm requires
arithmetic exact to at least 38~bits.

To accommodate different compiler/computer systems this program unit
contains code for~3 different ways of computing the new XCUR from the old
XCUR, each producing the same sequence of values. Initially we have MODE $=
1 $. When MODE $= 1$ the code does tests to see which of the three
implementation methods will be used, and sets MODE $= 2$, 3, or~4 to
indicate the choice.

Mode~2 will be used in machines such as the Cray that have at least a 38~bit
significand in single precision arithmetic. XCUR will be advanced using the long algorithm
in single precision arithmetic.

Mode~3 will be used on machines that don't meet the Mode~2 test, but can
maintain at least 56~bits exactly in computing mod(AFAC $\times $ XCUR,
MDIV) in double precision arithmetic. This includes VAX, UNISYS, IBM~30xx, and some IEEE
machines that have clever compilers that keep an extended precision
representation of the product AFAC $\times $ XCUR within the math processor
for use in the division by MDIV. XCUR will be advanced using the short
algorithm in double precision arithmetic.

Mode~4 will be used on machines that don't meet the Mode~2 or~3 tests, but
have at least a 38~bit significand in double precision arithmetic. This includes IEEE
machines that have not-so-clever compilers. XCUR is advanced using the long
algorithm in double precision arithmetic.

If a user wishes to know which mode has been selected, the statement
$$
\fbox{{\bf CALL RN2( MODE )}}
$$
can be used after at least one access has been made to one of the random
number generators or to RANPUT or RANGET. This call will set the integer
variable MODE to the mode value of~2, 3, or~4 that the package is using.

\subsubsection{Remarks on alternative algorithms}

From \cite{Learmonth:1973:STS} and \cite{Learmonth:1976:ETM} it appears
that a multiplicative congruential generator of the form of Eq.\,(1) using
the values MDIV $= 2147483647 = (2^{31}) - 1$ and AFAC $= 16807 = 7^5
\approx 0.513 \times 2^{15}$ has been widely used, and is quite
satisfactory.  However, the associated values of $\mu _i$ and $\nu _i$ are
smaller than those associated with the MDIV and AFAC we are using.  With
these values, XCUR would take all integer values in [1,\ MDIV $-$ 1].  The
maximum value of the product, AFAC $\times $ XCUR would be approximately
$0.361 \times 10^{14} \approx 0.51 \times 2^{46}$, so at least 46-bit
arithmetic would be needed.

The method of \cite{Wichmann:1982:AEP} is interesting in that it
illustrates techniques for getting a very long period generator
$(0.36\times 10^{14} \approx 0.51 \times 2^{46})$ using relatively
low-precision arithmetic.  This method uses at least three integer mod
operations, three integer multiplications, three floating divisions, two
floating additions, and a floating mod; there does not appear to be any
theoretical measure of the quality of the sequence, such as the $\mu _i$
and $\nu _i$ described in \cite{Knuth:1981:ACP}.

\subsubsection{Organization of the package}

The basic uniform pseudorandom number generation algorithm for the MATH77
library is contained in the program unit RANPK2 that returns an array of N
numbers when called at any one of the entry points, SRANUA, SRANUS, DRANUA,
or DRANUS. A single sequence of pseudorandom integers is managed within this
program unit. Calling any of these four entry points will cause updating of
this single pseudorandom integer sequence. The other random number
subprograms in Chapters~3.2 and~3.3 depend on uniform pseudorandom numbers
obtained by calling SRANUA or DRANUA.

The function SRANU is a separate program unit. SRANU references common
blocks that contain a REAL buffer array of length~97, and an index into the
array. If the value of the index indicates the buffer contains unused random
numbers, SRANU simply decrements the index and returns the next number from
the buffer. If the index indicates the buffer is empty, SRANU calls SRANUA
to fill the buffer with uniform random numbers and then reinitializes the
index and returns one random number.

The common blocks are also referenced and used in this way by other random
number subprograms needing single-precision uniform random numbers: SRANE,
SRANG, SRANR, and IRANP.

The double-precision subprograms, DRANU, DRANE, DRANG, and DRANR share
reference to a different common block that is used similarly to buffer an
array of double-precision uniform pseudorandom numbers.

RAN1 and RANPUT are entries in a program unit RANPK1. When either of these
entries is called, the pointers in common will be set to indicate the empty
state and a call will be made to the appropriate one of two private entry
points in program unit RANPK2 to make the requested change in the seed.

RANSIZ and RANGET are entries in RANPK2 that simply return stored values, a
constant in the case of RANSIZ, and a variable integer array in the case of
RANGET.

\subsubsection{Accuracy tests}

We have run tests of equidistribution in~1, 2, and~3 dimensions, as well
as the run test and gap test described in \cite{Knuth:1981:ACP}.  Results
were satisfactory.  The main basis for confidence in this algorithm is the
number-theoretic properties of the pair $(m,a)$ described above.

Values returned as double-precision random numbers will have random bits
throughout the word, however the quality of randomness should not be
expected to be as good in a low-order segment of the word as in a high-order
part.

\bibliography{math77}
\bibliographystyle{math77}

\subsection{Error Procedures and Restrictions}

If the argument N in SRANUA, SRANUS, DRANUA, or DRANUS is nonpositive the
subroutine will return immediately and make no reference to the array XTAB().

When using RANPUT or RANGET, the user must assure that the array, KSEED(),
has an adequate dimension. Violation of this condition will have
unpredictable effects. The user can call RANSIZ to determine the required
dimension.

If none of the three modes of computation (See MODE $= 2$, 3, or~4 in
Section~D.) succeeds, the program unit RANPK2 will write an error message
directly to the system output unit and stop. This is unlikely, as it would
only happen if the host system cannot at least do exact 38-bit arithmetic in
double precision.

\subsection{Supporting Information}

The source language is ANSI Fortran~77.

\begin{tabular}{@{\bf}l@{\hspace{5pt}}l}
\bf Entry & \hspace{.35in} {\bf Required Files}\vspace{2pt} \\
DRANU & \parbox[t]{2.7in}{\hyphenpenalty10000 \raggedright
DRANU, ERFIN, ERMSG, RANPK1, RANPK2\rule[-5pt]{0pt}{8pt}}\\
DRANUA & \parbox[t]{2.7in}{\hyphenpenalty10000 \raggedright
ERFIN, ERMSG, RANPK2\rule[-5pt]{0pt}{8pt}}\\
DRANUS & \parbox[t]{2.7in}{\hyphenpenalty10000 \raggedright
ERFIN, ERMSG, RANPK2\rule[-5pt]{0pt}{8pt}}\\
RAN1 & \parbox[t]{2.7in}{\hyphenpenalty10000 \raggedright
ERFIN, ERMSG, RANPK1, RANPK2\rule[-5pt]{0pt}{8pt}}\\
RANGET & \parbox[t]{2.7in}{\hyphenpenalty10000 \raggedright
ERFIN, ERMSG, RANPK2\rule[-5pt]{0pt}{8pt}}\\
RANPUT & \parbox[t]{2.7in}{\hyphenpenalty10000 \raggedright
ERFIN, ERMSG, RANPK1, RANPK2\rule[-5pt]{0pt}{8pt}}\\
RANSIZ & \parbox[t]{2.7in}{\hyphenpenalty10000 \raggedright
ERFIN, ERMSG, RANPK2\rule[-5pt]{0pt}{8pt}}\\
RN2 & \parbox[t]{2.7in}{\hyphenpenalty10000 \raggedright
ERFIN, ERMSG, RANPK2\rule[-5pt]{0pt}{8pt}}\\
SRANU & \parbox[t]{2.7in}{\hyphenpenalty10000 \raggedright
ERFIN, ERMSG, RANPK1, RANPK2, SRANU\rule[-5pt]{0pt}{8pt}}\\
SRANUA & \parbox[t]{2.7in}{\hyphenpenalty10000 \raggedright
ERFIN, ERMSG, RANPK2\rule[-5pt]{0pt}{8pt}}\\
SRANUS & \parbox[t]{2.7in}{\hyphenpenalty10000 \raggedright
ERFIN, ERMSG, RANPK2}\\\end{tabular}

Designed by C. L. Lawson and F. T. Krogh, JPL, April~1987. Programmed by C.
L. Lawson and S. Y. Chiu, JPL, April, 1987. November~1991: Lawson redesigned
RANPK2 to have MODES~2, 3, and~4 for better portability. Also reorganized
and renamed common blocks.


\begcodenp
\lstset{language=[77]Fortran,showstringspaces=false}
\lstset{xleftmargin=.8in}

\centerline{\bf \large DRSRANU}\vspace{10pt}
\lstinputlisting{\codeloc{sranu}}
\newpage
\vspace{30pt}\centerline{\bf \large ODSRANU}\vspace{10pt}
\lstset{language={}}
\lstinputlisting{\outputloc{sranu}}


\newpage
\enlargethispage*{30pt}
\lstset{language=[77]Fortran,showstringspaces=false}
\lstset{xleftmargin=.8in}

\centerline{\bf \large DRDRAN}\vspace{10pt}
\lstinputlisting{\codeloc{dran}}
\newpage
\vspace{30pt}\centerline{\bf \large ODDRAN}\vspace{10pt}
\lstset{language={}}
\lstinputlisting{\outputloc{dran}}

\end{document}
