\documentclass[twoside]{MATH77}
\usepackage{multicol}
\usepackage[fleqn,reqno,centertags]{amsmath}
\begin{document}

\begmath 10.5 Primitive Fast Fourier Transform

\silentfootnote{$^\copyright$1997 Calif. Inst. of Technology, \thisyear \ Math \`a la Carte, Inc.}

\subsection{Purpose}

This subroutine is not intended to be used directly in applications. It is
used by other library subroutines and is described here to supplement the
functional descriptions of these other routines. This subroutine computes
one-dimensional complex Fourier transforms using the Cooley-Tukey {\bf F}ast
{\bf F}ourier {\bf T}ransform ({\bf FFT}). Complex values $z$ and complex
Fourier coefficients $\zeta $ are related by
\begin{equation}
\label{def}z_j=\sum_{k=0}^{N-1}\zeta _kW^{jk},\quad j=0,1,...,N-1,
\end{equation}
where $N =2^M$, $W=e^{2\pi i/N}=\cos 2\pi /N+i\ \sin 2\pi /N$,
and $i^2=-1$. This subroutine is used by the library subroutines:

\begin{tabular}{ll}
   {\bf SRFT1} & One-dimensional {\bf R}eal {\bf F}ourier {\bf T}ransform  \\
  {\bf SCFT} & {\bf C}omplex {\bf F}ourier {\bf T}ransform \\
  {\bf SSCT} & {\bf S}ine, {\bf C}osine, or {\bf T}rigonometric Transform \\
  {\bf SRFT}  & {\bf R}eal {\bf F}ourier {\bf T}ransform \\
\end{tabular}

The last three of these do transformations in up to~6 dimensions. For any of
the above tasks, one should refer to the corresponding chapter.

\subsection{Usage}

\subsubsection{Program Prototype, Evaluate a Fourier Series, Single Precision}

\begin{description}
\item[\bf REAL]  \ {\bf A}$(\geq \text{IR}-1+\text{KS}\times
2^{\text{MM}})$, {\bf S}$(\geq \text{NT}-1)$

\item[\bf INTEGER]  \ {\bf MT, NT, MM, KS, ILAST, KE(30)}

\item[\bf LOGICAL]  {\bf NEEDST}

\item[\bf COMMON]  / {\bf CSFFTC} / Contains {\bf NEEDST, MT, NT, MM, KS,
ILAST, KE}
\end{description}
(Contents and structure of the common block are subject to change.)

If the sine table S has not been computed, set NEEDST=.TRUE., and
$$
\fbox{{\bf CALL SFFT(A(IR), A(II), S)}}
$$
To evaluate a complex Fourier series, assign values to IR (the index of
the real part of the first coefficient, $\zeta $, in the real array A()),
II (the index of the imaginary part of the first coefficient, $\zeta $, in
the real array A()), the Fourier coefficients, $\zeta $, in A(), and
values in the common block.
$$
\fbox{{\bf CALL SFFT(A(IR), A(II), S)}}
$$
On return the array A() contains the values $z$, and S() contains the sine
table.

\subsubsection{Argument and Common Definitions}

\begin{description}
\item[A()]  [inout] Real array holding the complex coefficients.

\item[S()]  [inout] Array used to hold the sine table. If NEEDST is false on
entry, then S should have been computed on a previous entry to SFFT.

\item[NEEDST]  [inout] If true the sine table S is computed and no
transform is computed. NEEDST is always false on exit.

\item[MT]  [in] base~2 log of NT.

\item[NT]  [inout] Number of entries in the sine table + 1. If NEEDST is
false, then NT should have been computed on a previous entry to SFFT.

\item[MM]  [in] log$_2$(number of complex Fourier coefficients).

\item[KS]  [in] distance in memory between successive entries in A. The
complex coefficients used in the computation are A(IR + $I\times
\text{KS})+i\ \text{A(II}+I\times \text{KS}),\ I=0$,
1, ..., $2^{\text{MM}-1}$, where $i^2=-1.$

\item[ILAST]  [in] = $\text{KS} \times 2^{\text{MM}}.$

\item[KE()]  [in] array with KE($L$) = $\text{KS} \times 2^{\text{MM}-L}.$
\end{description}

\subsubsection{Modifications to Obtain Coefficients Given Values of the
Series}

The complex values $z\times 2^{-{\text{MM}}}$ are stored in A(), and then
$$
\fbox{{\bf CALL SFFT(A(II), A(IR), S)}}
$$
On return the array A() contains the complex Fourier Coefficients $\zeta .$

\subsubsection{Modifications for Double Precision}

Change SFFT to DFFT, the REAL type statement to DOUBLE PRECISION, and
the name of the common block to CDFFTC.

\subsection{Examples and Remarks}

Examples of use can be found in source listings for the library subprograms
{\bf SRFT1}, {\bf SCFT}, {\bf SSCT}, and {\bf SRFT}.

\subsection{Functional Description}

This subprogram uses the Cooley-Tukey algorithm \cite{Cooley:1965:AAF} to
evaluate series of the form given in Eq.\,(\ref{def}) above.  The radix
$(4+2)$ algorithm is used, which means that $N$ is factored as $4^q2^r$,
$r=0$ or~1.  The algorithm involves $q$ radix~4 stages, and if $M$ is odd,
one radix~2 stage.  The input array is scrambled based on the KE() before
the first stage so that results come out in the correct order.  Details
can be found in \cite{Krogh:1970:FFT}.

The procedure for computing $\zeta _k$ given $z_j$ involves dividing by $N$,
and interchanging the roles of the real and imaginary parts for both the
input and output arrays. Since $i\overline{z}=\Im z+i \Re z\ (%
\overline{z}=$ conjugate of $z)$, it follows that using this procedure is
equivalent to writing%
\begin{equation*}
i\overline{\zeta }_k=\frac 1N\sum_{j=0}^{N-1}\overline{z}_jW^{jk},\quad
k=0,1,...,N-1,
\end{equation*}
which in turn can be verified by dividing both sides by $i$, conjugating
both sides $(\overline{W}=W^{-1})$, and substituting the expression for $z_j$
given by Eq.\,(\ref{def}), yielding%
\begin{equation*}
\begin{split}
   \zeta _K   &=\frac 1N\sum_{j=0}^{N-1}\sum_{k=0}^{N-1}\zeta _kW^{jk}W^{-jK}\\
   &=\frac 1N\sum_{k=0}^{N-1}\zeta _k\sum_{j=0}^{N-1}W^{j(k-K)}=\zeta _K
   \end{split}
\end{equation*}
since%
\begin{equation*}
\hspace{-10pt}\sum_{j=0}^{N-1}W^{jL}=\begin{cases}
N & \text{if }L=0\\
\dfrac{1-W^{LN}}{1-W^L}=0 &\text{if }L=\pm 1,...,\pm (N-1).
\end{cases}
\end{equation*}

\bibliography{math77}
\bibliographystyle{math77}

\subsection{Error Procedures and Restrictions}

No tests are made for bad argument values, e.g., MM $<0.$

\subsection{Supporting Information}

The source language is ANSI Fortran~77.

\begin{tabular}{@{\hspace{0pt}\bf}l@{\hspace{7pt}}p{2.7in}}
Entry & \hspace{.3in} {\bf Required Files}\vspace{2pt} \\
DFFT & \hspace{.35in}  DFFT\rule[-5pt]{0pt}{8pt}\\
SFFT & \hspace{.35in}   SFFT \\
\end{tabular}

\end{multicols}
\end{document}
