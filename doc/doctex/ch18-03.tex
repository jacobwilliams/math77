\documentclass[twoside]{MATH77}
\usepackage{multicol}
\usepackage[fleqn,reqno,centertags]{amsmath}
\begin{document}
\begmath 18.3 Sorting Partially Ordered Data of
\hbox{Arbitrary Structure in Memory}

\silentfootnote{$^\copyright$1997 Calif. Inst. of Technology, \thisyear \ Math \`a la Carte, Inc.}

\subsection{Purpose}

Sort data having an organization or structure not supported by one of the
subprograms in Chapter~18.1, for example, data having more than one key to
determine the sorted order. This subprogram has similar functionality
to GSORTP of Chapter~18.2 and is more efficient when the data are initially
partially ordered, or when the ordering criterion is expensive to
determine.

\subsection{Usage}

\subsubsection{Program Prototype}

\begin{description}
\item[INTEGER]  \ {\bf N, L}($\geq $N){\bf , L1, COMPAR}

\item[EXTERNAL]  \ {\bf COMPAR}
\end{description}

Assign values to N and data elements indexed by~1 through N. Require
N $\geq 1.$
$$
\fbox{{\bf CALL INSORT (COMPAR, N, L, L1)}}
$$
Following the call to INSORT the contents of L(1) through L(N) contain a
linked list that defines the sorted order of the data. L1 is the index of
the first record of the sorted sequence. Let I = L(J). If I = 0 record
J is the last record in the sorted sequence, else record I is the immediate
successor of record J in the sorted sequence.

\subsubsection{Argument Definitions}

\begin{description}
\item[COMPAR]  \ [in] An INTEGER FUNCTION subprogram that defines the
relative order of elements of the data. COMPAR is invoked as COMPAR(I, J),
and is expected to return $-$1 (or any negative integer) if the $\text{I}^{th}$
element of the original data is to precede the $\text{J}^{th}$ element in
the sorted sequence, +1 (or any positive integer) if the $\text{I}^{th}$
element is to follow the $\text{J}^{th}$
element, and zero if the order is immaterial. INSORT does not have access to
the data. It is the caller's responsibility to make the data known to
COMPAR. Since COMPAR is a dummy procedure, it may have any name. Its name
must appear in an EXTERNAL statement in the calling program unit.

\item[N]  \ [in] The upper bound of the indices to be presented to COMPAR.

\item[L()]  \ [out] An array to contain the definition of the sorted
sequence. L(1:N) are set so that the immediate successor of the $%
\text{J}^{th}$ record of the sorted sequence is L(J) if the $\text{J}^{th}$
record is not the last record in the sorted sequence, else L(J) is zero.

\item[L1]  \ [out] The index of the first record of the sorted sequence.
\end{description}

\subsubsection{Converting the Linked List in L() to a Permutation Vector}

The linked list produced by INSORT (or by INSRTX, see Chapter~18-04) in
the array L() may be converted to a permutation vector by
$$
\fbox{\bf CALL PVEC (L, L1)}
$$
where L() and L1 are as above. Upon return from PVEC, L() is a permutation
vector, as described for the argument IP() of GSORTP (Chapter~18.2).

\subsection{Examples and Remarks}

The program DRINSORT illustrates the use of INSORT to sort 1000 randomly
generated real numbers. The output should consist of the single line

\hspace{.2in}INSORT succeeded

\subparagraph{Stability}

A sorting method is said to be $stable$ if the original relative order of
equal elements is preserved. This subroutine uses a  merge sort algorithm,
which is not inherently stable. To impose stability, return COMPAR =
I $-$ J if the $\text{I}%
^{th}$ and $\text{J}^{th}$ elements are equal.

\subsection{Functional Description}

The INSORT subprogram uses an opportunistic merge sort algorithm, as
described by Sedgewick \cite{Sedgewick:1983:A}, with a modification
suggested by Power \cite{Power:1980:ISU}.  In the basic opportunistic
merge sort algorithm, the first step consists of detecting either
ascending or descending sequences of initially ordered data.  In the
second step, these sequences are merged in pairs to form half as many
sequences, each approximately twice as long as the original sequences
(descending sequences are considered in reverse order).  The second step
is repeated until only one sequence remains.  The Power modification
consists of putting each sequence into a ``bucket" indexed by the base-2
logarithm of its length.  When a third sequence is to be put into a
bucket, the two longest sequences are merged and put into the next bucket.
If this would require putting three sequences into the next bucket, the
process is repeated.  Finally, sequences remaining in the buckets after
the initial order-detecting stage are merged, starting with the smallest
sequences and proceeding to the largest, to produce a single sequence.

\bibliography{math77}
\bibliographystyle{math77}

\subsection{Error Procedures and Restrictions}

INSORT neither detects nor reports any erroneous conditions.

Limitations on the size of array that can be sorted are imposed by the
amount of memory available to hold the array, and the length of an internal
array to hold the ``buckets" used in the Power modification. The number of
buckets is given by a Fortran PARAMETER, currently set to~32. This permits
sorting at least 4,294,467,295 records.

\subsection{Supporting Information}

The source language is Fortran~77.

\begin{tabular}{@{\bf}l@{\hspace{5pt}}l}
\bf Entry & \hspace{.2in} {\bf Required Files}\vspace{2pt} \\
INSORT & \hspace{.35in} INSORT\\
PVEC & \hspace{.35in} PVEC\\\end{tabular}

Designed and coded by W. V. Snyder, JPL 1974. Power modification~1980.
Adapted to MATH77,~1990.


\begcode
\medskip\

\lstset{language=[77]Fortran,showstringspaces=false}
\lstset{xleftmargin=.8in}

\centerline{\bf \large DRINSORT}\vspace{10pt}
\lstinputlisting{\codeloc{insort}}
\end{document}
