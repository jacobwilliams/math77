\documentclass[twoside]{MATH77}
\usepackage{multicol}
\usepackage[fleqn,reqno,centertags]{amsmath}
\DeclareRobustCommand{\us}{\rule{.2pt}{0pt}\rule[-.8pt]{.4em}{.5pt}\rule{.7pt}{0pt}}
\begin{document}
\begmath 4.7  Sparse Square Nonsingular Systems of Equations

\silentfootnote{$^\copyright$ \thisyear \ Math \`a la Carte, Inc.}

\subsection{Purpose}

For a sparse square nonsingular matrix, $A$, of order N, and an
optional vector or matrix B, this routine will factor A, solve $A X =
B, \ A^T X = B$, compute the reciprocal of the condition number, and
obtain the determinant and the inverse.  This code has similar
functionality to the dense codes described in Chapter 4-1.  The
program in Chapter 19-6 can be used together with the code here to
solve problems with the matrix given in Matrix Market format.

\subsection{Usage}
With minimal options, this simply solves $A \mathbf{x} = \mathbf{b}$
for a single vector $\mathbf{b}$, and does not preserve the
factorization.  Other features are available through the options in
ISPEC.

\paragraph{Program Prototype, Single Precision}

\begin{description}

\item[INTEGER] \ {\bf N, ISPEC(?), IA(?)}

\item[DOUBLE PRECISION] \ {\bf A(?), B(?), OPT(3)}

\end{description}

Assign values to N, ISPEC(), IA(),  A(), and B().

\begin{center}
\fbox{\begin{tabular}{@{\bf }c}
CALL DSPGE (N, ISPEC, IA, A, B, OPT)\\
\end{tabular}}
\end{center}

Computed quantities are returned in IA(), A(), B(), and ISPEC.

\paragraph{Argument Definitions}

\begin{description}

\item[N] [in] The order of the matrix $A$ and number of rows in the
  $B$ and $X$.

\item[ISPEC()] [inout] An array used to return a flag, specify
  dimensions and specify options.  Locations are used as follows.
  \begin{description}
  \item[1] [inout] Must be 0 if the matrix is not factored.  The value
    returned on the first call must be preserved on later calls when
    the matrix is already factored.  If the returned value is $<$ 0,
    then the matrix was not factored.  See Section~\ref{sec:errors} on
    page~\pageref{sec:errors} for details.
  \item[2] [in] Declared dimension of IA() (or at least all that you
    want this routine to know about.  This number must be at least the
    space required for storing the factors of the matrix + twice the
    space needed for the original matrix + 7N.
  \item[3] [in] Declared dimension of A(), this can have a dimension
    4N less than that for IA.
  \item[4] [out]  The total free space in A at the solution.  If there
    was insufficient free space this gives the negative of the number
    of columns left to be factored.
  \item[5] [in] Start of options, see the next Section.
  \end{description}

\item[IA()] [inout] IA(1) gives the number of nonzeros in the first
  column.  IA(2) to IA(1 + IA(1)) give the row indexes for rows in the
  first column which must be in increasing order.  IA(2+IA(1)) then
  gives the number of nonzeros in the second column, etc.  Locations
  after this input data are used internally.

\item[A()] [inout] For each row index given in IA, the corresponding
  location in A contains the coefficient corresponding to that row and
  column.  The remaining locations in A are used internally.

\item[B()] [inout] Used to hold the right hand side(s) on entry and to
  hold the solution on exit.

\item[OPT()] [out] Used only for some options..\\
  OPT(1) = reciprocal condition number if requested.\\
  OPT$(2) \times 10^{OPT[3]}$ = the determinant if requested.
\end{description}

\subsubsection{Options}

First, a summary of the options.\vspace{2pt}

\begin{tabular}{|l@{}r|l|}\hline
  \multicolumn{2}{|c|}{\rule{0pt}{12pt}Name \hfill Value}&
  \multicolumn{1}{|c|}{Brief Description}
  \\[2pt]\hline\rule{0pt}{12pt}%
  IPS & 0 & Solve $A \mathbf{x} = \mathbf{b}$\\
  IPST & 1 & Solve $A^T \mathbf{x} = \mathbf{b}$\\
  IPBDIM & 2 & Specifies dimensions on B\\
  IPRCON & 3 & Compute reciprocal condition number\\
  IPDET & 4 & Compute determinant\\
\hline
\end{tabular}\vspace{5pt}

Starting in ISPEC(5) there is an index for an action which we denote
here by $I_a$.  Each $I_a$ has associated with it a parameter name, a
value for that parameter, and a (possibly empty) list of arguments,
which follow immediately after $I_a$ in ISPEC.  After the last
argument for an $I_a$, the next location in ISPEC contains the next
$I_a$, with its associated arguments.  This continues until the last
$I_a$ has been specified.  An option index $\leq 1$ ends the option
list (other options must precede these).  Setting ISPEC(5) = 0, gives
the default action.

It is assumed that B is a one dimensional of length N if option
IPBDIM is not used.

All the above options except for IPBDIM, take no arguments, and the
above table should be self-explanatory.  IPBDIM uses 3 locations in
ISPEC, the 2 to indicate this option, next $I_2$ to indicate the
number of columns (default is 1), and then $J_2$ to indicate the
distance between successive columns of B.  In addition one can get the
inverse by setting $I_2 = -1$, and DSPGE will initialize B to the
identity matrix computing a solution.  If no backsolve is desired, set
$I_2$ to 0.  In the case of a B with 5 columns one might have
ISPEC(5)=2, ISPEC(6)=5, and ISPEC(7)=N.  Note that the code always
assumes that B is dense.

If you want to compute the inverse matrix and are not an expert,
chances are you should ask an expert who is likely to tell you ---
don't.  Given the ability to backsolve using a previous factorization
the inverse should very rarely be needed.

\paragraph{Program Prototype, Single Precision}

Usage is the same as for double precision, except the ``DOUBLE
PRECISION'' statement is changed to ``REAL'', and the name DSPGE is
changed to SSPGE.

\subsection{Examples and Remarks}

Program DRDSPGE illustrates the use of DSPGE to solve a system
of linear equations and compute the reciprocal condition number of the
matrix of the system. Output is shown in ODDSPGE. The data for this problem
were chosen so the exact solution has components 2, $-$5, and~3.

\subsection{Functional Description}

To factor the matrix, Gaussian elimination is used on columns to get a
factorization of the form $A = U L^{-1}$.  Pivot selection permutes
rows and columns independently balancing a desire to keep an upper
triangular form of the matrix without sacrificing stability.

Internal scaling first computes scale factors to make all the columns
have an $L_\infty$ norm of 1, and follows this by doing the same to
the rows.

\subsubsection{Estimating reciprocal condition number}

The condition number of an $n\times n$ nonsingular matrix, $A$, is defined
as $\kappa = \|A\|\times \|A^{-1}\|$, a quantity that is never less than
unity.  This is the largest factor by which the relative error in a
vector, ${\bf y}$, can be amplified as a result of multiplication by $A$
or by $A^{-1}$.  Roughly speaking, if $\kappa = 10^k$ and the components
of the vector ${\bf b}$ in the problem, $A{\bf x} = {\bf b}$, are known to
$d$ decimal digits, and the components of $A$ are known to more than $d$
decimal digits, and the problem is solved using precision greater than $d$
decimal digits, then the solution will be known to about $d-k$ decimal
digits.  In particular, if $k \geq d$, the solution may have no reliable
digits at all.

We use the procedure described in \cite[pp.\ 128--130]{Golub:1996:GVL}
(based on work in \cite{Dongarra:1979:LUG}), which requires on the
order of $3n^2$ additional operations.  In a test with 1250~cases
reported in \cite{Dongarra:1979:LUG}, the maximum value for the
condition number / estimated condition number was just slightly more
than 16.

\subsubsection{Computing the determinant of $A$}

We have $Det(A) = Det(U) Det(L^{-1})$.  Since all row and column
permutations are done on both $L$ and $U$, and the diagonal of $L$
contains 1's, the determinant is simply the product of the diagonals
of $U$ times $-1$ if the sum of the number of exchanges done on rows
and columns of A is odd.\normalsize


For a matrix of large order it is not uncommon for the determinant to
be of extremely large or small magnitude. Consequently we compute and
store the determinant as a pair of numbers, permitting a very large
exponent range.  The second number of the pair is always in integer.

\bibliography{math77}
\bibliographystyle{math77}


\subsection{Error Procedures and Restrictions}
\label{sec:errors}
Error messages are printed using the facility described in Chapter
19-03.  Options there give one some control over actions taken on
errors, and on where the debug print goes.  Error indexes $<-4$ stop
by default, those $>-4$ only return an error flag, and the $-4$ case
will return only if there was enough space to get started on the
factorization.  The error flags returned are

\begin{description}
\setlength{\parsep}{0pt} \setlength{\itemsep}{-3pt}
\item[$-1$] Matrix appears singular.  (Sets OPT(1) = 0.)
\item[$-2$] Matrix has an empty column.
\item[$-3$] Matrix has an empty row.
\item[$-4$] No more space.
\item[$-5$] N $\leq$ 0
\item[$-6$] An unknown option.
\item[$-7$] Problem with column size.
\item[$-8$] Problem with row indexes.
\end{description}

\subsection{Supporting Information}

The source language is ANSI Fortran~77.

Design and programming by Fred T. Krogh, Math \`a la Carte, Inc.
March 2006.  We hope to revisit this in the future and improve
performance.

\begin{tabular}{@{\bf}l@{\hspace{5pt}}l}
\bf Entry & \hspace{.35in} {\bf Required Files}\vspace{2pt} \\
DSPGE & \parbox[t]{2.7in}{\hyphenpenalty10000 \raggedright
DMESS, DSPGE, MESS\rule[-5pt]{0pt}{8pt}}\\
SSPGE & \parbox[t]{2.7in}{\hyphenpenalty10000 \raggedright
MESS, SMESS, SSPGE\rule[-5pt]{0pt}{8pt}}\\
\ & \ \\\end{tabular}

\begcodenp
\vspace {10pt}
\lstset{language=[77]Fortran,showstringspaces=false}
\lstset{xleftmargin=.8in}

\centerline{\bf \large DRDSPGE}\vspace{0pt}

\vspace{10pt}
\lstinputlisting{\codeloc{dspge}}

\vspace{10pt}\centerline{\bf \large ODDSPGE}

\lstset{language={}}
\lstinputlisting{\outputloc{dspge}}

\end{document}
