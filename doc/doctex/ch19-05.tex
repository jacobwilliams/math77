\documentclass[twoside]{MATH77}
\usepackage{multicol}
\usepackage[fleqn,reqno,centertags]{amsmath}
\begin{document}
\begmath 19.5 Checking the Installed Library

\silentfootnote{$^\copyright$1997 Calif. Inst. of
 Technology, \thisyear \ Math \`a la Carte, Inc.}

\subsection{Purpose}

The programs, {\tt cdemo} and {\tt cdemom}, simplify comparing MATH77 (or
{\em mathc90}) demonstration driver results from different machines, or
from different languages or precisions on the same machine.  These
programs may also prove useful for others wishing to compare results
from running their codes on different machines.

\subsection{Usage}
\subsubsection{Checking MATH77 and {\em mathc90}}
We assume you have managed to get the source for MATH77 or {\em mathc90}
into some directory on your machine, and have the corresponding demonstration
drivers in another directory.  First check the source for {\tt amach},
see Chapter~19.1, and if necessary change it so that the machine constants
are computed correctly for your environment.  Then compile the source for
the library, and make up a library so that it is easy to link in the
required MATH77 or {\em mathc90} routines.

The {\tt support} directory of the distribution contains a number of files
that are meant to aid in getting the library checked out on your machine.

\begin{tabular}{lp{2.5in}}
\tt cdcfg & Configuration file -- Described below.\\
\tt cdemo & Stand alone program which compares results, described below
-- Compile and link this now.\\
\tt cdemom & Stand alone program which generates command files, described
below -- Compile and link this now.\\
\tt xxx.t & Sample template files used by {\tt cdemom}, see below.\\
\tt xxx.res & Results from some environments to use in comparisons
\end{tabular}

One uses {\tt cdemom} together with {\tt cdcfg} (or whatever name you put on
the first line of the {\tt cdjob} file) and a template file to
generate a run stream for compiling, running, and collecting the results
from the demonstration drivers.  Given these results, and results
from some other environment (several are provided in the {\tt support}
directory of the MATH77 \& {\em mathc90} distribution), one can use the
program {\tt cdemo} together with {\tt cdcfg} to compare results.

\subsubsection{Comparing Results for Your Codes}
If you are interested in comparing results you have computed in two
different environments for your own codes, you should read the next
section if you want some control over how the comparisons are made;
otherwise you can run {\tt cdemo} without having a file {\tt cdcfg}.  Each
section of what you want to compare should contain a header line of the
form `=name $\cdots$'.  No comparisons are done until both result files
have positioned themselves at such lines.  Thus your other results should
not contain an `=' in column 1.  (There are ways around this, see the
template file for the Univac in Section~B.5 for details.) The only other
section you need to read is Section~B.6.

\subsubsection{The Configuration File, {\tt cdcfg}}
The demonstration drivers distributed with MATH77 \& {\em mathc90}
generate output that is almost certain to be different for different
machines, or even on the same machine when comparing MATH77 and {\em
mathc90} results.  The configuration file contains specifications that
enable the program {\tt cdemo} to distinguish which differences are of a
nature that they should be called to ones attention.  One can skip the
rest of this section if one is just interested in using the configuration
file as provided in the distribution for the checkout of the MATH77 \&
{\em mathc90} libraries.

In the descriptions below, items in brackets are optional, extra spaces
between tokens are not significant, and upper and lower case letters are
equivalent.  For each demonstration driver this file contains an initial
line of the form

=demonstration\underline{ }driver\underline{ }name\hspace{.5in}[Fortran
only]

If the `[Fortran only]' appears, this demonstration driver is not
available in {\em mathc90}.  (`[C only]' can be used for routines that
are in C, but not in Fortran.)

This line is followed by 0 or more lines that indicate special conditions
that apply to the demonstration driver named just above.  The following
statements are possible.
\vspace{-10pt}
\begin{description}
\item[ ]Set Tolerances on {\bf on\underline{ }expression}
\item[ ]TOL = {\bf tol\underline{ }expression}
[on {\bf on\underline{ }expression}]
\item[ ]Ignore {\bf ignore\underline{ }list}
[on {\bf on\underline{ }expression}]
\item[ ]Reset [on {\bf on\underline{ }expression}]
\end{description}
where
\vspace{-10pt}
\begin{description}
\item[\bf tol\underline{ }expression] a Fortran expression restricted to
using
\item[ ]the operators `()*/'
\item[ ]the functions `sqrt', and `max'
\item[ ]the operands: integers, floating point numbers (`E' used for the
exponent), and the variable `TOL'.
\item[\bf on\underline{ }expression] is one of the following forms ($n_1$
and $n_2$ integers): `columns $n_1$:$[n_2]$', `lines $n_1$:$[n_2]$', or
a Fortran character literal (i.e. {\tt '}string of characters{\tt '}).  If
the $n_2$ is missing it is assumed to be a very large number.
\item[\bf ignore\underline{ }list] is either `extra lines', 'end failure' or a
  comma delimited list of one or more of the following: `integer', `floating',
  `floating sign', or `all'.
 \end{description}

In addition a line with `C', `c' or `\#' as a first character or a line starting
with 8 blanks (or a blank line) is treated as a comment line.

Two floating point numbers, $x$, and $y$, are assumed to be sufficiently
close to one another if
\begin{multline*}
    |x-y| \leq \max(\max(1, M)\times \text{TOL},\ M \times 10^{1-d})
\end{multline*}
where $M = \max(|x|, |y|),\ d = \min(n_x, n_y)$, and $n_x$ and $n_y$ are
the number of significant digits in $x$ and $y$ respectively.

The value used for TOL depends on four parameters: TOLSP1, TOLDP1, TOLSP2,
and TOLDP2, and actions taken with `TOL = $\cdots$' instructions.
Initially the code sets TOLSP1 = TOLSP2 = $4 \times 10^{-6}$, and TOLDP1 =
TOLDP2 = $7 \times 10^{-15}$.  These values can be overridden by
instructions in the file {\tt cdjob} described below.

Finally, these values can be overridden with the `Set Tolerances on
$\cdots$' command.  This instruction is meant to be used only with the
demonstration driver DRMACH.  As set up in the configuration file {\tt
cdcfg} this command sets these values to $(33 \times \text{machine }
\varepsilon )$ for the two results being compared.  There are separate
values for single and double precision being defined, and the machine
$\varepsilon $ is defined as the smallest positive floating point number
such that $1 + \varepsilon > 1$.  Note that if this latter mechanism is
used, the results for DRMACH must be the first ones given.  In {\tt cdcfg}
as distributed this is the case, and the remaining demonstration drivers
should be given in alphabetical order, since {\tt cdemo} assumes an
alphabetical order if it is trying to get a match on names, and there is a
mismatch with the names in the result files.

When first encountering the start of results for a demonstration driver,
it is assumed that the results are double precision if the third letter in
the name is a `D' or a `d' and otherwise the results are for single
precision.  The permanent value for TOL assigned to comparing results for
two demonstration drivers is then max(TOL?P1, TOL?P2), where the `?' is
replaced by `D' if the corresponding demonstration driver is in double
precision, and else is replaced by `S'.

A 'TOL = $\cdots$' instruction without an `on $\cdots $' part sets the
value
of TOL to the value defined in the expression.  When TOL appears in the
expression, its value just prior to evaluating the expression is used.
When an 'on $\cdots$' clause is present the value of TOL is not changed.
Rather if two floating point numbers appear to be too large, a check is
made to see if any `on $\cdots$' clause provides a larger TOL, and if so
that value is used to see if the numbers are close enough.

An `on $\cdots$' expression means that a test is to be used only when a
certain condition applies.  In the case of `on columns $\cdots$' the
exceptional case applies when $n_1 \leq i \leq n_2$, where $i$ is the
column index in the first result file where the current token starts.  The
first result file is the first one given in {\tt cdjob}, see the
description of {\tt cdemo} below.

In the case of `on lines $\cdots$', the exceptional case applies when
$n_1 \leq \ell \leq n_2$ where $\ell$ is the number of nonblank lines seen
in the first result file since the line containing the header for the
results for the demonstration driver, or if there has been a `reset'
instruction, the number of nonblank lines read since seeing this instruction.
(Thus the line processed just after a reset instruction caused as a result
of reading that line, will have an index of 0.) After reading an
instruction that contains `on lines $\cdots$', the configuration file is
read no further until $\ell > n_2$, or a `reset' instruction is executed
as a result of seeing another line.  When either of these events occurs,
further lines in the configuration file are processed.

In the case of `on {\tt '}string{\tt '}', the exceptional case applies
when the line from the first result file contains {\tt '}string{\tt '}.
The configuration file continues to be processed, except when the
associated action is `Reset', in which case processing of the
configuration file stops until the 'Reset' action occurs or the end of
output for this demonstration driver.  If more than one of these is
processed, only the last has any effect.

It is possible that all three of these mechanisms indicate that
comparisons are to be done in an exceptional way.  If any one of these
mechanisms would excuse a difference that would otherwise be reported, no
difference is reported.

For the `ignore' instruction: `integer' indicates that integers need not have
the same value, `floating' indicates that every floating point number is
assumed to be equal to any other number, `floating sign' indicates that
floating point numbers should be compared with their signs stripped off, and
`all' indicates that any text matches any other text.  The 'extra lines' means
no diagnostic is given if the files match, except that one file has some lines
embedded that are not in the other.  The 'end failure' results in extra work
if the lines match except for extra characters at the end of one line.  When
this occurs, an attempt is made to match the extra part of the longer line,
with the start of another line at some point below.

Finally, the `reset' instruction, sets everything to the values present
just after processing the initial line for the demonstration driver.  Thus
there are no exceptional cases defined, TOL has its initial value, and the
nonblank line counter is set to 0.

\subsubsection{The Template File}

You will probably need to make up your own template file that is used to
generate the command file used to generate results.  You will find various
files with extensions `.tf' (for Fortran), and `.tc' (for C), that you can
use as models.  The template files use `\#' as an escape character in the
following ways.
\begin{description}
\item[\#1] Substitute the name of the demonstration driver (default is
lower case) for the \#1.
\item[\#2] As for \#1, except the name here and everywhere a \#1 appears
is in upper case.
\item[\#3] This is the end of the last line for output occurring one time
at the very beginning.  (\#1 or \#2 should not be used prior to a \#3.)
\item[\#4] This is the end of the last line that is to be output for each
of the demonstration drivers.
\item[\#5] This and the text following on the line are treated as a
comment. \end{description}
The nature of the template file is probably best understood by looking at
an example.  The following is for the Lahey Fortran compiler running under
DOS on a PC.

\begin{tabular}{l}
del pcf77l.res /n\#3\\
echo =\#1 PC - Lahey F77L $>>$pcf77l.res\\
f77l k:$\backslash$math77$\backslash$demo$\backslash$\#1, \#1.obj /NS\\
d:$\backslash$progs$\backslash$link \#1,,,k:$\backslash$math77$\backslash$\\
fortran$\backslash$obj$\backslash$math77; \#1 $>>$pcf77l.res\\
del \#1.* /n\#4
\end{tabular}

The output for running the demonstration drivers is being sent to the file
pcf77l.res, and thus we start by deleting this file, since we don't want
to add to an existing file.  The \#3 signals that this is all that is done
one time only.

The echo statement writes a line starting with
`=demonstration\underline{ }driver\underline{ }name' to the result file.
The template file must arrange to write such a message ahead of the
results for each driver.  The `=' serves to signal the start of results
for one demonstration driver, and the end of the results for the previous
driver.

The next four lines compile, link, run the driver, and then delete the
files that were created.  Note that your file will almost certainly
require different prefixes for the location of the source files, the
location of library files, and the prefix used to find the location of the
linker.

At least on one system, Exec 8 running on a Univac, one cannot avoid
getting output from the compiling and linking into the output.  To show
how this is handled, we give the template we used for getting results on
the Univac.

When the first line containing an ``='' has text preceding the ``='' this
text is used as a sentinel that causes the current line and
following lines to be ignored as far as making comparisons is concerned.
The text after the ``='' is used as a sentinel to cause comparisons to
start being made again.

The way this file is setup, a `@@MSG,N =' at the start of a line in the
output file signals that this line and any following lines not containing
a header sentinel are to be ignored.  The header sentinel is the text
immediately following the `=' on the first line that contains an '='.  In
this case the start of output for a demonstration driver is flagged by a
line of the form

@@XQT demonstration\underline{ }driver\underline{ }name.

The `{\LARGE $_{\tilde{~}}$}' output in the result file indicates a blank
that occurs at the end of a line since {\tt cdemo} has no way to recognize
trailing blanks on a line.  Thus {\tt cdemo} will not use output in the
result file that is not between a line of the form `@@XQT $\cdots $' and a
line starting with `@@MSG,N~=' for the purposes of making comparisons.

\begin{tabular}{l}
@@DELETE,C UNIRES\\
@@ASG,PU UNIRES\\
@@BRKPT PRINT\$/UNIRES\\
@@MSG,N =@@XQT{\LARGE $_{\tilde{~}}$}\#3\\
@@FTN \#1\\
@@PREP\\
@@MAP,I ,\#1\\
IN \#1\\
LIB LIB*MATH77\$\\
@@EOF\\
@@XQT \#1\\
@@MSG,N = End of Fortran output on Univac for \#1.\\
@@DELETE,AR \#1\#4\\
@@BRKPT PRINT\$
\end{tabular}

A `@@MSG,N =' at the start of a line in the output file signals that
this line and any following lines not containing a header sentinel are to
be ignored.  The header sentinel is the text immediately following the
`=' on the first line that contains a '='.  In this case the start of
output for a demonstration driver is flagged by a line of the form

Finally for an HP workstation we used the following (the demonstration
drivers were compressed).

\begin{tabular}{l}
rm hpfor.res\#3\\
echo '='\#1 ' HP Fortran  ' `date` $>>$hpfor.res\\
zcat /math77/demo/\#1.f.Z $>$\#1.f\\
f77 $-$o \#1 +OPP \#1.f $-$lmath77\\
./\#1 $>>$hpfor.res\\
rm \#1*\#4
\end{tabular}

Something similar should work on most Unix systems.

\subsubsection{Obtaining the Result File Using {\tt cdemom}}
After getting the library installed and making up the template file for
your environment, execute the program {\tt cdemom}.  You will be prompted
for the name of your template file.  The program reads the file {\tt
cdcfg} together with your template file to generate the run stream.  Lines
starting with anything other than a `=' in {\tt cdcfg} are skipped.  If
the last character in the name of your template file is a `C' or a `c',
lines with `Fortran only' are also skipped.  The name of the file
containing the generated run stream is obtained as follows: If the file
name contains a `.', the `.' and all characters to the right are deleted,
and if the first characters of the template file are `PC' or `pc', the
characters `.bat' are appended.  If the file name does not contain a '.',
an 'M' is appended to the name.  The name used for the output file is
printed before the program stops.  Execute this file to get the result
file.

\subsubsection{Comparing Two Result Files Using {\tt cdemo}}
One needs to make up a file named {\tt cdjob} that contains lines as
follows.

\begin{tabular}{l}
 Name of configuration file (blank line if none)\\
 Name of first result file\\
 Name of second result file\\
 $[~F_1 ~ F_2~]$
\end{tabular}

Where if $F_1$ and $F_2$ are given they must be floating point numbers.
This replaces the default values of TOLSP1 \ldots TOLDP2 with TOLSP1 =
TOLSP2 = $\max(F_1,\ F_2)$, and TOLDP1 = TOLDP2 = $\min(F_1,\ F_2)$.
These values may in turn be overwritten by the `Set Tolerances on
$\cdots$' instruction in the configuration file.

One then executes the program {\tt cdemo}, which will print the names of
demonstration drivers as their results are being compared, and outputs all
results into the file {\tt cdres}.

If as a result of doing the comparison you find differences there are five
possibilities: there really is a problem in MATH77 or {\em mathc90}, the
problem is in your compiler (check if the comparison is satisfactory when
you turn off all compiler optimizations), the configuration file is not
perfectly defined (we have not done an exhaustive analysis in picking the
exceptions that are defined in this file), the result file you are
comparing with is corrupted, or {\tt cdemom} has a bug.

\subsection{Examples and Remarks}
Files ending with `.tf', and '.tc.' serve as examples of template files.
The configuration file `{\tt cdcfg}' can serve as an example of a
configuration file should you want one for another purpose.  The
exceptions for {\tt DRDSVA} illustrate the use of most of the features.

\begin{tabular}{l}
=drdsva\\
$\quad$Reset on '(Elements of V'\\
$\quad$Ignore FLOATING SIGN\\
$\quad$Reset on 'INDEX  SING. VAL.'\\
$\quad$Ignore FLOATING SIGN\\
$\quad$TOL = TOL * 10 on lines 4:4\\
$\quad$TOL = TOL * 100 on lines 5:5\\
$\quad\quad\cdots$\\
$\quad$Reset on '1 Earth'\\
$\quad$TOL = TOL * 10 on columns 25:38\\
$\quad$TOL = TOL * 100 on columns 40:53\\
$\quad$TOL = TOL * 1.E5 on columns 54:\\
$\quad$Reset on '1 Earth'\\
$\quad$TOL = TOL * 1.E7
\end{tabular}


\subsection{Functional Description}
When comparing DRMACH on the PC with that on the Hewlett Packard
workstation, the following line is printed.

drmach PC Lahey F77L  ==  drmach HP Fortran  FCOMP =  10

The `FCOMP' refers to the number of comparisons that were made by computing
and checking two floating point numbers.  (Identical numbers are compared as
text, and don't get counted in this comparison.)  If the programs had
differences, the `==' would have been replaced by `/=', and the FCOMP result
would have been printed on a separate line after all the results for comparing
these drivers had been printed.  The result of the comparison in this case is
a listing of the lines in the result files that were different.  Those from
the first file have a `1)' in the first two columns followed by the line
number in the file, a `:` and then the line from the file.  Those from the
second file are similar except for starting with `2)'.  If the difference was
due to floating point errors a line something like the following is printed

F1=$-$3.3...E$-$13\ \ F2=2.0...E$-$12\ \ DIFF=$-$2.3...E$-$12

The F1 refers to a number in the first result file, the F2 to a number in
the second result file, and DIFF to their difference.  These numbers are
printed from the value of the numbers and thus will typically not be
formatted in the same way as they are in the result files.  When numbers
have extremely large exponents, the number printed may have a exponent
that is less by 100.  This allows results from a machine with an extremely
large exponent range to be checked on a machine with a much smaller range.
Also if one has an `Ignore FLOATING SIGN' active, then the sign on one of
the numbers may have been changed so that F1 and F2 have the same sign.

\subsection{Error Procedures and Restrictions}
If an error condition occurs, a Fortran `STOP' is executed with a message
describing the nature of the problem.  Errors in the configuration file
will also print the line causing the problem with an indication of where
the problem was diagnosed.

\subsection{Supporting Information}

The source language is ANSI Fortran~77.

\begin{tabular}{@{\bf}l@{\hspace{5pt}}l}
\bf Program & \hspace{.35in} {\bf Required Files}\vspace{2pt} \\
CDEMO & \parbox[t]{2.7in}{\hyphenpenalty10000 \hspace{.7in}
none \rule[-5pt]{0pt}{8pt}}\\
CDEMOM & \parbox[t]{2.7in}{\hyphenpenalty10000 \hspace{.7in} none}\\
\end{tabular}

Design and code due to F. T. Krogh, 1995.

\end{multicols}
\end{document}
