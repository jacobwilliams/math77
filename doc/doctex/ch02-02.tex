\documentclass[twoside]{MATH77}
\usepackage[\graphtype]{mfpic}
\usepackage{multicol}
\usepackage[fleqn,reqno,centertags]{amsmath}
\begin{document}
\opengraphsfile{pl02-02}
\begmath 2.2 Error Function, Gaussian Probability Integral, etc.

\silentfootnote{$^\copyright$1997 Calif. Inst. of Technology, \thisyear \ Math \`a la Carte, Inc.}

\subsection{Purpose}

These subprograms compute values of the error function,
the complementary
error function, and an exponentially scaled complementary error
function, \cite{ams55:erf,Hart:1968:CA:erf}.  The closely related Gaussian or
normal probability integral can be computed by subprograms described in Chapter
15.2.

Error function:
\begin{align*}
\erf (x)&=\frac 2{\sqrt{\pi }}\int_0^xe^{-t^2}dt\\
\intertext{Complementary error function:}
\erfc (x)&=1-\erf (x)=\frac 2{\sqrt{\pi }}\int_x^\infty e^{-t^2}dt\\
\intertext{Exponentially scaled complementary error function:}
\erfce (x)&=\exp (x^2)\ \erfc (x)
\end{align*}
\subsection{Usage}

\subsubsection{Program Prototype, Single Precision}

\begin{description}
\item[REAL]  \ {\bf X, Y, SERF, SERFC, SERFCE}
\end{description}

Assign a value to X and obtain the value of erf, erfc, or erfce,
respectively, as follows:
$$
\fbox{{\bf Y = SERF(X)}} \hspace{.4in} \fbox{{\bf Y = SERFC(X)}}
$$
$$
\fbox{{\bf Y = SERFCE(X)}}
$$

\subsubsection{Argument Definitions}

\begin{description}
\item[X]  \ [in] Argument at which function evaluation is desired. Require
$\text{X} \geq 0$ when using SERFCE.

\item[Y]  \ [out] Value of function returned.
\end{description}

\subsubsection{Modifications for Double Precision}

For double precision change the REAL type statement to DOUBLE PRECISION and
change the function names to DERF, DERFC, and DERFCE.

\subsection{Examples and Remarks}

See DRSERF and ODSERF for an example of the usage of these subprograms.

The ANSI Fortran 77 standard does not include the error function, or the
related functions, erfc and erfce, as intrinsic functions.

Some Fortran systems, however, do provide such functions. For example,
UNISYS ASCII Fortran and IBM VS-Fortran provide ERF, ERFC, DERF, and DERFC,
with ERF and ERFC being ``generic," whereas VAX-11 Fortran does not provide
these functions.

In a system having DERF or DERFC as intrinsics, a reference to one of these
function names will cause the vendor-supplied code to be used. If one wishes
to override this and use the code from this library one must declare the
function name to be EXTERNAL in the referencing program unit.

Whenever the user's mathematical problem involves the expression, $1-\erf (x)$%
, $\erfc (x)$ should be used instead. This will give significantly better
relative accuracy as $x$ increases above 0.5, and no worse accuracy for $x <
0.5.$

In cases in which the user's problem involves the expression, $\exp (x^2)
[1-\erf (x)]$, with $x \geq 0$, $\erfce (x)$ should be used. This avoids the
problem of $\exp (x^2)$ reaching the overflow limit for fairly modest values
of $x$, since $\erfce (x)$ is asymptotic to $1/(x\ \pi ^{1/2})$ as $x$
approaches infinity.

\subsection{Functional Description}

The function $\erf (x)$ is an odd function, increasing monotonically from $-$1
at $x = -\infty $ to +1 at $x = +\infty .$
The function $\erfc (x) = 1 - \erf (x)$ approaches zero rapidly as $x$
increases. More specifically, $\erfc (x)$ is asymptotic to
\begin{equation*}
u(x) =  \exp (-x^2) / (x \sqrt \pi) \text{ as }x \rightarrow + \infty .
\end{equation*}
For $x > 5$, $\erfc (x)$ satisfies $0.98 u(x) < \erfc (x) < u(x)$.
\vspace{10pt}

\hspace{5pt}\mbox{\input pl02-02 }

The largest value of $x$ for which $\erfc (x)$ does not underflow is estimated
by this subprogram using the System Parameters subprogram on first entry.
Examples of this limiting value, called XMAX in the subprogram, are 9.18 for
IEEE single precision and 26.53 for IEEE double precision.

The mathematical function $\erfce (x)$ is defined for all $x$ but it is
expected to be used only for large positive $x$. Thus the subprograms SERFCE
and DERFCE are designed only for use with $x \geq 0.$

These subprograms are based on polynomial approximations and code due to L.
W. Fullerton, Los Alamos, 1977, and rational approximations due to W. J.
Cody, \cite{Cody:1969:RCA}. By use of the System Parameters subprogram the
accuracy adjusts to the machine precision. The stored coefficients provide for
precision to about 30 significant decimal digits for DERF, DERFC, SERF, and
SERFC. The accuracy for DERFCE and SERFCE is up to 30 significant decimal
digits for $0 \leq x \leq $ XMAX, and up to 18 digits for $x \geq $ XMAX.

\subparagraph{Accuracy Tests}

The single precision subprograms were tested on a UNISYS 1100 by comparison
with the corresponding double precision subprograms at over 45000 points.
The relative precision of the UNISYS single precision arithmetic is $\rho =
2^{-27} = 0.745$\,E$-8$. The test results may be summarized as follows:

\begin{center}
\begin{tabular}{lcr}
 & {\bf Argument} & \multicolumn{1}{c}{\bf Max. Rel.}\\
\multicolumn{1}{c}{\bf Function} & {\bf Interval} &
\multicolumn{1}{c}{\bf Error}\\
SERF & \multicolumn{1}{c}{$x \leq 1$} & $2.1\rho $\rule{.25in}{0pt}~\\
 & $x \geq 1$ & $1.2\rho $\hspace{.25in}\rule[-7pt]{0pt}{8pt}\\
SERFC & [$-$7, 0.5] & $2.1\rho $\rule{.25in}{0pt}\\
 & [0.5, 1.0] & $5.1\rho $\rule{.25in}{0pt}\\
 & [1.0, 2.8] & $8\rho $\rule{.25in}{0pt}\\
 & [2.8, 4.0] & $16\rho $\rule{.25in}{0pt}\\
 & [4.0, 5.7] & $32\rho $\rule{.25in}{0pt}\\
 & [5.7, 8.0] & $64\rho $\rule{.25in}{0pt}\\
 & [8.0, 8.6] & $128\rho $\rule{.25in}{0pt}\rule[-7pt]{0pt}{8pt}\\
SERFCE & [0, 2] & $4\rho $\rule{.25in}{0pt}\\
& [2, 4] & $8\rho $\rule{.25in}{0pt}\\
& [4, 50] & $2\rho $\rule{.25in}{0pt}
\end{tabular}
\end{center}

The increase of relative error in SERFC with increasing arguments is an
inherent property of the erfc function.

\bibliographystyle{math77}
\bibliography{math77}

\subsection{Error Procedures and Restrictions}

The argument for SERFCE or DERFCE must be nonnegative. If it is negative the
subprogram will issue an error message and return a zero result.

If the argument for SERFC or DERFC exceeds XMAX, defined above in Section D,
the subroutine will issue a warning message and return a zero result.

Error and warning messages are processed using the subroutines of
Chapter~19.2 with an error level of zero.

\subsection{Supporting Information}

\begin{tabular}{@{\bf}l@{\hspace{5pt}}l}
\bf Entry & \hspace{.35in} {\bf Required Files}\vspace{2pt} \\
DERF & \parbox[t]{2.7in}{\hyphenpenalty10000 \raggedright
AMACH, DCSEVL, DERF, DERM1, DERV1, DINITS, ERFIN, ERMSG, IERM1, IERV1\rule[-5pt]{0pt}{8pt}}\\DERFC & \parbox[t]{2.7in}{\hyphenpenalty10000 \raggedright
AMACH, DCSEVL, DERF, DERM1, DERV1, DINITS, ERFIN, ERMSG, IERM1, IERV1\rule[-5pt]{0pt}{8pt}}\\DERFCE & \parbox[t]{2.7in}{\hyphenpenalty10000 \raggedright
AMACH, DCSEVL, DERF, DERM1, DERV1, DINITS, ERFIN, ERMSG, IERM1, IERV1\rule[-5pt]{0pt}{8pt}}\\SERF & \parbox[t]{2.7in}{\hyphenpenalty10000 \raggedright
AMACH, ERFIN, ERMSG, IERM1, IERV1, SCSEVL, SERF, SERM1, SERVI, SINITS\rule[-5pt]{0pt}{8pt}}\\SERFC & \parbox[t]{2.7in}{\hyphenpenalty10000 \raggedright
AMACH, ERFIN, ERMSG, IERM1, IERV1, SCSEVL, SERF, SERM1, SERVI, SINITS\rule[-5pt]{0pt}{8pt}}\\SERFCE & \parbox[t]{2.7in}{\hyphenpenalty10000 \raggedright
AMACH, ERFIN, ERMSG, IERM1, IERV1, SCSEVL, SERF, SERM1, SERVI, SINITS\rule[-5pt]{0pt}{8pt}}\\\end{tabular}

Based on a 1977 program by L. W. Fullerton, Los Alamos, and a 1969 program
by E. W. Ng, JPL. Present version by C. L. Lawson and S. Y. Chiu, JPL, 1983.


\begcodenp
\medskip\
\lstset{language=[77]Fortran,showstringspaces=false}
\lstset{xleftmargin=.8in}

\centerline{\bf \large DRSERF}\vspace{10pt}
\lstinputlisting{\codeloc{serf}}
\newpage

\vspace{30pt}\centerline{\bf \large ODSERF}\vspace{10pt}
\lstset{language={}}
\lstinputlisting{\outputloc{serf}}

\closegraphsfile
\end{document}
