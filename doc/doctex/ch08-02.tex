\documentclass[twoside]{MATH77}
\usepackage{multicol}
\usepackage[fleqn,reqno,centertags]{amsmath}
\begin{document}
\begmath 8.2 Solve System of Nonlinear Equations

\silentfootnote{$^\copyright$1997 Calif. Inst. of Technology, \thisyear \ Math \`a la Carte, Inc.}

\subsection{Purpose}

This subroutine solves a system of $n$ nonlinear equations in $n$ unknowns.
The problem can be expressed as
\begin{equation*}
{\bf f}({\bf x}) = {\bf 0}
\end{equation*}
where ${\bf x}$ denotes an $n$-vector of unknowns and ${\bf f}$ denotes an $%
n $-dimensional vector-valued function, with components $f_i({\bf x})$, $i =
1$, ..., $n.$

It is assumed that the function ${\bf f}$ has continuous first partial
derivatives with respect to ${\bf x}$, at least in a reasonably sized
neighborhood of the solution.

The solution algorithm makes use of the $n\times n$ Jacobian matrix, $J({\bf %
x})$, whose $(i,j)$ element is $\partial f_i/\partial x_j$ evaluated at $%
{\bf x}$. The elements of $J$ may either be computed by user-provided code
or estimated by the algorithm using finite differences of values of ${\bf f}%
. $

An auxiliary subroutine DCKDER (Chapter~8.3) may be used as an aid for
checking the consistency of code the user may write for computing
derivatives.

The problem, ${\bf f}({\bf x}) = {\bf 0}$, can also be solved using the
nonlinear least-squares package DNLxxx of Chapter~9.3. Generally if a
problem can be solved by both DNQSOL and DNLxxx the execution time will be
significantly less using DNQSOL. Problems for which DNLxxx is needed are:
\begin{itemize}
\item[(1)] those in which the number of unknowns is not equal to the number of
equations,
\item[(2)] those in which one does not expect the system to have an
exact solution,
\item[(3)] those in which constraints on the variables are an
inherent aspect of the problem and one expects that the solution may lie on
a constraint boundary, or
\item[(4)] those in which it is necessary to constrain
the region of search to prevent the algorithm from wandering away from the
expected neighborhood of the solution.
\end{itemize}

\subsection{Usage}

\subsubsection{Program Prototype, Double Precision}
\begin{description}
\item[EXTERNAL]  \ {\bf DNQFJ}

\item[INTEGER]  \ {\bf N, IOPT(), IDIMW}

\item[DOUBLE PRECISION]  {\bf X}($\geq $N){\bf , FVEC}($\geq $N){\bf ,\\
XTOL, W}(IDIMW)
\end{description}
Assign values to N, X(), XTOL, IOPT(), IDIMW, and optionally, W().
\begin{center}
\fbox{\begin{tabular}{@{\bf }c}
CALL DNQSOL(DNQFJ, N, X, FVEC,\\
XTOL, IOPT, W, IDIMW)\\
\end{tabular}}
\end{center}
Computed results will be returned in X(), FVEC(), IOPT(1:3), and W(3).

\subsubsection{Argument Definitions}

\begin{description}
\item[DNQFJ]  \ [in] Name of subroutine provided by the user to compute the
value of ${\bf f}$, and optionally $J$, at any specified value of ${\bf x}$.
This subroutine must be of the form:
{\tt \begin{tabbing}
subroutine DNQFJ(N, X, FVEC, FJAC, IFLAG)\\
integer N, IFLAG\\
double precision X(N), FVEC(N), FJAC(N,N)\\
On entry X() contains the current value\\
of ${\bf x}.$\\
If IFLAG $= 0$, print X() and FVEC().\\
\ \ \ \=$\{$This action is needed only if the\\
\>$iprint$ option is selected with $nprint$\\
\>positive$.\}$\\
If IFLAG $= 1$, compute ${\bf f}({\bf x})$, storing $f_i$ in\\
\>FVEC(i) for $i = 1$, ...,N.\\
If IFLAG $= 2$, compute $J({\bf x})$, storing\\
\>$\partial f_i/\partial x_j$ in FJAC$(i,j)$ for $i = 1$, ..., N,\\
\>and $j = 1$, ..., N. $\{$This action is not\\
\>needed if the $inoj$ option is selected,\\
\>however FJAC must still appear in the\\
\>argument list$.\}$\\
return\\
end
\end{tabbing}}
The subroutine DNQFJ may set IFLAG to a negative integer value to cause
immediate termination of the solution procedure. Otherwise, the value of
IFLAG should not be changed.

Except at the initial ${\bf x}$, a call to DNQFJ with IFLAG $=2$ will almost
never be with the same ${\bf x}$ as the previous call with IFLAG $=1$. Thus,
it is not easy to save subexpressions from the computation of function
values for reuse in the computation of the Jacobian. The Jacobian updating
method used by this algorithm reduces the number of Jacobian evaluations
needed.

In some applications DNQFJ may need additional data that may have been input
by the user's main program. One way to handle this in Fortran~77 is by use
of named COMMON blocks.

\item[N]  \ [in] The number, $n$, of components in ${\bf x}$ and in ${\bf f}%
. $

\item[X()]  \ [inout] On entry X() must contain an initial value for the $%
{\bf x}$ vector. On successful termination X() contains the final estimate
of the solution vector, ${\bf x}.$

\item[FVEC()]  \ [out] On termination FVEC() contains the value of the ${\bf %
f}$ vector evaluated at the final ${\bf x}.$

\item[XTOL]  \ [in] Require XTOL $\geq 0.0$. The algorithm will report
successful termination when it estimates the relative error in ${\bf x}$, in
a weighted Euclidean vector norm sense, is less than $\max (\text{XTOL}%
,\epsilon )$, where $\epsilon $ is the machine accuracy, given by
D1MACH(4) (or R1MACH(4) for single precision).  D1MACH and R1MACH are
described in Chapter~19.1.

We suggest setting XTOL in the range from $\epsilon ^{0.50}$ to $%
\epsilon ^{0.75}.$

See Section D for further discussion of XTOL and the convergence test.

\item[IOPT()]  \ [inout] Array used to select options and to return
information. The first three locations are used to return information as
follows:

IOPT$(1)=info$. Indicator of status on termination. If $info<0$, termination
is due to IFLAG having been set negative in DNQFJ, and $info$ will have the
value that was assigned to IFLAG. Otherwise, $info$ may have the following
values:

\begin{itemize}
\item[0]  Successful termination. Either the XTOL test is satisfied or the
condition ${\bf f}={\bf 0}$ is satisfied exactly.

\item[1]  Improper input parameters. Require N $>0$, IDIMW $\geq $ the value
specified below, and valid entries in IOPT(i) for $i\geq 4.$

\item[2]  Number of calls to DNQFJ has reached the limit of {\em maxfev}. Default:
$maxfev =200\times (\text{N}+1).$

\item[3]  It appears to be impossible to satisfy the XTOL test. Possibly
XTOL should be set larger.

\item[4, 5]  The value 4 means there have been five successive evaluations
of the Jacobian matrix without any significant reduction in $\Vert {\bf f}%
\Vert $, while 5 means there have been ten successive evaluations of ${\bf f}
$ without any significant reduction in $\Vert {\bf f}\Vert $. Typically the
algorithm will terminate with IOPT(1) = 4 or 5 when it is trapped in the
neighborhood of a local minimum of $\Vert {\bf f}\Vert $ at which ${\bf f}$
is not the zero vector.
\end{itemize}

When IOPT(1) = 2, 4, or 5 it is advisable to check the validity of the user
code, if any, for computing the Jacobian. Once that is assured it may be
useful to try different initial values for ${\bf x}.$

IOPT$(2)=nfev$. Number of function evaluations used, $i.e.$, the number of
times DNQFJ was called with IFLAG = 1.

IOPT$(3)=njev$. Number of evaluations of the Jacobian matrix, either by
finite-difference approximation or by calling DNQFJ with IFLAG =
2.\vspace{5pt}

Locations in IOPT() indexed from~4 on are available for selecting options.
The sequence of option selections must be terminated with a zero. For the
simplest usage, if DNQFJ contains code for computing the Jacobian matrix
just set IOPT(4) = 0, while if DNQFJ does not contain code for the Jacobian
set IOPT(4) = 1 and IOPT(5) = 0.

Options have code numbers from~1 to~8, and some options have one or two
integer arguments. The code numbers of options to be exercised, each
followed immediately by its arguments, if any, must be placed in IOPT()
beginning at IOPT(4), with no gaps, and terminated by a zero value. The
ordering of different options in IOPT() is not significant. If an option is
repeated the last occurrence prevails.

As a convenience for altering the set of selected options, the negative of
an option code is interpreted to mean the option is to take its default
value. If this is an option that has arguments, space for the appropriate
number of arguments must still be allocated following the option code, even
though the arguments will not be used.

The option codes and their arguments are as follows:

\begin{itemize}
\item[1]  $inoj$. Select this option if the subroutine DNQFJ does not
contain code for computing the Jacobian matrix. In this case no calls to
DNQFJ will be made with IFLAG = 2, and the algorithm will estimate the
Jacobian matrix by computing finite differences of values of ${\bf f}.$

\item[2]  $isetd$. Argument: $dmode$. Selects manner of initializing and
subsequently altering the weighting array, $diag$(), which is stored in W($4:%
\text{N}+3)$ and discussed in Section D.  {\em dmode} may be set to~1,~2, or~3. The
default value is~1.

\begin{itemize}
\item[1] DNQSOL will set $diag$() to all ones and not subsequently alter
these values.  Reference \cite{Powell:AHM:1970} states that, for most
tested problems, the MINPACK code from which the present package was
derived was most successful using this option.

\item[2]  The user must assign positive values to $diag$() before calling
DNQSOL. DNQSOL will not subsequently alter these values.

\item[3]  DNQSOL will initially set $diag(j)$ to be the Euclidian norm of
the $j^{th}$ column of the initial Jacobian matrix, for $j=1$, ..., N. If
the $j^{th}$ column is all zeros $diag(j)$ will be set to one. The value of
$diag(j)$ will be increased subsequently if the norm of the $j^{th}$ column
increases.
\end{itemize}

\item[3]  $iprint$. Argument: $nprint$. The algorithm increments an internal
counter, $ibest$, whenever a new value of ${\bf x}$ is accepted as giving a
significant reduction in $\Vert {\bf f}\Vert $. If $nprint >0$, DNQFJ will
be called with IFLAG $=0$ after the first evaluation of ${\bf f}$, after
every $nprint^{th}$ time $ibest$ is incremented, and on termination. DNQFJ is
expected to print X() and FVEC() on these occasions. The values in X() and
FVEC() on these calls will be those that have given the smallest value of $%
\Vert {\bf f}\Vert $ up to that point. Thus, these will not always be the
values associated with the most recent function evaluation. Setting $nprint
\leq $ 0 has the same effect as not exercising option $iprint$, $i.e.$, no
calls will be made to DNQFJ with IFLAG = 0.

\item[4]  {\em imxfev}. Argument: {\em maxfev}. The algorithm will terminate with $%
info=2 $ if convergence has not been attained after {\em maxfev} function
evaluations. Default value: $200 \times (\text{N}+1).$

\item[5]  $iband$. Arguments: $ml$ and $mu$. If DNQFJ will not be computing the
Jacobian matrix, $J$, and thus option $inoj$ is selected, and if $J$ has a
sufficiently narrow band structure, a reduction in the number of function
evaluations needed to estimate $J$ by finite differences can be effected by
informing the algorithm of the band structure. In such a case set $ml$ and $mu$
to values such that all the nonzero elements of $J$ lie within the first $ml$
subdiagonals, the main diagonal, and the first $mu$ superdiagonals. This only
reduces the number of function evaluations if $ml+mu+1<$ N.

\item[6]  {\em iepsfn}. Select this option if DNQFJ will not be computing the
Jacobian matrix, $J$, and thus option $inoj$ is selected, and you are
providing an estimate, {\em epsfcn}, in W(1), of the relative error expected in
function evaluations. {\em epsfcn} is used by the algorithm in selecting the
increment used in computing finite difference approximations to derivatives.
Default: {\em epsfcn} = machine precision obtained as D1MACH(4) (or R1MACH(4) in
single precision.)

\item[7]  $ifactr$. Select this option if you are providing a value, $factor$,
in W(2). {\em factor} provides a bound on the weighted length of the first step
the algorithm takes. Default: $factor$ = 0.75. See Section D for more
information on the role of $factor$.

\item[8]  $itrace$. This option activates printing of intermediate diagnostic
output. The output statements use unit `` * " to direct the output to the
standard system output unit.
\end{itemize}

\item[W()]  \ [in, out, scratch] An array of length IDIMW.

If option {\em iepsfn} is selected the user must store {\em epsfcn} in W(1). This value
will not be altered.

If option $ifactr$ is selected the user must store $factor$ in W(2). This value
will not be altered.

On return W(3) will contain a quantity, $toltst$, that can be regarded as an
estimate of the relative error in the final ${\bf x}$ in a weighted norm
sense. If ${\bf x}\neq 0$, $toltst=\Delta /\Vert Dx\Vert $, which is the
value compared with XTOL for the convergence test (see Section D).

If ${\bf x}=0$, $toltst=0.$

If option $isetd$ is selected with $dmode$ = 2, the user must store positive
values for $diag(1:\text{N})$ in W($4:\text{N}+3)$. In this case the
contents of W($4:\text{N}+3)$ will not be altered by DNQSOL. If $isetd$ is
not selected or if it is selected with $dmode$ = 1 or~3, values for $diag$()
will be stored in W($4:\text{N}+3)$ by DNQSOL.

W($\text{N}+4:3+(15 \times \text{N}+3 \times \text{N}^2)/2)$ is used as working
space.

\item[IDIMW]  \ [in] Dimension of W() array. Require IDIMW $\geq 3+(15
\times \text{N}+3 \times \text{N}^2)/2.$
\end{description}

\subsubsection{Modifications for Single Precision}

For single precision usage change the DOUBLE PRECISION statements to REAL
and change the initial letters of the subroutine names from ``D'' to ``S.''
It is recommended that one use the double precision rather than the
single precision version of this
package for better reliability, except on computers such as the Cray Y/MP
that have precision of about 14~decimal places in single precision.

\subsection{Examples and Remarks}

Consider the sample problem:%
\begin{align*}
\exp (-x_1)+\sinh(2x_2)+\tanh(2x_3) &= 5.01\\
\exp (2x_1)+\sinh(-x_2)+\tanh(2x_3) &= 5.85\\
\exp (2x_1)+\sinh(2x_2)+\tanh(-x_3) &= 8.88
\end{align*}
To use DNQSOL this must be put in the form of expressions whose desired
values are zeros. Thus, for instance we may define%
\begin{alignat*}{2}
f_1 &\equiv \exp &(-x_1)+\sinh(2x_2)+\tanh(2x_3)-5.01&=0\\
f_2 &\equiv \exp &(2x_1)+\sinh(-x_2)+\tanh(2x_3)-5.85&=0\\
f_3 &\equiv \exp &(2x_1)+\sinh(2x_2)+\tanh(-x_3)-8.88&=0
\end{alignat*}
The program DRDNQSOL illustrates the use of DNQSOL to solve this problem.
DRDNQSOL also illustrates the use of DCKDER of Chapter~8.3 to check the
mutual consistency of the code for function and Jacobian evaluation. Results
are shown in ODDNQSOL.

A problem ${\bf f}({\bf x}) = {\bf 0}$ may have no solutions, or one or more
solutions. The residual norm $\|{\bf f}({\bf x})\|$ may have more than one
local minimum, some with ${\bf f}$ not zero. The norm $\|{\bf f}({\bf x})\|$
can have a nonzero minimum only at a point where the Jacobian matrix is
singular. The user should choose an initial ${\bf x}$ that is as close to
the desired solution as possible. For a problem whose properties with regard
to multiplicity of solutions or local minima are not known, it may be
advisable to apply the subroutine several times using different initial $%
{\bf x}$'s to see if different termination points occur.

It is often useful, or even necessary, to limit the search to a specified
region. Subroutine DNQSOL does not have a specific feature for bounding
variables. One can limit the size of the initial step by setting $factor$
using option $ifactr$. If bounding of variables is definitely needed, one can
use the nonlinear least-squares package of Chapter~9.3. If a problem can be
solved by DNQSOL, this approach will generally require less execution time
than would the use of the Chapter~9.3 package.

Since the uniformly weighted Euclidean norm of ${\bf f}$ plays a central
role in the algorithm, it is generally advantageous to have the separate
components of ${\bf f}$ scaled so that an error of one unit in one component
of ${\bf f}$ has about the same importance as an error of one unit in any
other component.

If the user has estimates of positive numbers $dx_j$ such that a change of
magnitude $dx_j$ in $x_j$ is expected to have about the same effect on $\|%
{\bf f}\| $ as a change of magnitude $dx_i$ in $x_i$ for all $i$ and $j$, it
may be useful to set $diag(j) = 1/dx_j$ for $j = 1$, ..., $n$, by using the
option $isetd$. The overall scaling of values for $diag$() must be coordinated
with the choice of XTOL and $factor$ to assure that the convergence test (see
end of Section D) makes sense. For example it may be convenient to scale
$diag$() so its largest element is~1.

\subsection{Functional Description}

The algorithm attempts to find a vector ${\bf \hat x}$ satisfying ${\bf f}(%
{\bf \hat x})={\bf 0}$, starting the search from a given point ${\bf x}_0$.

The algorithm is a trust-region method, using a Broyden update to reduce the
number of times the Jacobian matrix must be recomputed, and using a
double-dogleg method of solving the constrained linearized problem within
the trust-region. For detailed descriptions of these terms and techniques,
see \cite{Dennis:1983:NMU}. A brief description follows:

Given a point, ${\bf x}$, a positive diagonal scaling matrix, D, and a
trust-region radius, $\Delta $, the algorithm attempts to find a step
vector, ${\bf p}$, that solves the modified problem:%
\begin{equation*}
\min \{\Vert {\bf f}({\bf x}+{\bf p})\Vert ^2:\Vert D{\bf p}\Vert \leq
\Delta \}
\end{equation*}
where $\Vert \cdot \Vert $ denotes the Euclidean norm. The algorithm
replaces this problem by the linearization,%
\begin{equation*}
\min \{\Vert {\bf f}+J{\bf p}\Vert ^2:\Vert D{\bf p}\Vert \leq \Delta \}
\end{equation*}
where ${\bf f}$ and $J$ denote the function vector and Jacobian matrix
evaluated at ${\bf x}$. This problem is approximately solved using a
double-dogleg method that further restricts the step vector ${\bf p}$ to be
in the two-dimensional subspace spanned by the Gauss-Newton direction, $%
-J^{-1}{\bf f}$, and the scaled gradient descent direction, $-D^{-2}J^t{\bf f}.$

The region $\{{\bf x}+{\bf p} : \|D{\bf p}\| \leq \Delta \}$ is called the
trust-region associated with ${\bf x}$. The algorithm attempts to regulate
the size of $\Delta $ so the linear function ${\bf f}+J{\bf p}$ will be a
reasonably good approximation to the nonlinear function ${\bf f}({\bf x}+%
{\bf p})$ within the trust-region.

Diagonal elements of D are stored in the array $diag$(), which is initialized,
and optionally updated, as described in Section B. The trust-region radius $%
\Delta $ is initially computed as $\Delta = factor \times \|{\bf x}_0\|$, if $%
\|{\bf x}_0\| > 0$, and as $\Delta = factor$, otherwise. The initialization of
$factor$ is described in Section B.

Let ${\bf p}$ denote the step vector determined by this process and define $%
{\bf x}_+ = {\bf x} + {\bf p}$. If $\|{\bf f}({\bf x}_+)\| \geq
\|{\bf f}({\bf x})\|$ the algorithm will reduce $\Delta $ and solve for
a new ${\bf p}.$

If a smaller value of $\Vert {\bf f}\Vert $ is found, the algorithm computes
an estimate of the Jacobian at the new point and starts the step
determination process again. Rather than always computing the new Jacobian
by calling the user-supplied code or using finite differences, the algorithm
first tries to use an updated Jacobian which is more economical to compute.
The update formula, due to Broyden, is%
\begin{equation*}
J_{+}=J+\frac{\left( {\bf f}({\bf x}+{\bf p})-{\bf f}({\bf x})-J{\bf p}%
\right) \left( D^tD{\bf p}\right) ^t}{\Vert D{\bf p}\Vert ^2}
\end{equation*}
The algorithm computes and maintains a QR factorization of $J$. The Broyden
update process is applied to the QR factorization of J.

If progress is unsatisfactory when based on an updated Jacobian, the
algorithm will recompute the Jacobian using the user-supplied code for $J$,
if available, or otherwise by finite differences of user-computed values of $%
{\bf f}.$

As the algorithm makes progress toward a zero of ${\bf f}$, the step ${\bf p}
$ will generally become smaller. If also the trust-region seems reliable, in
the sense that the relative reduction in $\Vert {\bf f}\Vert ^2$ is within
10\% of the relative reduction predicted by the linear model, then $\Delta $
will be reset to $2\Vert D{\bf p}\Vert $ so that $\Delta $ decreases as $%
\Vert D{\bf p}\Vert $ decreases. Convergence to a zero of ${\bf f}$ is
assumed to have occurred when%
\begin{equation*}
\Delta \leq \max (\text{XTOL},\epsilon )\times \Vert D{\bf x}\Vert ,
\end{equation*}
or if an ${\bf x}$ is found for which ${\bf f}({\bf x})={\bf 0}$, exactly.
In these cases the subroutine returns with IOPT(1) = 1.

If the algorithm gets trapped in the neighborhood of a local minimum of $\|%
{\bf f}\|$ at which ${\bf f}$ is not zero, the step ${\bf p}$, and thus the
trust-region radius, $\Delta $, will generally not get small. The Jacobian $%
J $ will be singular at such a local minimum. These conditions generally
result in a return with IOPT$(1) = 4$ or~5 or possibly~2.

\bibliography{math77}
\bibliographystyle{math77}

\subsection{Error Procedures and Restrictions}

Notice of any detected error condition is sent to the error message printing
routines of Chapter~19.2 using error level 0, and the subroutine will return
with a value of $info$ in IOPT(1) other than~0. The action of the error
printing routines can be altered as described in Chapter~19.2.

\subsection{Supporting Information}

\begin{tabular}{@{\bf}l@{\hspace{5pt}}l}
\bf Entry & \hspace{.35in} {\bf Required Files}\vspace{2pt} \\
DNQSOL & \parbox[t]{2.7in}{\hyphenpenalty10000 \raggedright
AMACH, DERV1, DNQSOL, DNRM2, ERFIN, ERMSG, IERM1, IERV1\rule[-5pt]{0pt}{8pt}}\\
SNQSOL & \parbox[t]{2.7in}{\hyphenpenalty10000 \raggedright
AMACH, ERFIN, ERMSG, IERM1, IERV1, SERV1, SNQSOL, SNRM2}\\
\end{tabular}

The source language is ANSI Fortran~77.

The original MINPACK version of this code was designed and programmed by
the authors of \cite{More:1980:MIN} and \cite{Garbow:1980:MIN} where this
code is called HYBRD and HYBRJ.  They, in turn, cite
\cite{Powell:AHM:1970} for the general strategy of their approach and
numerous other authors for additional ideas.  The MINPACK code was
downloaded from $netlib$ to JPL by C.  L.  Lawson in February~1990.  The
subroutine names and the user interface were modified by Lawson and F.  T.
Krogh in~1990 and 1991 to conform to the MATH77 library style.  Changes to
the logic of the algorithm have been made that reduced the execution time
and improved the reliability of the code when applied to the set of
fourteen test problems included with the MINPACK package.


\begcode

\medskip\
\lstset{language=[77]Fortran,showstringspaces=false}
\lstset{xleftmargin=.8in}

\centerline{\bf \large DRDNQSOL}\vspace{0pt}
\lstinputlisting{\codeloc{dnqsol}}

\vspace{10pt}\centerline{\bf \large ODDNQSOL}\vspace{5pt}
\lstset{language={}}
\lstinputlisting{\outputloc{dnqsol}}
\end{document}
